\documentclass[fleqn,usenatbib]{mnras}
\usepackage[T1]{fontenc}
\usepackage{ae,aecompl}
\usepackage{graphicx}	
\usepackage{amsmath}	
\usepackage{amssymb}
\usepackage{pdflscape}
\usepackage{siunitx}
\def\lya{Ly$\alpha$~}
\usepackage{color}
\definecolor{notecolor}{rgb}{0.8,0,0}
\newcommand{\gk}[1]{{\bf \color{notecolor} [#1]}}
\def\HI{\hbox{H~$\scriptstyle\rm I$}}
\def\HII{\hbox{H~$\scriptstyle\rm II$}}
\def\nHI{{\rm HI}}
\def\nH{{\rm H}}
\def\nHII{{\rm HII}}
\def\nHe{{\rm He}}
\def\nHeI{{\rm HeI}}
\def\nHeII{{\rm HeII}}
\def\nHeIII{{\rm HeIII}}
\def\bHII{\hbox{\bf H~$\scriptstyle\bf II$}}
\def\HeI{\hbox{He~$\scriptstyle\rm I$}}
\def\HeII{\hbox{He~$\scriptstyle\rm II$}}
\def\HeIII{\hbox{He~$\scriptstyle\rm III$}}
\def\bHeII{\hbox{\bf He~$\scriptstyle\bf II$}}
\def\HeIII{\hbox{He~$\scriptstyle\rm III$}}
\usepackage{dcolumn}
\newcolumntype{.}{D{.}{.}{2.7}}
\newcolumntype{e}{D{.}{.}{2.10}}
\newcolumntype{d}{D{.}{.}{2.3}}

\title[AGN luminosity function]{Evolution of the AGN UV
  luminosity function from redshift 7.5}

\author[Kulkarni et al.]
       {{Girish Kulkarni$^{1}$\thanks{Email: kulkarni@ast.cam.ac.uk},
           G\'abor Worseck$^{2}$
           and Joseph F.~Hennawi$^{3}$} \\
         $^1$Institute of Astronomy and Kavli Institute of Cosmology,
         University of Cambridge, Madingley Road, Cambridge CB3 0HA,
         UK \\
         $^2$Institute f\"ur Physik und Astronomie, Universit\"at
         Potsdam, Karl-Liebknecht-Stra\ss e\ 24/25, 14476 Potsdam,
         Germany \\
         $^3$Department of Physics, Broida Hall, UC Santa Barbara,
         Santa Barbara, CA 93106-9530 USA}
       \date{Accepted ---. Received ---; in original form ---}

\pubyear{2016}

\begin{document}
\label{firstpage}
\pagerange{\pageref{firstpage}--\pageref{lastpage}}
\maketitle

\begin{abstract}
  Determinations of the UV luminosity function of AGN at high
  redshifts are important for constraining the AGN contribution to
  reionization and understanding the growth of supermassive black
  holes.  Recent inferences of the luminosity function suffer from
  inconsistencies arising from inhomogeneous selection and mixing of
  AGN data.  We address this problem by constructing a sample of more
  than 80,000 colour-selected AGN from redshift $z=0$ to $7.5$.  While
  this sample is composed of multiple data sets with spectroscopic
  redshifts and completeness estimates, we homogenise these data sets
  to identical cosmologies, intrinsic AGN spectra, and magnitude
  systems.  Using this sample, we derive the AGN UV luminosity
  function from redshifts $z=0$ to $7.5$.  The luminosity function has
  a double power law form at $z<5.5$, and a single power law form at
  higher redshifts.  The break luminosity brightens rapidly with
  redshift and the faint end of the luminosity function steepens.  In
  spite of this steepening, the contribution of AGN to the hydrogen
  photoionization rate at $z\sim 6$ is subdominant, although it can be
  non-negligible if these luminosity functions hold down to
  $M_{1450}=-18$.  Under reasonable assumptions, AGN can reionize
  \HeII\ by redshift $z=3.2$.  At low redshifts ($z<0.5$), AGN can
  produce all of the hydrogen photoionization rate inferred from the
  statistics of \HI\ absorption lines in the IGM.  Our global analysis
  of the luminosity function also reveals important systematic errors
  in the data, particularly at $z=2.2$--$3.5$, which need to be
  addressed in future in order to improve our results.  We make
  various fitting functions, luminosity function analysis codes, and
  homogenized AGN data publicly available.
\end{abstract}

\begin{keywords}
  dark ages, reionization, first stars -- intergalactic medium --
  quasars: general -- galaxies: active
\end{keywords}

\section{Introduction}

The luminosity function of active galactic nuclei (AGN) and its
evolution over cosmological time scales has been a matter of central
interest of a large body of work over the last four decades
\citep[e.g.,][]{1978A&A....68...17M, 1983ApJ...269..352S,
  1988ApJ...325...92K, 1988MNRAS.235..935B, 1993ApJ...406L..43H,
  1994ApJ...421..412W, 1995AJ....110...68S, 1995AJ....110.2553K,
  1995ApJ...438..623P, 2000MNRAS.317.1014B, 2001AJ....121...54F,
  2004AJ....128..515F, 2006AJ....131.2766R, 2007ApJ...654..731H,
  2009MNRAS.392...19C, 2010AJ....139..906W, 2011ApJ...728L..26G,
  2013ApJ...773...14R, 2013ApJ...768..105M, 2015AA...578A..83G,
  2015ApJ...798...28K, 2016ApJ...829...33Y, 2016ApJ...833..222J,
  2017MNRAS.466.1160M}.  Determination of the AGN luminosity function
constrains models of the build-up of supermassive black holes
\citep{2015MNRAS.452..575S, 2016MNRAS.462..190R}.
%% JFH Some historical papers should be cited for this SMBH buildup point, like
%% for example Soltan. 
Due to the
incidence of supermassive black holes in most galaxies, the tight
scaling relations observed to exist between the mass of these black
holes and properties of their host galaxies
\citep{2013ARA&A..51..511K, 2013ApJ...764..184M}, and the increasing
consensus that AGN activity feeds back on the host galaxy evolution,
the AGN luminosity function also constrains models of galaxy
formation.  Finally, thanks to their relative brightness and high
Lyman-continuum (LyC) escape fractions, the luminosity function of AGN
determines their influence on the temperature and ionization state of
the intergalactic medium (IGM) over large scales,
%% JFH Probably want to say something here about radiation background
%% instead of this ``large scales'' 
possibly even
primarily driving hydrogen and helium reionization.  

There is a renewed interest in understanding the evolution of the UV
luminosity function of AGN, triggered by the 19 low-luminosity
($M_{1450}>-22.6$)
%% JFH Probably more sensible to quote a luminosity range rather than just a > 
AGN candidates between redshifts $z=4.1$ and $6.3$
discovered by \citet{2015AA...578A..83G} using a novel X-ray/NIR
selection criterion.  This finding suggested that AGN brighter than
$M_{1450}=-18$ can potentially produce all of the metagalactic
hydrogen photoionization rate inferred from the \lya forest at
$4<z<6$.
%% JFH Probably should say something here about assuming all ionizing photons escape, since these
%% are galaxy like luminosities. Maybe quote L_star galaxy for comparison, which helps with your next
%% point about faint galaxies. 
A significant presence of AGN at high redshift ($z\sim 6$)
and a dominant contribution of AGN to reionization is appealing as the
LyC escape fraction of galaxies is uncertain.  High-redshift galaxies
down to rest-frame UV magnitude $M_\mathrm{UV}=-12.5$ ($L\sim
10^{-3}L^*$) at $z=6$ \citep{2017ApJ...835..113L} and redshifts up to
$z=11.1$ \citep{2016ApJ...819..129O} have now been reported.  But the
escape of LyC photons has been measured in only a few comparatively
bright ($L>0.5L^*$) galaxies at relatively low redshifts ($z < 4$).
The escape fraction in these galaxies is inferred to be typically
about 2--20\% \citep{2010ApJ...725.1011V, 2011ApJ...736...41B,
  2015ApJ...804...17S, 2015ApJ...810..107M, 2016A&A...585A..48G,
  2017MNRAS.468..389J, 2017MNRAS.465..316M}, whereas reionization
requires escape fraction of about 20\% in galaxies as faint as
$M_\mathrm{UV}=-13$ \citep{2016PASA...33...37F, 2015ApJ...802L..19R,
  2016MNRAS.457.4051K}.  An enhanced incidence of high-redshift AGN
may also be consistent with the shallow bright-end slopes of the
$z\sim 7$ UV luminosity function of galaxies relative to a Schechter
function \citep{2012MNRAS.426.2772B, 2014MNRAS.440.2810B,
  2014ApJ...792...76B, 2015MNRAS.452.1817B} and the hard spectra of
these galaxies \citep{2015MNRAS.450.1846S, 2015MNRAS.454.1393S,
  2017MNRAS.464..469S}.  Finally, AGN may also provide a natural
explanation for the large scatter in the \lya opacity between
different quasar sightlines close to redshift $z=6$
\citep{2015MNRAS.447.3402B, 2018arXiv180208177B}.
%% JFH Please cite new paper by eilers et al. here, not yet on archive. 
These opacity
fluctuations extend to substantially larger scales ($\gtrsim 50\,
h^{-1}$cMpc) than expected in galaxies-dominated reionization models
\citep{2015MNRAS.453.2943C}.
%% JFH You probably need a ``but see Davies et al. 201X'' since Fred claims
%% he can explain these fluctuations with galaxies. See also the new paper with George Becker and Fred. 
AGN clustering can result in these
fluctuations naturally if there is a significant contribution
($\gtrsim 50\%$) of AGN to the ionising emissivity at $z=5$--$6$
\citep{2017MNRAS.465.3429C}.

Several counter-arguements against a higher incidence of AGN at high
redshift have also been made. \citet{2017MNRAS.468.4691D} and
\citet{2018MNRAS.473.1416M} pointed out that an AGN population
consistent with the findings of \citet{2015AA...578A..83G} leads to a
much earlier \HeII\ reionization.  For instance, in the model of
\citet{2015ApJ...813L...8M}, He~\textsc{ii} reionization is complete
at $z=4.5$, compared to $z=3$ in the standard scenario
\citep{2012ApJ...746..125H}.  Such early \HeII\ reionization could
result in higher IGM temperatures due to the associated photoheating.
\citet{2017MNRAS.468.4691D} found that the temperature of the IGM at
mean density is twice as much in AGN-dominated reionization models as
the standard models at $z=3.5$--$5$, in conflict with constraints from
the \lya forest.  This inconsistency could be avoided by postulating a
reduced escape fraction of \HeII-ionizing photons in AGN, but it is
difficult to reconcile this with a unit escape fraction of
hydrogen-ionizing photons that is required to explain the Ly$\alpha$
opacity fluctuations.
%% JFH Probably need to clarify for this last point that you mean in a situation
%% where galaxies contribute no LL photons, and QSOs completely dominate. 
Further arguments against AGN-dominated
reionization have been presented by \citet{2016MNRAS.459.2299F} by
analysing the incidence of metal-line absorbers at $z\sim 6$.
\citet{2016MNRAS.459.2299F} find that in their cosmological radiation
hydrodynamical simulations AGN-dominated UV background results in too
many C~\textsc{iv} absorption systems relative to Si~\textsc{iv} and
C~\textsc{ii} at $z\sim 6$.  However, these simulations assume a
$L_\nu\propto\nu^{-1.57}$ AGN SED at extreme UV
\citep{2002ApJ...565..773T}, which somewhat harder than recent
measurements ($L_\nu\propto\nu^{-1.7}$) by
\citet{2015MNRAS.449.4204L}.  \citet{2016MNRAS.459.2299F} also found
that the N(Si~\textsc{iv})/N(C~\textsc{iv}) column density ratio
measurements prefer a harder and more intense $>4$~Ry background than
the standard model \citep{2012ApJ...746..125H}.  Finally, comparing
the \citet{2015AA...578A..83G} sample to X-ray-selected quasar data at
$z=0$--$6$, \citet{2017MNRAS.465.1915R} argued that the faint end of
the AGN UV luminosity function at $z\sim 6$ is probably shallower that
that reported by \citet{2015AA...578A..83G}.
\citet{2017MNRAS.465.1915R} argue that the apparent contradiction with
the results of \citet{2015AA...578A..83G} could be explained by
contamination from the AGN host galaxies for faint AGN.

While a straightforward way towards a more robust understanding of AGN
contribution to reionization is to determine the evolution of the AGN
UV luminosity function, many recent inferences of the luminosity
function suffer from inconsistencies arising from inhomogeneous
selection and mixing of AGN data.  This has resulted in a large
scatter in the inferred high-redshift hydrogen-ionizing AGN emissivity
between various recent studies.  For instance, there is an order of
magnitude scatter between various estimates of the hydrogen-ionizing
AGN emissivity at $z=5$--$6$ \citep{2011ApJ...728L..26G,
  2012ApJ...755..169M, 2015AA...578A..83G, 2018PASJ...70S..34A,
  2018AJ....155..131M, 2018MNRAS.474.2904P, 2017ApJ...847L..15O,
  2017MNRAS.466.1160M}.  Our aim in this paper is to address this
issue, by constructing an AGN sample with robust redshift and
completeness estimates and homogeneous assumptions of the cosmology
and intrinsic AGN spectrum.  We discuss our sample construction in
Section~\ref{sec:sample}.  Our derived luminosity functions are
presented in Section~\ref{sec:lf}.  Section~\ref{sec:reion} presents
our inference of the AGN contribution to hydrogen and helium
reionization.  We summarise our findings in Section~\ref{sec:conc}.

We assume a flat cosmology with density parameters
$\left(\Omega_\mathrm{m},\Omega_\Lambda\right)=\left(0.3,0.7\right)$
and a Hubble constant $H_0=70$\,km\,s$^{-1}$\,Mpc$^{-1}$. Comoving
distances are given explicitly in comoving Mpc (cMpc). Magnitudes
are reported in the AB system \citep{1983ApJ...266..713O}, and
observed magnitudes are point spread function (PSF) magnitudes
\citep{2002AJ....123..485S} corrected for Galactic extinction
\citep{1998ApJ...500..525S} unless otherwise noted.
Our homogenised sample (Section~\ref{sec:sample}) uses absolute
monochromatic AB magnitudes at a rest frame wavelength of 1450\,\AA.


\begin{figure*}
  \begin{center}
    \includegraphics[width=\textwidth]{qsos.pdf}
    % data.py 
  \end{center}
  \caption{Redshift distribution of the 83,488 AGN considered in this
    analysis.  Shown here are the observed AGN numbers, without
    correcting for incompleteness.  Further details on each of these
    data sets are in Table~\ref{tab:samples} and
    Section~\ref{sec:sample}.}
  \label{fig:qsos}
\end{figure*}

\begin{table*}
  \caption{AGN samples analysed in this work.}
  \label{tab:samples}
  \begin{tabular}{lcllrS}
    \hline
    Sample& $z$ range$^a$& Survey & Reference & Number & {Area} \\
    & & & & of quasars & {(deg$^2$)} \\
    \hline
    1 & 0.0--2.2 & SDSS DR7 & \citet{2010AJ....139.2360S} & 48664 & 6248.0 \\
    2$^b$ & 0.4--2.2 & 2SLAQ SGP & \citet{2009MNRAS.392...19C} & 2338 & 64.2 \\
    3$^b$ & 0.4--2.2 & --- NGP & \citet{2009MNRAS.392...19C} & 7027 & 127.7 \\
    4 & 2.2--3.5 & BOSS DR9 & \citet{2013ApJ...773...14R} & 23301 & 2236.0 \\
    5 & 3.7--4.7 & SDSS DR7 & \citet{2010AJ....139.2360S} & 1785 & 6248.0 \\
    6 & 3.6--5.2 & NDWFS & \citet{2011ApJ...728L..26G} & 12 & 1.71 \\
    7 & 3.8--5.3 & DLS & \citet{2011ApJ...728L..26G} & 12 & 2.05 \\
    8 & 4.7--5.4 & SDSS+WISE & \citet{2016ApJ...829...33Y} & 99 & 14555.0 \\
    9$^c$ & 4.7--5.5 & SDSS DR7 & \citet{2013ApJ...768..105M} & 103 & 6248.0 \\
    10$^c$ & 4.7--5.5 & --- Stripe 82 & \citet{2013ApJ...768..105M} & 59 & 235.0 \\
    11 & 5.7--6.5 & SDSS Main & \citet{2016ApJ...833..222J} & 24 & 11240.0 \\
    12 & 5.7--6.5 & --- Overlap & \citet{2016ApJ...833..222J} & 10 & 4223.0 \\
    13 & 5.7--6.5 & --- Stripe 82 & \citet{2016ApJ...833..222J} & 13 & 277.0 \\
    14 & 5.8--6.6 & CFHQS Deep & \citet{2010AJ....139..906W} & 1 & 4.47 \\
    15 & 5.8--6.6 & --- Very Wide & \citet{2010AJ....139..906W} & 16 & 494.0 \\
    16 & 5.8--6.5 & Subaru High-$z$ Quasar & \citet{2015ApJ...798...28K} & 2 & 6.5 \\
    17$^d$ &4.0--6.5 & CANDELS GOODS-S & \citet{2015AA...578A..83G} & 19 & 0.047 \\
    18$^e$ & 6.5--7.4 & UKIDSS & \citet{2011Natur.474..616M} & 1 & 3370.0 \\
    19$^e$ & 6.5--7.4 & UKIDSS & \citet{2015ApJ...801L..11V} & 1 & 3370.0 \\
    20$^e$ & 6.5--7.4 & ALLWISE+UKIDSS+DECaLS & \citet{2018Natur.553..473B} & 1 & 2500.0 \\
    \hline
  \end{tabular}\\
  \begin{minipage}{14.5cm}
    \textsuperscript{$a$}{Redshift range of the sample or, for small
      samples, approximate redshift range in which the survey is
      sensitive.}\\
    \textsuperscript{$b$}{Restricted to $z<2.2$.}\\
    \textsuperscript{$c$}{Restricted to $M_{1450}>-26.73$ quasars to
      avoid overlap with the \citet{2016ApJ...829...33Y} sample.}\\
    \textsuperscript{$d$}{Used only in Section~\ref{sec:global} and
      Appendix~\ref{sec:conv} due to lack of spectroscopic redshifts
      for majority of the sample.}\\
    \textsuperscript{$e$}{Used only in Section~\ref{sec:global} due to
      roughly estimated selection function.}
 \end{minipage}
\end{table*}

\section{Homogenised Quasar Sample}
\label{sec:sample}

\subsection{Sample Selection}

We started with
%% JFH we started with --> we started by
compiling the samples of recent photometric rest-frame UV-optical
quasar surveys. The restriction to UV-optical surveys was mainly
driven by our science goal to characterise the UV luminosity function
of Type~1 quasars. X-ray-selected samples are less suited for this
purpose due to spectroscopic incompleteness and the $\sim 0.4$\,dex
scatter in the conversion from X-ray to UV luminosity
\citep{2010A&A...512A..34L, 2015MNRAS.453.1946G, 2016ApJ...819..154L}
that contributes significantly to the error budget in the UV
luminosity function of X-ray-selected samples unless rest-frame UV
photometry is incorporated \citep{2015AA...578A..83G}.  The individual
surveys and their main characteristics are listed in
Table~\ref{tab:samples}.  Figure~\ref{fig:qsos} presents a redshift
histogram of the contributing surveys.

We included surveys based on a set of simple criteria:
\begin{enumerate}
\item Spectroscopic redshifts for the vast majority of targets.
  %% JFH Did we actually use any photometric redshifts? If not, I would omit this ``Vast majority'' and make
  %% it more definite. 
\item Accurate rest-frame UV-optical CCD photometry.
\item Statistical power (sample size, coverage in $z$ and/or absolute magnitude).
\item A characterised selection function.
  %% JFH well characterized
\end{enumerate}
As a prerequisite for a joint analysis of the QLF we obtained the
survey selection functions in electronic form, either from the
publication or
%% JFH by request 
from the authors. As a reference for future surveys we
publish them here
%% JFH rather than publish them here say ``we make them electronically available here''. Cite an appendix where
%% you do this. 
in modified and homogenised form
(Section~\ref{sect:datahom}).

Due to their selection criteria and their statistical power specific
surveys contribute to distinct redshift ranges. At $z<2.2$ we
considered quasars from the SDSS DR7 quasar catalogue
\citep{2010AJ....139.2360S} and the 2SLAQ survey catalogue
\citep{2009MNRAS.392...19C}. We restricted the SDSS DR7 sample to the
48,664 $0.1<z<2.2$ quasars selected with the final SDSS quasar
selection algorithm \citep{2002AJ....123.2945R, 2006AJ....131.2766R}
from a survey area of 6248\,deg$^2$ \citep{2012ApJ...746..169S}. We
adopted the SDSS targeting photometry corrected for Galactic
extinction \citep{2010AJ....139.2360S}. To limit systematic
uncertainties in the correction for host galaxy light
%% JFH What is Croom the reference for here, host galaxy contamination or the 2SLAQ sample. You basically
%% need a reference for 2SLAQ and it appears in the wrong place here. 
\citep{2009MNRAS.392...19C} we restricted the 2SLAQ sample to 11,816
$g<21.85$ $0.4<z<2.2$ quasars from its spectroscopic survey footprint
near the North Galactic Pole (NGP, 8153 quasars in $127.7$\,deg$^2$)
and the South Galactic Pole (SGP, 3663 quasars in
$64.2$\,deg$^2$). The small sample overlap between SDSS and 2SLAQ (112
quasars) has negligible impact on the QLF evaluation.
%% JFH Maybe state what the 2SLAQ survey goal was re quasars. 

At $2.2<z<3.5$ we used a single sample of 23,301 uniformly
colour-selected quasars from 2236\,deg$^2$ in BOSS DR9
\citep{2013ApJ...773...14R} due to several improvements compared to
previous surveys. First, it covers a similar magnitude range as 2SLAQ
but with $>20$ times as many quasars. Second, although the SDSS DR7
sample provides better coverage of the bright end of the QLF at these
redshifts, its selection function is highly dependent on the assumed
incidence of (partial) Lyman limit systems in the IGM
\citep{2009ApJ...705L.113P, 2011ApJ...728...23W}. While the BOSS DR9
selection function considers these improvements, the uncertainty in
the QLF remains dominated by assumptions in the selection function
given the large sample size
\citep{2013ApJ...773...14R}. Variability-selected quasar samples
circumvent this issue \citep{2013ApJ...773...14R, 2013A&A...551A..29P,
  2016A&A...587A..41P}, but may be affected by (i) single-epoch
imaging incompleteness at the faint end \citep{2013ApJ...773...14R},
and (ii) uncertainties in the selection function caused by the limited
number of known $z\ga 3$ quasars not selected by variability in the
same footprint \citep{2013A&A...551A..29P, 2016A&A...587A..41P}.

At $3.7<z<4.7$ we used a combination of SDSS DR7 \citep[1785 uniformly
  selected quasars from][]{2010AJ....139.2360S} and the NDWFS$+$DLS
survey \citep{2010ApJ...710.1498G,2011ApJ...728L..26G}. The lower cut
$z>3.7$ in SDSS limits the impact of systematic uncertainties in the
\citet{2006AJ....131.2766R} selection function
\citep{2009ApJ...705L.113P, 2011ApJ...728...23W}. We did not consider
the results from surveys for faint $z\sim 4$ quasars in the COSMOS
field \citep{2011ApJ...728L..25I, 2012ApJ...755..169M} due to
systematic errors in their selection functions\footnote{Both studies
  simulated quasar colours with a mean IGM attenuation curve
  \citep{1995ApJ...441...18M} that cannot account for stochastic Lyman
  continuum absorption, and therefore underpredicts the variance in
  quasar colours \citep{1999ApJ...518..103B, 2008MNRAS.387.1681I,
    2011ApJ...728...23W}. Modelling the colour variance in these
  surveys is essential, as most of the \citet{2011ApJ...728L..25I}
  quasars are near the edge of their colour selection region (see
  their Figure~1), and \citet{2012ApJ...755..169M} require modest
  attenuation of the $U$ band flux relative to the mid-infrared
  flux.}. Furthermore, 30 per cent of the \citet{2012ApJ...755..169M}
COSMOS sample have visually estimated photometric redshifts, and the
spectroscopic subsample reveals that 40 per cent of the visually
estimated redshifts are biased low
($z_\mathrm{spec}>z_\mathrm{est}+0.3$, see their Figure~9). These
unaccounted systematic redshift errors at least partly explain the
discrepancy in the $z\sim 4$ QLF between \citet{2011ApJ...728L..26G}
and \citet{2012ApJ...755..169M}, which justifies our preference for
the former sample that is 77 per cent spectroscopically complete at
$R$ magnitudes $<23.5$ \citep{2011ApJ...728L..26G}.


At $4.7\le z<5.5$ we combined several recent surveys, accounting for
sample overlap and updated selection functions. At the bright end of
the QLF we used the 99 quasars from the SDSS+WISE survey
\citep{2016ApJ...829...33Y} that have $M_{1450}<-26.73$ in our adopted
cosmology.  For these 99 quasars selected from 14,555\,deg$^2$ we
adopted the \citet{2016ApJ...829...33Y} selection function.  The
\citet{2016ApJ...829...33Y} sample partially overlaps with the SDSS
DR7 sample from \citet{2013ApJ...768..105M}, so to avoid
double-counting quasars we used the latter sample only at $M_{1450}\ge
-26.73$, yielding 104 additional $4.7\le z<5.5$ quasars selected in
6248\,deg$^2$. We used the $z\sim 5$ SDSS DR7 selection function from
\citet{2013ApJ...768..105M} that supersedes the one from
\citet{2006AJ....131.2766R} due to improved bandpass corrections and IGM
parameterization.  To these two bright-end samples we added the
faint-end sample from the \citet{2013ApJ...768..105M} SDSS Stripe~82
survey (62 uniformly selected $4.7\le z<5.5$ quasars in 235\,deg$^2$)
and two $4.7\le z<5.5$ quasars from \citet{2011ApJ...728L..26G},
adopting the respective selection functions.  We did not consider the
limit on the $z\sim 5$ QLF by \citet{2012ApJ...756..160I} due to
systematic errors in their selection
function\footnote{\citet{2012ApJ...756..160I} underestimated the
  dispersion in rest-frame UV quasar colours with respect to SDSS at
  all redshifts (their Figure~4). Contrary to their claim, photometric
  errors have a small effect on the colour distribution of SDSS
  quasars given the statistical errors of $<0.03$\,mag in $gri$ for 90
  per cent of the SDSS DR7 bright quasar sample ($i<19.1$) and a
  relative calibration error of $\sim 1$ per cent
  \citep{2008ApJ...674.1217P}.}.

The SDSS colours of $5.1<z<5.5$ quasars are similar to those of M and
L dwarf stars, resulting in a low and uncertain selection efficiency
\citep{2013ApJ...768..105M}. WISE mid-infrared selection performs
better \citep{2016ApJ...829...33Y}, but is restricted to the bright
end of the quasar population. Unlike \citet{2013ApJ...768..105M} we
include the 9 uniformly selected $M_{1450}\ge -26.73$ SDSS DR7 quasars
and the 10 SDSS Stripe~82 quasars at $z>5.1$, adopting their low
selection efficiency.
%% JFH efficiency and completeness are two different things and you seem to be conflating
%% the two concepts here. 
As we will show in Section~3.2, the resulting
QLF is consistent with those at lower and higher redshifts, indicating
that the \citet{2013ApJ...768..105M} selection functions are quite
reliable.

At $z\sim 6$ we combined the samples from all spectroscopic surveys
with a determined selection function as of June 2017.
\citet{2016ApJ...833..222J} recently compiled all quasars discovered
in several SDSS $z\sim 6$ surveys together with consistently derived
selection functions. Their uniform sample consists of 24 quasars from
the SDSS main survey (11,240\,deg$^2$), 10 additional quasars in
regions with two or more SDSS imaging scans (so-called overlap
regions, 4223\,deg$^2$), and 13 faint quasars from SDSS Stripe~82
(277\,deg$^2$). The CFHQS \citep{2010AJ....139..906W} provided a
uniform sample of 16 quasars in the Very Wide Survey (494\,deg$^2$)
and a single quasar in the Deep Survey ($4.47$\,deg$^2$). The one
quasar detected in both SDSS and CFHQS does not lead to underestimated
statistical errors in the QLF. Lastly, we included the two objects
from \citet{2015ApJ...798...28K}, one of which might be a galaxy due
to its narrow Ly$\alpha$ emission line (half width at half maximum
427\,km\,s$^{-1}$). With better photometry and additional spectroscopy
recently reported by \citet{2017ApJ...847L..15O} the
\citet{2015ApJ...798...28K} sample is complete.  Although we will
account for the slightly different redshift sensitivities for the
different surveys, we will quote a nominal redshift range $5.7<z<6.5$
for the combined $z\sim 6$ sample.

At the highest redshifts $z>6.5$ we considered two quasars found by
\citet{2011Natur.474..616M} and \citet{2015ApJ...801L..11V} in UKIDSS
imaging (3370\,deg$^2$)
%% JFH quote the individual redshifst of the mortlock and venemans objects
, in addition to the current highest-redshift
quasar at $z=7.54$ selected from a combination of UKIDSS, WISE and
DECaLS \citep[$\sim 2500$\,deg$^2$, ][]{2018Natur.553..473B}. Although
currently only rough estimates exist concerning their selection
functions, these quasars provide constraints on the evolution of the
integrated quasar space density from $z\sim 6$ to $z\sim 7$.  We use
them in our secondary analysis in Section~\ref{sec:global}.  Although
we do not include the highly debated \citet{2015AA...578A..83G} sample
in our main analysis due to its rough selection function and lacking
%% JFH lacking --> lack of 
spectroscopy for 17 of the 22 quasar candidates, we use it in
Section~\ref{sec:global} to constrain the faint end ($M_{1450}>-23$)
of the QLF at $z>4.1$. We restricted the \citet{2015AA...578A..83G}
sample to the 19/22 sources considered in their QLF.

\subsection{Sample Homogenisation}
\label{sect:datahom}

%% JFH This section is long tedious and hard to follow. Perhaps we
%% should put a brief description here, relegate these details to the
%% appendix, and in the appendix, structure this section by
%% sub-sections for each individual survey.

\begin{figure}
    \includegraphics[width=\columnwidth]{kcorr.pdf}
  \caption{Bandpass corrections $K_{m,1450}$ from a broadband
    magnitude $m=\{g,i,z_\mathrm{AB}\}$ to the monochromatic AB
    magnitude at 1450\,\AA\ as a function of redshift $z$ for the
    \citet{2015MNRAS.449.4204L} quasar SED used in this work, and for
    two quasar composite spectra \citep{2001AJ....122..549V,
      2002ApJ...565..773T}.  The redshift range has been restricted to
    exclude the Ly$\alpha$ forest and to account for the different
    rest frame wavelength coverage of the spectra.}
  \label{fig:kcorr}
\end{figure}

For a joint fit of the QLF it is necessary to homogenise the different
survey samples in absolute magnitude, and to convert their selection
functions to the same absolute magnitude system. For the analysis of
the quasar UV emissivity and to be consistent with published work at
$z>3$ we chose to convert all samples and selection functions to the
absolute AB magnitude at a rest frame wavelength $\lambda=1450$\,\AA
\begin{equation}\label{eq:absmag}
  M_{1450}\left(z\right) = m-5\log{\left(\frac{d_L\left(z\right)}
    {\mathrm{Mpc}}\right)}-25-K_{m,1450}\left(z\right),
\end{equation}
with the luminosity distance
\begin{equation}
  d_L(z)=(1+z)\frac{c}{H_0}\int_0^z\frac{\mathrm{d}z^\prime}
  {\sqrt{\Omega_\mathrm{m}(1+z^\prime)^3+\Omega_\Lambda}}
  \label{eqn:dl}
\end{equation}
to a quasar at redshift $z$, the apparent magnitude $m$ in a filter used
in the survey, and the bandpass correction $K_{m,1450}\left(z\right)$
\citep{1956AJ.....61...97H, 1968ApJ...154...21O, 2000A&A...353..861W,
  2002astro.ph.10394H}. For the bandpass correction we used a
combination of the \citet{2015MNRAS.449.4204L} stacked quasar spectrum at
$\lambda<2500$\,\AA, and the \citet{2001AJ....122..549V} quasar
composite spectrum at longer wavelengths to cover the lowest
redshifts. The samples from SDSS and BOSS are defined in the SDSS $i$
band, while 2SLAQ is defined in the $g$ band. At $z>4.7$ we adopted
the SDSS $z$ band magnitude (in the following denoted $z_\mathrm{AB}$)
for SDSS DR7 quasars to avoid additional corrections due to the
Ly$\alpha$ forest. Figure~\ref{fig:kcorr} shows our bandpass
corrections for SDSS, BOSS and 2SLAQ as a function of redshift. We
ignored the luminosity dependence of the bandpass correction due to
the known anticorrelation of emission line equivalent width and
luminosity \citep{1977ApJ...214..679B}.  While the
\citet{2015MNRAS.449.4204L} spectrum is for luminous ($M_{1450}\simeq
-27.2$) quasars, UV composite spectra including fainter quasars
\citep{2002ApJ...565..773T, 2012ApJ...752..162S, 2014ApJ...794...75S}
give similar values, such that our bandpass corrections remain
applicable at $M_{1450}\la -24$. Empirical luminosity-dependent
bandpass corrections show a $\la 0.2$\,mag variation over $\sim 4$
orders of absolute magnitude depending on redshift and the filter, and
with a $\sim 0.2$\,mag intrinsic scatter due to individual quasar-to-quasar
variations \citep{2013ApJ...773...14R, 2013ApJ...768..105M,
  2013A&A...551A..29P}.

%% JFH Let's make all these conversion factors and codes to do K-correction
%% publicly avaialble in an appendix. 

For the $z<2.2$ sample we corrected the SDSS $i$ and 2SLAQ $g$ band
magnitudes for host galaxy contamination following
\citet{2009MNRAS.392...19C}. Considering the different magnitude
limits of 2SLAQ and SDSS, the modelled host galaxy contamination is
small for $z>0.5$ quasars ($<0.1$\,mag in $g$, $<0.2$\,mag in $i$),
and is negligible at $z>0.8$.  In case the band defining the magnitude
limit of the survey undesirably overlaps with the Ly$\alpha$ forest
\citep{2010ApJ...710.1498G, 2011ApJ...728L..26G, 2013ApJ...768..105M}
we adopted their respective bandpass corrections to $M_{1450}$. In
particular, for the \citet{2010ApJ...710.1498G, 2011ApJ...728L..26G}
sample we recomputed $M_{1450}$ from the $R$ band photometry to be
consistent with the selection function defined in $R$, and to avoid
uncertainties in their spectrophotometry due to incomplete spectral
coverage. Since the \citet{2010ApJ...710.1498G, 2011ApJ...728L..26G}
$R$ band traces the rest frame UV, we assumed negligible host galaxy
contamination for their faint quasars. For the remaining high-redshift
surveys reporting $M_{1450}$ obtained by various methods
\citep{2010AJ....139..906W, 2011Natur.474..616M, 2015ApJ...798...28K,
  2015ApJ...801L..11V, 2016ApJ...829...33Y, 2016ApJ...833..222J,
  2018Natur.553..473B} we did not re-compute $M_{1450}$, but applied
appropriate shifts to correct to our adopted cosmology.

The selection functions were treated similarly, i.e.\ the photometric
selection function of survey $j$ given in observed magnitudes
$f_{\mathrm{p},j}\left(m,z\right)$ \citep{2006AJ....131.2766R,
  2009MNRAS.392...19C, 2010ApJ...710.1498G, 2013ApJ...773...14R} was
transformed to our absolute magnitudes
$f_{\mathrm{p},j}\left(M_{1450},z\right)$ with
Equation~\ref{eq:absmag}, while the ones given in $M_{1450}$ were
adjusted to our cosmology. Note, however, that many surveys report
additional sources of incompleteness that require modifications to the
photometric selection functions.

For 2SLAQ we corrected for magnitude-dependent spectroscopic coverage
in the two survey areas \citep[$f_\mathrm{c,NGP}\left(g\right)$ and
  $f_\mathrm{c,SGP}\left(g\right)$; Figure~4
  in][]{2009MNRAS.392...19C} and spectroscopic redshift success
\citep[$f_\mathrm{s,2SLAQ}\left(g\right)$; Figure~6b
  in][]{2009MNRAS.392...19C} by multiplying them with the photometric
selection function, resulting in two area-specific 2SLAQ selection
functions
$f_\mathrm{NGP}\left(M_{1450},z\right)=f_\mathrm{p,2SLAQ}f_\mathrm{c,NGP}f_\mathrm{s,2SLAQ}$
and
$f_\mathrm{SGP}\left(M_{1450},z\right)=f_\mathrm{p,2SLAQ}f_\mathrm{c,SGP}f_\mathrm{s,2SLAQ}$
that are relevant for the QLF.  The $z<4.7$ SDSS photometric selection
function was modified to include known imaging incompleteness to
$f_{\mathrm{SDSS},z<4.7}=0.95f_{\mathrm{p,SDSS},z<4.7}$
\citep{2006AJ....131.2766R}. The BOSS colour-selected sample contains
quasars with $f_\mathrm{c,BOSS}f_\mathrm{s,BOSS}\ge 0.85$
\citep{2013ApJ...773...14R}, and we adopted
%% JFH The overline looks weird here. I would suggest taht you rather
%% use angle brackets instead. It is not jiving with the equation above
%% and is very hard to read. 
$f_\mathrm{BOSS}=\overline{f_\mathrm{c,BOSS}f_\mathrm{s,BOSS}}f_\mathrm{p,BOSS}=0.962f_\mathrm{p,BOSS}$. \citet{2010ApJ...710.1498G}
presented two area-specific photometric selection functions due to
different filters employed, and more follow-up spectroscopy was
reported in \citet{2011ApJ...728L..26G}. We accounted for remaining
spectroscopic incompleteness at $R>23$, yielding the final selection
functions $f_\mathrm{NDWFS}$ and $f_\mathrm{DLS}$. The updated $z\sim
5$ SDSS photometric selection function \citep{2013ApJ...768..105M} was
modified to include imaging and spectroscopic incompleteness, yielding
$f_{\mathrm{SDSS},z\sim 5}=0.95^2f_{\mathrm{p,SDSS},z\sim 5}$. In the
deeper $z\sim 5$ SDSS Stripe~82 survey the spectroscopic
incompleteness is larger and magnitude-dependent \citep[Figure~14
  in][]{2013ApJ...768..105M}, resulting in $f_{\mathrm{S82},z\sim
  5}=0.95f_{\mathrm{s,S82},z\sim
  5}\left(i\right)f_{\mathrm{p,S82},z\sim 5}$. Likewise, imaging and
magnitude-dependent spectroscopic incompleteness was factored into the
\citet{2016ApJ...829...33Y} photometric selection function (their
Figures~5 and 7), resulting in
$f_\mathrm{SDSS+WISE}=0.97f_\mathrm{s,SDSS+WISE}\left(z_\mathrm{AB}\right)f_\mathrm{p,SDSS+WISE}$.
We obtained a rough estimate of the \citet{2015AA...578A..83G}
selection function by comparing the corrected and observed QLFs,
i.e.\ taking $f_\mathrm{GOODS-S}=\phi_\mathrm{obs}/\phi_\mathrm{corr}$
(see their Table~3).  Finally, for the three $z>6.5$ quasars we
assumed a selection function of unity in a range of $z$ and $M_{1450}$
estimated by the respective survey teams (private communication).

\section{Luminosity function}
\label{sec:lf}

%% JFH We need a better roadmap here for where this section is going. 1) Binned estimates 2) individual
%% fits in redshift bins. 3) examining redshift evolution. 4) attempting a joint fit.

After homogenising the samples and selection functions we are in a position
to compute the UV luminosity function of AGN. We begin our analysis by
deriving luminosity functions in bins of redshift and absolute magnitude.
For simplicity we will use the notation $M\equiv M_{1450}$ in the following.

\subsection{Binned luminosity function estimates}
\label{sec:binnedlf}

In a magnitude bin $[M_\mathrm{min}, M_\mathrm{max})$, and redshift
  bin $[z_\mathrm{min}, z_\mathrm{max})$, we define the luminosity
    %% JFH Notation error here. 
function as \citep{2000MNRAS.311..433P}
\begin{equation}
  \phi \equiv \frac{N}{V_\mathrm{bin}},
\end{equation}
where $N$ is the number of quasars with magnitude
$M_\mathrm{min}\leq M<M_\mathrm{max}$ and redshift
$z_\mathrm{min}\leq z<z_\mathrm{max}$, and
\begin{equation}
  V_\mathrm{bin} = \int_{M_\mathrm{min}}^{M_\mathrm{max}}\mathrm{d}M
  \int_{z_\mathrm{min}}^{z_\mathrm{max}}\mathrm{d}z\, f(M, z)\,\frac{\mathrm{d}V}{\mathrm{d}z},
  \label{eqn:vi}
\end{equation}
is the effective volume of the bin. The inclusion of the survey selection function
$f(M,z)$ (Section~\ref{sect:datahom}) in Equation~(\ref{eqn:vi}) accounts for
what are sometimes called ``incomplete bins''
\citep{2006AJ....131.2766R}.  The comoving volume element $\mathrm{d}V/\mathrm{d}z$ is
given by
\begin{equation}
  \frac{\mathrm{d}V}{\mathrm{d}z}=\frac{\mathrm{d}V}{\mathrm{d}z\,\mathrm{d}\Omega}\times A\times\frac{4\pi}{41253},
\end{equation}
where $A$ is the survey area in deg$^2$, and 
\begin{equation}
  \frac{\mathrm{d}V}{\mathrm{d}z\,\mathrm{d}\Omega}=\frac{c}{H_0}\frac{d_L^2\left(z\right)}
       {\left(1+z\right)^2\sqrt{\Omega_\mathrm{m}\left(1+z\right)^3+\Omega_\Lambda}}
  \label{eqn:dvdzdo}
\end{equation}
denotes the comoving volume element per unit solid angle
\citep{1999astro.ph..5116H}.  

The luminosity function $\phi$ has units of
$\mathrm{cMpc}^{-3}\mathrm{mag}^{-1}$.  We evaluate the double
integral in Equation~(\ref{eqn:vi}) by the Euler method, i.e., by
simply summing over the ``tiles'' of the selection function map,
without interpolating between the redshift and luminosity values of
neighbouring tiles. 
%%GW: We should add that interpolation of selection functions is not
%%straightforward (strong gradients) and not unique. Many selection
%%functions are very coarse. V=0 means that the selection function has
%%systematic errors (a QSO was found where the selection probability
%%is zero). By removing them we avoid numerical issues, but
%%underestimate the QLF. Surveys need to report selection functions
%%that are consistent with their actual samples. -- done (GK) 
This may result in $V_i=0$ for some quasars, in which case we remove
them from our analysis.  (Interpolation of selection functions is not
straightforward due to strong gradients.  The presence of objects with
$V_i=0$ implies that the selection function has systematic errors.)
%%GW: Poisson statistics will underestimate the error since
%%uncertainties in the selection function are not considered (all
%%surveys assume a fixed selection function). For small surveys the
%%Poisson error dominates, but for large surveys the error is driven
%%by systematics in the selection function. These facts have been
%%neglected so far, and we should make this point very clearly. -- done (GK)
In each bin, we estimate the uncertainty in the luminosity function by
assuming Poisson statistics \citep{1986ApJ...303..336G} for the number
of quasars, i.e.\ assuming negligible uncertainty in the selection function.
The resultant binned luminosity function estimates are shown by the
points in Figure~\ref{fig:mosaic}.

%%GW: I've sharpened the discussion below. -- done (GK)
As seen in Figure~\ref{fig:mosaic}, the distribution of luminosity
function values in each redshift bin are suggestive of a double power
law form for the QLF.  We will fit such a form below. 
However, in several redshift bins we note a suspicious decline of the
luminosity function at the faint limit of several surveys, for example in
the 2SLAQ sample at $z<2.2$, and in the SDSS sample at $z<1.8$ and $z\sim 4$.
The inconsistency between the SDSS faint end and the deeper
2SLAQ QLF indicates that the SDSS selection function is
systematically overestimated at its magnitude limit.
At $z\sim 4$ the SDSS faint end QLF is inconsistent with the fainter
\citet{2011ApJ...728L..26G} QLF. We identify such discrepant bins
by eye and discard them from our analysis.
%%GW: We should also set the selection functions to zero in these
%%bins. -- done (GK)
The discarded magnitude bins are shown in Figure~\ref{fig:mosaic} by
open circles.

\begin{figure*}
  \begin{center}
    \includegraphics[width=\textwidth,keepaspectratio]{mosaic_small.pdf}
    % mosaic.py 
  \end{center}
  \caption{Homogenised quasar luminosity functions at
    $\lambda=1450$\,\AA\ in redshift bins from $z=0.1$ to $6.5$.  The
    symbols show our inferred binned luminosity functions from various
    data sets:
    \citet[red]{2010AJ....139.2360S},
    \citet[green]{2009MNRAS.392...19C},
    \citet[dark blue]{2013ApJ...773...14R},
    \citet[light blue]{2011ApJ...728L..26G},
    \citet[yellow]{2016ApJ...829...33Y},
    \citet[DR7 brown]{2013ApJ...768..105M},
    \citet[Stripe 82 teal]{2013ApJ...768..105M},
    \citet[pink]{2016ApJ...833..222J},
    \citet[orange]{2010AJ....139..906W}, and
    \citet[grey]{2015ApJ...798...28K}.
    Open circles in corresponding colours indicate magnitude bins
    excluded due to incompleteness for the respective data sets.  The
    number of AGN before this selection is shown in parentheses; the
    selected number of AGN is shown outside parentheses.  In each
    redshift bin, the black curve shows our fiducial double power law
    model fit, which is represented by the median of the posterior
    probability distribution function.  The grey shaded area shows the
    one-sigma (68.26\%) uncertainty.  See Sections~\ref{sec:binnedlf}
    and \ref{sec:bins} for further details.}
  \label{fig:mosaic}
\end{figure*}

\begin{figure*}
  \begin{center}
    \includegraphics[width=0.7\textwidth]{evolution_individuals.pdf}
    % summary_bins.py 
  \end{center}
  \caption{Redshift evolution of the four double power law parameters from the
    redshift bins shown in Figure~\ref{fig:mosaic}.  Vertical error
    bars show one-sigma (68.26\%) statistical uncertainties, whereas horizontal
    error bars show widths of the redshift bins.  We identify the
    general evolutionary trends of each of these parameters from the
    bins shown by the filled symbols.  The open symbols show bins that
    appear to be offset from these trends, likely due to unknown
    systematic errors.  The open circles at $2.2<z<3.5$ show the
    BOSS sample, while the bins at $z < 0.6$ contain AGN from the SDSS
    and 2SLAQ data sets.  See Section~\ref{sec:bins} for further
    details.}
  \label{fig:evoln}
\end{figure*}

%%GW: I would delete the z_bin column from the table. Otherwise the
%%reader might be confused what you are plotting in the figure. The
%%bin centers and the mean values are very similar anyway.--done (GK)
%%GW: Again 2.5<z<2.6 is missing--done (GK)
%%GW: I would add a column or a comment which redshift bins we
%%consider reliable. I'm afraid lazy people will just take the values
%%without thinking.
\begin{table*}
  % bins_tabulate.py; Nqso added by hand
  \caption{Derived double power law luminosity function parameters in
    various redshift bins.  The luminosity function parameters
    $\phi_*$, $M_*$, $\alpha$, and $\beta$ are defined in
    Equation~(\ref{eqn:dpl}), with $\beta$ denoting the faint-end
    slope.  Quasars in each bin have redshifts $z_\mathrm{min}\leq z <
    z_\mathrm{max}$.  The number of quasars in each bin is given by
    $N_\mathrm{qso}$.  The bin centre and sample mean redshift are
    given by $z_\mathrm{bin}$ and $\langle z\rangle$, respectively.
    Errors indicate one-sigma (68.26\%) uncertainties.  These values
    are shown in Figure~\ref{fig:evoln}.  The corresponding luminosity
    functions are shown in Figure~\ref{fig:mosaic}.  See
    Section~\ref{sec:bins} for further details.}
  \label{tab:bins}
  \begin{tabular}{cccr....}
    \hline
    $\langle z\rangle$ &
    $z_\mathrm{min}$ &
    $z_\mathrm{max}$ &
    $N_\mathrm{qso}$ &
    \multicolumn{1}{c}{$\log_{10}(\phi_*/$} &
    \multicolumn{1}{c}{$M_*$} &
    \multicolumn{1}{c}{$\alpha$} &
    \multicolumn{1}{c}{$\beta$} \\
    
    &
    &
    &
    &
    \multicolumn{1}{c}{cMpc$^{-3}$mag$^{-1}$)} &
    &
    & \\
    \hline
    0.31 & 0.10 & 0.40 & 3632  & -5.61^{+0.09}_{-0.11} & -21.36^{+0.20}_{-0.23} & -2.94^{+0.09}_{-0.10} & -1.21^{+0.18}_{-0.17} \\
    0.50 & 0.40 & 0.60 & 4686  & -6.12^{+0.04}_{-0.04} & -23.00^{+0.07}_{-0.08} & -3.37^{+0.08}_{-0.08} & -1.36^{+0.04}_{-0.04} \\
    0.72 & 0.60 & 0.80 & 4684  & -6.42^{+0.09}_{-0.08} & -24.04^{+0.15}_{-0.12} & -3.56^{+0.12}_{-0.12} & -1.75^{+0.05}_{-0.04} \\
    0.91 & 0.80 & 1.00 & 5248  & -6.39^{+0.08}_{-0.07} & -24.51^{+0.12}_{-0.11} & -3.69^{+0.11}_{-0.11} & -1.73^{+0.06}_{-0.05} \\
    1.10 & 1.00 & 1.20 & 6566  & -6.53^{+0.03}_{-0.03} & -25.14^{+0.04}_{-0.04} & -4.17^{+0.08}_{-0.09} & -1.73^{+0.02}_{-0.02} \\
    1.30 & 1.20 & 1.40 & 7132  & -6.43^{+0.04}_{-0.04} & -25.29^{+0.06}_{-0.06} & -3.97^{+0.08}_{-0.08} & -1.74^{+0.03}_{-0.03} \\
    1.50 & 1.40 & 1.60 & 7771  & -6.51^{+0.03}_{-0.03} & -25.68^{+0.04}_{-0.04} & -4.28^{+0.08}_{-0.08} & -1.76^{+0.02}_{-0.02} \\
    1.71 & 1.60 & 1.80 & 7421  & -6.29^{+0.04}_{-0.03} & -25.56^{+0.05}_{-0.05} & -3.97^{+0.07}_{-0.07} & -1.61^{+0.03}_{-0.03} \\
    1.98 & 1.80 & 2.20 & 10876 & -6.72^{+0.03}_{-0.03} & -26.27^{+0.04}_{-0.04} & -4.21^{+0.07}_{-0.07} & -1.87^{+0.02}_{-0.02} \\
    2.30 & 2.20 & 2.40 & 8419  & -6.16^{+0.07}_{-0.06} & -25.49^{+0.12}_{-0.11} & -3.33^{+0.11}_{-0.12} & -1.60^{+0.04}_{-0.04} \\
    2.45 & 2.40 & 2.50 & 3403  & -6.40^{+0.08}_{-0.07} & -25.86^{+0.15}_{-0.12} & -3.60^{+0.20}_{-0.23} & -1.60^{+0.05}_{-0.05} \\
    2.55 & 2.50 & 2.60 & 2640  & -6.15^{+0.08}_{-0.09} & -25.34^{+0.18}_{-0.17} & -3.31^{+0.16}_{-0.16} & -1.39^{+0.08}_{-0.08} \\
    2.65 & 2.60 & 2.70 & 1883  & -5.98^{+0.06}_{-0.06} & -25.15^{+0.14}_{-0.13} & -3.13^{+0.12}_{-0.12} & -1.05^{+0.09}_{-0.08} \\
    2.75 & 2.70 & 2.80 & 1135  & -6.29^{+0.07}_{-0.07} & -25.95^{+0.15}_{-0.13} & -3.79^{+0.26}_{-0.29} & -1.34^{+0.07}_{-0.06} \\
    2.85 & 2.80 & 2.90 & 1069  & -6.46^{+0.11}_{-0.10} & -26.23^{+0.22}_{-0.18} & -3.62^{+0.36}_{-0.41} & -1.46^{+0.08}_{-0.07} \\
    2.95 & 2.90 & 3.00 & 1104  & -6.76^{+0.07}_{-0.06} & -26.52^{+0.10}_{-0.09} & -5.04^{+0.59}_{-0.63} & -1.71^{+0.05}_{-0.04} \\
    3.05 & 3.00 & 3.10 & 1127  & -6.77^{+0.08}_{-0.07} & -26.48^{+0.11}_{-0.11} & -4.72^{+0.44}_{-0.51} & -1.70^{+0.05}_{-0.05} \\
    3.15 & 3.10 & 3.20 & 1041  & -7.25^{+0.16}_{-0.12} & -27.10^{+0.23}_{-0.18} & -4.42^{+0.79}_{-1.17} & -1.96^{+0.07}_{-0.05} \\
    3.25 & 3.20 & 3.30 & 815   & -7.34^{+0.13}_{-0.13} & -27.20^{+0.19}_{-0.23} & -4.37^{+0.68}_{-0.74} & -1.94^{+0.06}_{-0.05} \\
    3.34 & 3.30 & 3.40 & 510   & -7.53^{+0.22}_{-0.21} & -27.38^{+0.29}_{-0.36} & -4.79^{+1.34}_{-1.47} & -2.08^{+0.09}_{-0.07} \\
    3.44 & 3.40 & 3.50 & 155   & -6.78^{+0.19}_{-0.20} & -26.64^{+0.40}_{-0.34} & -3.74^{+0.64}_{-0.79} & -1.25^{+0.28}_{-0.23} \\
    3.88 & 3.70 & 4.10 & 1204  & -7.92^{+0.12}_{-0.11} & -27.26^{+0.14}_{-0.12} & -4.84^{+0.34}_{-0.36} & -2.07^{+0.10}_{-0.09} \\
    4.35 & 4.10 & 4.70 & 603   & -8.32^{+0.28}_{-0.26} & -27.37^{+0.36}_{-0.31} & -4.20^{+0.40}_{-0.49} & -2.20^{+0.16}_{-0.14} \\
    4.92 & 4.70 & 5.50 & 263   & -9.04^{+0.30}_{-0.22} & -27.90^{+0.37}_{-0.26} & -4.56^{+0.70}_{-0.81} & -2.31^{+0.11}_{-0.08} \\
    6.00 & 5.50 & 6.50 & 66    & -10.64^{+0.72}_{-1.08} & -29.18^{+1.05}_{-1.81} & -5.00^{+0.76}_{-1.21} & -2.40^{+0.09}_{-0.08} \\
    \hline
  \end{tabular}
\end{table*}

\subsection{Double power law fits}
\label{sec:bins}

%%GW: Not sure what you mean here. I've cut the next two sentences
%As our next step, we combine magnitude bins by binning the quasars
%only in redshifts.  This allows us to estimate the luminosity
%function at each redshift.--done(GK)
In each redshift bin, we model the quasar
luminosity function as a double power law 
\citep[e.g.][]{1988MNRAS.235..935B}
\begin{equation}
  \phi(M) =
  \frac{\phi_*}{10^{0.4(\alpha+1)(M-M_*)}+10^{0.4(\beta+1)(M-M_*)}}
  \label{eqn:dpl}
\end{equation}
with four free parameters: (i) the amplitude $\phi_*$, (ii) the break magnitude $M_*$,
(iii)  the bright-end slope $\alpha$, and (iv) the faint-end slope $\beta$.
%%GW: Maybe give the range of the uniform priors.
By assuming broad, uniform priors, we obtain posterior probability
distributions for these parameters using MCMC \citep[e.g.,][]{jaynes}.
The joint posterior probability distribution of the model parameters
is then written as
\begin{multline}
  p(\phi_*, M_*, \alpha, \beta | \{M_i, z_i\}) \propto \\ p(\phi_*, M_*,
  \alpha, \beta)p(\{M_i, z_i\} | \phi_*, M_*, \alpha, \beta),
\end{multline}
where the constant of proportionality is independent of the luminosity
function parameters, and $\{M_i, z_i\}$ denotes the magnitudes and
redshifts of quasars falling in a redshift bin $[z_\mathrm{min},
  z_\mathrm{max})$.  We use a uniform prior distribution $p(\phi_*,
  M_*, \alpha, \beta)$ and assume that the likelihood
\begin{equation}
  \mathcal{L}\equiv p(\{M_i, z_i\} | \phi_*, M_*, \alpha, \beta)
\end{equation}
is given by $\phi(M)$
%% JFH  ``is given by $\phi(M)$ itself''
with suitable normalisation.  The negative
logarithm of the likelihood $S\equiv -2\ln\mathcal{L}$ can then be
written as
\begin{multline}
  S = -2\sum_{i=1}^N\ln\phi(M_i, z_i)\\+2\int_{M_\mathrm{min}}^{M_\mathrm{max}}\mathrm{d}M
  \int_{z_\mathrm{min}}^{z_\mathrm{max}}\mathrm{d}z\, \phi(M,z) f(M, z)\,\frac{\mathrm{d}V}{\mathrm{d}z}.
  \label{eqn:S}
\end{multline}
where $N$ is the total number of quasars in the redshift bin and the
luminosity integral in the second term on the right hand side is on
the surveyed range of $M$.  We use the \texttt{emcee} code
\citep{2013PASP..125..306F} for MCMC.

%% JFH SOmewhere indicate taht this procedure of ubninned LF estimates is standard practice. 
The above likelihood can also be understood as the limit of the
Poisson likelihood in luminosity and redshift bins
\citep{1983ApJ...269...35M, 2001AJ....121...54F}.  We can write the
probability of observing $n_{ij}$ quasars in the $(M_i, z_j)$ bin as
the Poisson distribution
\begin{equation}
  \mathcal{L}=\prod_{i,j}\frac{e^{-\mu_{ij}}\mu_{ij}^{n_{ij}}}{n_{ij}!},
  \label{eqn:lhood}
\end{equation}
where 
\begin{equation}
  \mu_{ij}= \int_{M_i}^{M_{i+1}}\mathrm{d}M\int_{z_j}^{z_{j+1}}\mathrm{d}z\, \phi(M,z) f(M, z)
  \,\frac{\mathrm{d}V}{\mathrm{d}z},
\end{equation}
is the average number of quasars expected in the $(M_i, z_j)$ bin
given the luminosity function $\phi(M,z)$.  In the limit of
infinitesimal bins, $n_{ij}=0$ or $1$, and Equation~(\ref{eqn:lhood})
can be simplified to obtain Equation~(\ref{eqn:S}).

Our estimates for the double power law luminosity function are shown
in Figure~\ref{fig:mosaic} for 25 redshift bins.
%% JFH you are confusing terminology. posterior median and best-fit are not generally the same and you should
%% not refer to them as such. You are also not explaning what you mean by posterior median. Is this the
%% median of the individual marginalized posteriors? Or is it the median of the function that results from
%% the posterior distribution? Those are not generally the same, unless the model is linear in the parameters,
%% which it is not. Finally, best fit always means maximum likelihood, and that is certainly not what you are
%% plotting here, so don't say best-fit. You can say as our ``adopted model'' or ``fiducial model fit'' or
%% something like that. I would just explain waht you plot. 
The posterior median is shown as our adopted model at each redshift,
while the grey shaded region shows the one-sigma (68.26\%)
uncertainty.  Consistent with previous studies, the double power law
model provides an excellent description of the luminosity function
model over almost the complete range of redshifts spanned by the data.
It is only in the highest redshift bin ($z=5.5$--$6.5$) that the data
seem to favour a single power law.  In this bin, the resultant
posterior distribution of the break magnitude $M_*$ is bimodal with
favoured values at the faint ($M_*>-18$) and bright end of the data
($M_*<-30$).  While in the literature the $z\sim 6$ AGN have been
assumed to lie on the bright end of the luminosity function
\citep[e.g.,][]{2016ApJ...833..222J}, a comparison with the luminosity
function at lower redshifts ($z<5.5$) suggests that these AGN should
instead be understood to describe the faint-end of a double power law.
%% JFH Okay so you need to be more clear here that what you mean is that the trend
%% from the plot is that M_star gets progressively brighter with redshift. This is not
%% obvious given what you write here. Indeed, it is not super obvious until one looks at FIgure 4, which
%% anyway has this prior already in it. 
Therefore, in light of the
low-redshift data, we use restricted priors in this
%% JFH Which redshift bin?? Make it more clear that this prior is put in to avoid bi-modal distributions
%% and it is informed by inspection of the lower redshift data. 
redshift bin: we
restrict the bright-end slope $\alpha$ to values less than $-4$,
%% JFH Something does not jive here, since the error bars you are showing on the high-z poitns cross
%% the -4 value of your prior? So either your prior is not a hard prior, or it is not exactly at -4
which
is equivalent to forcing $M_*$ to be at the bright end of the data.
Other parameters continue to have wide uniform priors.  This also
illustrates the importance of analysing the evolution of the QLF with
redshift.

The redshift evolution of the four double power law parameters is
shown in Figure~\ref{fig:evoln} and tabulated in Table~\ref{tab:bins}.
We find interesting evolutionary trends in each of the four
parameters.  The break magnitude $M_*$ evolves by more than eight
magnitudes from redshift $z=0$ to $7$.  The amplitude of the
luminosity function $\phi_*$ evolves moderately from $z=0$ to $z\sim
3$ and then drops by six orders of magnitude to about
$10^{-12}$\,cMpc$^{-3}$mag$^{-1}$ at $z\sim 7$.  The bright end slope
$\alpha$ has significant scatter,
%% JFH most likely attributable to systematic erorrs in the measurements.
but still shows a trend towards more
negative values, i.e., steeper luminosity function bright ends, at
high redshifts.  Finally, the faint end slope of the luminosity
function also shows signs of increasing steepness towards high
redshifts.
%%GW: I would not write that beta has a break.--done(GK)
Somewhat similar to the amplitude $\phi_*$, the faint-end slope
$\beta$ also shows rapid evolution at $z\sim 3$.  It stays roughly
constant up to this redshift, and the drops to increasingly negative
values of close to $-2.5$ at high redshifts, indicating a steeper
faint end.

%% JFH I think before imposing this prior by hand it might be useful to show the degeneracies that
%% you get at high-z without imposing the prior on the faint-end/bright-end slope. 


%%GW: I've commented the following paragraph and replaced it by a
%%sharper version
%Figure~\ref{fig:evoln} reveals further likely systematic errors.
%The behaviour of BOSS quasars, shown by open circles in
%Figure~\ref{fig:evoln} at redshifts $z=2$--$4$ is striking.  In this
%redshift range covered by BOSS the values of the four double power law
%parameters deviate from the general trends.  The faint end of the
%luminosity function of the BOSS quasars is much shallower and the
%break luminosity is fainter by a magnitude compared to the general
%trend expected from a smooth evolution.  The bright end of the
%luminosity function of quasars in this sample is also much flatter
%than that for other samples.  The conspicuousness of the deviation of
%the BOSS sample from the general trends makes it unlikely that the
%luminosity function evolution indicated by BOSS in the redshift range
%$z=2$--$3$ is physical.  Indeed, in this redshift range BOSS quasar
%selection is known to be affected by a colour bias
%\citep{2011ApJ...728...23W}, which can make estimating completeness
%difficult \citep{2006AJ....131.2766R}. --done(GK)

Discontinuities and scatter in the QLF parameters over short redshift
intervals in Figure~\ref{fig:evoln} reveal further likely systematic
errors in the survey selection functions.  Quasars at $2.2\le z<3.5$
taken solely from BOSS \citep{2013ApJ...773...14R} follow the redshift
trends in $\phi_*$ and $M_*$, but with significant scatter in $\Delta
z=0.1$ intervals that is much larger than the statistical error.  The
discontinuity at $z=2.2$ indicates a mismatch of BOSS and
SDSS$+$2SLAQ. The most striking feature, however, is the apparent
rapid redshift evolution of the faint-end slope revealed in the BOSS
sample, which is also highlighted in Figure~\ref{fig:mosaic}. Given
the relatively smooth evolution of all QLF parameters at lower and
higher redshifts it is unlikely that the QLF evolution at $2.2\le
z<3.5$ indicated by BOSS is physical.  Rather it reflects the
systematics limit of the large BOSS sample induced by a fixed
selection function that critically depends on the assumed quasar
spectral energy distribution and the IGM parameterization
\citep{2011ApJ...728...23W,2013ApJ...773...14R}.  Consequently, we
exclude all BOSS quasars from further analysis.

%% JFH I think another point worth mentioning is that the BOSS survey is never working at high completeness,
%% (this you can quantify from your selection function). That means that the resevoir of QSOs that can
%% scatter into or out of the sample is large, making one susceptible to errors in the selection function
%% (or at the faint end variations in the photometric senstivity 


The imperfect match between SDSS and 2SLAQ (Figure~\ref{fig:mosaic})
causes low-level systematics, as evidenced by the apparent
discontinuities in $\alpha$ at $z<2.2$ and the jump in $\beta$ at
$z=1.8$ in Figure~\ref{fig:evoln}.  At $z<0.6$ the faint-end slope
shows a sharp increase which we attribute to remaining uncertainties
in the correction for host galaxy light and potentially missed AGN in
extended sources. we exclude $z<0.6$ quasars from further
consideration. Due to different selection function parameters
inter-survey systematics are definitely present at $z>3.5$ as well,
but Poisson errors of the limited samples dominate.

\begin{figure*}
  \begin{center}
    \includegraphics[width=\textwidth,keepaspectratio]{mosaic_small_global.pdf}
    % mosaic.py 
  \end{center}
  \caption{Luminosity function estimates from $z=0.6$ to $6.5$.
    Similar to Figure~\ref{fig:mosaic}, the symbols show our inferred
    binned luminosity functions.  In each redshift bin, yellow curves
    show our fiducial double power law luminosity function model in
    that redshift bin.  Other curves show the three global evolution
    models.  Shaded regions show the one-sigma (68.26\%)
    uncertainties.}
  \label{fig:mosaic_global}
\end{figure*}

\begin{figure*}
  \begin{center}
    \includegraphics[width=0.7\textwidth]{evolution_global.pdf}
    % summary_fromFile.py 
  \end{center}
  \caption{Luminosity function parameter evolution in the global
    models.  The symbols show the posterior median values of
    parameters with one-sigma (68.26\%) uncertainties in redshift bins
    from Figure~\ref{fig:evoln}.  Redshift bins deemed to be affected
    by systematics and removed from the global analysis are shown in
    grey.  In each panel, the solid curves and shaded regions show the
    three derived global models with one-sigma uncertainties.  Model~1
    provides a better fit, but requires a rapid change in the
    faint-end slope at $z\sim 3.5$.}
  \label{fig:evoln_global}
\end{figure*}

\begin{figure}
  \begin{center}
    \includegraphics[width=\columnwidth,keepaspectratio]{rhoqso_withGlobal.pdf}
    % rhoqso.py -- draw_withGlobal_multiple()
  \end{center}
  \caption{AGN number density evolution in the global model, when the
    luminosity function is integrated (from top to bottom) down to
    $M_\mathrm{1450}=-18, -21, -24,$ and $-27$.  Filled black circles
    show the estimates from the our fiducual double power law
    luminosity function models in various redshift bins, with vertical
    error bars denoting one-sigma (68.26\%) uncertainties.  Open
    circles show the same in redshifts bins affected by systematics.
    Curves show the estimated density from the global models, with the
    accompanying shaded regions denoting the one-sigma uncertainty. }
  \label{fig:rhoqso}
\end{figure}

\begin{figure*}
  \begin{center}
    \begin{tabular}{cc}
      \includegraphics[width=0.47\textwidth]{emissivity_18.pdf} & 
      \includegraphics[width=0.47\textwidth]{emissivity_21.pdf} \\
      % gammapi.py -- draw_emissivity_18()
      % gammapi.py -- draw_emissivity_21()
    \end{tabular}
  \end{center}
  \caption{The 912~\AA\ emissivity of AGN assuming 100\% escape
    fraction down to a limiting magnitude for a luminosity function
    integration limit of $M_{1450}=-18$ (left panel) and
    $M_{1450}=-21$ (right panel).  Black filled circles with error
    bars in both panels show the emissivity determinations in redshift
    bin deemed to have low systematic errors.  Open circles show
    emissivities for redshift bins that we remove from analysis due to
    high systematic errors.  Solid red curves in both panels show the
    derived posterior median emissivity evolution model, with the
    shaded area showing the one-sigma (68.26\%) uncertainty.  Also
    shown for comparison in both panels are models by \citet[pentagon
      symbol]{2009A&A...507..781S}, \citet[star]{2012ApJ...755..169M},
    \citet[square]{2015AA...578A..83G},
    \citet[triangle]{2018PASJ...70S..34A},
    \citet[circle]{2018AJ....155..131M},
    \citet[diamond]{2018MNRAS.474.2904P}, \citet[open
      rectangle]{2017ApJ...847L..15O}, \citet[dotted green
      curve]{2015ApJ...813L...8M}, \citet[dashed
      brown]{2012ApJ...746..125H}, \citet[dashed
      grey]{2017MNRAS.466.1160M}, \citet[dashed
      blue]{2013A&A...551A..29P}, \citet[solid
      cyan]{2016A&A...587A..41P}, \citet[solid
      blue]{2017A&A...608A..64C}, and
    \citet[brown]{2007A&A...472..443B}.}
  \label{fig:e912_2}
\end{figure*}

\subsection{Evolution of the luminosity function}
\label{sec:global}

The smooth evolution of the luminosity function parameters apparent in
Figure~\ref{fig:evoln} suggests that it should be possible to describe
the quasar UV luminosity function evolution out to $z\sim 7$ using
fewer parameters.  Such descriptions have been developed in the
literature for the X-ray \citep[e.g.,][]{2015MNRAS.451.1892A} and
bolometric luminosity functions \citep[e.g.,][]{2007ApJ...654..731H}.
Such ``global'' models of the luminosity function evolution are useful
as they give a continuous description of the luminosity function.
This allows one to reduce the bias introduced by binning the data in
arbitrary redshift bins.  By potentially allowing for extrapolations
beyond the redshift range spanned by the data, such models are
valuable for understanding of the physics behind the luminosity
function.  Ideally, one would want to use physically meaningful
parameters that govern the formation and evolution of the AGN
population.  Unfortunately, such physical parameterisation is yet to
be developed.  We therefore set up an empirical parameterisation to
describe the evolution of the four parameters of the double power law
model in Equation~(\ref{eqn:dpl}) as
\begin{align}
  &\phi_*(z) = f_0(\{c_{0,i}\}, z)\nonumber\\
  &M_*(z) = f_1(\{c_{1,i}\}, z)\nonumber\\
  &\alpha(z) = f_2(\{c_{2,i}\}, z)\nonumber\\
  &\beta(z) = f_3(\{c_{3,i}\}, z),
  \label{eqn:global}
\end{align}
where the $c_{j,i}$'s are the new model parameters, and the $f_j$'s
are functions that vary smoothly with redshift $z$.  The joint
posterior probability distribution of these parameters can be now
written as
\begin{equation}
  p(\{c_i\} | \{M_i, z_i\}) \propto p(\{c_i\})p(\{M_i, z_i\} | \{c_i\}),
\end{equation}
where the likelihood 
\begin{equation}
  \mathcal{L}\equiv p(\{M_i, z_i\} | \{c_i\}),
\end{equation}
is now given by $\phi(M,z)$ with suitable normalisation.  Note that
$\phi(M,z)$ is given by Equation~(\ref{eqn:dpl}), but now the four
parameters in that equation are redshift-dependent, so that
\begin{multline}
  \phi(M,z) = \phi_*(z) \left[10^{0.4(\alpha(z)+1)(M-M_*(z))}\right. \\ \left.+ 10^{0.4(\beta(z)+1)(M-M_*(z))}\right]^{-1}.
\end{multline}
The negative logarithm of the likelihood $S\equiv -2\ln\mathcal{L}$ is
a straightforward generalisation of Equation~(\ref{eqn:S}), given by
\begin{multline}
  S = -2\sum_{i=1}^N\ln\phi(M_i, z_i)\\+2\int_{M_\mathrm{min}}^{M_\mathrm{max}}dM\int_{z_\mathrm{min}}^{z_\mathrm{max}}dz\, \phi(M,z) f(M, z)\,\frac{dV}{dz},
  \label{eqn:S2}
\end{multline}
where now $N$ is the total number of quasars included in the analysis,
and the integrals in the second term on the right-hand side are over
the complete surveyed range of luminosity and redshift, instead of
redshift and luminosity bins.  We consider models in which the
evolution of the four double power law parameters is modelled
independently as in Equations~(\ref{eqn:global}), an approach
sometimes termed as ``flexible double power law''
\citep{2015MNRAS.451.1892A}.  We present three such models in this
paper.  These are shown in Figure~\ref{fig:evoln_global}.
%% JFH Something is wrong with the flow here. You are referencing Figure 6, but have not yet discussed Figure 5. 
The three
models differ in the way they describe the evolution of the faint-end
slope $\beta$ and in the selection of the AGN data.  The models are
defined as follows.

\begin{itemize}

\item In Model 1, for the parameters $\phi_*$, $M_*$ and $\alpha$ we
  assume that the functions $f_i$'s
  %% JFH don't need the apostrophe s. 
  in Equations~(\ref{eqn:global})
  are Chebyshev polynomials in $(1+z)$.  Thus the functions $f_i$s
  %% JFH This s should be here after f_i
  are
  written as
  \begin{equation}
    f_i=\sum_{j=0}^{n_i}c_{i,j}T_j(1+z),
    \label{eqn:cbs}
  \end{equation}
  where $c_{i,j}$ are the parameters that appear in
  Equations~(\ref{eqn:global}) and $T_j(1+z)$ are Chebyshev
  polynomials of the first kind.  We try successively higher orders of
  Chebyshev polynomials in order to arrive at a good fit with the
  data.
  %% JFH State that you assess this by eye
  As we discuss below, we find that $\phi_*$, $M_*$ and
  $\alpha$ prefer quadratic, cubic, and linear evolutions in $(1+z)$.
  %% JFH , respecitvely. 
  The faint-end slope $\beta$ seems to require a break in its
  evolution at $z\sim 3$, as discussed in the previous section.  As a
  result, it is difficult to model the evolution of $\beta$ using the
  Chebyshev expansion of Equation~(\ref{eqn:cbs}).  Instead, we
  consider a double power law model for $\beta$ and write
  \begin{equation}
    f_3(1+z)=c_{3,0}+\frac{c_{3,1}}{10^{c_{3,3}\zeta}+10^{c_{3,4}\zeta}},
    \label{eqn:beta}
  \end{equation}
  where
  \begin{equation}
    \zeta = \log_{10}\left(\frac{1+z}{1+c_{3,2}}\right),
  \end{equation}
  thus resulting in a five-parameter model with parameters $c_{3,i}$.
  The parameters $c_{3,3}$ and $c_{3,4}$ thus determine the low and
  high redshift slopes of this evolution, with a break at redshift
  $c_{3,2}$.
  %% JFH Consider calling this z_3,2 
  This is similar to the model of
  \citet{2007ApJ...654..731H}, who also favoured a broken power law
  model for the evolution of the faint-end slope of the bolometric
  luminosity function of quasars.  Model 1 thus has 14 parameters.
  Excluding those deemed to be dominated by systematic errors, as
  discussed in the previous section, all of the remaining AGN from
  Table~\ref{tab:samples} are included while fitting this model.
  %% JFH where was this discussed?? Nowhere have I seen explicitly stated in the
  %% text which luminosity/redshift ranges you are excluding from the analysis in performing
  %% these fits because you think they are systematics dominated. This just needs to be clearly stated.
  %% Add this to the end of Section 3.2 after you discuss the systematics. State what is excluded and why.
  %% It is perfectly fine if you do this by eye, just state that. 

\item Model 2 parameterises the luminosity function evolution in the
  same way as Model 1, so that the faint-end slope evolution is
  described by a double power law while the evolution of the other
  parameters $\phi_*$, $M_*$ and $\alpha$ is described by,
  respectively, quadratic, cubic, and linear polynomials in $(1+z)$.
  The total number of parameter is 14.  However, while fitting this
  model, we drop samples 7, 19, and 20 from the analysis.  The
  selection function for these samples are not yet well-characterised.
  Removing them allows us to understand the effect this has on the
  favoured evolution model.
  %% JFH Do not put the discussion here. Add a separate discussion where you address the samples you leave out. Referencing them by
  %% number here is also a bit annoying to me. Just state the samples with a name of the survey. 

\item In Model 3, we again exclude samples 7, 19, and 20 from the
  analysis.  We also continue to describe the evolution of $\phi_*$,
  $M_*$ and $\alpha$ by quadratic, cubic, and linear polynomials in
  $(1+z)$, respectively.  But in this model, the evolution of the
  faint-end slope $\beta$ is also assumed to be linear in $(1+z)$.
  This model thus has just 11 parameters.
\end{itemize}

%% JFH I'm lost. According to this discussion Model 1 and 2 treat the faint end slope with a broke power law, but your figures
%% suggests that only the gray band Model 1 is using this double power law? What you are calling Model 2 appears to not be a double
%% power law? Am I getting fooled by the shape of that green curve?

Figure~\ref{fig:mosaic_global}
shows the three
%% JFH global 
models in comparison
with the binned fits from the previous section.  The shaded regions
show the one-sigma (68.26\%) uncertainty.  The symbols show the
luminosity function binned in luminosity and redshift, and the yellow
shaded regions show the posterior distribution of the double power law
luminosity functions in various redshift bins, as in
Figure~\ref{fig:mosaic}.  Bins containing data with large systematic
error are excluded from Figure~\ref{fig:mosaic_global}.  All three
global models are in excellent agreement with the binned models,
although Model~1 performs better.
%% JFH by eye model 1 performs better

Figure~\ref{fig:evoln_global} shows the parameter evolution in the
global models, by comparing it with the results from the fits in
individual redshift bins shown in Figure~\ref{fig:evoln}.  All three
models capture the steepening of the faint end of the luminosity
function towards higher redshifts.  The adopted posterior median form
of Model~1 is shown by the black curves in
Figure~\ref{fig:evoln_global}.  The accompanying grey shaded area
depicts the one-sigma (68.26\%) uncertainty.  The model is in
excellent agreement with the results of the fits in redshift bins
discussed in the previous section.
%% JFH Reference Figure 5 here as well.  
The
deviation of the BOSS quasars at $z=2$--$4$ from the smooth evolution
is again strikingly apparent.  So is
%% JFH , as is the deviation
the deviation of the SDSS and
2SLAQ quasars at $z<0.6$.  This is an indirect justification for the
data selection discussion previously in Section~\ref{sec:bins}.
Unfortunately, this model suffers
%% JFH suffers from 
with a remarkably sharp break in the
evolution of the faint-end slope $\beta$ at about $z\sim 3.5$.  As
seen in Figure~\ref{fig:evoln_global}, the data seem to require this
break, although it seems unlikely that such a sharp break at this
redshift would be physical.  Model 1 thus serves to emphasize the
necessity of better quality data at these redshifts.  Models 2 and 3
are shown in Figure~\ref{fig:evoln_global} by the green and orange
curves, respectively.  The corresponding parameter values are
tabulated in Table~\ref{tab:global}.

The evolution of the comoving number density of quasars is shown in
Figure~\ref{fig:rhoqso} when the luminosity function is integrated
down to different limits.  Similar to Figure~\ref{fig:evoln_global},
symbols show the posterior median values with one-sigma uncertainties from the
double power law fits to the data in redshift bins.
%% JFH I'm bothered by the fact that the symbols and error bars here arise from the parametric fits. In my opinion, you
%% should be able to plot the data on here with Poisson errors in exactly the same way that you do when you compute the binned
%% luminosity function data points? These symbols with errors should be the data to me, and not the result of your fitting procedure. 
Solid curves and
shaded regions show the global models.  This number density evolution
again highlights the systematic error in data at $z\sim 3$.  The
number density of AGN down to the limit of
%% JFH the deepest spectroscopic surveys 
spectroscopic data
($M_{1450}<-21$) is about $10^{-5}$ cMpc$^{-3}$ at its peak.  This
density rapidly increases
%% JFH I think you mean decreases here
at low redshifts and then drops gradually at
high redshifts.  Figure~\ref{fig:rhoqso} also shows the familiar
downsizing feature in which the number density of faint AGN peaks at
lower redshifts than that of the bright AGN.  While the number density
of AGN with $M_*<-27$ peaks at $z\sim 2.5$, the number density of AGN
with $M_*<-24$ peaks at $z\sim 2$.  At the faintest luminosity at
which spectroscopic data exist, $M_*<-21$, the number density of qsos
peaks at $z\sim 1$.
%% JFH The figure is not consistent with this statement. The model showing M < 21 is peaking
%% just below z = 2. 
However, the difference between our three models
is dramatically evident in Figure~\ref{fig:rhoqso}.  Model~1 prefers a
decrease in the number density of faint quasars at $z\sim 3$ followed
by an increase at higher redshift.  This is caused by the rapid
steepening of the faint-end slope in this model at this redshift.
%% JFH Reference lower right panel of figure 6 here. 
Figure~\ref{fig:rhoqso} reveals another property of these models: when
extrapolated, the AGN number density diverges in all three models at
high redshifts.
%% JFH I think this is just an artifiact of your fitting procedure,
%% i.e. that these polynomial coefficients are unconstrained at really
%% high-z. Do these fits include the banados and MOrtlock quasars? In
%% that case the divergence should not occur, at least for the
%% brightest bin.  Furthermore, if you would make this plot with the
%% data instead of of integrating fits, you would be able to plot
%% those objects on here as well.
This is result of the steep faint-end slope at high
redshifts combined with the rapid brightening of the break luminosity.
While no data exist at redshift $z>7.5$, this divergent behaviour is
shared by previous models in the literature
\citep{2007ApJ...654..731H}.  Figure~\ref{fig:rhoqso} also shows that
although Models 2 and 3 exhibit regular behaviour in the evolution of
the AGN number density at $z\sim 3$, they do not fit the $z\sim
4$--$5$ data as well as Model 1.

\begin{table}
  \caption{Derived luminosity function evolution models.  These
    parameters are defined in Equations~(\ref{eqn:cbs}) and
    (\ref{eqn:beta}).  See Section~\ref{sec:global} for further
    details and the redshift range of validity of these models.
    Errors indicate one-sigma (68.26\%) uncertainties.  Model 2 is our
    preferred model.}
  \label{tab:global}
  \begin{tabular}{p{1.0cm}eee}
    \hline 
    Param. &
    \multicolumn{1}{c}{Model 1} &
    \multicolumn{1}{c}{Model 2} &
    \multicolumn{1}{c}{Model 3} \\
    \hline
    $c_{0,0}$ & -7.559^{+0.131}_{-0.139} & -7.084^{+0.136}_{-0.142} & -6.842^{+0.077}_{-0.076} \\     
    $c_{0,1}$ & 1.013^{+0.079}_{-0.072} & 0.753^{+0.080}_{-0.073} & 0.590^{+0.039}_{-0.041} \\        
    $c_{0,2}$ & -0.113^{+0.005}_{-0.005} & -0.096^{+0.004}_{-0.005} & -0.083^{+0.003}_{-0.003} \\
    \\
    $c_{1,0}$ & -17.006^{+0.230}_{-0.243} & -15.423^{+0.263}_{-0.261} & -15.140^{+0.144}_{-0.136} \\  
    $c_{1,1}$ & -5.548^{+0.156}_{-0.143} & -6.725^{+0.156}_{-0.171} & -6.910^{+0.091}_{-0.093} \\     
    $c_{1,2}$ & 0.588^{+0.016}_{-0.019} & 0.737^{+0.018}_{-0.015} & 0.750^{+0.012}_{-0.013} \\        
    $c_{1,3}$ & -0.023^{+0.001}_{-0.001} & -0.029^{+0.001}_{-0.001} & -0.029^{+0.001}_{-0.001} \\
    \\
    $c_{2,0}$ & -3.246^{+0.121}_{-0.123} & -2.973^{+0.117}_{-0.133} & -2.950^{+0.105}_{-0.097} \\     
    $c_{2,1}$ & -0.250^{+0.048}_{-0.050} & -0.347^{+0.050}_{-0.050} & -0.363^{+0.040}_{-0.043} \\
    \\
    $c_{3,0}$ & -2.350^{+0.051}_{-0.060} & -2.545^{+0.123}_{-0.290} & -1.424^{+0.031}_{-0.031} \\     
    $c_{3,1}$ & 0.647^{+0.072}_{-0.060} & 1.581^{+0.520}_{-0.264} & -0.111^{+0.010}_{-0.010} \\       
    $c_{3,2}$ & 3.857^{+0.092}_{-0.073} & 2.102^{+0.450}_{-0.283} & \multicolumn{1}{c}{---} \\        
    $c_{3,3}$ & 27.534^{+41.364}_{-9.405} & 1.965^{+0.461}_{-0.464} & \multicolumn{1}{c}{---} \\      
    $c_{3,4}$ & -0.002^{+0.074}_{-0.082} & -0.641^{+0.169}_{-0.154} & \multicolumn{1}{c}{---} \\      
    \hline 
  \end{tabular}
\end{table}

%% JFH My main feedback on this section is that you need to elaborate more on which samples you included and excluded and why. That
%% is fine if it is all based on by eye discarding things. We need to make the point more clearly that the various surveys suffer
%% systematics, and hence cannot yet really be combined to give a sensible global evolution without some kind of massaging or choices
%% about which datasets to trust. 


\section{Contribution of AGN to reionization}
\label{sec:reion}

%% JFH I think you could possibly break of this last section as a separate letter. The paper above would just be a data analysis and
%% compilation paper. 

We now discuss the contribution of AGN to the hydrogen and helium
reionization in our luminosity function model.  We first derive the
expected evolution of the 912\,\AA\ emissivity of AGN in our model,
and then use this to estimate the \HI\ photoionization rate and the
average \HeIII\ fraction.
%% JFH Mention low-z UVB here as well in intro. 
\subsection{Hydrogen-ionizing emissivity}
\label{sec:e912}

We assume that all quasars have a universal UV SED, parameterised by
\citet{2015MNRAS.449.4204L} as a power law with a break at 912~{\AA},
\begin{equation}
  f_\nu\propto\begin{cases}
  \nu^{-0.61\pm 0.01} & \text{if}~\lambda\geq 912~\text{\AA},\\
  \nu^{-1.70\pm 0.61} & \text{if}~600~\text{\AA}<\lambda<912~\text{\AA}.\\                
  \end{cases}
  \label{eqn:sed}
\end{equation}
This fit was derived by \citet{2015MNRAS.449.4204L} from a composite
spectrum of 53 quasars at $z\sim 2.4$.  While four published composite
quasar UV SEDs \citep{2002ApJ...565..773T, 2001AJ....122..549V,
  2012ApJ...752..162S, 2014ApJ...794...75S} agree with
Equation~(\ref{eqn:sed}), the composite of
\citet{2004ApJ...615..135S}, which uses more than 100 quasars at
$z<0.1$, is somewhat steeper, with a EUV slope of $-0.56$.  The
quasars considered by \citet{2004ApJ...615..135S} are fainter
($M_i(z=2)>-26$)
%% JFH Something is wrong with this faint number here. The bulk of the scott sample I thought was much fainter than this.
%% Ask Gabor? 
than those considered by \citet{2015MNRAS.449.4204L},
which suggest that faint quasars may have a steeper EUV spectrum.  We
therefore assume an SED of the form
\begin{equation}
  f_\nu\propto\begin{cases}
  \nu^{-0.61\pm 0.01} & \text{if}~\lambda\geq 912~\text{\AA},\\
  \nu^{-0.56\pm 0.61} & \text{if}~600~\text{\AA}<\lambda<912~\text{\AA}.\\                
  \end{cases}
  \label{eqn:sed_faint}
\end{equation}
for AGN with $M_{1450}<-23$.

The rate of emission of photons of frequency $\nu$ by quasars can be
written as
\begin{equation}
  \dot n_\nu = \int dM_\nu \phi(M_\nu) \frac{L_\nu(M_\nu)}{h\nu},
\end{equation}
where the integral is over a suitable range of magnitudes, the
monochromatic luminosity $L_\nu(M)$ is related to the (absolute AB)
magnitude by \citep{1983ApJ...266..713O}
\begin{equation}
  M_\nu = -2.5\log_{10}L_\nu+51.60,
\end{equation}
and $\phi$ is the luminosity function.  The 912\,\AA\ luminosity is
related to the UV luminosity at 1450\,\AA\ by Equation~(\ref{eqn:sed})
or (\ref{eqn:sed_faint})
\begin{equation}
  L_{912}=L_{1450}\left(\frac{\nu_{912}}{\nu_{1450}}\right)^{-0.61}=L_{1450}\left(\frac{912}{1450}\right)^{0.61},
\end{equation}
depending on the AGN luminosity.  The corresponding comoving volume
emissivity is
\begin{equation}
  \epsilon_\nu = \dot n_\nu h\nu.
  \label{eqn:epsilon}
\end{equation}
Figure~\ref{fig:e912_2} shows the comoving 912\,\AA\ emissivity when
the luminosity function is integrated down to $M_{1450}<-21$ (right
panel) and when it is integrated down to $M_{1450}<-18$ (left panel).
%% JFH Figure 8 is a nightmare to look at. You need to separate this into multiple
%% figures somehow. Perhaps one with curves and one with points. 
Redshift bins that were removed from analysis due to large systematic
errors are shown as open circles.  The emissivity peaks at $z=2$--$3$,
depending on the luminosity integration limit, at
$\epsilon_{912}=10^{25}\,\mathrm{erg\, s^{-1}\, Hz^{-1}\, cMpc^{-3}}$
and decreased
%% JFH decreases 
rapidly towards high redshifts.  A continuous
description of this emissivity evolution is valuable in deriving the
hydrogen photoionizing flux in the IGM.  In principle, the luminosity
function evolution models discussed in the previous section provide
such a description.  Unfortunately, as discussed above, all of these
models suffer from pathologies such as non-monotonic or divergent
%% JFH divergence is just becuase of how you do the fits, and not having data at high-z. I don't think
%% that should be highlighted. 
AGN
number densities, likely resulting from residual systematic errors in
the data.  Therefore, here we describe the emissivity evolution by the
five-parameter functional form used by \citet{2012ApJ...746..125H}
\begin{equation}
  \epsilon_{912}=\epsilon_0(1+z)^a\frac{\exp(-bz)}{\exp(cz)+d}.
  \label{eqn:e912fit}
\end{equation}
We fit this model to the emissivity estimates in the redshift bins,
shown in Figure~\ref{fig:e912_2}, assuming a Gaussian likelihood for
the emissivity in each bin.  The resultant curves and the
corresponding one-sigma uncertainties are shown in
Figure~\ref{fig:e912_2} for $M_{1450}<-18$ and $M_{1450}<-21$.  The
912\,\AA\ emissivities in various redshift bins and in these 
models are tabulated in Appendix~\ref{sec:tables}.  The adopted
form of Equation~\ref{eqn:e912fit} is given by
\begin{multline}
  \epsilon_{912}=(10^{24.49}\mathrm{erg\, s^{-1}\, Hz^{-1}\, cMpc^{-3}})(1+z)^{7.57}\\\times\frac{\exp(-1.78z)}{\exp(1.01z)+29.20}
\end{multline}
for $M_{1450}<-18$, and by 
\begin{multline}
  \epsilon_{912}=(10^{24.11}\mathrm{erg\, s^{-1}\, Hz^{-1}\, cMpc^{-3}})(1+z)^{6.64}\\\times\frac{\exp(-0.65z)}{\exp(1.82z)+20.45}
\end{multline}
for $M_{1450}<-21$.

%% JFH So if I understand correctly you have given up on the global model and are now doing things
%% with the binned fits, since your global model produces weird non-monotonic behavior. You need to state
%% what you are doing here, so the reader understands where these data points come from on the plot.


Before proceeding to derive a hydrogen photoionization rate estimate
from these emissivities, we now comment on the comparison of our
results with estimates from the literature.  Several of these are
shown in Figure~\ref{fig:e912_2}.  The brown dashed curve in
Figure~\ref{fig:e912_2} shows the AGN 912\,\AA\ emissivity model of
\citet{2012ApJ...746..125H}, which is based on the bolometric
luminosity function model of \citet{2007ApJ...654..731H}.  Their
assumed emissivity agrees reasonably well with our $M_{1450}<-21$
case, although at its maximum the emissivity in the
\citet{2007ApJ...654..731H} model is lower by about 20\%.  The
differences between the two models are much larger for $M_{1450}<-18$.
Figure~\ref{fig:e912_2} also shows the model presented by
\citet{2015ApJ...813L...8M}.  This model disagrees with ours rather
severely at almost all redshifts in the range considered here, for
both of our integration limits.  It is important to note here that a
direct comparison of our model with \citet{2012ApJ...746..125H} and
\citet{2015ApJ...813L...8M} is difficult because these models
integrate the luminosity function to a fixed fraction of the break
luminosity $L_*$ instead of a fixed absolute magnitude.  The chosen
limite
%% JFH limite --> limit 
by \citet{2012ApJ...746..125H} and \citet{2015ApJ...813L...8M}
is $0.01L_*$.  Using a limit that depends on the break luminosity is
problematic as different quasar surveys used to develop these
emissivity models often report strikingly different break luminosity
values.  Even when these samples are homogeneized appropriately, have
%% JFH adopting such an integratino limit....
such an integration limit is likely only justified if the faint-end
slope of the luminosity function is sufficiently shallow.  This issue
also affects some other models in the literature, such as those
presented by \citet{2015AA...578A..83G}, \citet{2015MNRAS.451L..30K}
and \citet{2018arXiv180104931P}.

\citet{2017MNRAS.466.1160M} integrated the luminosity function down to
$M_{1450}=-19$.  However, their model strongly disagrees with our
inference, as seen in Figure~\ref{fig:e912_2}, potentially due to
biases resulting from inhomogeneously combining binned luminosity
function estimates from different data sets.  The emissivity
determinations from the luminosity function derived from a
variability-selected sample observed using the Sloan
%% JFH Sloan telescopee --> SDSS telescope
telescope and the
Multiple Mirror Telescope (MMT) by \citet{2013A&A...551A..29P} is in
agreement with our $M_{1450}<-21$ determination for $z<2.2$.  At other
redshifts and for $M_{1450}<-18$, the models disagree by 10--30\%.
The sample of 1877 quasars presented by \citet{2013A&A...551A..29P} is
not included in our analysis, and therefore provides a cross-check.
(This sample also does not overlap with the variability-selected
sample reported by the BOSS survey \citep{2013ApJ...773...14R}.)
\citet{2013A&A...551A..29P} assume constant double power law
luminosity function slopes with a break at $z=2.2$ and assume a pure
luminosity evolution.  This leads to a discontinuity in the emissivity
evolution in their model at $z=2.2$.  In Figure~\ref{fig:e912_2}, we
convert their $g$-band magnitudes to $M_{1450}$ and integrate the
resultant luminosity function down to our two chosen integration
limits with our chosen SED for a direct comparison.
Figure~\ref{fig:e912_2} also shows the model presented by
\citet{2016A&A...587A..41P}.  We consider their PLE$+$LEDE model, in
which the faint-end slope is assumed to stay constant with redshift.
Although the emissivity evolution in this model is continuous at
$z=2.2$, unlike in the \citet{2013A&A...551A..29P} model, it is not
smooth.  This model agrees with our $M_{1450}<-18$ result to a better
degree relative to the \citet{2013A&A...551A..29P} model, bjt
%% JFH bjt --> but
it
results in a higher emissivity in the $M_{1450}<-21$ case.
\citet{2017A&A...608A..64C} reanalysed the \citet{2016A&A...587A..41P}
sample with an improved $K$-correction implementation, but, as seen in
Figure~\ref{fig:e912_2}, their estimate is higher by 50\% or more
throughout the redshift range of their data.

%% JFH For the Palanque and Caditz etc. comparisons, it occurs to me that these guys
%% probably used the luminosity function and not the epsilon_912. I'm just wondering why you
%% choose to do the comparisons in this quantity if other authors didn't compute this quantity. I understand
%% that one is the integral of the other, but it would seem more sensible to me to compare to the Palangue-luminosity
%% function when you discuss the individual fits to luminosity functions in say their same redshift bins, then when you
%% compute the emissivity evolution here. This just seems out of place. I think comparing to other peoples emissivities
%% here is sensible, but integrating other peoples lum functions here is less transparent. 

At higher redshifts, \citet{2017ApJ...847L..15O} recently fit a double
power law luminosity function model to a combined binned samples of
\citet{2016ApJ...833..222J}, \citet{2010AJ....139..906W}, and
\citet{2015ApJ...798...28K}, and an X-ray-selected AGN candidate with
photometric redshift $z\sim 6$ reported by
\citet{2018MNRAS.474.2904P}.  We show the resultant
912\,\AA\ emissivity by a rectangle in Figure~\ref{fig:e912_2}.  The
lower side of the rectangle corresponds to the emissivity for the
\citet{2017ApJ...847L..15O} model excluding the
\citet{2018MNRAS.474.2904P} AGN.  When this AGN candidate is included,
the emissivity is higher; this is shown by the upper end of the
rectangle.  Our emissivities broadly in agreement with the
determination of \citet{2017ApJ...847L..15O}.  Figure~\ref{fig:e912_2}
shows the determination from the fits of \citet{2015AA...578A..83G}.
These authors also choose a break luminosity-dependent integration
limit of $0.01L_*$, which corresponds to $M_{1450}\sim -18$ in their
model.  The resultant emissivities are factors of 2 to 3 higher than
our determinations for $z=4$--$6$ for $M_{1450}<-21$.  This is
puzzling at first sight, because the AGN candidates reported by
\citet{2015AA...578A..83G} are consistent with our luminosity function
estimates obtained without these candidates (see
Appendix~\ref{sec:conv}).  Our luminosity function are also somewhat
steeper than those reported by \citet{2015AA...578A..83G}.  The
discrepancy is explained by the difference in the two luminosity
function models at intermediate magnitudes.  We find that the data
prefer a much brighter break luminosity than that inferred by
\citet{2015AA...578A..83G}.  This reduces our emissivities relative to
their estimates.  We go into this issue in further detail in
Appendix~\ref{sec:conv}.

\citet{2018AJ....155..131M} recently extended their previous Stripe~82
AGN sample by 1.5 magnitude and fit a double power law luminosity
function model to a combined SDSS $+$ Stripe~82 $+$ CFHTLS sample at
$4.7 < z < 5.1$.  These authors fixed the bright-end slope to
$\alpha=-4$ and obtained a shallower faint-end slope ($\beta=-1.97$)
than our models at this redshift ($\beta\sim -2.3$).  Rescaling their
LF to our cosmology and assuming our SED result in the emissivities
shown in Figure~\ref{fig:e912_2}.  Our emissivity agrees with the
measurement of \citet{2018AJ....155..131M} for $M_{1450}<-21$ but is
higher than their measurement for $M_{1450}<-18$, which is not
surprising due to the difference in the luminosity function slopes.
The emissivity derived from the luminosity function of
\citet{2018PASJ...70S..34A} is also shown in Figure~\ref{fig:e912_2}.
These authors report 1668 AGN candidates at $3.5<z<4.3$, of which 76
have spectroscopic redshifts.  Their reported faint-end slope is very
flat ($\beta=-1.3\pm 0.05$), inconsistent with our determination at
all redshifts.  Their inferred emissivity is consistent with our
estimate for $M_{1450}< -21$.  Their emissivity does not change
significantly with the integration limit due to the shallow faint-end
slope.  As a result, our $M_{1450}< -18$ determination of the
emissivity is considerably higher than their estimate.  We also
compare our emissivities with the determinations reported by
\citet{2012ApJ...755..169M} at $z\sim 3.2$ and $z\sim 4$.  Their
emissivity in reasonable agreement with our model for $M_{1450}<-21$
at $z\sim 3.2$.  Our emissivity estimates are in good agreement with
recent estimates of \citet{2018MNRAS.474.2904P} at $z=4$--$5$.

At low redshifts, \citet{2009A&A...507..781S} combined data from the
Hamburg/ESO survey and the SDSS to measure a quasar luminosity
function at $z=0$ in their $B_J$ band to minimize host galaxy
contribution.  We convert their $B_J$ magnitudes (Vega system) to
$M_{1450}$ (AB) as
\begin{equation}
  M_{1450,\mathrm{AB}}=M_{B_J, \mathrm{Vega}}+0.59.  
\end{equation}
In this magnitude system, the luminosity function reported by
\citet{2009A&A...507..781S} has a very faint break luminosity of
$M_*=-18.87$.  The faint-end slope is also steep ($\beta=-2$).  We
integrate this luminosity function down to our limiting magnitudes of
$M_{1450}=-18$ and $-21$.  The resultant values are shown in
Figure~\ref{fig:e912_2}.  Our emissivity values at $z=0$ lie in
between the \citet{2009A&A...507..781S} measurements.  The convergence
in the emissivity is slower in their measurements than in our models
because of the shallower faint-end slope at this redshift in our model
($\beta\sim -1.22$).  

\begin{figure*}
  \begin{center}
    % rtg2.draw_g_paper() 
    \includegraphics[scale=0.65]{g.pdf}
  \end{center}
  \caption{AGN contribution to the hydrogen photoionisation rate,
    assuming unit escape fraction, when the AGN luminosity function is
    integrated down to $M_{1450}=-21$ (blue curve and shaded region)
    and $M_{1450}=-18$ (red curve and shaded region).  The shaded
    regions show the one-sigma (68.26\%) uncertainty.  Also shown are
    the photoionization rate measurements by \citet[filled
      circles]{2013MNRAS.436.1023B}, \citet[inverted
      triangles]{2011MNRAS.412.2543C}, and
    \citet[pentagons]{2017MNRAS.467.3172G}, and models of
    \citet[dotted brown curve]{2012ApJ...746..125H}, the QSO
    contribution in this model (dashed grey), \citet[dashed
      brown]{2015ApJ...813L...8M}, the QSO contribution from the model
    of \citet[dashed orange]{2015MNRAS.451L..30K}, \citet[dotted
      grey]{2017ApJ...837..106O}, and \citet[dashed
      grey]{2018arXiv180104931P}.}
  \label{fig:gammapi}
\end{figure*}

\subsection{Hydrogen photoionization rate}
\label{sec:gammahi}

The 912\,\AA\ emissivity can now be used to calculate the hydrogen
ionizing flux and the hydrogen photoionization rate contributed by
AGN.  This adds an additional layer of uncertainty as the \HI\ column
density structure of the IGM has not been measured at all redshifts of
interests, thereby requiring uncertain extrapolation.  The frequency
dependence of the AGN SED above 1~Ry is given by
Equation~(\ref{eqn:sed}) or (\ref{eqn:sed_faint}), so that for $\nu >
\nu_{912}$
\begin{equation}
  \epsilon_\nu = \epsilon_{912}\left(\frac{\nu}{\nu_{912}}\right)^\alpha,
  \label{eqn:epsilon_freq}
\end{equation}
where $\alpha=-1.70$ (Equation~\ref{eqn:sed}) or $-0.56$
(Equation~\ref{eqn:sed_faint}).  We assume that faint AGN
($M_{1450}>-23$) have the shallower spectral slope.  The flux in the
IGM is written as a solution of the cosmological radiative transfer
equation as \citep{2012ApJ...746..125H}
\begin{multline}
  j(\nu_0, z_0)=\frac{1}{4\pi}\int_{z_0}^\infty dz\frac{dl}{dz}
  \frac{(1+z_0)^3}{(1+z)^3}\epsilon(\nu,z)\\
  \times\exp{(-\tau_\mathrm{eff}(\nu_0, z_0, z))},
  \label{eqn:flux}
\end{multline}
where
\begin{equation}
  \nu = \nu_0\left(\frac{1+z}{1+z_0}\right).
\end{equation}
Here the effective optical depth is estimated as
\begin{equation}
  \tau_\mathrm{eff}(\nu_0, z_0, z) = \int_{z_0}^z dz^\prime\int_0^\infty
  dN_\mathrm{HI} f(N_\mathrm{HI}, z^\prime) (1-e^{-\tau_\nu}),
\end{equation}
where $\tau_\nu=\sigma_\nu N_\mathrm{HI}$.  (We ignore the small
contribution of helium to the opacity.)  We assume the \HI\ column
density distribution given by \citet{2012ApJ...746..125H}.
%% JFH elaborate on this. Say it is chosen to reproduce Worseck's MFP measurements
%% and that it sensible extrapolate or something. You need to make it clear that it is indeed
%% Gabors MFP that you are using here. 
The
\HI\ photoionization rate is then given by
\begin{equation}
  \Gamma_\mathrm{HI}=\int_{\nu_{912}}^\infty d\nu
  \frac{4\pi j(\nu,z)}{h\nu} \sigma(\nu),
\end{equation}
where $\sigma$ is the photoionization cross-section.
%% JFH Is it obvious that the uncertaintites in the MFP which propagate into the HM fit for the column density
%% distribution do not matter? If so this needs to be explicitly stated. 
We assume an
upper limit of $z=15$ while integrating Equation~(\ref{eqn:flux}).
The result is shown in Figure~\ref{fig:gammapi} for the luminosity
function integration limits of $M_{1450}=-21$ and $-18$.  Also shown
for comparison are the measurements of \citet{2013MNRAS.436.1023B},
derived from the Lyman-$\alpha$ forest, and the measurements by
\citet{2011MNRAS.412.2543C} from quasar proximity zones.
%% JFH Please add the data from this paper by Davies:
%% http://adsabs.harvard.edu/cgi-bin/nph-ref_query?bibcode=2018ApJ...855..106D&amp;refs=CITATIONS&amp;db_key=AST
For an
integration limit of $M_{1450}=-21$ the AGN contribution to the
hydrogen photoionization rate falls short of 100\% across the redshift
range.  It is marginally consistent with the measured photoionization
rate at $z=2.4$.  The photoionization rate in our model for
$M_{1450}=-21$ has the same evolution but a higher amplitude by a
factor of $\sim 2$ as the QSO contribution to the \HI\ photoionization
rate in the model of \citet{2012ApJ...746..125H}.  For both of our
integration limits, the photoionisation rate peaks at $z\sim 2$.  For
the integration limit of $M_{1450}=-18$, AGN can provide all the flux
necessary to explain the observed \lya forest between $z=2.4$ and
$3.2$, with a contribution from other sources necessary only at higher
redshifts.  The photoionization rate in the $M_{1450}<-18$ case also
agrees with the inference of \citet{2017MNRAS.467.3172G} from the
low-redshift ($z<0.6$) \lya\ data.  We discuss this low-redshift
evolution in greater detail in the next section.

An important conclusion from Figure~\ref{fig:gammapi} is that the AGN
contribution to hydrogen reionization is likely subdominant, although
it can be non-negligible if faint AGN down to $M_{1450}=-18$ emit
hydrogen-ionizing photons with our assumed SED and a unit escape
fraction.  At $z=6.1$, AGN with $M_{1450}<-18$ contribute about $10\%$
of the required \HI\ ionizing flux.  The contribution of
$M_{1450}<-21$ at this redshift is $\sim 3\%$.  At $z=6$, our
determinations are lower than those by \citet{2015AA...578A..83G} by
almost an order of magnitude.  This difference arises from the
difference in the inferred emissivities in our models relative to
\citet{2015AA...578A..83G}, as discussed in the previous section.  The
photoionization rate evolution in the model of
\citet{2015ApJ...813L...8M} agrees with our determination at low
redshifts ($z<0.5$) for $M_{1450}<-18$ but is much higher at $z>4$, as
expected from the higher emissivities assumed by these authors.  At
$3<z<6$ our model photoionization rates are understandably lower than
those in the models of \citet{2017ApJ...837..106O} and
\citet{2018arXiv180104931P} as these authors include contribution to
the photoionization rate from galaxies in their models. The
differences in our model from that of \citet{2015MNRAS.451L..30K} are
a result of our differences from the reported emissivities of
\citet{2009MNRAS.392...19C} and \citet{2013A&A...551A..29P}, which are
used by \citet{2015MNRAS.451L..30K} to derive an emissivity evolution
model.

\subsection{Photon underproduction at $z=0$?}
% Kollmeier number of ~10 times higher than HM12 in figure.

%% JFH I disagree with your approach here. It is not interesting to simply extrapolate your z > 0.6 to these
%% redshifts in my opinion, when you have the ability to fit them or at least to show the binned fits if you cannot
%% get a global model to work. Why is it interesting to write an entire section on an extrapolation?

It is instructive to closely examine if the corresponding hydrogen
photoionization rate is consistent with the \HI\ column density
distribution function measured from the \lya\ forest
\citep{2016ApJ...817..111D} at low redshifts ($z<0.5$).
\citet{2014ApJ...789L..32K} argued that in order to match the
\HI\ column density distribution function observed by
\citet{2016ApJ...817..111D} at these redshifts, hydrodynamical
cosmological simulations require a hydrogen photoionization rate that
is a factor of five larger than that in the UV background model of
\citet{2012ApJ...746..125H}.  Several recent studies have addressed
this `photon underproduction crisis' \citep{2015MNRAS.451L..30K,
  2015ApJ...811....3S, 2017MNRAS.467.3172G, 2017MNRAS.467.4802F,
  2017MNRAS.466..838G, 2017MNRAS.467L..86V}.  On the one hand, these
studies emphasised the uncertainty in the \citet{2012ApJ...746..125H}
UVB model at these redshifts due to the lack of certainty in the UV
photon emissivities of galaxies and AGN \citep{2015MNRAS.451L..30K,
  2015ApJ...811....3S}.  On the other hand, they noted the uncertainty
in the results of the cosmological simulations at these redshifts, due
to effects such as AGN feedback and limited numerical resolution
\citep{2015ApJ...811....3S, 2017MNRAS.467L..86V, 2017MNRAS.471.1056N,
  2017ApJ...837..106O, 2017MNRAS.466..838G, 2017MNRAS.467.3172G,
  2017ApJ...835..175G}.  A general conclusion of these studies was
that the discrepancy between the photoionization rate required by the
observed \HI\ column density distribution and predicted by the UVB
model of \citet{2012ApJ...746..125H} is likely to be smaller than that
found by \citet{2014ApJ...789L..32K}.

\begin{figure}
  \begin{center}
    % rtg2.draw_g_puc()
    \includegraphics[width=\columnwidth,keepaspectratio]{g_puc.pdf}
  \end{center}
  \caption{Evolution of the hydrogen photoionisation rate at low
    redshifts, when the AGN luminosity function is integrated down to
    $M_{1450}=-21$ (blue curve and shaded region) and $M_{1450}=-18$
    (red curve and shaded region).  Shaded regions show the one-sigma
    (68.26\%) uncertainty.  Also shown are the models and inferences
    from \citet[dotted brown curve]{2012ApJ...746..125H}, the QSO
    contribution in this model (dashed grey), \citet[dotted
      green]{2015ApJ...813L...8M}, \citet[dashed
      black]{2015ApJ...811....3S}, the QSO contribution from the model
    of \citet[dashed orange]{2015MNRAS.451L..30K}, \citet[dotted
      grey]{2017ApJ...837..106O}, \citet[dashed
      brown]{2018arXiv180104931P}, \citet[yellow
      box]{2017MNRAS.467.4802F}, \citet[black
      box]{2017MNRAS.467L..86V}, \citet[inverted
      triangle]{2013MNRAS.436.1023B},
    \citet[pentagon]{2014ApJ...789L..32K}, and
    \citet[circle]{2017MNRAS.467.3172G}. Note that we use the the
    \HI\ column density distribution model from
    \citet{2012ApJ...746..125H} to derive the photoionization rate.
    \label{fig:puc}}
\end{figure}

Figure~\ref{fig:puc} shows the evolution of the \HI\ photoionization
rate due to AGN in our model for luminosity function integration
limits of $M_{1450}=-21$ and $-18$ at redshifts $z<3$.  Note that this
photoionization rate is derived by fitting the model from
Equation~(\ref{eqn:e912fit}) to the emissivities obtained from
luminosity functions in various redshift bins.  While doing this, as
discussed above, redshift bins that were interpreted as being affected
by systematic errors were ignored.  As a result, the emissivity model
used in calculating the photoionization rate is an extrapolation at
$z<0.6$.  However, it is interesting to note that even if we disregard
the $z<0.6$ data due to systematic errors, the extrapolated
emissivities at these redshifts are consistent with the emissivities
calculated from our luminosity function fits to these data for either
of our chosen integration limits.  This can be seen in
Figure~\ref{fig:e912_2}.  We find that the low-redshift
\HI\ photoionization rate in our model is higher than that in the
\citet{2012ApJ...746..125H} UVB model by a factor of 1.2 for
$M_{1450}<-21$ and a factor of 2 for $M_{1450}<-18$.  The
photoionization rate obtained when the faint AGN ($M_{1450}<-18$) are
included is consistent with a variety of recent estimates.  This rate
is in reasonable agreement with the photoionization rate found by
\citet{2017MNRAS.467.3172G}, who found that this photoionisation rate
is sufficient to explain the \HI\ column density distribution measured
by \citet{2016ApJ...817..111D}, thus alleviating the photon
underproduction problem without requiring additional contribution of
hydrogen-ionizing photons from galaxies.  At $z<0.5$, this
photoionization rate evolution in our model appears consistent also
with that derived by \citet{2015MNRAS.451L..30K} using the quasar
luminosity function measurements of \citet{2009MNRAS.392...19C} and
\citet{2013A&A...551A..29P}, although the evolution is somewhat
shallower in our model at $z>0.5$.  The model of
\citet{2015ApJ...813L...8M} also agrees with our $M_{1450}<-18$
inference at $z<0.7$.  \citet{2015ApJ...811....3S} compared the
\HI\ column density distribution measurements to cosmological
simulations with an enhanced photoionization rate relative to the
\citet{2012ApJ...746..125H} model to find that
$\Gamma_\mathrm{HI}=4.6\times 10^{-14}(1+z)^{4.4}\,\mathrm{s}^{-1}$
produces the observed \HI\ column densities.  This photoionization
rate was achieved in the simulations of \citet{2015ApJ...811....3S} by
a combination of quasars and galaxies (with an escape fraction of
hydrogen-ionizing photons assumed to be $f_\mathrm{esc}=0.05$).  As
seen in Figure~\ref{fig:puc}, this photoionization rate agrees with
our model at $z\sim 0$ for $M_{1450}<-18$, although it evolves
somewhat more rapidly at higher redshifts.  The photoionization rate
estimate by \citet{2017MNRAS.467.4802F} from the H$\alpha$ surface
brightness of a $z\sim 0$ galaxy observed by VLT/MUSE is somewhat
higher than even our $M_{1450}<-18$ determination.  However, it is
possible that the \citet{2017MNRAS.467.4802F} estimate is an upper
limit, as the contribution of local sources to the photoionization
rate is neglected in their modelling.  The requirement of an enhanced
photoionization rate at $z\sim 0$ relative to
\citet{2012ApJ...746..125H} UVB model is also confirmed by the
simulations presented by \citet{2017MNRAS.467L..86V},
\citet{2017ApJ...837..106O}, and \citet{2018arXiv180104931P}.  But
note that, as discussed above in Section~\ref{sec:e912}, a direct
comparison of our results with the models of
\citet{2015ApJ...813L...8M}, \citet{2015MNRAS.451L..30K}, and
\citet{2018arXiv180104931P} is difficult because of the inhomogenous
redshift-dependent integration limits adopted by these authors.

\subsection{Helium reionization}

\begin{figure}
  \begin{center}
    % qhe.py 
    \includegraphics[width=\columnwidth,keepaspectratio]{q.pdf}
  \end{center}
  \caption{Evolution of the \HeIII\ ionization fraction in our model,
    when the AGN luminosity function is integrated down to
    $M_{1450}=-21$ (blue curve and shaded region) and $M_{1450}=-18$
    (red curve and shaded region).  The shaded regions show the
    one-sigma (68.26\%) uncertainty.  Other curves show models of
    \citet[solid grey]{2012ApJ...746..125H}, \citet[dashed
      grey]{2015ApJ...813L...8M}, \citet[brown with shaded
      region]{2016ApJ...828...90L}, \citet[blue]{2018arXiv180104931P},
    \citet[maroon]{2009ApJ...694..842M}, and
    \citet[green]{2014MNRAS.445.4186C}.  Note that we derive
    $Q_\nHeIII$ using Equation~(\ref{eqn:qHedot}), which ignores the
    presence of \HeII\ Lyman-limit systems and becomes inaccurate
    towards $Q_\nHeIII\sim 1$. \label{fig:qhe}}
\end{figure}

We now consider the implications of our AGN luminosity function models
for \HeII\ reionization.  We use the emissivity model developed in
Section~\ref{sec:e912} for this purpose.  The evolution of the
volume-averaged fraction $Q_\nHeIII$ of \HeIII\ is given by
\citep{2012ApJ...746..125H}
\begin{equation}
  {dQ_\nHeIII\over dt}={\dot n_{\rm ion,4}\over \langle n_\nHe \rangle} -{Q_\nHeIII\over
    \langle t_{\rm rec,He}\rangle},
  \label{eqn:qHedot}
\end{equation}
where $\dot n_{\rm ion,4}$ is the number density of photons with
energy 4~Ry and above produced per unit time, $n_\nHe$ is the Helium
number density, and $\langle t_{\rm rec,He}\rangle$ is an average
recombination time scale for \HeIII.  We obtain the evolution of $\dot
n_{\rm ion,4}$ for AGN with $M_\mathrm{1450}<-21$ and
$M_\mathrm{1450}<-18$ by integrating the contribution of double power
law luminosity function obtained in distinct redshifts bins and
fitting Equation~(\ref{eqn:e912fit}) to the resultant \HeII-ionizing
emissivities.  We extrapolate the resultant fit to $z>7$, although the
contribution from this extrapolated emissivity to \HeII-reionization
is small.  As before, we assume the SED given by
Equations~(\ref{eqn:sed}), which has a spectral index of
$\alpha_\mathrm{EUV}=-1.7$.  The recombination time scale in
Equation~(\ref{eqn:qHedot}) is given by
\begin{equation}
  \langle t_{\rm rec}\rangle=[(1+2\chi) \langle n_\nH\rangle \alpha_B\,C]^{-1},
  \label{eq:trec}
\end{equation}
where $\langle n_\nH\rangle$ is the average physical number density of
hydrogen, $\alpha_B$ is the case-B \HeIII\ recombination rate, and
$\chi=0.079$ is the cosmic number fraction of helium for a cosmic
helium mass fraction of $Y_\mathrm{He}=0.24$.  The hydrogen number
density is given by $\langle n_\nH\rangle=1.881\times
10^{-7}(1+z)^3$~cm$^{-3}$.  We assume that the helium clumping factor
$C$ is given by \citep{2015ApJ...813L...8M}
\begin{equation}
  C = 2.9\left(\frac{1+z}{6}\right)^{-1.1}.
\end{equation}
We use the fitting function provided by \citet{1997MNRAS.292...27H}
for the recombination coefficient.

Equation~(\ref{eqn:qHedot}) neglects the \HeII\ Lyman-limit systems
that play an increasingly important role towards the end of Helium
reionization \citep{2009MNRAS.395..736B, 2017ApJ...851...50M}.  In the
absence of these self-shielded systems of high-density \HeII,
$Q_\nHeIII$ can continue to increase beyond unity.  The mean free path
of \HeII-ionizing photon diverges.  We set $Q_\nHeIII=1$ when this
happens.  This can be corrected by accounting for the \HeII\ column
density distribution and filtering the \HeII-ionizing radiation
background through it.  Unfortunately, the \HeII\ column density
distribution is itself uncertain \citep{2018arXiv180104931P}.  We
consider our treatment of $Q_\nHeIII$ accurate enough for the purpose
of this work, while noting that the redshift of \HeII\ reionization is
likely an overestimate in our model.

The resultant reionization histories of \HeII\ are shown in
Figure~\ref{fig:qhe} for $M_\mathrm{1450}<-18$ and
$M_\mathrm{1450}<-21$.  When only AGN with $M_\mathrm{1450}<-21$ are
accounted, Helium reionization is complete at $z\sim
3.20^{+0.08}_{-0.12}$ where the uncertainty is one-sigma.  When the
luminosity function is extrapolated down to $M_\mathrm{1450}<-18$,
\HeII\ is reionized at $z=3.47^{+0.22}_{-0.26}$.  \HeII\ reionization
begins at $z\gtrsim 5$ in both models.  Observational constraints on
the \HeII\ reionization history are derived from measurements of the
effective \HeII\ \lya optical depth \citep{2001Sci...293.1112K,
  2004ApJ...600..570S, 2006A&A...455...91F, 2010ApJ...722.1312S,
  2011ApJ...733L..24W, 2016ApJ...825..144W}.  These are considered to
imply a redshift of \HeII\ reionization of $z=2.7$
\citep{2011ApJ...733L..24W}.  While we do not compute the effective
\HeII\ \lya optical depth in our model, the higher reionization
redshifts in our models are partly explained by the absence of
\HeII\ Lyman-limit systems, as discussed above.  Furthermore, the
reionization histories shown in Figure~\ref{fig:qhe} may likely be
consistent with the optical depth measurements of
\citet{2016ApJ...825..144W}, who find that the effective optical depth
rises at $z>2.7$, but low optical depth sightlines continue to be seen
beyond this redshift.  The incidence of these low opacity sightlines
implies that the substantial volume of \HeII\ was ionized already at
$z=3.4$, consistent with our models.

%% JFH I think these next two paragraphs here needs to be shortened considerably.
%% You are spending way to to much time and text discussing the literature. 
In Figure~\ref{fig:qhe}, we also show the reionization histories from
other models of \HeII\ reionization in the literature.  Our model
curve for $M_\mathrm{1450}<-21$ is closest in agreement to the
simulations of \citet{2009ApJ...694..842M} and the analytical model of
\citet{2018arXiv180104931P}.  \citet{2018arXiv180104931P} model the
\HeII-ionizing emissivity of AGN by fitting the functional form in
Equation~(\ref{eqn:e912fit}) to the emissivities reported by
\citet{2009A&A...507..781S}, \citet{2007A&A...472..443B}, and
\citet{2012ApJ...755..169M}.  The resultant emissivity in the model of
\citet{2018arXiv180104931P} is similar in shape but higher in
amplitude than the emissivity in the \citet{2012ApJ...746..125H}
model.  The \HeII-ionizing emissivity of \citet{2018arXiv180104931P}
is closer to our inference for $M_\mathrm{1450}<-21$ than the model of
\citet{2012ApJ...746..125H}.  (However, note that both these models
suffer from inhomogeneous luminosity function integration limits, as
the integration limit is set to a fixed fraction of the break
luminosity, as discussed above in Section~\ref{sec:e912}.)  A spectral
index of $\alpha_\mathrm{EUV}=-1.7$ is assumed.
\citet{2018arXiv180104931P} compute the evolution of $Q_\nHeIII$ by
evolving the \HeIII\ ionization fraction in a gas cloud with the mean
cosmic baryonic density exposed to the average \HeII-ionizing flux.
The ionizing flux is obtained by filtering the ionizing emissivity
through an inhomogeneous IGM in which the column density distribution
of \HeIII\ is derived from that of \HI\ in a manner similar to
\citet{2012ApJ...746..125H}.  As a result, this model incorporates the
effect of \HeII\ Lyman-limit systems, and the $Q_\nHeIII$ evolution in
this model deviates from our predictions towards the end of
reionization, as seen in Figure~\ref{fig:qhe}.  Comparing the
emissivity model in the cosmological radiative transfer simulations of
\citet{2009ApJ...694..842M} with our models is less straightforward.
Quasars are modelled in the simulations of \citet{2009ApJ...694..842M}
by relating quasar luminosities and lifetimes to halo masses following
the galaxy-merger-induced quasar activity model of
\citet{2005ApJ...630..705H}.  The resultant bolometric quasar
luminosity functions are somewhat shallower than those of
\citet{2007ApJ...654..731H}, but with a higher amplitude.  A spectral
index of $\alpha_\mathrm{EUV}=-1.6$ is assumed.  Figure~\ref{fig:qhe}
shows that the resultant reionization history in these simulations
agrees with our model for $M_\mathrm{1450}<-21$, except towards the
end of reionization, where, as before, \HeII\ Lyman-limit systems
become important.

Helium reionization is delayed in the model of
\citet{2012ApJ...746..125H} due to their reduced emissivity at $z\sim
3$.  In the \citet{2015ApJ...813L...8M} model, a significant
\HeII\ fraction is ionized already at $z=6$, and reionization is
complete by $z=4.1$ due to the very high emissivity assumed in this
model.  In the radiation-hydrodynamic simulations by
\citet{2016ApJ...828...90L}, the AGN population is calibrated to the
SDSS \citep{2013ApJ...768..105M} and COSMOS
\citep{2012ApJ...755..169M} data at $z>3.5$, combined with a model for
quasar light curves.  The fiducial \HeII\ reionization history
obtained by these authors is shown in Figure~\ref{fig:qhe} along with
an estimate of the uncertainty.  Reionization ends considerably later
in this model ($z\sim 2.6$).  The \HeII\ reionization history in our
model for $M_\mathrm{1450}<-18$ agrees closely with that in the
radiative transfer simulations presented by
\citet{2014MNRAS.445.4186C}.  The AGN population in these simulations
is `turned on' at $z=5$ and evolved assuming a pure luminosity
evolution to match the inferred luminosity function of
\citet{2011ApJ...728L..26G}.  The somewhat early reionization in the
simulations of \citet{2014MNRAS.445.4186C} relative to most other
models shown in Figure~\ref{fig:qhe} may be driven by the
significantly higher emissivity from AGN with intermediate brightness
($M_{1450}\sim -23$) in the model of \citet{2011ApJ...728L..26G}.
However, the limited numerical resolution of these simulations may
also play a role \citep{2014MNRAS.445.4186C}.

An important constraint on \HeII\ reionization can be derived from the
thermal history of the IGM.  \citet{2018MNRAS.473.1416M} constrained
the evolution of the 912\,\AA\ emissivity of AGN by assuming a
spectral index of $\alpha_\mathrm{EUV}=-1.57$ and using the
measurements of the effective \HeII\ \lya optical depth
\citep{2016ApJ...825..144W} and the IGM temperature
\citep{2011MNRAS.410.1096B}.  \citet{2018MNRAS.473.1416M} found that
although these data favour a reionization history in which \HeII\ is
reionized at $z>4$, the 912\,\AA\ emissivity of AGN is nevertheless
required to drop rapidly at $z>3$.  This emissivity evolution is much
more rapid than that in our model for $M_\mathrm{1450}<-21$, with the
caveat that the constraint derived by \citet{2018MNRAS.473.1416M} is
model-dependent and assumes a shallower EUV spectrum.  A steeper
spectrum with $\alpha_\mathrm{EUV}=-1.8$ was assumed in a recent model
by \citet{2018arXiv180109693K}.  The resultant emissivity evolution is
consistent with our model for $M_\mathrm{1450}<-21$, although somewhat
more rapid at $z>2$.  The AGN luminosity function is derived by
combining determinations from various data sets
\citep{2015MNRAS.451L..30K}.  An inhomogeneous integration limit of
$0.01L_*$ is applied to calculate the emissivity.  \HeII\ reionization
ends at $z=2.8$ in the model of \citet{2018arXiv180109693K}.
(\citealt{2017MNRAS.471..255K} argued that for their luminosity
function model, spectra with $\alpha_\mathrm{EUV}>-1.6$ are unlikely
to be consistent with the \HeII\ \lya data.)
\citet{2017MNRAS.468.4691D} considered the effect of an AGN-dominated
reionization on the \HI\ Ly$\alpha$ opacity at $z>5$,
\HeII\ Ly$\alpha$ opacity at $z\sim 3.1$--$3.3$, and the thermal
history of the IGM.  These authors found that that any scenario in
which reionization of \HeII\ completes at $z>3$ is inconsistent with
the measurements of the IGM temperature at $z=4$--$5$.  We leave the
analysis of the IGM thermal history implied by our AGN luminosity
models for future work.

\section{Conclusions}
\label{sec:conc}

We have presented an analysis of the evolution of the UV luminosity
function of AGN from redshift $z=0$ to $7.5$ using a combined sample
of close to 85,000 colour-selected AGN from 12 data sets, homogenized
with respect to the assumed cosmology, intrinsic AGN spectrum, and the
magnitude system.  AGN from 11 of these 12 samples have reported
spectroscopic redshifts, and all but two AGN have known completeness
estimates.  These data span magnitudes down to $M_{1450}=-22$ at high
redshifts ($z\gtrsim 3.5$).

%% JFH Make some figure references below to remind the reader of where these
%% results are. Also some people just jump to conclusions so they need to know where to look. 
We find that the UV luminosity function of AGN prefers a double power
law description at all redshifts considered in this study, except
possibly at very high redshifts ($z\sim 6$) where a single power law
seems to be preferred.  The behaviour of the luminosity function at
lower redshifts ($z=4$--$5.5$) suggests that this single power law at
$z\sim 6$ can be interpreted as the faint end of the double power law
with a break at $M_*=-29$.  The break magnitude $M_*$ of the double
power law luminosity function shows a steep brightening towards high
redshifts.  The break evolves from $M_*<-25$ at $z\sim 1$ to $M_*\sim
-29$ at $z\sim 6$.  Correspondingly, the amplitude $\phi_*$ of the
luminosity function drops rapidly towards high redshifts from
$\phi_*\sim 10^{-6}$~mag$^{-1}$cMpc$^{-3}$ at $z\sim 1$ to $\phi_*<
10^{-10}$~mag$^{-1}$cMpc$^{-3}$ at $z\sim 6$.  The faint-end slope
$\beta$ becomes increasingly negative towards high redshifts,
resulting in the a steepening of the luminosity function.  The
faint-end slope is $\beta\sim -1.8$ at $z\sim 1$ and steepens to
$\beta\sim-2.5$ at $z\sim 6$.  The bright-end slope $\alpha$ also
shows moderate steepening towards high redshifts.  However, the
constraints on $\alpha$ are weak, particularly at the high redshift
because of the bright break luminosity.

%% JFH Figure references here as well. 
An important finding of this work is that there are severe systematic
errors in observed luminosity function at certain redshifts.  We find
that the most significant systematic bias is at redshift $z\sim 3$ in
the BOSS sample of quasars.  There is also some evidence of a
systematic bias in the lowest redshifts $z\lesssim 0.5$.  At $z<2.2$
there is some indication of a systematic error in completeness
estimates at the faintest magnitudes $M_{1450}\sim -19$.  Systematic
errors make it harder to construct simple empirical models to describe
the luminosity function evolution.  We develop three such models in
this paper.  We find that a fourteen-parameter model (Model 1)
describes the redshift evolution of the luminosity function rather
well.  However, this model prefers a very rapid evolution in the
faint-end slope at $z=3.5$, which is probably unphysical.  Our other
models (Models 2 and 3) do not have this feature, but these models do
not fit the data as well as Model 1 at $z=4$--$6$.  The integrated
quasar number density diverges at high redshifts in all three models.

In spite of the rapid steepening of the faint end of the luminosity
function at high redshifts, our derived hydrogen-ionizing emissivity
from AGN is lower that recent determinations from the literature
\citep{2015AA...578A..83G}.  This is because the bright break
luminosities reduce the contribution of intermediate-brightness AGN to
the LyC emissivity.  We find that the LyC emissivity peaks at $z\sim
2$.  We derive the average hydrogen photoionization rate by filtering
the hydrogen-ionizing emissivity through the intergalactic \HI\ column
density distribution.  We find that the HI photoionization rate is
lower than that required by the Lyman-$\alpha$ forest data at $z>2$.
Our hydrogen photoionization rate estimates are also considerably
lower (by a factor of 10) than that of \citet{2015AA...578A..83G}.
This indicates that hydrogen reionization is unlikely to have caused
by AGN alone.  Extrapolating the \HI\ column density distribution
function to $z>3.5$, we find that even when AGN down to $M_{1450}=-18$
are assumed to supply hydrogen-ionizing photons with a unit escape
fraction, the resultant photoionization rate is only about 10\% of
what is observed at $z\sim 6$.  At the lowest redshifts ($z\sim 0$),
we find that the hydrogen photoionization rate is about a factor of
two higher than the estimate of \citet{2007ApJ...654..731H} and
\citet{2012ApJ...746..125H}.  When compared with the recent analysis
of \citet{2017MNRAS.467.3172G}, this photoionisation rate appears to
alleviate the photon underproduction crisis
\citep{2014ApJ...789L..32K}.  Helium reionization in our model occurs
at redshift $z=3.2$ or $z=3.5$ if the luminosity function is
integrated down to $M_{1450}=-21$ and $-18$, respectively.

These results will be made more robust with new data from deep
large-area surveys to discover faint quasars at high redshifts, such
as the Subaru High-z Exploration of Low-Luminosity Quasars (SHELLQs)
project \citep{2016ApJ...828...26M} and the VISTA Extragalactic
Infrared Legacy Survey (VEILS; \citealt{2017MNRAS.464.1693H}),
followed by the Wide Field Infrared Survey Telescope (WFIRST;
\citealt{2013arXiv1305.5422S}) and Euclid \citep{2011arXiv1110.3193L}.
This study highlights the need for a better understanding of
luminosity function systematics.  In particular, at $z>4$, it seems
important to include extended objects at faint magnitudes that could
be impacted by host galaxy light, take follow-up spectroscopy to
confirm a fraction of these as high-redshift broad-line AGN, and
carefully assess host galaxy contamination.

\section*{Acknowledgements}

We thank Eilat Glikman, Linhua Jiang, Nobunari Kashi\-kawa, Ian
McGreer, Nick Ross, Chris Willott, and Jinyi Yang for sharing data and
especially for sharing their quasar selection functions.  It is a
pleasure to acknowledge useful discussions with James Aird, Eduardo
Ba\~nados, Manda Banerji, Tirthankar Roy Choudhury, George Efstathiou,
Xiaohui Fan, Andrea Ferrara, Prakash Gaikwad, Francesco Haardt, Martin
Haehnelt, Paul Hewett, David Hogg, Vikram Khaire, Sergey Koposov,
Donald Lynden-Bell, Roberto Maiolino, Richard McMahon, Daniel
Mortlock, Ewald Puchwein, Gordon Richards, Alberto Rorai, Bram
Venemans and Stephen Warren.  GK acknowledges support from ERC
Advanced Grant 320596 `The Emergence of Structure During the Epoch of
Reionization'.
%% Joe, do you need to add anything to acknowledgements? 

\appendix

\section{Posterior distributions}

Figure~\ref{fig:corner} shows marginalised one-dimensional and
two-dimensional posterior probability distribution functions (PDFs) of
the four parameters, $\phi_*$, $M_*$, $\alpha$ and $\beta$, of the
double-power-law luminosity function in the $3.7\leq z < 4.1$ redshift
bin.  The procedure adopted for fitting this model is described in
Section~\ref{sec:bins}.  Figure~\ref{fig:corner} illustrates that the
four parameters are well-constrainted.  This figure also shows the
degeneracies between the parameters.  There is a relatively strong
correlation between the amplitude of the luminosity function $\phi_*$
and the break magnitude $M_*$.  The faint-end slope $\beta$ is
positively correlated with the other three parameters.  Similar
behaviour of the posterior distributions is seen in all the redshift
bins defined in Section~\ref{sec:bins}.  In our highest-redshift bin
($5.5\leq z < 6.5$), we impose a prior $\alpha < -4$, which changes
the posterior distributions.  However, various parameter correlations
remain qualitatively unchanged.

\begin{figure*}
  \begin{center}
    % corner.corner() with bins.lfs[-4]
    \includegraphics[width=\textwidth]{corner_z3p9.pdf}
  \end{center}
  \caption{Posterior distributions of the four double-power-law
    parameters in the $3.7\leq z < 4.1$ redshift bin.  The blue
    squares indicate median values.  Similar behaviour of the
    posterior distributions is seen in all other redshift bins defined
    in Section~\ref{sec:bins}.
    \label{fig:corner}}
\end{figure*}


\section{Comparison with other luminosity function determinations}

Figure~\ref{fig:params_grand} compares the parameters of the double
power law luminosity function from our analysis of
Section~\ref{sec:bins} to values reported in the literature.  In
general, our break luminosity is brighter and the faint-end slope is
steeper than other determinations.  Note however that several results
from the literature shown in Figure~\ref{fig:params_grand} make
restrictive assumptions while fitting double power law models to data.
For example, \citet{2013ApJ...768..105M} fix the bright-end slope
$\alpha$ to $-4$ at $z=4.7$--$5.1$.  \citet{2016ApJ...833..222J} fix
the faint-end slope $\beta$ to $-2.8$ at $z=5.7$--$6.4$.
\citet{2017ApJ...847L..15O} fix the bright-end slope $\alpha$ to
$-2.8$ at $z=5.5$--$6.5$.  \citet{2015AA...578A..83G} fix the
faint-end slope and the break luminosity in their highest-redshift bin
($z=5.0$--$6.5$).  The choice of data sets is also often different.
For instance, \citet{2015AA...578A..83G} do not include the SDSS
Stripe 82 data from \citet{2013ApJ...768..105M} and the AGN samples of
\citet{2010AJ....139..906W} and \citet{2015ApJ...798...28K} in their
analysis.  (We discuss the results of \citet{2015AA...578A..83G} in
greater detail in Appendix~\ref{sec:conv}.)  Of the remaining
determinations, our parameter values are closest to those reported by
\citet{2016ApJ...829...33Y} at $z=4.7$--$5.4$.

\begin{figure*}
  \begin{center}
    % summary_grand.py
    \includegraphics[width=0.7\textwidth]{evolution_grand.pdf}
  \end{center}
  \caption{A comparison of our inferred double power law parameter
    values with those reported in the literature.  Black points show
    our determinations from Figure~\ref{fig:evoln}.  Blue points show
    other values, from \citet[triangles]{2012ApJ...755..169M},
    \citet[squares]{2016ApJ...833..222J},
    \citet[diamonds]{2011ApJ...728L..26G},
    \citet[crosses]{2013ApJ...768..105M}, \citet[downward
      triangles]{2015AA...578A..83G},
    \citet[pentagons]{2018PASJ...70S..34A}, \citet[leftward
      triangles]{2017ApJ...847L..15O},
    \citet[asterisks]{2016ApJ...829...33Y},
    \citet[circles]{2013ApJ...773...14R}, and \citet[rightward
      triangles]{2009A&A...507..781S}.  Where necessary, values from
    the literature have been converted to our cosmology.
    \label{fig:params_grand}}
\end{figure*}

\section{Comparison with G15}
\label{sec:conv}
%% Define 'G15' somewhere above.
In the double power law luminosity function models presented in
distinct redshift bins in Section~\ref{sec:bins}, we did not include
the 19 low-luminosity ($M_{1450}>-22.6$) AGN between redshifts $z=4.1$
and $6.3$ reported by G15.  While this was done in order to restrict
our sample to quasars with spectroscopic redshift determinations, it
is instructive to consider how our results are affected if the G15 AGN
are added to the analysis.  In their work, G15 found a shallower
faint-end slope ($\beta\sim -1.5$ to $-1.8$) for the luminosity
function at $4.1 < z < 6.3$, relative to our result from other AGN
samples at these redshifts ($\beta\sim -2.0$ to $-2.5$).  Still, G15
derived a higher 912\,\AA\ emissivity than our estimates
(cf.\ Figure~\ref{fig:e912_2}), so that in their analysis AGN can
produce all ionizing photons necessary to keep hydrogen ionized and
explain the \lya data.  The black points in
Figure~\ref{fig:params_giallongo} show the parameters of the double
power law luminosity function from our analysis of
Section~\ref{sec:bins}.  The red open circles show the parameter
values obtained when the G15 sample is added to the analysis.  We find
that the two results are highly consistent, showing that the G15
sample is consistent with our double power law fit obtained from other
AGN samples are comparable redshifts.  This is surprising as the
integrated 912\,\AA\ emissivities in our model are smaller than those
derived by G15.  Figure~\ref{fig:lf_giallongo} provides an
explanation.  As seen in this figure, the double power law fits
favoured by G15 (red dashed curves) are quite different from our fits
(black curves).  The characteristic luminosity $M_*$ obtained by G15
is much fainter ($\sim -23$ at $z = 5$) than that resulting out of our
analysis ($\sim -29$ at $z = 5$).  Thus the 912\,\AA\ emissivities are
enhanced in G15 because of the increase contribution from
intermediate-luminosity AGN in their model.
Figure~\ref{fig:lf_giallongo} suggests that this is possibly because
of the inclusion of SDSS Stripe 82 data from
\citet{2013ApJ...768..105M} and the AGN samples of
\citet{2010AJ....139..906W} and \citet{2015ApJ...798...28K} in our
analysis.  Additionally, the homogenisation of data in our work may
also cause part of the difference.

\begin{figure*}
  \begin{center}
    % summary_giallongo.py
    \includegraphics[width=0.7\textwidth]{evolution_g.pdf}
  \end{center}
  \caption{Effect of the 19 AGN reported by \citet{2015AA...578A..83G}
    on the double power law luminosity function parameters in
    redshift bins from $z=0$ to $7$.  Black points show parameter
    values from Figure~\ref{fig:evoln}.  Red open circles show the
    parameter values obtained when the sample of G15 is added to the
    analysis.  In both cases, vertical error bars show one-sigma
    (68.26\%) uncertainties, and horizontal error bars show widths of
    the redshift bins. \label{fig:params_giallongo}}
\end{figure*}

\begin{figure*}
  \begin{center}
    % bins_withg.py
    % drawlf_giallongocompare.py 
    % giallongo_compare.py 
    \includegraphics[width=\textwidth]{giallongo_compare.pdf}
  \end{center}
  \caption{Luminosity functions in three redshift bins at $z>4.1$.
    Black curves in each panel show the double power law,
    with the corresponding one-sigma (68.26\%) uncertainty shown by
    the grey shaded area.  There are 451, 270, and 69 AGN in each
    redshift bin from left to right, respectively.  These numbers are
    higher than those in Figure~\ref{fig:mosaic} because they include,
    respectively, 9, 7, and 3 AGN from G15.  The magnitude bins
    containing these AGN are shown in purple.  The red dashed curves
    show the double power law fits reported by G15 at $z=4.25, 4.75,$
    and $5.75$. \label{fig:lf_giallongo}}
\end{figure*}

\section{Tables of emissivities and photoionisation rates}
\label{sec:tables}

Table~\ref{tab:emissivity_bins} shows the 912\,\AA\ and
1450\,\AA\ comoving emissivities obtained in various redshift bins
with the one-sigma (68.26\%) uncertainties.  Redshift bins severely
affected by systematic errors are also shown.  The result of the model
presented in Equation~(\ref{eqn:e912fit}), which describes the
emissivity evolution using a smooth function, is tabulated in
Table~\ref{tab:gamma2}, along with the hydrogen photoionization rate
computed in Section~\ref{sec:gammahi}.  In both tables, we show
results for our two integration limits of $M_{1450}<-18$ and
$M_{1450}<-21$.  These tables describe the curves shown in
Figures~\ref{fig:e912_2} and \ref{fig:gammapi}.  Note that the
photoionisation rate calculation assumes an \HI\ column density
distribution given by \citet{2012ApJ...746..125H} and extrapolates
this to high redshifts.

\begin{table*}
  % gammapi.py and tabulate_emissivities.py. 
  \caption{The 912\,\AA\ and 1450\,\AA\ comoving emissivities
    corresponding to the double power law luminosity function models
    in redshift bins presented in Table~\ref{tab:bins} for two
    integration limits.  These emissivities are shown in
    Figure~\ref{fig:e912_2}.  The units are
    $10^{24}$\ erg\ s$^{-1}$\ Hz$^{-1}$\ cMpc$^{-3}$.  Uncertainties
    are one-sigma (68.26\%).}
  \label{tab:emissivity_bins}
  \begin{tabular}{cccc....}
    \hline
    $\langle z\rangle$ &
    $z_\mathrm{bin}$ &
    $z_\mathrm{min}$ &
    $z_\mathrm{max}$ &
    \multicolumn{1}{c}{$\epsilon_{912}$} &
    \multicolumn{1}{c}{$\epsilon_{1450}$} &
    \multicolumn{1}{c}{$\epsilon_{912}$} &
    \multicolumn{1}{c}{$\epsilon_{1450}$} \\
    &
    &
    &
    &
    \multicolumn{1}{c}{$(M_{1450}<-18)$} &
    \multicolumn{1}{c}{$(M_{1450}<-18)$} &
    \multicolumn{1}{c}{$(M_{1450}<-21)$} &
    \multicolumn{1}{c}{$(M_{1450}<-21)$} \\
    \hline
    0.31 & 0.25 & 0.10 & 0.40 & 0.53^{+0.02}_{-0.02} & 0.71^{+0.03}_{-0.03} & 0.30^{+0.01}_{-0.01} & 0.40^{+0.01}_{-0.01} \\
    0.50 & 0.50 & 0.40 & 0.60 & 0.75^{+0.01}_{-0.01} & 1.00^{+0.02}_{-0.02} & 0.58^{+0.01}_{-0.01} & 0.78^{+0.01}_{-0.01} \\
    0.72 & 0.70 & 0.60 & 0.80 & 1.73^{+0.06}_{-0.05} & 2.31^{+0.07}_{-0.06} & 1.19^{+0.02}_{-0.02} & 1.57^{+0.02}_{-0.02} \\
    0.91 & 0.90 & 0.80 & 1.00 & 2.85^{+0.16}_{-0.16} & 3.77^{+0.20}_{-0.19} & 2.10^{+0.05}_{-0.05} & 2.79^{+0.06}_{-0.06} \\
    1.10 & 1.10 & 1.00 & 1.20 & 3.71^{+0.11}_{-0.10} & 4.94^{+0.15}_{-0.13} & 2.91^{+0.05}_{-0.05} & 3.87^{+0.06}_{-0.06} \\
    1.30 & 1.30 & 1.20 & 1.40 & 5.69^{+0.23}_{-0.24} & 7.52^{+0.32}_{-0.30} & 4.46^{+0.10}_{-0.09} & 5.91^{+0.11}_{-0.12} \\
    1.50 & 1.50 & 1.40 & 1.60 & 7.04^{+0.22}_{-0.22} & 9.34^{+0.26}_{-0.31} & 5.59^{+0.09}_{-0.11} & 7.41^{+0.13}_{-0.13} \\
    1.71 & 1.70 & 1.60 & 1.80 & 7.69^{+0.18}_{-0.18} & 10.23^{+0.24}_{-0.26} & 6.71^{+0.10}_{-0.11} & 8.90^{+0.13}_{-0.14} \\
    1.98 & 2.00 & 1.80 & 2.20 & 10.59^{+0.36}_{-0.36} & 14.08^{+0.43}_{-0.42} & 7.93^{+0.13}_{-0.14} & 10.52^{+0.19}_{-0.17} \\
    2.25 & 2.25 & 2.20 & 2.30 & 11.34^{+0.35}_{-0.37} & 14.97^{+0.48}_{-0.50} & 10.06^{+0.20}_{-0.20} & 13.37^{+0.26}_{-0.23} \\
    2.35 & 2.35 & 2.30 & 2.40 & 9.30^{+0.30}_{-0.30} & 12.29^{+0.36}_{-0.35} & 8.02^{+0.15}_{-0.15} & 10.65^{+0.22}_{-0.20} \\
    2.45 & 2.45 & 2.40 & 2.50 & 8.06^{+0.24}_{-0.24} & 10.67^{+0.30}_{-0.26} & 7.16^{+0.13}_{-0.15} & 9.49^{+0.17}_{-0.18} \\
    2.65 & 2.65 & 2.60 & 2.70 & 6.55^{+0.16}_{-0.17} & 8.70^{+0.21}_{-0.20} & 6.46^{+0.15}_{-0.15} & 8.55^{+0.20}_{-0.20} \\
    2.75 & 2.75 & 2.70 & 2.80 & 7.44^{+0.24}_{-0.22} & 9.93^{+0.30}_{-0.31} & 7.17^{+0.20}_{-0.20} & 9.51^{+0.27}_{-0.25} \\
    2.85 & 2.85 & 2.80 & 2.90 & 7.94^{+0.28}_{-0.29} & 10.45^{+0.42}_{-0.41} & 7.48^{+0.29}_{-0.28} & 9.95^{+0.36}_{-0.35} \\
    2.95 & 2.95 & 2.90 & 3.00 & 7.93^{+0.34}_{-0.34} & 10.51^{+0.45}_{-0.46} & 6.83^{+0.20}_{-0.21} & 9.09^{+0.28}_{-0.28} \\
    3.05 & 3.05 & 3.00 & 3.10 & 7.18^{+0.38}_{-0.36} & 9.59^{+0.47}_{-0.50} & 6.24^{+0.20}_{-0.21} & 8.29^{+0.34}_{-0.30} \\
    3.15 & 3.15 & 3.10 & 3.20 & 9.99^{+0.95}_{-1.03} & 13.14^{+1.38}_{-1.56} & 7.01^{+0.35}_{-0.35} & 9.34^{+0.41}_{-0.47} \\
    3.25 & 3.25 & 3.20 & 3.30 & 8.27^{+0.94}_{-1.01} & 11.05^{+1.27}_{-1.17} & 6.06^{+0.35}_{-0.36} & 8.07^{+0.49}_{-0.50} \\
    3.34 & 3.35 & 3.30 & 3.40 & 12.00^{+2.57}_{-2.55} & 16.10^{+3.44}_{-3.31} & 7.21^{+0.61}_{-0.74} & 9.59^{+0.91}_{-0.93} \\
    3.44 & 3.45 & 3.40 & 3.50 & 4.47^{+0.52}_{-0.51} & 5.85^{+0.61}_{-0.64} & 4.37^{+0.45}_{-0.47} & 5.73^{+0.56}_{-0.67} \\
    3.88 & 3.90 & 3.70 & 4.10 & 4.36^{+1.44}_{-1.39} & 5.76^{+1.98}_{-1.84} & 2.54^{+0.44}_{-0.44} & 3.45^{+0.72}_{-0.66} \\
    4.35 & 4.40 & 4.10 & 4.70 & 4.13^{+2.15}_{-1.95} & 5.40^{+2.60}_{-2.57} & 1.81^{+0.58}_{-0.52} & 2.42^{+0.64}_{-0.68} \\
    4.92 & 5.10 & 4.70 & 5.50 & 2.51^{+0.89}_{-0.86} & 3.45^{+1.28}_{-1.31} & 0.97^{+0.20}_{-0.18} & 1.29^{+0.25}_{-0.26} \\
    6.00 & 6.00 & 5.50 & 6.50 & 0.65^{+0.30}_{-0.24} & 0.96^{+0.36}_{-0.37} & 0.21^{+0.04}_{-0.04} & 0.27^{+0.06}_{-0.06} \\
    \hline
  \end{tabular}
\end{table*}

\begin{table*}
  % rtg2.py and tabulate_emissivities_global.py.
  % Check these values once again.
  \caption{The 912\,\AA\ and 1450\,\AA\ comoving emissivities obtained
    by fitting Equation~(\ref{eqn:e912fit}) to the selected redshift
    bins from Table~\ref{tab:emissivity_bins}, and the corresponding
    hydrogen photoionisation rates with the one-sigma (68.26\%)
    uncertainties.  The emissivities are extrapolated here up to
    $z=15$.  Emissivity units are
    erg\ s$^{-1}$\ Hz$^{-1}$\ cMpc$^{-3}$, and photoionisation rate
    units are s$^{-1}$.  These values are shown in
    Figures~\ref{fig:e912_2} and \ref{fig:gammapi}.  See
    Sections~\ref{sec:e912} and \ref{sec:gammahi} for more details.}
  \label{tab:gamma2}
  \begin{tabular}{d......}
    \hline
    \multicolumn{1}{c}{$z$} &
    \multicolumn{1}{c}{$\log_{10}\epsilon_{1450}$} &
    \multicolumn{1}{c}{$\log_{10}\epsilon_{1450}$} &
    \multicolumn{1}{c}{$\log_{10}\epsilon_{912}$} &
    \multicolumn{1}{c}{$\log_{10}\epsilon_{912}$} & 
    \multicolumn{1}{c}{$\log_{10}\Gamma_\mathrm{HI}$} &
    \multicolumn{1}{c}{$\log_{10}\Gamma_\mathrm{HI}$} \\ 
    &
    \multicolumn{1}{c}{$(M_{1450}<-18)$} &
    \multicolumn{1}{c}{$(M_{1450}<-21)$} &
    \multicolumn{1}{c}{$(M_{1450}<-18)$} &
    \multicolumn{1}{c}{$(M_{1450}<-21)$} &
    \multicolumn{1}{c}{$(M_{1450}<-18)$} &
    \multicolumn{1}{c}{$(M_{1450}<-21)$} \\
    \hline
    0.0 & 23.36^{+0.38}_{-0.20} & 23.18^{+0.25}_{-0.22} & 23.01^{+0.23}_{-0.16} & 22.83^{+0.12}_{-0.15} & -13.38^{+0.10}_{-0.06} & -13.57^{+0.06}_{-0.05} \\
    0.1 & 23.54^{+0.29}_{-0.16} & 23.35^{+0.18}_{-0.17} & 23.24^{+0.18}_{-0.13} & 23.06^{+0.09}_{-0.11} & -13.22^{+0.08}_{-0.06} & -13.41^{+0.05}_{-0.04} \\
    0.2 & 23.71^{+0.21}_{-0.14} & 23.51^{+0.13}_{-0.12} & 23.45^{+0.14}_{-0.10} & 23.27^{+0.07}_{-0.09} & -13.07^{+0.06}_{-0.05} & -13.26^{+0.04}_{-0.03} \\
    0.3 & 23.85^{+0.16}_{-0.10} & 23.66^{+0.10}_{-0.08} & 23.64^{+0.10}_{-0.07} & 23.46^{+0.05}_{-0.06} & -12.94^{+0.05}_{-0.04} & -13.12^{+0.03}_{-0.03} \\
    0.4 & 23.99^{+0.11}_{-0.08} & 23.80^{+0.07}_{-0.06} & 23.81^{+0.07}_{-0.06} & 23.63^{+0.04}_{-0.04} & -12.82^{+0.04}_{-0.04} & -12.99^{+0.03}_{-0.02} \\
    0.5 & 24.11^{+0.08}_{-0.06} & 23.93^{+0.05}_{-0.04} & 23.95^{+0.05}_{-0.04} & 23.78^{+0.03}_{-0.03} & -12.71^{+0.03}_{-0.03} & -12.87^{+0.02}_{-0.02} \\
    0.6 & 24.23^{+0.05}_{-0.04} & 24.06^{+0.03}_{-0.02} & 24.09^{+0.04}_{-0.03} & 23.93^{+0.02}_{-0.02} & -12.60^{+0.03}_{-0.03} & -12.76^{+0.02}_{-0.02} \\
    0.7 & 24.34^{+0.03}_{-0.03} & 24.18^{+0.02}_{-0.02} & 24.21^{+0.02}_{-0.03} & 24.06^{+0.01}_{-0.01} & -12.51^{+0.03}_{-0.03} & -12.66^{+0.02}_{-0.02} \\
    0.8 & 24.44^{+0.03}_{-0.02} & 24.30^{+0.01}_{-0.01} & 24.32^{+0.02}_{-0.02} & 24.18^{+0.01}_{-0.01} & -12.42^{+0.03}_{-0.03} & -12.57^{+0.02}_{-0.02} \\
    0.9 & 24.54^{+0.02}_{-0.02} & 24.41^{+0.01}_{-0.01} & 24.42^{+0.02}_{-0.01} & 24.29^{+0.01}_{-0.01} & -12.35^{+0.02}_{-0.03} & -12.48^{+0.02}_{-0.02} \\
    1.0 & 24.63^{+0.02}_{-0.02} & 24.51^{+0.01}_{-0.01} & 24.51^{+0.02}_{-0.01} & 24.38^{+0.01}_{-0.01} & -12.28^{+0.02}_{-0.03} & -12.40^{+0.02}_{-0.02} \\
    1.1 & 24.71^{+0.02}_{-0.02} & 24.60^{+0.01}_{-0.01} & 24.59^{+0.01}_{-0.01} & 24.47^{+0.01}_{-0.01} & -12.21^{+0.02}_{-0.03} & -12.33^{+0.02}_{-0.02} \\
    1.2 & 24.78^{+0.02}_{-0.02} & 24.68^{+0.01}_{-0.01} & 24.66^{+0.01}_{-0.01} & 24.55^{+0.01}_{-0.01} & -12.16^{+0.03}_{-0.03} & -12.27^{+0.02}_{-0.02} \\
    1.3 & 24.85^{+0.01}_{-0.02} & 24.75^{+0.01}_{-0.01} & 24.72^{+0.01}_{-0.01} & 24.63^{+0.01}_{-0.01} & -12.11^{+0.03}_{-0.03} & -12.22^{+0.02}_{-0.02} \\
    1.4 & 24.91^{+0.01}_{-0.02} & 24.82^{+0.01}_{-0.01} & 24.78^{+0.01}_{-0.01} & 24.69^{+0.01}_{-0.01} & -12.07^{+0.03}_{-0.03} & -12.18^{+0.02}_{-0.02} \\
    1.5 & 24.95^{+0.01}_{-0.01} & 24.87^{+0.01}_{-0.01} & 24.83^{+0.01}_{-0.01} & 24.74^{+0.01}_{-0.01} & -12.04^{+0.03}_{-0.03} & -12.15^{+0.02}_{-0.02} \\
    1.6 & 24.99^{+0.02}_{-0.01} & 24.91^{+0.01}_{-0.01} & 24.87^{+0.01}_{-0.01} & 24.79^{+0.01}_{-0.01} & -12.02^{+0.03}_{-0.03} & -12.12^{+0.02}_{-0.02} \\
    1.7 & 25.03^{+0.02}_{-0.02} & 24.95^{+0.01}_{-0.01} & 24.91^{+0.01}_{-0.01} & 24.83^{+0.01}_{-0.01} & -12.00^{+0.03}_{-0.03} & -12.11^{+0.02}_{-0.02} \\
    1.8 & 25.05^{+0.02}_{-0.02} & 24.98^{+0.01}_{-0.01} & 24.94^{+0.02}_{-0.01} & 24.86^{+0.01}_{-0.01} & -11.99^{+0.04}_{-0.04} & -12.10^{+0.02}_{-0.02} \\
    1.9 & 25.07^{+0.02}_{-0.02} & 24.99^{+0.01}_{-0.01} & 24.96^{+0.02}_{-0.02} & 24.88^{+0.01}_{-0.01} & -11.99^{+0.04}_{-0.04} & -12.09^{+0.03}_{-0.03} \\
    2.0 & 25.08^{+0.02}_{-0.02} & 25.01^{+0.01}_{-0.01} & 24.98^{+0.02}_{-0.02} & 24.90^{+0.01}_{-0.01} & -11.99^{+0.04}_{-0.05} & -12.09^{+0.03}_{-0.03} \\
    2.1 & 25.09^{+0.03}_{-0.02} & 25.01^{+0.02}_{-0.02} & 25.00^{+0.03}_{-0.02} & 24.91^{+0.02}_{-0.02} & -11.99^{+0.05}_{-0.05} & -12.10^{+0.03}_{-0.03} \\
    2.2 & 25.09^{+0.03}_{-0.03} & 25.01^{+0.02}_{-0.02} & 25.01^{+0.03}_{-0.03} & 24.91^{+0.02}_{-0.02} & -12.00^{+0.05}_{-0.06} & -12.11^{+0.04}_{-0.03} \\
    2.3 & 25.09^{+0.04}_{-0.04} & 25.01^{+0.02}_{-0.02} & 25.01^{+0.03}_{-0.03} & 24.91^{+0.02}_{-0.03} & -12.01^{+0.05}_{-0.07} & -12.12^{+0.04}_{-0.04} \\
    2.4 & 25.08^{+0.04}_{-0.04} & 25.00^{+0.03}_{-0.03} & 25.02^{+0.04}_{-0.04} & 24.91^{+0.03}_{-0.03} & -12.03^{+0.06}_{-0.08} & -12.14^{+0.04}_{-0.04} \\
    2.5 & 25.07^{+0.04}_{-0.05} & 24.99^{+0.03}_{-0.03} & 25.01^{+0.04}_{-0.04} & 24.89^{+0.03}_{-0.04} & -12.04^{+0.06}_{-0.08} & -12.16^{+0.05}_{-0.05} \\
    2.6 & 25.06^{+0.05}_{-0.06} & 24.97^{+0.04}_{-0.04} & 25.01^{+0.05}_{-0.05} & 24.88^{+0.04}_{-0.04} & -12.06^{+0.07}_{-0.09} & -12.18^{+0.05}_{-0.05} \\
    2.7 & 25.04^{+0.05}_{-0.07} & 24.96^{+0.04}_{-0.04} & 25.00^{+0.05}_{-0.05} & 24.86^{+0.04}_{-0.04} & -12.09^{+0.07}_{-0.10} & -12.21^{+0.05}_{-0.05} \\
    2.8 & 25.02^{+0.06}_{-0.07} & 24.93^{+0.05}_{-0.04} & 24.99^{+0.06}_{-0.06} & 24.84^{+0.05}_{-0.04} & -12.11^{+0.07}_{-0.11} & -12.24^{+0.06}_{-0.06} \\
    2.9 & 25.01^{+0.06}_{-0.08} & 24.91^{+0.05}_{-0.05} & 24.98^{+0.06}_{-0.07} & 24.81^{+0.05}_{-0.05} & -12.14^{+0.08}_{-0.12} & -12.27^{+0.06}_{-0.06} \\
    3.0 & 24.98^{+0.06}_{-0.09} & 24.88^{+0.05}_{-0.05} & 24.96^{+0.06}_{-0.07} & 24.78^{+0.05}_{-0.05} & -12.17^{+0.08}_{-0.12} & -12.31^{+0.06}_{-0.07} \\
    3.1 & 24.96^{+0.07}_{-0.10} & 24.85^{+0.05}_{-0.06} & 24.94^{+0.07}_{-0.08} & 24.75^{+0.06}_{-0.05} & -12.20^{+0.09}_{-0.13} & -12.35^{+0.07}_{-0.07} \\
    3.2 & 24.93^{+0.07}_{-0.10} & 24.82^{+0.06}_{-0.06} & 24.91^{+0.07}_{-0.09} & 24.71^{+0.06}_{-0.05} & -12.23^{+0.09}_{-0.14} & -12.39^{+0.07}_{-0.08} \\
    3.3 & 24.91^{+0.08}_{-0.11} & 24.79^{+0.06}_{-0.06} & 24.89^{+0.08}_{-0.09} & 24.68^{+0.06}_{-0.06} & -12.26^{+0.10}_{-0.15} & -12.43^{+0.07}_{-0.08} \\
    3.4 & 24.88^{+0.08}_{-0.12} & 24.75^{+0.06}_{-0.07} & 24.86^{+0.08}_{-0.09} & 24.64^{+0.07}_{-0.06} & -12.30^{+0.10}_{-0.17} & -12.47^{+0.07}_{-0.09} \\
    3.5 & 24.85^{+0.09}_{-0.14} & 24.72^{+0.07}_{-0.07} & 24.83^{+0.08}_{-0.10} & 24.59^{+0.07}_{-0.06} & -12.34^{+0.11}_{-0.18} & -12.52^{+0.08}_{-0.09} \\
    3.6 & 24.82^{+0.10}_{-0.15} & 24.68^{+0.07}_{-0.08} & 24.80^{+0.09}_{-0.11} & 24.55^{+0.07}_{-0.06} & -12.38^{+0.11}_{-0.19} & -12.56^{+0.08}_{-0.10} \\
    3.7 & 24.79^{+0.10}_{-0.16} & 24.64^{+0.07}_{-0.08} & 24.77^{+0.09}_{-0.11} & 24.51^{+0.07}_{-0.06} & -12.42^{+0.12}_{-0.20} & -12.61^{+0.08}_{-0.10} \\
    3.8 & 24.75^{+0.10}_{-0.16} & 24.59^{+0.07}_{-0.09} & 24.73^{+0.10}_{-0.12} & 24.47^{+0.07}_{-0.07} & -12.46^{+0.12}_{-0.21} & -12.66^{+0.08}_{-0.11} \\
    3.9 & 24.72^{+0.11}_{-0.17} & 24.55^{+0.07}_{-0.09} & 24.70^{+0.10}_{-0.13} & 24.42^{+0.07}_{-0.07} & -12.50^{+0.13}_{-0.22} & -12.72^{+0.08}_{-0.11} \\
    4.0 & 24.68^{+0.11}_{-0.18} & 24.51^{+0.07}_{-0.09} & 24.66^{+0.10}_{-0.14} & 24.37^{+0.07}_{-0.08} & -12.54^{+0.13}_{-0.24} & -12.77^{+0.08}_{-0.12} \\
    4.1 & 24.65^{+0.11}_{-0.20} & 24.46^{+0.07}_{-0.10} & 24.63^{+0.11}_{-0.16} & 24.33^{+0.06}_{-0.09} & -12.58^{+0.13}_{-0.25} & -12.82^{+0.08}_{-0.12} \\
    4.2 & 24.61^{+0.12}_{-0.21} & 24.42^{+0.07}_{-0.10} & 24.58^{+0.12}_{-0.16} & 24.28^{+0.06}_{-0.10} & -12.63^{+0.14}_{-0.27} & -12.88^{+0.08}_{-0.12} \\
    4.3 & 24.58^{+0.13}_{-0.23} & 24.37^{+0.07}_{-0.11} & 24.54^{+0.12}_{-0.17} & 24.23^{+0.07}_{-0.10} & -12.67^{+0.14}_{-0.29} & -12.94^{+0.09}_{-0.13} \\
    4.4 & 24.54^{+0.12}_{-0.24} & 24.32^{+0.08}_{-0.11} & 24.50^{+0.13}_{-0.18} & 24.17^{+0.07}_{-0.11} & -12.72^{+0.14}_{-0.30} & -13.00^{+0.09}_{-0.13} \\
    4.5 & 24.50^{+0.13}_{-0.26} & 24.27^{+0.08}_{-0.11} & 24.45^{+0.14}_{-0.19} & 24.12^{+0.07}_{-0.12} & -12.76^{+0.15}_{-0.32} & -13.06^{+0.09}_{-0.14} \\
    4.6 & 24.46^{+0.13}_{-0.28} & 24.22^{+0.08}_{-0.12} & 24.40^{+0.15}_{-0.19} & 24.07^{+0.08}_{-0.12} & -12.81^{+0.15}_{-0.34} & -13.12^{+0.10}_{-0.14} \\
    4.7 & 24.42^{+0.14}_{-0.30} & 24.16^{+0.09}_{-0.12} & 24.36^{+0.16}_{-0.21} & 24.01^{+0.08}_{-0.13} & -12.86^{+0.16}_{-0.36} & -13.18^{+0.11}_{-0.14} \\
    4.8 & 24.38^{+0.15}_{-0.31} & 24.11^{+0.09}_{-0.13} & 24.31^{+0.17}_{-0.22} & 23.96^{+0.09}_{-0.14} & -12.91^{+0.17}_{-0.38} & -13.25^{+0.11}_{-0.15} \\
    4.9 & 24.34^{+0.15}_{-0.33} & 24.05^{+0.10}_{-0.13} & 24.26^{+0.17}_{-0.24} & 23.90^{+0.09}_{-0.15} & -12.96^{+0.18}_{-0.39} & -13.32^{+0.12}_{-0.15} \\
    5.0 & 24.30^{+0.16}_{-0.35} & 23.99^{+0.11}_{-0.13} & 24.21^{+0.17}_{-0.25} & 23.84^{+0.10}_{-0.17} & -13.01^{+0.19}_{-0.41} & -13.38^{+0.13}_{-0.16} \\
    \hline
  \end{tabular}
\end{table*}

\begin{table*}
  \contcaption{}
  \begin{tabular}{d......}
    \hline
    \multicolumn{1}{c}{$z$} &    
    \multicolumn{1}{c}{$\log_{10}\epsilon_{1450}$} &
    \multicolumn{1}{c}{$\log_{10}\epsilon_{1450}$} &
    \multicolumn{1}{c}{$\log_{10}\epsilon_{912}$} &
    \multicolumn{1}{c}{$\log_{10}\epsilon_{912}$} & 
    \multicolumn{1}{c}{$\log_{10}\Gamma_\mathrm{HI}$} &
    \multicolumn{1}{c}{$\log_{10}\Gamma_\mathrm{HI}$} \\ 
    &
    \multicolumn{1}{c}{$(M_{1450}<-18)$} &
    \multicolumn{1}{c}{$(M_{1450}<-21)$} &
    \multicolumn{1}{c}{$(M_{1450}<-18)$} &
    \multicolumn{1}{c}{$(M_{1450}<-21)$} &
    \multicolumn{1}{c}{$(M_{1450}<-18)$} &
    \multicolumn{1}{c}{$(M_{1450}<-21)$} \\
    \hline
    5.1 & 24.26^{+0.17}_{-0.36} & 23.93^{+0.11}_{-0.14} & 24.16^{+0.18}_{-0.26} & 23.78^{+0.10}_{-0.19} & -13.06^{+0.20}_{-0.42} & -13.45^{+0.13}_{-0.16} \\
    5.2 & 24.21^{+0.19}_{-0.37} & 23.88^{+0.12}_{-0.14} & 24.10^{+0.19}_{-0.27} & 23.72^{+0.11}_{-0.20} & -13.12^{+0.21}_{-0.44} & -13.52^{+0.14}_{-0.17} \\
    5.3 & 24.17^{+0.20}_{-0.39} & 23.82^{+0.13}_{-0.15} & 24.05^{+0.19}_{-0.29} & 23.66^{+0.12}_{-0.22} & -13.17^{+0.22}_{-0.46} & -13.59^{+0.15}_{-0.18} \\
    5.4 & 24.13^{+0.21}_{-0.41} & 23.76^{+0.13}_{-0.16} & 24.00^{+0.20}_{-0.30} & 23.60^{+0.12}_{-0.23} & -13.22^{+0.23}_{-0.47} & -13.67^{+0.16}_{-0.19} \\
    5.5 & 24.09^{+0.22}_{-0.43} & 23.70^{+0.14}_{-0.17} & 23.94^{+0.21}_{-0.31} & 23.54^{+0.12}_{-0.24} & -13.29^{+0.24}_{-0.49} & -13.75^{+0.17}_{-0.20} \\
    5.6 & 24.04^{+0.23}_{-0.45} & 23.63^{+0.16}_{-0.17} & 23.89^{+0.22}_{-0.33} & 23.48^{+0.12}_{-0.25} & -13.37^{+0.25}_{-0.51} & -13.85^{+0.18}_{-0.21} \\
    5.7 & 24.00^{+0.24}_{-0.46} & 23.57^{+0.16}_{-0.18} & 23.84^{+0.23}_{-0.34} & 23.42^{+0.13}_{-0.26} & -13.45^{+0.26}_{-0.53} & -13.94^{+0.18}_{-0.22} \\
    5.8 & 23.95^{+0.25}_{-0.48} & 23.51^{+0.16}_{-0.20} & 23.78^{+0.24}_{-0.35} & 23.35^{+0.14}_{-0.28} & -13.53^{+0.27}_{-0.55} & -14.04^{+0.19}_{-0.23} \\
    5.9 & 23.91^{+0.26}_{-0.50} & 23.45^{+0.17}_{-0.20} & 23.73^{+0.24}_{-0.38} & 23.29^{+0.15}_{-0.30} & -13.62^{+0.28}_{-0.57} & -14.15^{+0.20}_{-0.24} \\
    6.0 & 23.87^{+0.27}_{-0.52} & 23.38^{+0.18}_{-0.20} & 23.67^{+0.25}_{-0.40} & 23.23^{+0.15}_{-0.32} & -13.70^{+0.29}_{-0.58} & -14.26^{+0.21}_{-0.24} \\
    6.1 & 23.82^{+0.29}_{-0.54} & 23.32^{+0.19}_{-0.21} & 23.61^{+0.27}_{-0.42} & 23.16^{+0.16}_{-0.34} & -13.80^{+0.31}_{-0.60} & -14.37^{+0.22}_{-0.25} \\
    6.2 & 23.76^{+0.30}_{-0.56} & 23.25^{+0.20}_{-0.22} & 23.55^{+0.28}_{-0.42} & 23.10^{+0.16}_{-0.36} & -13.90^{+0.32}_{-0.62} & -14.48^{+0.23}_{-0.26} \\
    6.3 & 23.71^{+0.32}_{-0.57} & 23.19^{+0.21}_{-0.23} & 23.49^{+0.29}_{-0.43} & 23.03^{+0.17}_{-0.38} & -14.00^{+0.34}_{-0.64} & -14.59^{+0.24}_{-0.28} \\
    6.4 & 23.66^{+0.33}_{-0.59} & 23.13^{+0.22}_{-0.24} & 23.42^{+0.30}_{-0.44} & 22.96^{+0.18}_{-0.39} & -14.10^{+0.35}_{-0.66} & -14.71^{+0.25}_{-0.29} \\
    6.5 & 23.62^{+0.34}_{-0.61} & 23.06^{+0.23}_{-0.25} & 23.36^{+0.31}_{-0.46} & 22.90^{+0.19}_{-0.40} & -14.20^{+0.36}_{-0.68} & -14.83^{+0.26}_{-0.31} \\
    6.6 & 23.56^{+0.36}_{-0.62} & 23.00^{+0.24}_{-0.26} & 23.30^{+0.33}_{-0.48} & 22.83^{+0.19}_{-0.42} & -14.31^{+0.38}_{-0.69} & -14.96^{+0.27}_{-0.32} \\
    6.7 & 23.52^{+0.37}_{-0.64} & 22.93^{+0.25}_{-0.28} & 23.23^{+0.34}_{-0.50} & 22.76^{+0.20}_{-0.44} & -14.42^{+0.39}_{-0.72} & -15.08^{+0.28}_{-0.34} \\
    6.8 & 23.47^{+0.38}_{-0.67} & 22.87^{+0.26}_{-0.29} & 23.15^{+0.36}_{-0.51} & 22.70^{+0.20}_{-0.45} & -14.52^{+0.40}_{-0.74} & -15.22^{+0.29}_{-0.36} \\
    6.9 & 23.42^{+0.40}_{-0.68} & 22.80^{+0.27}_{-0.30} & 23.08^{+0.39}_{-0.53} & 22.63^{+0.21}_{-0.47} & -14.64^{+0.41}_{-0.77} & -15.35^{+0.30}_{-0.37} \\
    7.0 & 23.37^{+0.41}_{-0.71} & 22.73^{+0.28}_{-0.31} & 23.01^{+0.41}_{-0.54} & 22.56^{+0.22}_{-0.49} & -14.74^{+0.42}_{-0.79} & -15.48^{+0.32}_{-0.40} \\
    7.1 & 23.33^{+0.41}_{-0.73} & 22.66^{+0.29}_{-0.33} & 22.94^{+0.42}_{-0.56} & 22.49^{+0.23}_{-0.52} & -14.86^{+0.43}_{-0.82} & -15.62^{+0.33}_{-0.42} \\
    7.2 & 23.28^{+0.43}_{-0.76} & 22.60^{+0.29}_{-0.35} & 22.87^{+0.44}_{-0.57} & 22.42^{+0.23}_{-0.54} & -14.97^{+0.45}_{-0.85} & -15.75^{+0.34}_{-0.45} \\
    7.3 & 23.22^{+0.45}_{-0.77} & 22.53^{+0.31}_{-0.36} & 22.80^{+0.47}_{-0.59} & 22.35^{+0.24}_{-0.57} & -15.09^{+0.47}_{-0.87} & -15.88^{+0.36}_{-0.47} \\
    7.4 & 23.17^{+0.46}_{-0.80} & 22.46^{+0.32}_{-0.38} & 22.73^{+0.49}_{-0.60} & 22.29^{+0.25}_{-0.59} & -15.20^{+0.48}_{-0.90} & -16.02^{+0.37}_{-0.50} \\
    7.5 & 23.13^{+0.47}_{-0.83} & 22.39^{+0.34}_{-0.39} & 22.67^{+0.51}_{-0.61} & 22.21^{+0.25}_{-0.62} & -15.30^{+0.49}_{-0.93} & -16.15^{+0.39}_{-0.52} \\
    7.6 & 23.09^{+0.47}_{-0.87} & 22.32^{+0.35}_{-0.41} & 22.60^{+0.52}_{-0.63} & 22.14^{+0.26}_{-0.64} & -15.41^{+0.49}_{-0.97} & -16.29^{+0.41}_{-0.55} \\
    7.7 & 23.04^{+0.48}_{-0.90} & 22.26^{+0.37}_{-0.42} & 22.53^{+0.54}_{-0.64} & 22.07^{+0.27}_{-0.66} & -15.52^{+0.50}_{-1.00} & -16.42^{+0.43}_{-0.57} \\
    7.8 & 22.99^{+0.50}_{-0.92} & 22.19^{+0.38}_{-0.44} & 22.46^{+0.57}_{-0.65} & 22.00^{+0.28}_{-0.68} & -15.63^{+0.52}_{-1.02} & -16.55^{+0.45}_{-0.60} \\
    7.9 & 22.94^{+0.51}_{-0.94} & 22.12^{+0.41}_{-0.46} & 22.38^{+0.58}_{-0.65} & 21.92^{+0.29}_{-0.70} & -15.74^{+0.53}_{-1.05} & -16.68^{+0.47}_{-0.63} \\
    8.0 & 22.88^{+0.53}_{-0.97} & 22.05^{+0.43}_{-0.48} & 22.31^{+0.60}_{-0.66} & 21.85^{+0.30}_{-0.73} & -15.84^{+0.54}_{-1.07} & -16.80^{+0.50}_{-0.66} \\
    8.1 & 22.83^{+0.54}_{-0.99} & 21.97^{+0.44}_{-0.49} & 22.24^{+0.61}_{-0.68} & 21.78^{+0.31}_{-0.76} & -15.95^{+0.56}_{-1.10} & -16.93^{+0.52}_{-0.68} \\
    8.2 & 22.78^{+0.56}_{-1.02} & 21.90^{+0.46}_{-0.51} & 22.17^{+0.63}_{-0.70} & 21.71^{+0.32}_{-0.78} & -16.05^{+0.57}_{-1.12} & -17.06^{+0.54}_{-0.71} \\
    8.3 & 22.73^{+0.57}_{-1.04} & 21.82^{+0.48}_{-0.52} & 22.10^{+0.65}_{-0.71} & 21.63^{+0.33}_{-0.81} & -16.15^{+0.58}_{-1.15} & -17.20^{+0.56}_{-0.72} \\
    8.4 & 22.68^{+0.58}_{-1.07} & 21.74^{+0.50}_{-0.53} & 22.02^{+0.68}_{-0.72} & 21.56^{+0.33}_{-0.84} & -16.25^{+0.59}_{-1.18} & -17.33^{+0.58}_{-0.74} \\
    8.5 & 22.63^{+0.59}_{-1.10} & 21.66^{+0.52}_{-0.54} & 21.94^{+0.71}_{-0.73} & 21.49^{+0.34}_{-0.86} & -16.35^{+0.61}_{-1.21} & -17.45^{+0.60}_{-0.76} \\
    8.6 & 22.59^{+0.60}_{-1.13} & 21.59^{+0.53}_{-0.55} & 21.86^{+0.75}_{-0.73} & 21.41^{+0.35}_{-0.89} & -16.45^{+0.61}_{-1.24} & -17.57^{+0.61}_{-0.78} \\
    8.7 & 22.54^{+0.61}_{-1.16} & 21.52^{+0.54}_{-0.57} & 21.78^{+0.78}_{-0.73} & 21.34^{+0.36}_{-0.92} & -16.55^{+0.63}_{-1.26} & -17.69^{+0.63}_{-0.80} \\
    8.8 & 22.48^{+0.63}_{-1.18} & 21.45^{+0.56}_{-0.58} & 21.69^{+0.81}_{-0.73} & 21.26^{+0.37}_{-0.94} & -16.65^{+0.64}_{-1.28} & -17.82^{+0.64}_{-0.82} \\
    8.9 & 22.42^{+0.65}_{-1.20} & 21.37^{+0.58}_{-0.59} & 21.61^{+0.84}_{-0.74} & 21.19^{+0.38}_{-0.97} & -16.75^{+0.66}_{-1.31} & -17.94^{+0.67}_{-0.84} \\
    9.0 & 22.37^{+0.66}_{-1.23} & 21.30^{+0.60}_{-0.61} & 21.53^{+0.87}_{-0.75} & 21.11^{+0.39}_{-0.99} & -16.85^{+0.67}_{-1.33} & -18.06^{+0.69}_{-0.86} \\
    9.1 & 22.33^{+0.67}_{-1.26} & 21.22^{+0.62}_{-0.62} & 21.45^{+0.90}_{-0.76} & 21.04^{+0.40}_{-1.02} & -16.94^{+0.68}_{-1.36} & -18.18^{+0.72}_{-0.89} \\
    9.2 & 22.27^{+0.68}_{-1.28} & 21.15^{+0.65}_{-0.64} & 21.36^{+0.93}_{-0.77} & 20.96^{+0.41}_{-1.05} & -17.04^{+0.69}_{-1.38} & -18.30^{+0.74}_{-0.91} \\
    9.3 & 22.22^{+0.69}_{-1.31} & 21.07^{+0.67}_{-0.66} & 21.29^{+0.95}_{-0.80} & 20.89^{+0.43}_{-1.08} & -17.14^{+0.71}_{-1.41} & -18.41^{+0.77}_{-0.93} \\
    9.4 & 22.16^{+0.70}_{-1.33} & 21.00^{+0.69}_{-0.67} & 21.21^{+0.97}_{-0.82} & 20.81^{+0.44}_{-1.10} & -17.23^{+0.72}_{-1.43} & -18.53^{+0.79}_{-0.95} \\
    9.5 & 22.11^{+0.72}_{-1.36} & 20.92^{+0.71}_{-0.68} & 21.13^{+1.00}_{-0.83} & 20.73^{+0.45}_{-1.13} & -17.33^{+0.74}_{-1.45} & -18.65^{+0.81}_{-0.96} \\
    9.6 & 22.06^{+0.73}_{-1.38} & 20.84^{+0.73}_{-0.70} & 21.05^{+1.03}_{-0.85} & 20.65^{+0.46}_{-1.16} & -17.42^{+0.75}_{-1.47} & -18.77^{+0.83}_{-0.98} \\
    9.7 & 22.00^{+0.74}_{-1.40} & 20.76^{+0.75}_{-0.72} & 20.97^{+1.06}_{-0.86} & 20.57^{+0.48}_{-1.18} & -17.52^{+0.76}_{-1.49} & -18.89^{+0.85}_{-1.01} \\
    9.8 & 21.95^{+0.76}_{-1.42} & 20.69^{+0.77}_{-0.73} & 20.88^{+1.10}_{-0.88} & 20.49^{+0.49}_{-1.21} & -17.61^{+0.78}_{-1.51} & -19.01^{+0.87}_{-1.03} \\
    9.9 & 21.90^{+0.76}_{-1.45} & 20.61^{+0.79}_{-0.75} & 20.80^{+1.13}_{-0.89} & 20.42^{+0.50}_{-1.24} & -17.70^{+0.79}_{-1.54} & -19.12^{+0.89}_{-1.06} \\
    10.0 & 21.85^{+0.77}_{-1.48} & 20.53^{+0.81}_{-0.77} & 20.72^{+1.16}_{-0.90} & 20.34^{+0.51}_{-1.26} & -17.79^{+0.80}_{-1.57} & -19.23^{+0.91}_{-1.09} \\
    \hline
  \end{tabular}
\end{table*}

\begin{table*}
  \contcaption{}
  \begin{tabular}{d......}
    \hline
    \multicolumn{1}{c}{$z$} &    
    \multicolumn{1}{c}{$\log_{10}\epsilon_{1450}$} &
    \multicolumn{1}{c}{$\log_{10}\epsilon_{1450}$} &
    \multicolumn{1}{c}{$\log_{10}\epsilon_{912}$} &
    \multicolumn{1}{c}{$\log_{10}\epsilon_{912}$} & 
    \multicolumn{1}{c}{$\log_{10}\Gamma_\mathrm{HI}$} &
    \multicolumn{1}{c}{$\log_{10}\Gamma_\mathrm{HI}$} \\ 
    &
    \multicolumn{1}{c}{$(M_{1450}<-18)$} &
    \multicolumn{1}{c}{$(M_{1450}<-21)$} &
    \multicolumn{1}{c}{$(M_{1450}<-18)$} &
    \multicolumn{1}{c}{$(M_{1450}<-21)$} &
    \multicolumn{1}{c}{$(M_{1450}<-18)$} &
    \multicolumn{1}{c}{$(M_{1450}<-21)$} \\
    \hline
    10.0 & 21.85^{+0.77}_{-1.48} & 20.53^{+0.81}_{-0.77} & 20.72^{+1.16}_{-0.90} & 20.34^{+0.51}_{-1.26} & -17.79^{+0.80}_{-1.57} & -19.23^{+0.91}_{-1.09} \\
    10.1 & 21.80^{+0.80}_{-1.50} & 20.46^{+0.83}_{-0.79} & 20.63^{+1.19}_{-0.91} & 20.26^{+0.52}_{-1.29} & -17.88^{+0.83}_{-1.58} & -19.35^{+0.93}_{-1.11} \\
    10.2 & 21.74^{+0.83}_{-1.52} & 20.38^{+0.85}_{-0.81} & 20.55^{+1.23}_{-0.93} & 20.18^{+0.53}_{-1.31} & -17.97^{+0.85}_{-1.60} & -19.46^{+0.96}_{-1.13} \\
    10.3 & 21.69^{+0.85}_{-1.54} & 20.31^{+0.88}_{-0.83} & 20.47^{+1.26}_{-0.94} & 20.10^{+0.54}_{-1.34} & -18.07^{+0.87}_{-1.62} & -19.57^{+0.98}_{-1.16} \\
    10.4 & 21.63^{+0.87}_{-1.56} & 20.23^{+0.90}_{-0.85} & 20.39^{+1.29}_{-0.95} & 20.03^{+0.55}_{-1.36} & -18.16^{+0.89}_{-1.64} & -19.69^{+1.01}_{-1.18} \\
    10.5 & 21.58^{+0.89}_{-1.59} & 20.15^{+0.93}_{-0.87} & 20.31^{+1.31}_{-0.96} & 19.95^{+0.56}_{-1.39} & -18.25^{+0.90}_{-1.67} & -19.80^{+1.03}_{-1.21} \\
    10.6 & 21.52^{+0.90}_{-1.62} & 20.07^{+0.95}_{-0.89} & 20.22^{+1.35}_{-0.97} & 19.87^{+0.58}_{-1.42} & -18.34^{+0.91}_{-1.69} & -19.91^{+1.05}_{-1.23} \\
    10.7 & 21.47^{+0.90}_{-1.65} & 19.99^{+0.97}_{-0.90} & 20.13^{+1.38}_{-0.98} & 19.79^{+0.59}_{-1.44} & -18.43^{+0.92}_{-1.72} & -20.03^{+1.08}_{-1.25} \\
    10.8 & 21.42^{+0.91}_{-1.67} & 19.91^{+0.99}_{-0.92} & 20.04^{+1.42}_{-0.99} & 19.71^{+0.60}_{-1.47} & -18.52^{+0.93}_{-1.74} & -20.15^{+1.10}_{-1.27} \\
    10.9 & 21.36^{+0.93}_{-1.70} & 19.83^{+1.01}_{-0.94} & 19.95^{+1.46}_{-1.00} & 19.63^{+0.61}_{-1.50} & -18.62^{+0.95}_{-1.76} & -20.26^{+1.13}_{-1.29} \\
    11.0 & 21.30^{+0.94}_{-1.72} & 19.75^{+1.04}_{-0.96} & 19.86^{+1.50}_{-1.01} & 19.55^{+0.62}_{-1.53} & -18.71^{+0.96}_{-1.78} & -20.38^{+1.16}_{-1.32} \\
    11.1 & 21.24^{+0.96}_{-1.74} & 19.67^{+1.08}_{-0.98} & 19.78^{+1.53}_{-1.03} & 19.47^{+0.63}_{-1.56} & -18.81^{+0.98}_{-1.81} & -20.49^{+1.20}_{-1.34} \\
    11.2 & 21.18^{+0.97}_{-1.76} & 19.59^{+1.12}_{-1.00} & 19.69^{+1.56}_{-1.04} & 19.39^{+0.64}_{-1.59} & -18.90^{+0.99}_{-1.82} & -20.60^{+1.24}_{-1.36} \\
    11.3 & 21.12^{+0.99}_{-1.78} & 19.51^{+1.15}_{-1.01} & 19.60^{+1.60}_{-1.05} & 19.31^{+0.66}_{-1.62} & -18.99^{+1.01}_{-1.84} & -20.72^{+1.28}_{-1.39} \\
    11.4 & 21.06^{+1.00}_{-1.79} & 19.43^{+1.20}_{-1.03} & 19.51^{+1.63}_{-1.06} & 19.23^{+0.67}_{-1.65} & -19.09^{+1.02}_{-1.85} & -20.83^{+1.32}_{-1.40} \\
    11.5 & 21.00^{+1.02}_{-1.82} & 19.35^{+1.23}_{-1.05} & 19.42^{+1.67}_{-1.07} & 19.15^{+0.68}_{-1.68} & -19.18^{+1.04}_{-1.88} & -20.95^{+1.35}_{-1.42} \\
    11.6 & 20.94^{+1.03}_{-1.84} & 19.27^{+1.27}_{-1.06} & 19.32^{+1.71}_{-1.08} & 19.07^{+0.69}_{-1.71} & -19.27^{+1.05}_{-1.90} & -21.07^{+1.39}_{-1.44} \\
    11.7 & 20.88^{+1.05}_{-1.86} & 19.18^{+1.31}_{-1.08} & 19.23^{+1.75}_{-1.09} & 18.99^{+0.70}_{-1.74} & -19.37^{+1.07}_{-1.92} & -21.18^{+1.43}_{-1.45} \\
    11.8 & 20.82^{+1.06}_{-1.88} & 19.10^{+1.35}_{-1.09} & 19.14^{+1.79}_{-1.10} & 18.91^{+0.71}_{-1.77} & -19.46^{+1.09}_{-1.94} & -21.30^{+1.46}_{-1.47} \\
    11.9 & 20.76^{+1.08}_{-1.90} & 19.01^{+1.38}_{-1.10} & 19.04^{+1.83}_{-1.11} & 18.83^{+0.72}_{-1.80} & -19.56^{+1.10}_{-1.96} & -21.41^{+1.49}_{-1.48} \\
    12.0 & 20.70^{+1.10}_{-1.93} & 18.93^{+1.41}_{-1.11} & 18.95^{+1.86}_{-1.13} & 18.75^{+0.73}_{-1.82} & -19.65^{+1.12}_{-1.98} & -21.52^{+1.52}_{-1.50} \\
    12.1 & 20.64^{+1.12}_{-1.95} & 18.86^{+1.43}_{-1.14} & 18.87^{+1.90}_{-1.14} & 18.67^{+0.74}_{-1.85} & -19.74^{+1.14}_{-2.00} & -21.63^{+1.55}_{-1.52} \\
    12.2 & 20.58^{+1.14}_{-1.97} & 18.78^{+1.46}_{-1.16} & 18.78^{+1.93}_{-1.16} & 18.59^{+0.76}_{-1.88} & -19.83^{+1.16}_{-2.03} & -21.74^{+1.57}_{-1.54} \\
    12.3 & 20.52^{+1.16}_{-1.99} & 18.70^{+1.50}_{-1.18} & 18.69^{+1.96}_{-1.18} & 18.51^{+0.77}_{-1.91} & -19.92^{+1.18}_{-2.05} & -21.85^{+1.61}_{-1.56} \\
    12.4 & 20.46^{+1.16}_{-2.02} & 18.62^{+1.53}_{-1.20} & 18.60^{+1.99}_{-1.20} & 18.42^{+0.78}_{-1.93} & -20.01^{+1.19}_{-2.08} & -21.96^{+1.64}_{-1.58} \\
    12.5 & 20.41^{+1.17}_{-2.05} & 18.54^{+1.56}_{-1.22} & 18.52^{+2.02}_{-1.22} & 18.34^{+0.80}_{-1.96} & -20.10^{+1.20}_{-2.10} & -22.07^{+1.67}_{-1.60} \\
    12.6 & 20.35^{+1.19}_{-2.07} & 18.46^{+1.59}_{-1.24} & 18.43^{+2.05}_{-1.24} & 18.25^{+0.81}_{-1.98} & -20.19^{+1.22}_{-2.12} & -22.18^{+1.70}_{-1.63} \\
    12.7 & 20.29^{+1.21}_{-2.09} & 18.38^{+1.62}_{-1.26} & 18.35^{+2.07}_{-1.27} & 18.17^{+0.82}_{-2.01} & -20.28^{+1.24}_{-2.14} & -22.29^{+1.72}_{-1.65} \\
    12.8 & 20.23^{+1.23}_{-2.12} & 18.29^{+1.65}_{-1.28} & 18.27^{+2.10}_{-1.29} & 18.09^{+0.83}_{-2.04} & -20.37^{+1.26}_{-2.16} & -22.41^{+1.76}_{-1.66} \\
    12.9 & 20.17^{+1.25}_{-2.14} & 18.21^{+1.68}_{-1.29} & 18.19^{+2.12}_{-1.32} & 18.01^{+0.84}_{-2.07} & -20.46^{+1.28}_{-2.19} & -22.52^{+1.79}_{-1.68} \\
    13.0 & 20.11^{+1.28}_{-2.16} & 18.12^{+1.71}_{-1.31} & 18.11^{+2.15}_{-1.35} & 17.92^{+0.85}_{-2.09} & -20.55^{+1.30}_{-2.21} & -22.64^{+1.82}_{-1.69} \\
    13.1 & 20.05^{+1.30}_{-2.18} & 18.04^{+1.75}_{-1.32} & 18.02^{+2.18}_{-1.37} & 17.84^{+0.86}_{-2.12} & -20.64^{+1.31}_{-2.23} & -22.75^{+1.86}_{-1.71} \\
    13.2 & 19.99^{+1.31}_{-2.21} & 17.95^{+1.79}_{-1.34} & 17.93^{+2.21}_{-1.39} & 17.76^{+0.88}_{-2.15} & -20.73^{+1.33}_{-2.26} & -22.86^{+1.89}_{-1.72} \\
    13.3 & 19.93^{+1.32}_{-2.24} & 17.87^{+1.83}_{-1.36} & 17.84^{+2.24}_{-1.41} & 17.67^{+0.89}_{-2.17} & -20.81^{+1.34}_{-2.29} & -22.98^{+1.93}_{-1.74} \\
    13.4 & 19.88^{+1.33}_{-2.27} & 17.79^{+1.86}_{-1.37} & 17.76^{+2.27}_{-1.43} & 17.59^{+0.91}_{-2.20} & -20.90^{+1.34}_{-2.32} & -23.09^{+1.96}_{-1.75} \\
    13.5 & 19.83^{+1.34}_{-2.30} & 17.70^{+1.90}_{-1.39} & 17.66^{+2.30}_{-1.45} & 17.51^{+0.92}_{-2.23} & -20.98^{+1.35}_{-2.35} & -23.21^{+2.00}_{-1.77} \\
    13.6 & 19.78^{+1.34}_{-2.33} & 17.62^{+1.93}_{-1.41} & 17.57^{+2.33}_{-1.46} & 17.42^{+0.93}_{-2.26} & -21.06^{+1.36}_{-2.38} & -23.32^{+2.03}_{-1.78} \\
    13.7 & 19.72^{+1.35}_{-2.36} & 17.53^{+1.97}_{-1.42} & 17.48^{+2.36}_{-1.48} & 17.34^{+0.95}_{-2.29} & -21.15^{+1.37}_{-2.40} & -23.43^{+2.06}_{-1.79} \\
    13.8 & 19.67^{+1.37}_{-2.38} & 17.45^{+2.00}_{-1.44} & 17.39^{+2.39}_{-1.50} & 17.25^{+0.96}_{-2.32} & -21.24^{+1.38}_{-2.43} & -23.54^{+2.09}_{-1.81} \\
    13.9 & 19.61^{+1.38}_{-2.41} & 17.37^{+2.03}_{-1.46} & 17.30^{+2.42}_{-1.52} & 17.17^{+0.97}_{-2.35} & -21.32^{+1.39}_{-2.46} & -23.65^{+2.12}_{-1.83} \\
    14.0 & 19.55^{+1.39}_{-2.44} & 17.29^{+2.06}_{-1.48} & 17.21^{+2.45}_{-1.54} & 17.09^{+0.99}_{-2.38} & -21.41^{+1.40}_{-2.48} & -23.76^{+2.15}_{-1.84} \\
    14.1 & 19.50^{+1.40}_{-2.46} & 17.20^{+2.10}_{-1.50} & 17.12^{+2.48}_{-1.55} & 17.00^{+1.00}_{-2.41} & -21.49^{+1.41}_{-2.51} & -23.87^{+2.18}_{-1.86} \\
    14.2 & 19.44^{+1.41}_{-2.50} & 17.12^{+2.13}_{-1.52} & 17.03^{+2.51}_{-1.56} & 16.92^{+1.01}_{-2.44} & -21.58^{+1.42}_{-2.54} & -23.98^{+2.21}_{-1.87} \\
    14.3 & 19.39^{+1.42}_{-2.53} & 17.04^{+2.16}_{-1.54} & 16.94^{+2.54}_{-1.57} & 16.83^{+1.02}_{-2.47} & -21.66^{+1.43}_{-2.57} & -24.09^{+2.24}_{-1.89} \\
    14.4 & 19.34^{+1.43}_{-2.56} & 16.96^{+2.19}_{-1.56} & 16.85^{+2.57}_{-1.58} & 16.75^{+1.04}_{-2.50} & -21.75^{+1.44}_{-2.60} & -24.20^{+2.26}_{-1.91} \\
    14.5 & 19.29^{+1.44}_{-2.60} & 16.88^{+2.22}_{-1.58} & 16.76^{+2.60}_{-1.59} & 16.66^{+1.05}_{-2.53} & -21.84^{+1.45}_{-2.63} & -24.32^{+2.29}_{-1.93} \\
    14.6 & 19.24^{+1.46}_{-2.63} & 16.80^{+2.25}_{-1.60} & 16.67^{+2.63}_{-1.60} & 16.58^{+1.06}_{-2.56} & -21.93^{+1.47}_{-2.66} & -24.43^{+2.32}_{-1.94} \\
    14.7 & 19.18^{+1.47}_{-2.65} & 16.72^{+2.28}_{-1.62} & 16.57^{+2.66}_{-1.61} & 16.50^{+1.07}_{-2.59} & -22.03^{+1.48}_{-2.68} & -24.56^{+2.34}_{-1.96} \\
    14.8 & 19.13^{+1.48}_{-2.69} & 16.64^{+2.31}_{-1.64} & 16.48^{+2.69}_{-1.63} & 16.41^{+1.09}_{-2.61} & -22.15^{+1.48}_{-2.71} & -24.69^{+2.37}_{-1.97} \\
    14.9 & 19.08^{+1.48}_{-2.72} & 16.56^{+2.34}_{-1.66} & 16.39^{+2.72}_{-1.64} & 16.33^{+1.10}_{-2.64} & -22.28^{+1.48}_{-2.74} & -24.85^{+2.39}_{-1.99} \\
    15.0 & 19.02^{+1.49}_{-2.75} & 16.48^{+2.37}_{-1.68} & 16.30^{+2.75}_{-1.66} & 16.24^{+1.11}_{-2.67} & -22.50^{+1.49}_{-2.76} & -25.08^{+2.41}_{-2.00} \\
    \hline
  \end{tabular}
\end{table*}

\section{Code and data}

We make the code and data used in this work publicly available at
\url{https://github.com/gkulkarni/QLF}.  This includes homogenised AGN
catalogues and selection functions and the code used for developing
and analysing luminosity function models.
%% Make URL public. Set appropriate license.  Take permission from
%% other authors.  Set up a DOI.

\bibliographystyle{mnras}
\bibliography{refs}

\bsp
\label{lastpage}
\end{document}


