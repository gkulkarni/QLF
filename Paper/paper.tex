\documentclass[fleqn,usenatbib]{mnras}
\usepackage[T1]{fontenc}
\usepackage{ae,aecompl}
\usepackage{graphicx}	
\usepackage{amsmath}	
\usepackage{amssymb}
\usepackage{pdflscape}
\usepackage{siunitx}
\usepackage{color}
\definecolor{notecolor}{rgb}{0.8,0,0}
\newcommand{\gk}[1]{{\bf \color{notecolor} [#1]}}
\def\lya{Ly$\alpha$~}
\def\HI{\hbox{H~$\scriptstyle\rm I$}}
\def\HII{\hbox{H~$\scriptstyle\rm II$}}
\def\nHI{{\rm HI}}
\def\nH{{\rm H}}
\def\nHII{{\rm HII}}
\def\nHe{{\rm He}}
\def\nHeI{{\rm HeI}}
\def\nHeII{{\rm HeII}}
\def\nHeIII{{\rm HeIII}}
\def\bHII{\hbox{\bf H~$\scriptstyle\bf II$}}
\def\HeI{\hbox{He~$\scriptstyle\rm I$}}
\def\HeII{\hbox{He~$\scriptstyle\rm II$}}
\def\HeIII{\hbox{He~$\scriptstyle\rm III$}}
\def\bHeII{\hbox{\bf He~$\scriptstyle\bf II$}}
\def\HeIII{\hbox{He~$\scriptstyle\rm III$}}
\usepackage{dcolumn}
\newcolumntype{.}{D{.}{.}{2.7}}
\newcolumntype{e}{D{.}{.}{2.10}}
\newcolumntype{d}{D{.}{.}{2.3}}

\title[AGN luminosity function]{Evolution of the AGN UV
  luminosity function from redshift 7.5}

\author[Kulkarni et al.]
       {{Girish Kulkarni$^{1,2}$\thanks{Email: kulkarni@ast.cam.ac.uk},
           G\'abor Worseck$^{3}$
           and Joseph F.~Hennawi$^{4}$} \\
         $^1$Institute of Astronomy,
         University of Cambridge, Madingley Road, Cambridge CB3 0HA,
         UK \\
         $^2$Kavli Institute of Cosmology,
         University of Cambridge, Madingley Road, Cambridge CB3 0HA,
         UK \\
         $^2$Institute f\"ur Physik und Astronomie, Universit\"at
         Potsdam, Karl-Liebknecht-Stra\ss e\ 24/25, 14476 Potsdam,
         Germany \\
         $^3$Department of Physics, Broida Hall, UC Santa Barbara,
         Santa Barbara, CA 93106-9530, USA}
       \date{Accepted ---. Received ---; in original form ---}

\pubyear{2018}

\begin{document}
\label{firstpage}
\pagerange{\pageref{firstpage}--\pageref{lastpage}}
\maketitle

\begin{abstract}
  Determinations of the UV luminosity function of AGN at high
  redshifts are important for constraining the AGN contribution to
  reionization and understanding the growth of supermassive black
  holes.  Recent inferences of the luminosity function suffer from
  inconsistencies arising from inhomogeneous selection and mixing of
  AGN data.  We address this problem by constructing a sample of more
  than 80,000 colour-selected AGN from redshift $z=0$ to $7.5$.  While
  this sample is composed of multiple data sets with spectroscopic
  redshifts and completeness estimates, we homogenise these data sets
  to identical cosmologies, intrinsic AGN spectra, and magnitude
  systems.  Using this sample, we derive the AGN UV luminosity
  function from redshifts $z=0$ to $7.5$.  The luminosity function has
  a double power law form at $z<5.5$, and a single power law form at
  higher redshifts.  The break luminosity brightens rapidly with
  redshift and the faint end of the luminosity function steepens.  In
  spite of this steepening, the contribution of AGN to the hydrogen
  photoionization rate at $z\sim 6$ is subdominant, although it can be
  non-negligible if these luminosity functions hold down to
  $M_{1450}=-18$.  Under reasonable assumptions, AGN can reionize
  \HeII\ by redshift $z=3.2$.  At low redshifts ($z<0.5$), AGN can
  produce all of the hydrogen photoionization rate inferred from the
  statistics of \HI\ absorption lines in the IGM.  Our global analysis
  of the luminosity function also reveals important systematic errors
  in the data, particularly at $z=2.2$--$3.5$, which need to be
  addressed in future in order to improve our results.  We make
  various fitting functions, luminosity function analysis codes, and
  homogenised AGN data publicly available.
\end{abstract}

\begin{keywords}
  dark ages, reionization, first stars -- intergalactic medium --
  quasars: general -- galaxies: active
\end{keywords}

\section{Introduction}

The luminosity function of active galactic nuclei (AGN) and its
evolution over cosmological time scales has been a matter of central
interest of a large body of work over the last four decades
\citep{1968ApJ...151..393S, 1978A&A....68...17M, 1983ApJ...269..352S,
  1988ApJ...325...92K, 1988MNRAS.235..935B, 1993ApJ...406L..43H,
  1994ApJ...421..412W, 1995AJ....110...68S, 1995AJ....110.2553K,
  1995ApJ...438..623P, 2000MNRAS.317.1014B, 2001AJ....121...54F,
  2004AJ....128..515F, 2006AJ....131.2766R, 2007ApJ...654..731H,
  2009MNRAS.392...19C, 2010AJ....139..906W, 2011ApJ...728L..26G,
  2013ApJ...773...14R, 2013ApJ...768..105M, 2015AA...578A..83G,
  2015ApJ...798...28K, 2016ApJ...829...33Y, 2016ApJ...833..222J}.
Determination of the AGN luminosity function constrains models of the
build-up of supermassive black holes \citep{1982MNRAS.200..115S,
  2004ApJ...602..603Y, 2007ApJ...654..731H, 2008ApJ...679..118S,
  2009A&A...493...55E, 2000ApJ...531...42H, 2010MNRAS.401.2531A,
  2013ApJ...773...14R, 2014ApJ...787...73D, 2015MNRAS.448.3603D,
  2015MNRAS.452..575S, 2016MNRAS.462..190R}.  Due to the incidence of
supermassive black holes in most galaxies, the tight scaling relations
observed to exist between the mass of these black holes and properties
of their host galaxies, and the increasing consensus that AGN activity
feeds back on the host galaxy evolution, the AGN luminosity function
also constrains models of galaxy formation \citep{2006ApJ...650...42L,
  2008MNRAS.385.1846M}.  Finally, thanks to their relative brightness
and high Lyman-continuum (LyC) escape fractions, the luminosity
function of AGN determines their contribution to the radiation
background that influences the temperature and ionization state of the
intergalactic medium (IGM), possibly even primarily driving hydrogen
and helium reionization in the last twelve billion years
\citep{2012ApJ...746..125H, 2015AA...578A..83G, 2017ApJ...847L..15O,
  2018MNRAS.474.2904P, 2018arXiv180104931P}.

There is a renewed interest in understanding the evolution of the UV
luminosity function of AGN, triggered by the discovery of 19
low-luminosity ($-18.9>M_{1450}>-22.6$) AGN candidates between
redshifts $z=4.1$ and $6.3$ by \citet{2015AA...578A..83G} using a
novel X-ray/NIR selection criterion.  This finding suggested that if
100\% LyC escape fraction is assumed AGN brighter than $M_{1450}=-18$
can potentially produce all of the metagalactic hydrogen
photoionization rate inferred from the \lya forest at $4<z<6$, despite
the fact that these AGN are fainter than the brightest star-forming
galaxies at these redshifts.  (The UV luminosity function of galaxies
at $z=6$ has $M^*_{1450}\sim -20$.)  A significant presence of AGN at
high redshift ($z\sim 6$) and a dominant contribution of AGN to
reionization is appealing as the LyC escape fraction of galaxies is
uncertain.  High-redshift galaxies down to rest-frame UV magnitude
$M_\mathrm{UV}=-12.5$ ($L\sim 10^{-3}L^*_\mathrm{galaxies}$) at $z=6$
\citep{2017ApJ...835..113L} and redshifts up to $z=11.1$
\citep{2016ApJ...819..129O} have now been reported.  But the escape of
LyC photons has been measured in only a few comparatively bright
($L>0.5L^*_\mathrm{galaxies}$) galaxies at relatively low redshifts
($z < 4$).  The escape fraction in these galaxies reveals a broad
distribution from less than 2\% to more than 80\%
\citep{2010ApJ...725.1011V, 2011ApJ...736...41B, 2015ApJ...804...17S,
  2015ApJ...810..107M, 2016A&A...585A..48G, 2017MNRAS.468..389J,
  2018arXiv180506071S, 2018arXiv180601741F, 2018arXiv180511621M,
  2018arXiv180303655C}, but the average escape fraction is typically
lower than 20\%.  Statistical constraints from \HI\ column density
measurements in gamma-ray burst (GRB) afterglow spectra suggest an
even lower escape fraction of 0.5\% \citep{2007ApJ...667L.125C,
  2009ApJS..185..526F, 2018arXiv180507318T}.  This is a challenge for
reionization models, which require an escape fraction of about 20\% in
galaxies as faint as $M_\mathrm{UV}=-13$ \citep{2016PASA...33...37F,
  2015ApJ...802L..19R, 2016MNRAS.457.4051K}.  An enhanced incidence of
high-redshift AGN may also be consistent with the shallow bright-end
slopes of the $z\sim 7$ UV luminosity function of galaxies relative to
a Schechter function \citep{2012MNRAS.426.2772B, 2014MNRAS.440.2810B,
  2014ApJ...792...76B, 2015MNRAS.452.1817B} and the hard spectra of
these galaxies \citep{2015MNRAS.450.1846S, 2015MNRAS.454.1393S,
  2017MNRAS.464..469S}.  Finally, AGN may also provide a natural
explanation for the large scatter in the \lya opacity between
different quasar sightlines close to redshift $z=6$
\citep{2015MNRAS.447.3402B, 2018arXiv180208177B}.
These opacity fluctuations extend to substantially larger scales
($\gtrsim 50\, h^{-1}$cMpc) than expected in galaxies-dominated
reionization models (\citealt{2015MNRAS.453.2943C}; although see
\citealt{2016MNRAS.460.1328D, 2015ApJ...813L..38D,
  2018arXiv180308932B}).  AGN clustering can result in these
fluctuations naturally if there is a significant contribution
($\gtrsim 50\%$) of AGN to the ionising emissivity at $z=5$--$6$
\citep{2017MNRAS.465.3429C}.

Several counter-arguments against a higher incidence of AGN at high
redshift have also been made. \citet{2017MNRAS.468.4691D} and
\citet{2018MNRAS.473.1416M} pointed out that an enhanced number
density of AGN at $z\sim 6$ can lead to a much earlier
\HeII\ reionization (see also \citealt{2017MNRAS.471..255K}).  For
instance, He~\textsc{ii} reionization is complete at $z=4.5$ in the
model of \citet{2015ApJ...813L...8M}, compared to $z=3$ in the
standard scenario \citep{2012ApJ...746..125H}.  Such early
\HeII\ reionization will also result in higher IGM temperatures due to
the associated photoheating.  \citet{2017MNRAS.468.4691D} found that
the temperature of the IGM at mean density is twice as much in
AGN-dominated reionization models as the standard models at
$z=3.5$--$5$, in conflict with constraints from the \lya forest.  This
inconsistency could be avoided by postulating a reduced escape
fraction of \HeII-ionizing photons in AGN, but it is difficult to
reconcile this with a unit escape fraction of hydrogen-ionizing
photons that is required to explain the Ly$\alpha$ opacity
fluctuations.  Further arguments against AGN-dominated reionization
have been presented by \citet{2016MNRAS.459.2299F}, who analysed the
simulations of metal-line absorbers at $z\sim 6$.
\citet{2016MNRAS.459.2299F} find that in their cosmological radiation
hydrodynamical simulations AGN-dominated UV background results in too
many C~\textsc{iv} absorption systems relative to Si~\textsc{iv} and
C~\textsc{ii} at $z\sim 6$.  (However, note that these simulations
assume a $L_\nu\propto\nu^{-1.57}$ AGN SED at extreme UV
\citep{2002ApJ...565..773T}, which is somewhat harder than recent
measurements ($L_\nu\propto\nu^{-1.7}$) by
\citet{2015MNRAS.449.4204L}.  \citet{2016MNRAS.459.2299F} found that,
relative to the the standard model \citep{2012ApJ...746..125H}, the
observed N(Si~\textsc{iv})/N(C~\textsc{iv}) column density ratio
measurements seem to require a harder and more intense $>4$~Ry
background.)  Finally, comparing the \citet{2015AA...578A..83G} sample
to X-ray-selected quasar data at $z=0$--$6$,
\citet{2017MNRAS.465.1915R} argued that the faint end of the AGN UV
luminosity function at $z\sim 6$ is probably shallower that that
reported by \citet{2015AA...578A..83G}.  \citet{2017MNRAS.465.1915R}
argue that the apparent contradiction with the results of
\citet{2015AA...578A..83G} could be explained by contamination from
the host galaxies for faint AGN \citep[see
  also][]{2015MNRAS.453.1946G,2015MNRAS.448.3167W,2016MNRAS.463..348V}.

A straightforward way towards a more robust understanding of AGN
contribution to UV background is to determine the evolution of the AGN
UV luminosity function.  To this end, several estimates of the
luminosity function at various redshifts have been published in the
last decade \citep{2009A&A...507..781S, 2011ApJ...728L..26G,
  2012ApJ...755..169M, 2013ApJ...773...14R, 2013ApJ...768..105M,
  2015AA...578A..83G, 2016ApJ...833..222J, 2016ApJ...829...33Y,
  2017ApJ...847L..15O, 2018PASJ...70S..34A}.  However, many of these
inferences of the luminosity function suffer from inconsistencies
arising from inhomogeneous selection and mixing of AGN data.  Some of
the data sets analysed in these studies consist of photometric
samples, which is likely to increase sample contamination.  In some
studies, distinct data sets binned differently in redshift and
magnitudes were inhomogeneously combined.  Some authors imposed
restricted priors on parameters while fitting luminosity function
models, e.g., by fixing luminosity function slopes to certain values,
which may bias the result.  Finally, some studies arbitrarily excluded
certain data sets.  This has resulted in a large scatter in the
inferred high-redshift hydrogen-ionizing AGN emissivity between
various recent studies.  For instance, there is an order of magnitude
scatter between various estimates of the hydrogen-ionizing AGN
emissivity at $z=5$--$6$ \citep{2011ApJ...728L..26G,
  2012ApJ...755..169M, 2015AA...578A..83G, 2018PASJ...70S..34A,
  2018AJ....155..131M, 2018MNRAS.474.2904P, 2017ApJ...847L..15O}.  Our
aim in this paper is to address this issue, by constructing an AGN
sample with robust redshift and completeness estimates and homogeneous
assumptions of the cosmology and intrinsic AGN spectrum.  After
constructing such a sample, we derive the UV luminosity function of
AGN from redshift $z=0$--$7$ and estimate the AGN contribution to the
UV background.  We discuss our sample construction in
Section~\ref{sec:sample}.  Our derived luminosity functions are
presented in Section~\ref{sec:lf}.  Section~\ref{sec:reion} presents
our inference of the AGN contribution to hydrogen and helium
reionization.  We summarise our findings in Section~\ref{sec:conc}.

We assume a flat cosmology with density parameters
$\left(\Omega_\mathrm{m},\Omega_\Lambda\right)=\left(0.3,0.7\right)$
and a Hubble constant $H_0=70$\,km\,s$^{-1}$\,Mpc$^{-1}$. Comoving
distances are given explicitly in comoving Mpc (cMpc). Magnitudes
are reported in the AB system \citep{1983ApJ...266..713O}, and
observed magnitudes are point spread function (PSF) magnitudes
\citep{2002AJ....123..485S} corrected for Galactic extinction
\citep{1998ApJ...500..525S} unless otherwise noted.
Our homogenised sample (Section~\ref{sec:sample}) uses absolute
monochromatic AB magnitudes at a rest frame wavelength of 1450\,\AA.


\begin{figure*}
  \begin{center}
    \includegraphics[width=\textwidth]{qsos.pdf}
    % data.py 
  \end{center}
  \caption{Redshift distribution of the 83,488 AGN considered in this
    analysis.  Shown here are the observed AGN numbers, without
    correcting for incompleteness.  Further details on each of these
    data sets are in Table~\ref{tab:samples} and
    Section~\ref{sec:sample}.}
  \label{fig:qsos}
\end{figure*}

\begin{table*}
  \caption{AGN samples analysed in this work.}
  \label{tab:samples}
  \begin{tabular}{lcllrS}
    \hline
    Sample& $z$ range$^a$& Survey & Reference & Number & {Area} \\
    & & & & of quasars & {(deg$^2$)} \\
    \hline
    1 & 0.0--2.2 & SDSS DR7 & \citet{2010AJ....139.2360S} & 48664 & 6248.0 \\
    2$^b$ & 0.4--2.2 & 2SLAQ SGP & \citet{2009MNRAS.392...19C} & 2338 & 64.2 \\
    3$^b$ & 0.4--2.2 & --- NGP & \citet{2009MNRAS.392...19C} & 7027 & 127.7 \\
    4 & 2.2--3.5 & BOSS DR9 & \citet{2013ApJ...773...14R} & 23301 & 2236.0 \\
    5 & 3.7--4.7 & SDSS DR7 & \citet{2010AJ....139.2360S} & 1785 & 6248.0 \\
    6 & 3.6--5.2 & NDWFS & \citet{2011ApJ...728L..26G} & 12 & 1.71 \\
    7 & 3.8--5.3 & DLS & \citet{2011ApJ...728L..26G} & 12 & 2.05 \\
    8 & 4.7--5.4 & SDSS+WISE & \citet{2016ApJ...829...33Y} & 99 & 14555.0 \\
    9$^c$ & 4.7--5.5 & SDSS DR7 & \citet{2013ApJ...768..105M} & 103 & 6248.0 \\
    10$^c$ & 4.7--5.5 & --- Stripe 82 & \citet{2013ApJ...768..105M} & 59 & 235.0 \\
    11 & 5.7--6.5 & SDSS Main & \citet{2016ApJ...833..222J} & 24 & 11240.0 \\
    12 & 5.7--6.5 & --- Overlap & \citet{2016ApJ...833..222J} & 10 & 4223.0 \\
    13 & 5.7--6.5 & --- Stripe 82 & \citet{2016ApJ...833..222J} & 13 & 277.0 \\
    14 & 5.8--6.6 & CFHQS Deep & \citet{2010AJ....139..906W} & 1 & 4.47 \\
    15 & 5.8--6.6 & --- Very Wide & \citet{2010AJ....139..906W} & 16 & 494.0 \\
    16 & 5.8--6.5 & Subaru High-$z$ Quasar & \citet{2015ApJ...798...28K} & 2 & 6.5 \\
    17$^d$ &4.0--6.5 & CANDELS GOODS-S & \citet{2015AA...578A..83G} & 19 & 0.047 \\
    18$^e$ & 6.5--7.4 & UKIDSS & \citet{2011Natur.474..616M} & 1 & 3370.0 \\
    19$^e$ & 6.5--7.4 & UKIDSS & \citet{2015ApJ...801L..11V} & 1 & 3370.0 \\
    20$^e$ & 6.5--7.4 & ALLWISE+UKIDSS+DECaLS & \citet{2018Natur.553..473B} & 1 & 2500.0 \\
    \hline
  \end{tabular}\\
  \begin{minipage}{14.5cm}
    \textsuperscript{$a$}{Redshift range of the sample or, for small
      samples, approximate redshift range in which the survey is
      sensitive.}\\
    \textsuperscript{$b$}{Restricted to $z<2.2$.}\\
    \textsuperscript{$c$}{Restricted to $M_{1450}>-26.73$ quasars to
      avoid overlap with the \citet{2016ApJ...829...33Y} sample.}\\
    \textsuperscript{$d$}{Used only in Section~\ref{sec:global} and
      Appendix~\ref{sec:conv} due to lack of spectroscopic redshifts
      for majority of the sample.}\\
    \textsuperscript{$e$}{Used only in Section~\ref{sec:global} due to
      roughly estimated selection function.}
 \end{minipage}
\end{table*}

\section{Homogenised Quasar Sample}
\label{sec:sample}

\subsection{Sample Selection}
\label{sect:samplesel}

We started by compiling the samples of recent photometric rest-frame
UV-optical quasar surveys. The restriction to UV-optical surveys was
mainly driven by our science goal to characterise the UV luminosity
function of Type~1 quasars. X-ray-selected samples are less suited for
this purpose due to spectroscopic incompleteness and the $\sim
0.4$\,dex scatter in the conversion from X-ray to UV luminosity
\citep{2010A&A...512A..34L, 2015MNRAS.453.1946G, 2016ApJ...819..154L}
that contributes significantly to the error budget in the UV
luminosity function of X-ray-selected samples unless rest-frame UV
photometry is incorporated \citep{2015AA...578A..83G}.  The individual
surveys and their main characteristics are listed in
Table~\ref{tab:samples}.  Figure~\ref{fig:qsos} presents a redshift
histogram of the contributing surveys.

We included surveys based on a set of simple criteria:
\begin{enumerate}
\item High spectroscopic completeness of the target sample.
\item Accurate rest-frame UV-optical CCD photometry.
\item Statistical power (sample size, coverage in $z$ and/or absolute magnitude).
\item A well-characterised selection function.
\end{enumerate}
As a prerequisite for a joint analysis of the QLF we obtained the
survey selection functions in electronic form, either from the
publication or by request from the authors. As a reference for future
surveys we make them electronically available here in modified and
homogenised form (see Section~\ref{sect:datahom} and
Appendix~\ref{sec:code}).

Due to their selection criteria and their statistical power specific
surveys contribute to distinct redshift ranges. At $z<2.2$ we
considered quasars from the SDSS DR7 quasar catalogue
\citep{2010AJ....139.2360S} and the 2SLAQ survey catalogue
\citep{2009MNRAS.392...19C}. We restricted the SDSS DR7 sample to the
48,664 $0.1<z<2.2$ quasars selected with the final SDSS quasar
selection algorithm \citep{2002AJ....123.2945R, 2006AJ....131.2766R}
from a survey area of 6248\,deg$^2$ \citep{2012ApJ...746..169S}. We
adopted the SDSS targeting photometry corrected for Galactic
extinction \citep{2010AJ....139.2360S}. To limit systematic
uncertainties in the correction for host galaxy light \citep[detailed
  in][]{2009MNRAS.392...19C} we restricted the 2SLAQ sample to 9365
$g<21.85$ $0.4<z<2.2$ quasars from its spectroscopic survey footprint
near the North Galactic Pole (NGP, 7027 quasars in $127.7$\,deg$^2$)
and the South Galactic Pole (SGP, 2338 quasars in
$64.2$\,deg$^2$). The small sample overlap between SDSS and 2SLAQ (102
quasars) has negligible impact on the QLF evaluation.

At $2.2<z<3.5$ we used a single sample of 23,301 uniformly
colour-selected quasars from 2236\,deg$^2$ in BOSS DR9
\citep{2013ApJ...773...14R} due to several improvements compared to
previous surveys. First, it covers a similar magnitude range as 2SLAQ
but with $>20$ times as many quasars. Second, although the SDSS DR7
sample provides better coverage of the bright end of the QLF at these
redshifts, its selection function is highly dependent on the assumed
incidence of (partial) Lyman limit systems in the IGM
\citep{2009ApJ...705L.113P, 2011ApJ...728...23W}. While the BOSS DR9
selection function considers these improvements, the uncertainty in
the QLF remains dominated by assumptions in the selection function
given the large sample size
\citep{2013ApJ...773...14R}. Variability-selected quasar samples
circumvent this issue \citep{2013ApJ...773...14R, 2013A&A...551A..29P,
  2016A&A...587A..41P}, but may be affected by (i) single-epoch
imaging incompleteness at the faint end \citep{2013ApJ...773...14R},
and (ii) uncertainties in the selection function caused by the limited
number of known $z\ga 3$ quasars not selected by variability in the
same footprint \citep{2013A&A...551A..29P, 2016A&A...587A..41P}.

At $3.7<z<4.7$ we used a combination of SDSS DR7 \citep[1785 uniformly
  selected quasars from][]{2010AJ....139.2360S} and the NDWFS$+$DLS
survey \citep{2010ApJ...710.1498G,2011ApJ...728L..26G}. The lower cut
$z>3.7$ in SDSS limits the impact of systematic uncertainties in the
\citet{2006AJ....131.2766R} selection function
\citep{2009ApJ...705L.113P, 2011ApJ...728...23W}. We did not consider
the results from surveys for faint $z\sim 4$ quasars in the COSMOS
field \citep{2011ApJ...728L..25I, 2012ApJ...755..169M} due to
systematic errors in their selection functions\footnote{Both studies
  simulated quasar colours with a mean IGM attenuation curve
  \citep{1995ApJ...441...18M} that cannot account for stochastic Lyman
  continuum absorption, and therefore underpredicts the variance in
  quasar colours \citep{1999ApJ...518..103B, 2008MNRAS.387.1681I,
    2011ApJ...728...23W}. Modelling the colour variance in these
  surveys is essential, as most of the \citet{2011ApJ...728L..25I}
  quasars are near the edge of their colour selection region (see
  their Figure~1), and \citet{2012ApJ...755..169M} require modest
  attenuation of the $U$ band flux relative to the mid-infrared
  flux.}. Furthermore, 30 per cent of the \citet{2012ApJ...755..169M}
COSMOS sample have visually estimated photometric redshifts, and the
spectroscopic subsample reveals that 40 per cent of the visually
estimated redshifts are biased low
($z_\mathrm{spec}>z_\mathrm{est}+0.3$, see their Figure~9). These
unaccounted systematic redshift errors at least partly explain the
discrepancy in the $z\sim 4$ QLF between \citet{2011ApJ...728L..26G}
and \citet{2012ApJ...755..169M}, which justifies our preference for
the former sample that is 77 per cent spectroscopically complete at
$R$ magnitudes $<23.5$ \citep{2011ApJ...728L..26G}.


At $4.7\le z<5.5$ we combined several recent surveys, accounting for
sample overlap and updated selection functions. At the bright end of
the QLF we used the 99 quasars from the SDSS+WISE survey
\citep{2016ApJ...829...33Y} that have $M_{1450}\le -26.73$ in our adopted
cosmology.  For these 99 quasars selected from 14,555\,deg$^2$ we
adopted the \citet{2016ApJ...829...33Y} selection function.  The
\citet{2016ApJ...829...33Y} sample partially overlaps with the SDSS
DR7 sample from \citet{2013ApJ...768..105M}, so to avoid
double-counting quasars we used the latter sample only at
$M_{1450}>-26.73$, yielding 103 additional $4.7\le z<5.5$ quasars selected in
6248\,deg$^2$. We used the $z\sim 5$ SDSS DR7 selection function from
\citet{2013ApJ...768..105M} that supersedes the one from
\citet{2006AJ....131.2766R} due to improved bandpass corrections and IGM
parameterization.  To these two bright-end samples we added the
faint-end sample from the \citet{2013ApJ...768..105M} SDSS Stripe~82
survey (59 uniformly selected $M_{1450}>-26.73$ $4.7\le z<5.5$ quasars in 235\,deg$^2$)
and two $4.7\le z<5.5$ quasars from \citet{2011ApJ...728L..26G},
adopting the respective selection functions.  We did not consider the
limit on the $z\sim 5$ QLF by \citet{2012ApJ...756..160I} due to
systematic errors in their selection
function\footnote{\citet{2012ApJ...756..160I} underestimated the
  dispersion in rest-frame UV quasar colours with respect to SDSS at
  all redshifts (their Figure~4). Contrary to their claim, photometric
  errors have a small effect on the colour distribution of SDSS
  quasars given the statistical errors of $<0.03$\,mag in $gri$ for 90
  per cent of the SDSS DR7 bright quasar sample ($i<19.1$) and a
  relative calibration error of $\sim 1$ per cent
  \citep{2008ApJ...674.1217P}.}.

The SDSS colours of $5.1<z<5.5$ quasars are similar to those of M and
L dwarf stars, resulting in a low and uncertain completeness
\citep{2013ApJ...768..105M}. WISE mid-infrared selection performs
better \citep{2016ApJ...829...33Y}, but is restricted to the bright
end of the quasar population. Unlike \citet{2013ApJ...768..105M} we
include the 9 uniformly selected $M_{1450}>-26.73$ SDSS DR7 quasars
and the 10 SDSS Stripe~82 quasars at $z>5.1$, adopting their low
completeness. As we will show in Section~3.2, the resulting
QLF is consistent with those at lower and higher redshifts, indicating
that the \citet{2013ApJ...768..105M} selection functions are quite
reliable.

At $z\sim 6$ we combined the samples from all spectroscopic surveys
with a determined selection function as of June 2017.
\citet{2016ApJ...833..222J} recently compiled all quasars discovered
in several SDSS $z\sim 6$ surveys together with consistently derived
selection functions. Their uniform sample consists of 24 quasars from
the SDSS main survey (11,240\,deg$^2$), 10 additional quasars in
regions with two or more SDSS imaging scans (so-called overlap
regions, 4223\,deg$^2$), and 13 faint quasars from SDSS Stripe~82
(277\,deg$^2$). The CFHQS \citep{2010AJ....139..906W} provided a
uniform sample of 16 quasars in the Very Wide Survey (494\,deg$^2$)
and a single quasar in the Deep Survey ($4.47$\,deg$^2$). The one
quasar detected in both SDSS and CFHQS does not lead to underestimated
statistical errors in the QLF. Lastly, we included the two objects
from \citet{2015ApJ...798...28K}, one of which might be a galaxy due
to its narrow Ly$\alpha$ emission line (half width at half maximum
427\,km\,s$^{-1}$). With better photometry and additional spectroscopy
recently reported by \citet{2017ApJ...847L..15O} the
\citet{2015ApJ...798...28K} sample is complete.  Although we will
account for the slightly different redshift sensitivities for the
different surveys, we will quote a nominal redshift range $5.7<z<6.5$
for the combined $z\sim 6$ sample.

At the highest redshifts $z>6.5$ we considered ULAS~J1120$+$0641
\citep[$z=7.085$,][]{2011Natur.474..616M} and PSO~J$036.5078+03.0498$
\citep[$z=6.527$,][]{2015ApJ...801L..11V}, both discovered in UKIDSS imaging
(3370\,deg$^2$), in addition to the current highest-redshift quasar J1342$+$0928
at $z=7.54$ selected from a combination of UKIDSS, WISE and DECaLS
\citep[$\sim 2500$\,deg$^2$,][]{2018Natur.553..473B}. Although
currently only rough estimates exist concerning their selection
functions, these quasars provide constraints on the evolution of the
integrated quasar space density from $z\sim 6$ to $z\sim 7$.  We use
them in our secondary analysis in Section~\ref{sec:global}.  Although
we do not include the highly debated \citet{2015AA...578A..83G} sample
in our main analysis due to its rough selection function and lack of
spectroscopy for 17 of the 22 quasar candidates, we use it in
Section~\ref{sec:global} to constrain the faint end ($M_{1450}>-23$)
of the QLF at $z>4.1$. We restricted the \citet{2015AA...578A..83G}
sample to the 19/22 sources considered in their QLF.

\subsection{Sample Homogenisation}
\label{sect:datahom}

\begin{figure}
    \includegraphics[width=\columnwidth]{kcorr.pdf}
  \caption{Bandpass corrections $K_{m,1450}$ from a broadband
    magnitude $m=\{g,i,z_\mathrm{AB}\}$ to the monochromatic AB
    magnitude at 1450\,\AA\ as a function of redshift $z$ for the
    \citet{2015MNRAS.449.4204L} quasar SED used in this work, and for
    two quasar composite spectra \citep{2001AJ....122..549V,
      2002ApJ...565..773T}.  The redshift range has been restricted to
    exclude the Ly$\alpha$ forest and to account for the different
    rest frame wavelength coverage of the spectra.}
  \label{fig:kcorr}
\end{figure}

For a joint fit of the QLF it is necessary to homogenise the different
survey samples in absolute magnitude, and to convert their selection
functions to the same absolute magnitude system. For the analysis of
the quasar UV emissivity and to be consistent with published work at
$z>3$ we chose to convert all samples and selection functions to the
absolute AB magnitude at a rest frame wavelength $\lambda=1450$\,\AA
\begin{equation}\label{eq:absmag}
  M_{1450}\left(z\right) = m-5\log{\left(\frac{d_L\left(z\right)}
    {\mathrm{Mpc}}\right)}-25-K_{m,1450}\left(z\right),
\end{equation}
with the luminosity distance
\begin{equation}
  d_L(z)=(1+z)\frac{c}{H_0}\int_0^z\frac{\mathrm{d}z^\prime}
  {\sqrt{\Omega_\mathrm{m}(1+z^\prime)^3+\Omega_\Lambda}}
  \label{eqn:dl}
\end{equation}
to a quasar at redshift $z$, the apparent magnitude $m$ in a filter used
in the survey, and the bandpass correction $K_{m,1450}\left(z\right)$
\citep{1956AJ.....61...97H, 1968ApJ...154...21O, 2000A&A...353..861W,
  2002astro.ph.10394H}. For the bandpass correction we used a
combination of the \citet{2015MNRAS.449.4204L} stacked quasar spectrum at
$\lambda<2500$\,\AA, and the \citet{2001AJ....122..549V} quasar
composite spectrum at longer wavelengths to cover the lowest
redshifts. The samples from SDSS and BOSS are defined in the SDSS $i$
band, while 2SLAQ is defined in the $g$ band. At $z>4.7$ we adopted
the SDSS $z$ band magnitude (in the following denoted $z_\mathrm{AB}$)
for SDSS DR7 quasars to avoid additional corrections due to the
Ly$\alpha$ forest. Figure~\ref{fig:kcorr} shows our bandpass
corrections for SDSS, BOSS and 2SLAQ as a function of redshift. We
ignored the luminosity dependence of the bandpass correction due to
the known anticorrelation of emission line equivalent width and
luminosity \citep{1977ApJ...214..679B}.  While the
\citet{2015MNRAS.449.4204L} spectrum is for luminous ($M_{1450}\simeq
-27.2$) quasars, UV composite spectra including fainter quasars
\citep{2002ApJ...565..773T, 2012ApJ...752..162S, 2014ApJ...794...75S}
give similar values, such that our bandpass corrections remain
applicable at $M_{1450}\la -24$. Empirical luminosity-dependent
bandpass corrections show a $\la 0.2$\,mag variation over $\sim 4$
orders of absolute magnitude depending on redshift and the filter, and
with a $\sim 0.2$\,mag intrinsic scatter due to individual quasar-to-quasar
variations \citep{2013ApJ...773...14R, 2013ApJ...768..105M,
  2013A&A...551A..29P}.

%% JFH Let's make all these conversion factors and codes to do K-correction
%% publicly avaialble in an appendix. 

For the $z<2.2$ sample we corrected the SDSS $i$ and 2SLAQ $g$ band
magnitudes for host galaxy contamination following
\citet{2009MNRAS.392...19C}. Considering the different magnitude
limits of 2SLAQ and SDSS, the modelled host galaxy contamination is
small for $z>0.5$ quasars ($<0.1$\,mag in $g$, $<0.2$\,mag in $i$),
and is negligible at $z>0.8$.  In case the band defining the magnitude
limit of the survey undesirably overlaps with the Ly$\alpha$ forest
\citep{2010ApJ...710.1498G, 2011ApJ...728L..26G, 2013ApJ...768..105M}
we adopted their respective bandpass corrections to $M_{1450}$. In
particular, for the \citet{2010ApJ...710.1498G, 2011ApJ...728L..26G}
sample we recomputed $M_{1450}$ from the $R$ band photometry to be
consistent with the selection function defined in $R$, and to avoid
uncertainties in their spectrophotometry due to incomplete spectral
coverage. Since the \citet{2010ApJ...710.1498G, 2011ApJ...728L..26G}
$R$ band traces the rest frame UV, we assumed negligible host galaxy
contamination for their faint quasars. For the remaining high-redshift
surveys reporting $M_{1450}$ obtained by various methods
\citep{2010AJ....139..906W, 2011Natur.474..616M, 2015ApJ...798...28K,
  2015ApJ...801L..11V, 2016ApJ...829...33Y, 2016ApJ...833..222J,
  2018Natur.553..473B} we did not re-compute $M_{1450}$, but applied
appropriate shifts to correct to our adopted cosmology.

The selection functions were treated similarly, i.e.\ the photometric
selection function of survey $j$ given in observed magnitudes
$f_{\mathrm{p},j}\left(m,z\right)$ \citep{2006AJ....131.2766R,
  2009MNRAS.392...19C, 2010ApJ...710.1498G, 2013ApJ...773...14R} was
transformed to our absolute magnitudes
$f_{\mathrm{p},j}\left(M_{1450},z\right)$ with
Equation~\ref{eq:absmag}, while the ones given in $M_{1450}$ were
adjusted to our cosmology. Note, however, that many surveys report
additional sources of incompleteness that require modifications to the
photometric selection functions.

For 2SLAQ we corrected for magnitude-dependent spectroscopic coverage
in the two survey areas \citep[$f_\mathrm{c,NGP}\left(g\right)$ and
  $f_\mathrm{c,SGP}\left(g\right)$; Figure~4
  in][]{2009MNRAS.392...19C} and spectroscopic redshift success
\citep[$f_\mathrm{s,2SLAQ}\left(g\right)$; Figure~6b
  in][]{2009MNRAS.392...19C} by multiplying them with the photometric
selection function, resulting in two area-specific 2SLAQ selection
functions
$f_\mathrm{NGP}\left(M_{1450},z\right)=f_\mathrm{p,2SLAQ}f_\mathrm{c,NGP}f_\mathrm{s,2SLAQ}$
and
$f_\mathrm{SGP}\left(M_{1450},z\right)=f_\mathrm{p,2SLAQ}f_\mathrm{c,SGP}f_\mathrm{s,2SLAQ}$
that are relevant for the QLF.  The $z<4.7$ SDSS photometric selection
function was modified to include known imaging incompleteness to
$f_{\mathrm{SDSS},z<4.7}=0.95f_{\mathrm{p,SDSS},z<4.7}$
\citep{2006AJ....131.2766R}. The BOSS colour-selected sample contains
quasars with $f_\mathrm{c,BOSS}f_\mathrm{s,BOSS}\ge 0.85$
\citep{2013ApJ...773...14R}, and we adopted
%% JFH The overline looks weird here. I would suggest taht you rather
%% use angle brackets instead. It is not jiving with the equation above
%% and is very hard to read. 
$f_\mathrm{BOSS}=\overline{f_\mathrm{c,BOSS}f_\mathrm{s,BOSS}}f_\mathrm{p,BOSS}=0.962f_\mathrm{p,BOSS}$. \citet{2010ApJ...710.1498G}
presented two area-specific photometric selection functions due to
different filters employed, and more follow-up spectroscopy was
reported in \citet{2011ApJ...728L..26G}. We accounted for remaining
spectroscopic incompleteness at $R>23$, yielding the final selection
functions $f_\mathrm{NDWFS}$ and $f_\mathrm{DLS}$. The updated $z\sim
5$ SDSS photometric selection function \citep{2013ApJ...768..105M} was
modified to include imaging and spectroscopic incompleteness, yielding
$f_{\mathrm{SDSS},z\sim 5}=0.95^2f_{\mathrm{p,SDSS},z\sim 5}$. In the
deeper $z\sim 5$ SDSS Stripe~82 survey the spectroscopic
incompleteness is larger and magnitude-dependent \citep[Figure~14
  in][]{2013ApJ...768..105M}, resulting in $f_{\mathrm{S82},z\sim
  5}=0.95f_{\mathrm{s,S82},z\sim
  5}\left(i\right)f_{\mathrm{p,S82},z\sim 5}$. Likewise, imaging and
magnitude-dependent spectroscopic incompleteness was factored into the
\citet{2016ApJ...829...33Y} photometric selection function (their
Figures~5 and 7), resulting in
$f_\mathrm{SDSS+WISE}=0.97f_\mathrm{s,SDSS+WISE}\left(z_\mathrm{AB}\right)f_\mathrm{p,SDSS+WISE}$.
We obtained a rough estimate of the \citet{2015AA...578A..83G}
selection function by comparing the corrected and observed QLFs,
i.e.\ taking $f_\mathrm{GOODS-S}=\phi_\mathrm{obs}/\phi_\mathrm{corr}$
(see their Table~3).  Finally, for the three $z>6.5$ quasars we
assumed a selection function of unity in a range of $z$ and $M_{1450}$
estimated by the respective survey teams (private communication).

\section{Luminosity function}
\label{sec:lf}

After homogenising the samples and selection functions we are in a
position to compute the UV luminosity function of AGN. We begin our
analysis by computing binned estimates of the luminosity function as a
function of magnitude in several redshift intervals.  We then perform
parametric maximum-likelihood fits of the luminosity function in the
individual redshift bins, examine the resulting parameters as a
function of redshift, and attempt a joint fit in magnitude and
redshift.  For simplicity we will use the notation $M\equiv M_{1450}$
in the following.

\subsection{Binned luminosity function estimates}
\label{sec:binnedlf}

In a magnitude bin $[M_\mathrm{min}, M_\mathrm{max})$, and redshift
  bin $[z_\mathrm{min}, z_\mathrm{max})$, we define the binned luminosity
    %% JFH Notation error here. 
function as \citep{2000MNRAS.311..433P}
\begin{equation}
  \phi \equiv \frac{N_\mathrm{QSO}}{V_\mathrm{bin}},
\end{equation}
where $N_\mathrm{QSO}$ is the number of quasars with magnitude
$M_\mathrm{min}\leq M<M_\mathrm{max}$ and redshift
$z_\mathrm{min}\leq z<z_\mathrm{max}$, and
\begin{equation}
  V_\mathrm{bin} = \int_{M_\mathrm{min}}^{M_\mathrm{max}}\mathrm{d}M
  \int_{z_\mathrm{min}}^{z_\mathrm{max}}\mathrm{d}z\, f(M, z)\,\frac{\mathrm{d}V}{\mathrm{d}z},
  \label{eqn:vi}
\end{equation}
is the effective volume of the bin. The inclusion of the survey selection function
$f(M,z)$ (Section~\ref{sect:datahom}) in Equation~(\ref{eqn:vi}) accounts for
what are sometimes called ``incomplete bins''
\citep{2006AJ....131.2766R}.  The comoving volume element $\mathrm{d}V/\mathrm{d}z$ is
given by
\begin{equation}
  \frac{\mathrm{d}V}{\mathrm{d}z}=\frac{\mathrm{d}V}{\mathrm{d}z\,\mathrm{d}\Omega}\times A\times\frac{4\pi}{41253},
\end{equation}
where $A$ is the survey area in deg$^2$, and 
\begin{equation}
  \frac{\mathrm{d}V}{\mathrm{d}z\,\mathrm{d}\Omega}=\frac{c}{H_0}\frac{d_L^2\left(z\right)}
       {\left(1+z\right)^2\left[\Omega_\mathrm{m}\left(1+z\right)^3+\Omega_\Lambda\right]^{1/2}}
  \label{eqn:dvdzdo}
\end{equation}
denotes the comoving volume element per unit solid angle
\citep{1999astro.ph..5116H}. The resulting luminosity function $\phi$ has units of
$\mathrm{cMpc}^{-3}\mathrm{mag}^{-1}$. 

We evaluate the double integral in Equation~(\ref{eqn:vi}) by the Euler method,
i.e., by simply summing over the ``tiles'' of the selection function in $M$ and $z$
without interpolation.
This results in $V_i=0$ for a few quasars, which are subsequently removed
from our analysis\footnote{
  We note that the interpolation of sometimes coarse
  selection functions is not straightforward due to their strong gradients.
  The presence of objects with $V_i=0$ implies that the selection function has systematic errors.}.
In each bin we estimate the uncertainty in the luminosity function by
assuming Poisson statistics \citep{1986ApJ...303..336G} for the number
of quasars, i.e.\ assuming negligible uncertainty in the selection function.
While this is a reasonable approximation for small surveys, large surveys
with negligible Poisson errors are instead limited by rarely quantified systematic
errors due to implicit assumptions in their selection functions.
The resultant binned luminosity function estimates
are shown by the circles in Figure~\ref{fig:mosaic}.

From Figure~\ref{fig:mosaic} we see that the distribution of luminosity
function values in each redshift bin are suggestive of a double power
law form for the QLF.  We will fit such a form below. 
However, in several redshift bins we note a suspicious decline of the
luminosity function at the faint limit of several surveys, for example in
the 2SLAQ sample at $z<2.2$, and in the SDSS sample at $z<1.8$ and $z\sim 4$.
The inconsistency between the SDSS faint end and the deeper
2SLAQ QLF indicates that the SDSS selection function is
systematically overestimated at its magnitude limit.
At $z\sim 4$ the SDSS faint end QLF is inconsistent with the fainter
\citet{2011ApJ...728L..26G} QLF. We identify such discrepant bins
`by eye', discard the contributing quasars from our analysis, and set their
selection function values to zero.
The discarded magnitude bins are shown in Figure~\ref{fig:mosaic} by
open circles.

%% GW: Maybe say what we mean by one-sigma uncertainty: This is a credible interval from the posterior distribution. Do we use the highest posterior density interval or the equal-tailed interval?
\begin{figure*}
  \begin{center}
    \includegraphics[width=\textwidth,keepaspectratio]{mosaic_small.pdf}
    % mosaic.py 
  \end{center}
  \caption{Homogenised quasar luminosity functions at rest-frame wavelength
    $\lambda=1450$\,\AA\ in redshift bins from $z=0.1$ to $6.5$.
    The symbols show our inferred binned luminosity functions from various
    data sets:
    \citet[red]{2010AJ....139.2360S},
    \citet[green]{2009MNRAS.392...19C},
    \citet[dark blue]{2013ApJ...773...14R},
    \citet[light blue]{2011ApJ...728L..26G},
    \citet[yellow]{2016ApJ...829...33Y},
    \citet[DR7 brown]{2013ApJ...768..105M},
    \citet[Stripe 82 teal]{2013ApJ...768..105M},
    \citet[pink]{2016ApJ...833..222J},
    \citet[orange]{2010AJ....139..906W}, and
    \citet[grey]{2015ApJ...798...28K}.
    Open circles in corresponding colours indicate magnitude bins
    excluded due to incompleteness in the respective data sets.  The
    number of AGN before this selection is shown in parentheses; the
    selected number of AGN is shown outside parentheses.  In each
    redshift bin, the black curve shows our fiducial double power law
    model fit, which is represented by the median of the posterior
    probability distribution function.  The grey shaded area shows the
    one-sigma (68.26\%) uncertainty.  See Sections~\ref{sec:binnedlf}
    and \ref{sec:bins} for further details.}
  \label{fig:mosaic}
\end{figure*}

\begin{figure*}
  \begin{center}
    \includegraphics[width=0.7\textwidth]{evolution_individuals.pdf}
    % summary_bins.py 
  \end{center}
  \caption{Redshift evolution of the four double power law parameters
    from the redshift bins shown in Figure~\ref{fig:mosaic}.  Vertical
    error bars show one-sigma (68.26\%) statistical uncertainties,
    whereas horizontal error bars show widths of the redshift bins.
    We identify the general evolutionary trends of each of these
    parameters from the bins shown by the filled symbols.  The open
    symbols show bins that appear to be offset from these trends,
    likely due to unknown systematic errors.  The open circles at
    $2.2<z<3.5$ show the BOSS sample, while the bins at $z < 0.6$
    contain AGN from the SDSS and 2SLAQ data sets.  See
    Section~\ref{sec:bins} for further details.}
  \label{fig:evoln}
\end{figure*}

%%GW: Added a note on discarded redshift bins
%%GW: Check on final Nqso needed! The number at z~5 is inconsistent with Fig. 3
\begin{table*}
  % bins_tabulate.py; Nqso added by hand
  \caption{\textbf{Check on final Nqso needed! The number at $z\sim5$ is inconsistent with Fig. 3. We could also have two columns for the number of quasars: before and after magnitude restrictions.}
    Posterior median double power law luminosity function parameters
    and their $1\sigma$ (68.26\%) statistical errors in
    various redshift bins shown in Figure~\ref{fig:evoln}.
    The luminosity function parameters
    $\phi_*$, $M_*$, $\alpha$, and $\beta$ are defined in
    Equation~(\ref{eqn:dpl}), with $\beta$ denoting the faint-end
    slope.  Quasars in each bin have redshifts $z_\mathrm{min}\leq z <
    z_\mathrm{max}$ with a sample mean $\langle z\rangle$.
    The number of quasars in each bin is given by $N_\mathrm{QSO}$.
    %The corresponding luminosity
    %functions are shown in Figure~\ref{fig:mosaic}.  See
    %Section~\ref{sec:bins} for further details.
    }
  \label{tab:bins}
  \begin{tabular}{lccr....}
    \hline
    $\langle z\rangle$ &
    $z_\mathrm{min}$ &
    $z_\mathrm{max}$ &
    $N_\mathrm{QSO}$ &
    \multicolumn{1}{c}{$\log_{10}(\phi_*/$} &
    \multicolumn{1}{c}{$M_*$} &
    \multicolumn{1}{c}{$\alpha$} &
    \multicolumn{1}{c}{$\beta$} \\
    
    &
    &
    &
    &
    \multicolumn{1}{c}{cMpc$^{-3}$mag$^{-1}$)} &
    &
    & \\
    \hline
    0.31$^a$ & 0.10 & 0.40 & 3632 & -5.62^{+0.09}_{-0.11} & -21.37^{+0.21}_{-0.23} & -2.94^{+0.09}_{-0.10} & -1.22^{+0.18}_{-0.16} \\
    0.50$^a$ & 0.40 & 0.60 & 4686 & -6.13^{+0.04}_{-0.04} & -23.00^{+0.08}_{-0.07} & -3.38^{+0.08}_{-0.08} & -1.36^{+0.04}_{-0.04} \\
    0.72 & 0.60 & 0.80 & 4684 & -6.42^{+0.09}_{-0.08} & -24.03^{+0.14}_{-0.13} & -3.56^{+0.12}_{-0.13} & -1.75^{+0.04}_{-0.04} \\
    0.91 & 0.80 & 1.00 & 5248 & -6.39^{+0.08}_{-0.08} & -24.50^{+0.12}_{-0.12} & -3.69^{+0.11}_{-0.11} & -1.73^{+0.06}_{-0.05} \\
    1.10 & 1.00 & 1.20 & 6566 & -6.54^{+0.03}_{-0.03} & -25.15^{+0.04}_{-0.04} & -4.18^{+0.08}_{-0.08} & -1.73^{+0.02}_{-0.02} \\
    1.30 & 1.20 & 1.40 & 7132 & -6.42^{+0.04}_{-0.04} & -25.29^{+0.06}_{-0.05} & -3.97^{+0.08}_{-0.08} & -1.74^{+0.03}_{-0.03} \\
    1.50 & 1.40 & 1.60 & 7771 & -6.51^{+0.03}_{-0.03} & -25.67^{+0.04}_{-0.04} & -4.28^{+0.08}_{-0.08} & -1.76^{+0.02}_{-0.02} \\
    1.71 & 1.60 & 1.80 & 7421 & -6.29^{+0.03}_{-0.03} & -25.56^{+0.05}_{-0.05} & -3.96^{+0.07}_{-0.07} & -1.60^{+0.03}_{-0.03} \\
    1.98 & 1.80 & 2.20 & 10876 & -6.72^{+0.03}_{-0.03} & -26.26^{+0.04}_{-0.04} & -4.21^{+0.07}_{-0.07} & -1.87^{+0.02}_{-0.02} \\
    2.30$^a$ & 2.20 & 2.40 & 8419 & -6.16^{+0.07}_{-0.06} & -25.49^{+0.12}_{-0.11} & -3.34^{+0.12}_{-0.12} & -1.61^{+0.04}_{-0.04} \\
    2.45$^a$ & 2.40 & 2.50 & 3403 & -6.40^{+0.08}_{-0.07} & -25.86^{+0.14}_{-0.12} & -3.61^{+0.21}_{-0.22} & -1.60^{+0.05}_{-0.04} \\
    2.55$^a$ & 2.50 & 2.60 & 2640 & -6.15^{+0.08}_{-0.08} & -25.34^{+0.17}_{-0.16} & -3.32^{+0.15}_{-0.16} & -1.39^{+0.08}_{-0.07} \\
    2.65$^a$ & 2.60 & 2.70 & 1883 & -5.98^{+0.05}_{-0.06} & -25.15^{+0.13}_{-0.14} & -3.12^{+0.12}_{-0.12} & -1.05^{+0.09}_{-0.08} \\
    2.75$^a$ & 2.70 & 2.80 & 1135 & -6.29^{+0.08}_{-0.07} & -25.94^{+0.15}_{-0.13} & -3.78^{+0.27}_{-0.30} & -1.34^{+0.07}_{-0.06} \\
    2.85$^a$ & 2.80 & 2.90 & 1069 & -6.45^{+0.11}_{-0.10} & -26.22^{+0.21}_{-0.18} & -3.61^{+0.35}_{-0.41} & -1.46^{+0.08}_{-0.07} \\
    2.95$^a$ & 2.90 & 3.00 & 1104 & -6.76^{+0.07}_{-0.06} & -26.53^{+0.11}_{-0.09} & -5.05^{+0.59}_{-0.63} & -1.71^{+0.05}_{-0.04} \\
    3.05$^a$ & 3.00 & 3.10 & 1127 & -6.77^{+0.08}_{-0.07} & -26.48^{+0.12}_{-0.11} & -4.68^{+0.44}_{-0.52} & -1.70^{+0.06}_{-0.05} \\
    3.15$^a$ & 3.10 & 3.20 & 1041 & -7.26^{+0.16}_{-0.11} & -27.11^{+0.22}_{-0.17} & -4.38^{+0.76}_{-1.30} & -1.96^{+0.07}_{-0.05} \\
    3.25$^a$ & 3.20 & 3.30 & 815 & -7.34^{+0.13}_{-0.13} & -27.20^{+0.20}_{-0.22} & -4.39^{+0.69}_{-0.79} & -1.94^{+0.06}_{-0.05} \\
    3.34$^a$ & 3.30 & 3.40 & 510 & -7.54^{+0.21}_{-0.21} & -27.39^{+0.28}_{-0.37} & -4.67^{+1.38}_{-1.44} & -2.08^{+0.09}_{-0.07} \\
    3.44$^a$ & 3.40 & 3.50 & 155 & -6.76^{+0.19}_{-0.20} & -26.62^{+0.40}_{-0.36} & -3.67^{+0.62}_{-0.83} & -1.24^{+0.28}_{-0.23} \\
    3.88 & 3.70 & 4.10 & 1204 & -7.91^{+0.11}_{-0.11} & -27.26^{+0.13}_{-0.12} & -4.83^{+0.33}_{-0.38} & -2.07^{+0.10}_{-0.09} \\
    4.35 & 4.10 & 4.70 & 603 & -8.32^{+0.29}_{-0.25} & -27.37^{+0.37}_{-0.30} & -4.20^{+0.41}_{-0.50} & -2.21^{+0.15}_{-0.13} \\
    4.92 & 4.70 & 5.50 & 266 & -9.02^{+0.30}_{-0.21} & -27.89^{+0.37}_{-0.26} & -4.58^{+0.71}_{-0.80} & -2.30^{+0.11}_{-0.08} \\
    6.00 & 5.50 & 6.50 & 66 & -10.66^{+0.74}_{-1.11} & -29.21^{+1.08}_{-1.89} & -5.05^{+0.76}_{-1.18} & -2.41^{+0.10}_{-0.08} \\
    \hline
  \end{tabular}\\
  \begin{minipage}{13.0cm}
    \textsuperscript{$a$}{Redshift bin not considered in the joint QLF fit due to systematic errors (open circles in Figure~\ref{fig:mosaic}).}
  \end{minipage}
\end{table*}

\subsection{Double power law fits}
\label{sec:bins}

In each redshift bin, we model the quasar
luminosity function as a double power law 
\citep[e.g.][]{1988MNRAS.235..935B}
\begin{equation}
  \phi(M) =
  \frac{\phi_*}{10^{0.4(\alpha+1)(M-M_*)}+10^{0.4(\beta+1)(M-M_*)}}
  \label{eqn:dpl}
\end{equation}
with four free parameters: (i) the amplitude $\phi_*$, (ii) the break magnitude $M_*$,
(iii)  the bright-end slope $\alpha$, and (iv) the faint-end slope $\beta$.
%%GW: Maybe give the range of the uniform priors.
%% JFH SOmewhere indicate taht this procedure of ubninned LF estimates is standard practice. 
%% GW: Agreed. Maybe add references to Marshall et al. and other seminal papers further up in the text?
By assuming broad, uniform priors, we obtain posterior probability
distributions for these parameters using the Markov Chain Monte Carlo technique \citep[MCMC, e.g.,][]{jaynes}.
The joint posterior probability distribution of the model parameters
is then written as
\begin{multline}
  p(\phi_*, M_*, \alpha, \beta | \{M_i, z_i\}) \propto \\ p(\phi_*, M_*,
  \alpha, \beta)p(\{M_i, z_i\} | \phi_*, M_*, \alpha, \beta),
\end{multline}
where the constant of proportionality is independent of the luminosity
function parameters, and $\{M_i, z_i\}$ denotes the magnitudes and
redshifts of quasars falling in a redshift bin $[z_\mathrm{min},
  z_\mathrm{max})$.  We use a uniform prior distribution $p(\phi_*,
  M_*, \alpha, \beta)$ and assume that the likelihood
\begin{equation}
  \mathcal{L}\equiv p(\{M_i, z_i\} | \phi_*, M_*, \alpha, \beta)
\end{equation}
is given by $\phi(M)$
%% JFH  ``is given by $\phi(M)$ itself''
with suitable normalisation.  The negative
logarithm of the likelihood $S\equiv -2\ln\mathcal{L}$ can then be
written as
\begin{multline}
  S = -2\sum_{i=1}^{N_\mathrm{QSO}}\ln\phi(M_i, z_i)\\+2\int_{M_\mathrm{min}}^{M_\mathrm{max}}\mathrm{d}M
  \int_{z_\mathrm{min}}^{z_\mathrm{max}}\mathrm{d}z\, \phi(M,z) f(M, z)\,\frac{\mathrm{d}V}{\mathrm{d}z},
  \label{eqn:S}
\end{multline}
where the integral over magnitude is on the surveyed range of $M$.  We
use the \texttt{emcee} code \citep{2013PASP..125..306F} for MCMC.

The above likelihood can also be understood as the limit of the
Poisson likelihood in luminosity and redshift bins
\citep{1983ApJ...269...35M, 2001AJ....121...54F}.  We can write the
probability of observing $n_{ij}$ quasars in the $(M_i, z_j)$ bin as
the Poisson distribution
\begin{equation}
  \mathcal{L}=\prod_{i,j}\frac{e^{-\mu_{ij}}\mu_{ij}^{n_{ij}}}{n_{ij}!},
  \label{eqn:lhood}
\end{equation}
where 
\begin{equation}
  \mu_{ij}= \int_{M_i}^{M_{i+1}}\mathrm{d}M\int_{z_j}^{z_{j+1}}\mathrm{d}z\, \phi(M,z) f(M, z)
  \,\frac{\mathrm{d}V}{\mathrm{d}z},
\end{equation}
is the average number of quasars expected in the $(M_i, z_j)$ bin
given the luminosity function $\phi(M,z)$.  In the limit of
infinitesimal bins, $n_{ij}=0$ or $1$, and Equation~(\ref{eqn:lhood})
can be simplified to obtain Equation~(\ref{eqn:S}).

Our estimates for the double power law luminosity function are shown
in Figure~\ref{fig:mosaic} for 25 redshift bins.  We adopt the
posterior median as our fiducial model fit, and the 68.26\%
equal-tailed credible interval as the uncertainty on $\phi$.  The
resultant parameter values are listed in Table~\ref{tab:bins}.
Consistent with previous studies, the double power law model provides
an excellent description of the luminosity function model over almost
the complete range of redshifts spanned by the data.  It is only in
the highest redshift bin ($z=5.5$--$6.5$) that the data seem to favour
a single power law.  In this bin, the resultant posterior distribution
of the break magnitude $M_*$ is bimodal with favoured values at the
faint ($M_*>-18$) and bright end of the data ($M_*<-30$).  While in
the literature $z\sim 6$ quasars have been assumed to lie on the
bright end of the luminosity function
\citep[e.g.,][]{2016ApJ...833..222J}, a comparison with the luminosity
function at lower redshifts ($z<5.5$) suggests that these AGN should
instead be understood to describe the faint-end of a double power law.
As we discuss below, $M_*$ gets progressively brighter with redshift.
Therefore, after inspecting the data at lower redshifts, we use
restricted priors in the $z=5.5$--$6.5$ redshift bin in order to avoid
bimodal distributions.  In this bin, we restrict the bright-end slope
$\alpha$ to values less than $-4$, which is equivalent to forcing
$M_*$ to be at the bright end of the data.  Other parameters continue
to have wide uniform priors.  This also illustrates the importance of
analysing the evolution of the QLF with redshift.

The redshift evolution of the four double power law parameters is
shown in Figure~\ref{fig:evoln} and tabulated in Table~\ref{tab:bins}.
We find interesting evolutionary trends in each of the four
parameters.  The break magnitude $M_*$ evolves by more than eight
magnitudes from redshift $z=0$ to $7$.  The amplitude of the
luminosity function $\phi_*$ evolves moderately from $z=0$ to $z\sim
3$ and then drops by six orders of magnitude to about
$10^{-12}$\,cMpc$^{-3}$mag$^{-1}$ at $z\sim 7$.  The bright end slope
$\alpha$ has significant scatter but still shows a trend towards more
negative values, i.e., steeper luminosity function bright ends, at
high redshifts. Finally, the faint end slope also shows signs of
increasing steepness towards high redshifts, but with a marked
discontinuity at $2.2\le z<3.5$.
%% JFH I think before imposing this prior by hand it might be useful to show the degeneracies that
%% you get at high-z without imposing the prior on the faint-end/bright-end slope. 

Discontinuities and scatter in the QLF parameters over short redshift
intervals in Figure~\ref{fig:evoln} reveal further likely systematic
errors in the survey selection functions.  Quasars at $2.2\le z<3.5$
taken solely from BOSS \citep{2013ApJ...773...14R} follow the redshift
trends in $\phi_*$ and $M_*$, but with significant scatter in $\Delta
z=0.1$ intervals that is much larger than the statistical error.  The
discontinuity at $z=2.2$ indicates a mismatch of BOSS and
SDSS$+$2SLAQ. The most striking feature, however, is the apparent
rapid redshift evolution of the faint-end slope revealed in the BOSS
sample, which is also highlighted in Figure~\ref{fig:mosaic}. Given
the relatively smooth evolution of all QLF parameters at lower and
higher redshifts it is unlikely that the QLF evolution at $2.2\le
z<3.5$ indicated by BOSS is physical.  Rather it reflects the
systematics limit of the large BOSS sample induced by a fixed
selection function that critically depends on the assumed quasar
spectral energy distribution, the IGM parameterization, and the
modeled photometric errors at the targeted magnitude limit of the
single-epoch SDSS imaging
\citep{2011ApJ...728...23W, 2012ApJS..199....3R, 2013ApJ...773...14R}.
Consequently, we exclude all BOSS quasars from further analysis.

The imperfect match between SDSS and 2SLAQ (Figure~\ref{fig:mosaic})
causes low-level systematics, as evidenced by the apparent
discontinuities in $\alpha$ at $z<2.2$ and the jump in $\beta$ at
$z=1.8$ in Figure~\ref{fig:evoln}.  At $z<0.6$ the faint-end slope
shows a sharp increase which we attribute to remaining uncertainties
in the correction for host galaxy light and potentially missed AGN in
extended sources. We exclude $z<0.6$ quasars from further
consideration. Due to different selection function parameters
inter-survey systematics are definitely present at $z>3.5$ as well,
but Poisson errors of the limited samples dominate.

%% Bins considered
%% to be dominated by systematics are marked in Table~\ref{tab:bins}.
%% These bins should be avoided while evaluating the evolution of the
%% luminosity function.  The remaining redshift can 

\subsection{Evolution of the luminosity function}
\label{sec:global}

After excluding the redshift bins that are most obviously affected by
systematic errors, the remaining redshift bins (filled symbols in
Figure~\ref{fig:evoln}) are consistent with a smooth redshift
evolution of the luminosity function parameters that may be described
by parametric models \citep{1976A&A....53...15M, 1983ApJ...269..352S,
  1988ApJ...325...92K, 1988MNRAS.235..935B, 1993ApJ...406L..43H,
  1994ApJ...421..412W, 1995AJ....110...68S, 1995AJ....110.2553K,
  1995ApJ...438..623P, 2000MNRAS.317.1014B, 2001AJ....121...54F,
  2006AJ....131.2766R, 2007A&A...472..443B, 2009MNRAS.399.1755C,
  2013ApJ...773...14R, 2013A&A...551A..29P}.  Such descriptions have
also been developed in the literature for the X-ray
\citep[e.g.,][]{2015MNRAS.451.1892A} and so-called bolometric
luminosity functions \citep[e.g.,][]{2007ApJ...654..731H}.  Such
``global'' models of the luminosity function evolution are useful as
they give a continuous description of the luminosity function.  This
allows one to reduce the bias introduced by binning the data in
arbitrary redshift bins.  By potentially allowing for extrapolations
beyond the redshift range spanned by the data, such models are
valuable for understanding of the physics behind the luminosity
function.  Ideally, one would want to use physically meaningful
parameters that govern the formation and evolution of the AGN
population.  Unfortunately, such physical parameterisation is yet to
be developed.  We therefore set up an empirical parameterisation to
describe the evolution of the four parameters of the double power law
model in Equation~(\ref{eqn:dpl}) as 
\begin{align}
  &\phi_*(z) = F_0(\{c_{0,j}\}, z)\nonumber\\
  &M_*(z) = F_1(\{c_{1,j}\}, z)\nonumber\\
  &\alpha(z) = F_2(\{c_{2,j}\}, z)\nonumber\\
  &\beta(z) = F_3(\{c_{3,j}\}, z),
  \label{eqn:global}
\end{align}
where the $\{c_{n,j}\}$ are the new model parameters, and the $\{F_j\}$
are functions that vary smoothly with redshift $z$.  The joint
posterior probability distribution of these parameters can be now
written as
\begin{equation}
  p(\{c_{n,j}\} | \{M_i, z_i\}) \propto p(\{c_{n,j}\})p(\{M_i, z_i\} | \{c_{n,j}\}),
\end{equation}
where the likelihood 
\begin{equation}
  \mathcal{L}\equiv p(\{M_i, z_i\} | \{c_{n,j}\}),
\end{equation}
is now given by $\phi(M,z)$ with suitable normalisation.  Note that
$\phi(M,z)$ is given by Equation~(\ref{eqn:dpl}), but now the four
parameters in that equation are redshift-dependent.  The negative
logarithm of the likelihood $S\equiv -2\ln\mathcal{L}$ is a
straightforward generalisation of Equation~(\ref{eqn:S}).  We consider
models in which the evolution of the four double power law parameters
is modelled independently as in Equations~(\ref{eqn:global}), an
approach sometimes termed as ``flexible double power law''
\citep{2015MNRAS.451.1892A}.  We present three such models in this
paper.  These are shown in Figure~\ref{fig:evoln_global}.  The three
models differ in the way they describe the evolution of the faint-end
slope $\beta$ and in the selection of the AGN data.  The models are
defined as follows.

\begin{itemize}

\item
  In Model 1 we assume that the functions $F_0$, $F_1$ and $F_2$ from
  Equation~\eqref{eqn:global} are Chebyshev polynomials in
  $\left(1+z\right)$, written as
  \begin{equation}
    F_i\left(1+z\right)=\sum_{j=0}^{n_i}c_{i,j}T_j\left(1+z\right)
    \label{eqn:cbs}
  \end{equation}
  for $i\in\{0,1,2\}$, where $c_{i,j}$ are the parameters from
  Equation~\eqref{eqn:global} and $T_j\left(1+z\right)$ are Chebyshev
  polynomials of the first kind. We try successively higher orders of
  Chebyshev polynomials in order to arrive at a good fit with the
  data.  As we discuss below, we find that $\phi_*$, $M_*$ and
  $\alpha$ prefer quadratic, cubic, and linear evolutions in
  $\left(1+z\right)$, respectively.  For the faint-end slope $\beta$
  we adopt a double power law to account for a possible break at
  $z\sim 3$ that is currently not covered with credible data.  Thus we
  write
  \begin{equation}
    F_3\left(1+z\right)=c_{3,0}+\frac{c_{3,1}}{10^{c_{3,3}\zeta}+10^{c_{3,4}\zeta}},
    \label{eqn:beta}
  \end{equation}
  where
  \begin{equation}
    \zeta = \log_{10}\left(\frac{1+z}{1+c_{3,2}}\right),
  \end{equation}
  thus resulting in a five-parameter model with parameters $c_{3,i}$.
  The parameters $c_{3,3}$ and $c_{3,4}$ thus determine the low and
  high redshift slopes of this evolution, with a break at redshift
  $c_{3,2}$.  This is similar to the model of
  \citet{2007ApJ...654..731H}, who also favoured a broken power law
  model for the evolution of the faint-end slope of the bolometric
  luminosity function of quasars.  Model 1 thus has 14 parameters.
  Excluding those deemed to be dominated by systematic errors, as
  discussed in the previous section, all of the remaining AGN from
  Table~\ref{tab:samples} are included while fitting this model.
  \textbf{GW agrees with JFH that this is too short and
    cryptic. Table~\ref{tab:samples} refers to all samples, but we
    exclude some samples, some redshift ranges, and some magnitude
    ranges. I've put a clarification in the first sentence of the
    section, but we might elaborate more.}
  %% JFH where was this discussed?? Nowhere have I seen explicitly stated in the
  %% text which luminosity/redshift ranges you are excluding from the analysis in performing
  %% these fits because you think they are systematics dominated. This just needs to be clearly stated.
  %% Add this to the end of Section 3.2 after you discuss the systematics. State what is excluded and why.
  %% It is perfectly fine if you do this by eye, just state that. 

\item Model 2 parameterises the luminosity function evolution in the
  same way as Model 1, so that the faint-end slope evolution is
  described by a double power law while the evolution of the other
  parameters $\phi_*$, $M_*$ and $\alpha$ is described by,
  respectively, quadratic, cubic, and linear polynomials in $(1+z)$.
  The total number of parameter is 14.  However, while fitting this
  model, we drop samples 7, 19, and 20 from the analysis.  The
  selection function for these samples are not yet well-characterised.
  Removing them allows us to understand the effect this has on the
  favoured evolution model.
  %% JFH Do not put the discussion here. Add a separate discussion where you address the samples you leave out. Referencing them by
  %% number here is also a bit annoying to me. Just state the samples with a name of the survey. 

\item In Model 3, we again exclude samples 7, 19, and 20 from the
  analysis.  We also continue to describe the evolution of $\phi_*$,
  $M_*$ and $\alpha$ by quadratic, cubic, and linear polynomials in
  $(1+z)$, respectively.  But in this model, the evolution of the
  faint-end slope $\beta$ is also assumed to be linear in $(1+z)$.
  This model thus has just 11 parameters.
\end{itemize}

%% JFH I'm lost. According to this discussion Model 1 and 2 treat the faint end slope with a broke power law, but your figures
%% suggests that only the gray band Model 1 is using this double power law? What you are calling Model 2 appears to not be a double
%% power law? Am I getting fooled by the shape of that green curve?

\begin{figure*}
  \begin{center}
    \includegraphics[width=0.7\textwidth]{evolution_global.pdf}
    % summary_fromFile.py 
  \end{center}
  \caption{Luminosity function parameter evolution in the global
    models.  The symbols show the posterior median values of
    parameters with one-sigma (68.26\%) uncertainties in redshift bins
    from Figure~\ref{fig:evoln}.  Redshift bins deemed to be affected
    by systematics and removed from the global analysis are shown in
    grey.  In each panel, the solid curves and shaded regions show the
    three derived global models with one-sigma uncertainties.  Model~1
    provides a better fit, but requires a rapid change in the
    faint-end slope at $z\sim 3.5$.}
  \label{fig:evoln_global}
\end{figure*}

\begin{figure*}
  \begin{center}
    \includegraphics[width=\textwidth,keepaspectratio]{mosaic_small_global.pdf}
    % mosaic_selected.py and drawlf_selected.py 
  \end{center}
  \caption{Luminosity function estimates from $z=0.6$ to $6.5$.
    Similar to Figure~\ref{fig:mosaic}, the symbols show our inferred
    binned luminosity functions.  In each redshift bin, yellow curves
    show our fiducial double power law luminosity function model in
    that redshift bin.  Other curves show the three global evolution
    models.  Shaded regions show the one-sigma (68.26\%)
    uncertainties.}
  \label{fig:mosaic_global}
\end{figure*}

\begin{figure}
  \begin{center}
    \includegraphics[width=\columnwidth,keepaspectratio]{rhoqso_withGlobal.pdf}
    % rhoqso.py -- draw_withGlobal_multiple()
  \end{center}
  \caption{AGN number density evolution in the global model, when the
    luminosity function is integrated (from top to bottom) down to
    $M_\mathrm{1450}=-18, -21, -24,$ and $-27$.  Filled black circles
    show the estimates from the our fiducual double power law
    luminosity function models in various redshift bins, with vertical
    error bars denoting one-sigma (68.26\%) uncertainties.  Open
    circles show the same in redshifts bins affected by systematics.
    Curves show the estimated density from the global models, with the
    accompanying shaded regions denoting the one-sigma uncertainty. }
  \label{fig:rhoqso}
\end{figure}

Figure~\ref{fig:mosaic_global} shows the three global models in
comparison with the binned fits from the previous section.  The shaded
regions show the one-sigma (68.26\%) uncertainty.  The symbols show
the luminosity function binned in luminosity and redshift, and the
yellow shaded regions show the posterior distribution of the double
power law luminosity functions in various redshift bins, as in
Figure~\ref{fig:mosaic}.  Bins containing data with large systematic
error are excluded from Figure~\ref{fig:mosaic_global}.  All three
global models are in excellent agreement with the binned models,
although Model~1 performs better.

Figure~\ref{fig:evoln_global} shows the parameter evolution in the
global models, by comparing it with the results from the fits in
individual redshift bins shown in Figure~\ref{fig:evoln}.  All three
models capture the steepening of the faint end of the luminosity
function towards higher redshifts.  The derived form of Model~1 is
shown by the black curves in Figure~\ref{fig:evoln_global}.  The
accompanying grey shaded area depicts the one-sigma (68.26\%)
uncertainty.  The model is in excellent agreement with the results of
the fits in redshift bins discussed in the previous section.
The deviation of the BOSS quasars at $z=2$--$4$ from the smooth
evolution is again strikingly apparent, as is the deviation of the
SDSS and 2SLAQ quasars at $z<0.6$.  This is an indirect justification
for the data selection discussion previously in
Section~\ref{sec:bins}.  Unfortunately, this model suffers from a
remarkably sharp break in the evolution of the faint-end slope $\beta$
at about $z\sim 3.5$.  As seen in Figure~\ref{fig:evoln_global}, the
data seem to require this break, although it seems unlikely that such
a sharp break at this redshift would be physical.  Model 1 thus serves
to emphasize the necessity of better quality data at these redshifts.
Models 2 and 3 are shown in Figure~\ref{fig:evoln_global} by the green
and orange curves, respectively.  The corresponding parameter values
are tabulated in Table~\ref{tab:global}.

The evolution of the comoving number density of quasars is shown in
Figure~\ref{fig:rhoqso} when the luminosity function is integrated
down to different limits.  Similar to Figure~\ref{fig:evoln_global},
symbols show the posterior median values with one-sigma uncertainties
from the double power law fits to the data in redshift bins.
%% JFH I'm bothered by the fact that the symbols and error bars here arise from the parametric fits. In my opinion, you
%% should be able to plot the data on here with Poisson errors in exactly the same way that you do when you compute the binned
%% luminosity function data points? These symbols with errors should be the data to me, and not the result of your fitting procedure. 
Solid curves and shaded regions show the global models.  This number
density evolution again highlights the systematic error in data at
$z\sim 3$.  The number density of AGN down to the limit of the deepest
spectroscopic surveys ($M_{1450}<-21$) is about $10^{-5}$ cMpc$^{-3}$
at its peak.  This density rapidly increases with redshift at low
redshifts and then drops with redshift gradually at high redshifts.
Figure~\ref{fig:rhoqso} also shows the familiar downsizing feature in
which the number density of faint AGN peaks at lower redshifts than
that of the bright AGN.  While the number density of AGN with
$M_*<-27$ peaks at $z\sim 2.5$, the number density of AGN with
$M_*<-24$ peaks at $z\sim 2$.  At the faintest luminosity at which
spectroscopic data exist, $M_*<-21$, the number density of qsos peaks
at a slightly lower redshift.  However, the difference between our
three models is dramatically evident in Figure~\ref{fig:rhoqso}.
Model~1 prefers a decrease in the number density of faint quasars at
$z\sim 3$ followed by an increase at higher redshift.  This is caused
by the rapid steepening of the faint-end slope in this model at this
redshift (Figure~\ref{fig:evoln_global}).  Figure~\ref{fig:rhoqso}
reveals another property of these models: when extrapolated, the AGN
number density diverges in all three models at high redshifts.
%% JFH I think this is just an artifiact of your fitting procedure,
%% i.e. that these polynomial coefficients are unconstrained at really
%% high-z. Do these fits include the banados and MOrtlock quasars? In
%% that case the divergence should not occur, at least for the
%% brightest bin.  Furthermore, if you would make this plot with the
%% data instead of of integrating fits, you would be able to plot
%% those objects on here as well.
This is result of the steep faint-end slope at high
redshifts combined with the rapid brightening of the break luminosity.
While no data exist at redshift $z>7.5$, this divergent behaviour is
shared by previous models in the literature
\citep{2007ApJ...654..731H}.  Figure~\ref{fig:rhoqso} also shows that
although Models 2 and 3 exhibit regular behaviour in the evolution of
the AGN number density at $z\sim 3$, they do not fit the $z\sim
4$--$5$ data as well as Model 1.

\begin{table}
  \caption{Derived luminosity function evolution models.  These
    parameters are defined in Equations~(\ref{eqn:cbs}) and
    (\ref{eqn:beta}).  See Section~\ref{sec:global} for further
    details and the redshift range of validity of these models.
    Errors indicate one-sigma (68.26\%) uncertainties.  Model 2 is our
    preferred model.}
  \label{tab:global}
  \begin{tabular}{p{1.0cm}eee}
    \hline 
    Param. &
    \multicolumn{1}{c}{Model 1} &
    \multicolumn{1}{c}{Model 2} &
    \multicolumn{1}{c}{Model 3} \\
    \hline
    $c_{0,0}$ & -7.559^{+0.131}_{-0.139} & -7.084^{+0.136}_{-0.142} & -6.842^{+0.077}_{-0.076} \\     
    $c_{0,1}$ & 1.013^{+0.079}_{-0.072} & 0.753^{+0.080}_{-0.073} & 0.590^{+0.039}_{-0.041} \\        
    $c_{0,2}$ & -0.113^{+0.005}_{-0.005} & -0.096^{+0.004}_{-0.005} & -0.083^{+0.003}_{-0.003} \\
    \\
    $c_{1,0}$ & -17.006^{+0.230}_{-0.243} & -15.423^{+0.263}_{-0.261} & -15.140^{+0.144}_{-0.136} \\  
    $c_{1,1}$ & -5.548^{+0.156}_{-0.143} & -6.725^{+0.156}_{-0.171} & -6.910^{+0.091}_{-0.093} \\     
    $c_{1,2}$ & 0.588^{+0.016}_{-0.019} & 0.737^{+0.018}_{-0.015} & 0.750^{+0.012}_{-0.013} \\        
    $c_{1,3}$ & -0.023^{+0.001}_{-0.001} & -0.029^{+0.001}_{-0.001} & -0.029^{+0.001}_{-0.001} \\
    \\
    $c_{2,0}$ & -3.246^{+0.121}_{-0.123} & -2.973^{+0.117}_{-0.133} & -2.950^{+0.105}_{-0.097} \\     
    $c_{2,1}$ & -0.250^{+0.048}_{-0.050} & -0.347^{+0.050}_{-0.050} & -0.363^{+0.040}_{-0.043} \\
    \\
    $c_{3,0}$ & -2.350^{+0.051}_{-0.060} & -2.545^{+0.123}_{-0.290} & -1.424^{+0.031}_{-0.031} \\     
    $c_{3,1}$ & 0.647^{+0.072}_{-0.060} & 1.581^{+0.520}_{-0.264} & -0.111^{+0.010}_{-0.010} \\       
    $c_{3,2}$ & 3.857^{+0.092}_{-0.073} & 2.102^{+0.450}_{-0.283} & \multicolumn{1}{c}{---} \\        
    $c_{3,3}$ & 27.534^{+41.364}_{-9.405} & 1.965^{+0.461}_{-0.464} & \multicolumn{1}{c}{---} \\      
    $c_{3,4}$ & -0.002^{+0.074}_{-0.082} & -0.641^{+0.169}_{-0.154} & \multicolumn{1}{c}{---} \\      
    \hline 
  \end{tabular}
\end{table}

\begin{figure*}
  \begin{center}
    \begin{tabular}{cc}
      \includegraphics[width=0.47\textwidth]{emissivity_18.pdf} & 
      \includegraphics[width=0.47\textwidth]{emissivity_21.pdf} \\
      % gammapi.py -- draw_emissivity_18()
      % gammapi.py -- draw_emissivity_21()
    \end{tabular}
  \end{center}
  \caption{The 912~\AA\ emissivity of AGN assuming 100\% escape
    fraction down to a limiting magnitude for a luminosity function
    integration limit of $M_{1450}=-18$ (left panel) and
    $M_{1450}=-21$ (right panel).  Black filled circles with error
    bars in both panels show the emissivity determinations in redshift
    bin deemed to have low systematic errors.  Open circles show
    emissivities for redshift bins that we remove from analysis due to
    high systematic errors.  Solid red curves in both panels show the
    derived posterior median emissivity evolution model, with the
    shaded area showing the one-sigma (68.26\%) uncertainty.  Also
    shown for comparison in both panels are models by \citet[pentagon
      symbol]{2009A&A...507..781S}, \citet[star]{2012ApJ...755..169M},
    \citet[square]{2015AA...578A..83G},
    \citet[triangle]{2018PASJ...70S..34A},
    \citet[circle]{2018AJ....155..131M},
    \citet[diamond]{2018MNRAS.474.2904P}, \citet[open
      rectangle]{2017ApJ...847L..15O}, \citet[dotted green
      curve]{2015ApJ...813L...8M}, \citet[dashed
      brown]{2012ApJ...746..125H}, \citet[dashed
      grey]{2017MNRAS.466.1160M}, \citet[dashed
      blue]{2013A&A...551A..29P}, \citet[solid
      cyan]{2016A&A...587A..41P}, \citet[solid
      blue]{2017A&A...608A..64C}, and
    \citet[brown]{2007A&A...472..443B}.}
  \label{fig:e912_2}
\end{figure*}

\section{The AGN contribution to reionization}
\label{sec:reion}

We now discuss the contribution of AGN to the hydrogen and helium
reionization in our luminosity function model.  We first derive the
redshift evolution of the 912\,\AA\ emissivity of AGN, and then use
this to estimate the contribution of AGN to the \ion{H}{i}
photoionization rate between $z=0$ and $z=7$, as well as the redshift
evolution of the average \ion{He}{iii} fraction in the IGM for a
quasar-driven \ion{He}{ii} reionization.

\subsection{Quasar emissivity at the hydrogen Lyman limit}
\label{sec:e912}

For simplicity we assume that all quasars have a universal UV SED,
parameterised as a power law $f_\nu\propto\nu^{\alpha_\nu}$
with a break at 912\,\AA,
\begin{equation}
  f_\nu\propto\begin{cases}
  \nu^{-0.61} & \text{if}~\lambda\ge 912~\text{\AA},\\
  \nu^{-1.70} & \text{if}~600~\text{\AA}<\lambda<912~\text{\AA}                
  \end{cases}
  \label{eqn:sed}
\end{equation}
as derived from a stacked spectrum of 53 luminous ($M_{1450}\simeq
-27$) $z\simeq 2.4$ quasars \citep{2015MNRAS.449.4204L}, and
consistent with recent composite spectra of low-$z$ quasars with a
wide range in luminosity \citep{2012ApJ...752..162S,
  2014ApJ...794...75S}.  The extreme UV SED of faint ($M_{1100}\approx
-22$) AGN may be significantly harder
\citep[$\alpha_\nu=-0.56$,][]{2004ApJ...615..135S}, but the exact
value of the slope critically depends on the total rest-frame
wavelength coverage, the adopted continuum windows, the correction for
IGM line blanketing, and the ability to distinguish
low-equivalent-width emission lines from the underlying continuum
\citep{2014ApJ...794...75S,2015MNRAS.449.4204L,2016ApJ...817...56T}.
For the computation of the hydrogen Lyman limit emissivity the choice
of the spectral index at $\lambda<912$\,\AA\ is inconsequential,
whereas for the calculation of the \ion{H}{i} photoionization rate in
the IGM and the \ion{He}{ii} reionization history other sources of
uncertainty dominate (see below). We note that all composite AGN
spectra are consistent with an AGN Lyman limit escape fraction of
unity (see also \citealt{2018A&A...613A..44G} for a recent sample of
faint $z\sim 4$ AGN).  We therefore adopt an escape fraction of unity
for Lyman continuum photons.

With our adopted SED the specific comoving volume emissivity of
quasars at 912\,\AA\ can be written as
\begin{eqnarray}\nonumber
\epsilon_{912}\left(z\right)&=&\int\limits_{-\infty}^{M_{1450}^\mathrm{lim}}\phi\left(M_{1450},z\right)10^{-0.4\left(M_{1450}-51.60\right)}\,\mathrm{d}M_{1450}\\
& &\times\left(\frac{912}{1450}\right)^{0.61}\quad,
\label{eqn:epsilon}
\end{eqnarray}
which depends on the adopted magnitude limit $M_{1450}^\mathrm{lim}$
and on the faint-end slope of the QLF if $M_{1450}^\mathrm{lim}\gg
M_*$.  The black circles in Figure~\ref{fig:e912_2} show the comoving
912\,\AA\ emissivity obtained from the individual QLF fits
(Table~\ref{tab:bins}) for $M_{1450}<-18$ (left panel) and
$M_{1450}<-21$ (right panel), respectively. Redshift bins that were
removed from the analysis due to large systematic errors in the QLF
parameters are shown as open circles.  \textbf{GW: Add a sentence on
  error bars.}  Our derived emissivity values at 912\,\AA\ and at
1450\,\AA\ are listed in Table~\ref{tab:emissivity_bins} for
reference.  The emissivity peaks between $z=2$ and 3 at
$\epsilon_{912}\simeq 10^{25}\,\mathrm{erg\, s^{-1}\, Hz^{-1}\,
  cMpc^{-3}}$ depending on the magnitude limit, and decreases rapidly
towards lower and higher redshifts.  Systematic errors in the
faint-end slope derived from BOSS data are more pronounced for
$M_{1450}<-18$ due to extrapolation of the QLF.

To account for redshift effects in the calculation of the \ion{H}{i} photoionization rate,
a continuous function $\epsilon_{912}\left(z\right)$ is required. As our parametric QLF models
from the previous section suffer from non-monotonic or divergent AGN number densities,
we do not use them to derive the corresponding $\epsilon_{912}\left(z\right)$, but
instead fit the individual emissivity values derived from credible data
(filled black circles in Figure~\ref{fig:e912_2}) with a 
five-parameter functional form used by \citet{2012ApJ...746..125H}
\begin{equation}
  \epsilon_{912}\left(z\right)=\epsilon_0(1+z)^a\frac{\exp(-bz)}{\exp(cz)+d}\quad,
  \label{eqn:e912fit}
\end{equation}
assuming a Gaussian likelihood for the emissivity values.  The
resultant curves and the corresponding one-sigma uncertainties are
shown in Figure~\ref{fig:e912_2}. For $M_{1450}<-18$ we obtain
\begin{multline}
  \epsilon_{912}\left(z\right)=\left(10^{24.49}\mathrm{erg\, s^{-1}\, Hz^{-1}\, cMpc^{-3}}\right)\left(1+z\right)^{7.57}\\\times\frac{\exp(-1.78z)}{\exp(1.01z)+29.20}\quad,
  \label{eqn:e912_18}
\end{multline}
 and for $M_{1450}<-21$ we get
\begin{multline}
  \epsilon_{912}\left(z\right)=\left(10^{24.11}\mathrm{erg\, s^{-1}\, Hz^{-1}\, cMpc^{-3}}\right)\left(1+z\right)^{6.64}\\\times\frac{\exp(-0.65z)}{\exp(1.82z)+20.45}\quad.
  \label{eqn:e912_21}
\end{multline}
Table~\ref{tab:gamma2} provides the derived emissivities at
912\,\AA\ and 1450\,\AA\ together with their derived errors at
$0<z<15$, extrapolating at $z<0.6$ and at $z>6.5$.

\subsection{Comparison to the literature}

Before proceeding to derive estimates of the quasar contribution to
the IGM \ion{H}{i} photoionization rate from the fitted
912\,\AA\ emissivity, it is instructive to compare our results to
recent estimates from the literature\footnote{We do not compare to
  \citet{2017MNRAS.466.1160M} due to an error in their analysis.
  Integration of the 912\,\AA\ emissivity to
  $M_{1450}^\mathrm{lim}=-19$ with their double power law QLF
  parameterization as a function of redshift yields values 13--75 per
  cent higher than those implied by their Equation 9.  This also
  explains the striking discrepancy to literature values at $z<3$
  (their Figure 4).}.  The various blue symbols in
Figure~\ref{fig:e912_2} show emissivity values that we have computed
from other recent QLF determinations in narrow redshift ranges.  The
various curves show emissivities derived from parametric QLF model
fits over larger redshift ranges, or fits to
$\epsilon_{912}\left(z\right)$. All QLFs using different magnitude
systems have been converted to $M_{1450}$ with the
\citet{2015MNRAS.449.4204L} SED, and all QLFs have been adjusted to
our cosmology.  For the $z\simeq 0$ QLF reported by
\citet{2009A&A...507..781S} we convert their $B_J$ magnitudes in the
Vega system to our AB magnitudes as $M_{1450,\mathrm{AB}}=M_{B_J,
  \mathrm{Vega}}+0.59$.  Lyman limit emissivities have been
consistently derived using Equation~\eqref{eqn:sed} and for our
adopted magnitude limits whenever possible.  We note that due to
strong covariance in the QLF parameters it is not straightforward to
compute statistical errors of $\epsilon$ from given QLF fits, and we
refrained from using the incorrect procedures applied by
\citet{2015MNRAS.451L..30K}.  \textbf{GW: Emphasize again that our
  error bars are correct and superior to Khaire's!}

The brown dashed curve in Figure~\ref{fig:e912_2} shows the AGN
912\,\AA\ emissivity model adopted by \citet{2012ApJ...746..125H},
which is based on the bolometric luminosity function from
\citet{2007ApJ...654..731H}.  The bolometric emissivity derived by
\citet{2007ApJ...654..731H} converges for luminosities $L>0$ due to
their shallow faint-end QLF slope, which should yield a higher
912\,\AA\ emissivity for both our adopted magnitude limits. We
attribute much of the discrepancy to the conversion from bolometric to
912\,\AA\ emissivity assumed by \citet{2007ApJ...654..731H}.
Figure~\ref{fig:e912_2} also shows the emissivity curve from
\citet{2015ApJ...813L...8M} that was inspired by recent QLF fits
including the highly debated \citet{2015AA...578A..83G} results.  The
emissivity adopted by \citet{2015ApJ...813L...8M} exceeds our fits by
more than a factor of two at $z\la 1$ and $z\ga 4$. Moreover, we
stress that our emissivities have been derived for fixed magnitude
limits $M_{1450}^\mathrm{lim}$ at all redshifts
(Equation~\eqref{eqn:epsilon}), whereas other authors have adopted
magnitude limits
$M_{1450}^\mathrm{lim}\left(z\right)=M_*\left(z\right)+5$ that vary
with break magnitude and redshift
\citep{2015AA...578A..83G,2015ApJ...813L...8M,2015MNRAS.451L..30K,2018arXiv180104931P}.
We deem the latter convention to be physically unfounded, because (i)
the customary QLF double power-law parameterization lacks a deeper
physical meaning, (ii) $M_*$ decreases by more than five magnitudes
with redshift (Figure~\ref{fig:evoln}, see also
\citealt{2013ApJ...768..105M} and \citealt{2016ApJ...829...33Y}), and
(iii) the various QLF fits at the same redshift are highly discordant
(Appendix~\ref{sec:qlfliterature}).  Inhomogeneous magnitude limits
lead to artificial scatter in the derived emissivities if the
faint-end slope of the QLF is not sufficiently shallow. Our fixed
magnitude limits bracket a reasonable range of QLF extrapolations
beyond the range covered by current data, whereas a varying limit
$M_*+5$ includes feeble $M_{1450}\simeq -18$ AGN at $z<0.6$, but
excludes verified $M_{1450}\simeq -24$ quasars at $z\simeq 6$
(Figure~\ref{fig:evoln}).

\citet{2013A&A...551A..29P,2016A&A...587A..41P} presented two
variability-selected quasar samples that are not included in our
analysis, and therefore provide a valuable cross-check.  In
Figure~\ref{fig:e912_2} we show the emissivities computed from their
parametric model fits to binned QLFs at
$0.68<z<4$. \citet{2013A&A...551A..29P} fitted pure luminosity
evolution models to binned QLFs from their data set and the one by
\citet{2009MNRAS.399.1755C}, but with a discontinuity in the QLF
slopes at $z=2.2$ that cause artificial discontinuities in the QSO
number density and the emissivity. The good agreement with our
inferences at $z<2.2$ is partially due to sample
overlap. \citet{2016A&A...587A..41P} presented an independent sample
of 13876 variability-selected quasars. They fitted their binned QLFs
with a pure luminosity evolution model at $z<2.2$ and a luminosity and
density evolution model at higher redshifts, imposing continuity at
$z=2.2$. The emissivity computed from their QLF is in good agreement
with our results at $z<2.2$, but is systematically lower at higher
redshifts. \citet{2017A&A...608A..64C} corrected an apparent error in
the bandpass correction applied by \citet{2016A&A...587A..41P} that
results in a higher QLF at $z>3$.  However, his higher inferred
$\phi_*$ at $z=0$ causes a 30--60 per cent higher emissivity at
$z<2.2$ compared to our inferences. Since neither
\citet{2013A&A...551A..29P,2016A&A...587A..41P} nor
\citet{2017A&A...608A..64C} fitted the QLF in narrow redshift bins
from unbinned quasar data, it remains unclear whether there is a
systematic difference between colour-selected and variability-selected
samples at $z<2.2$. As variability-selected samples to not probe the
faint end of the QLF at $z>3$, inferences of the high-redshift quasar
number density and emissivity are highly uncertain at present.

We also calculate the 912\,\AA\ emissivities from various
determinations of the UV QLF in narrow redshift ranges
\citep{2009A&A...507..781S,2012ApJ...755..169M,2015AA...578A..83G,2017ApJ...847L..15O,2018PASJ...70S..34A,2018AJ....155..131M,2018MNRAS.474.2904P}.
All these QLF determinations are not fully independent from ours due
to partial sample overlap, mostly from SDSS at the bright end. We use
both QLF determinations from \citet{2017ApJ...847L..15O} with and
without including a faint X-ray-selected $z\sim 6$ AGN candidate
\citep{2018MNRAS.474.2904P}, and indicate the emissivities with a box
in Figure~\ref{fig:e912_2}.  For $M_{1450}<-21$ our
$\epsilon_{912}\left(z\right)$ parameterization and the individual
values are generally consistent with those computed from other QLFs in
narrow redshift ranges, apart from the results by
\citet{2015AA...578A..83G} that have been disputed by several studies
\citep[][see Appendix~\ref{sec:conv} for further
  discussion]{2015MNRAS.453.1946G,2016MNRAS.463..348V,2017MNRAS.465.1915R,2018AJ....155..131M,2018MNRAS.474.2904P}.
As detailed in Section~\ref{sect:samplesel}, the
\citet{2012ApJ...755..169M} QLF at $z\sim 4$ -- and hence the derived
emissivity -- is underestimated due to systematic error in their
photometric redshifts, while their $z\sim 3.2$ results likely suffer
from systematic uncertainty in the SDSS selection function
\citep{2011ApJ...728...23W,2012ApJS..199....3R,2013ApJ...773...14R}.

Integration to $M_{1450}=-18$ results in larger discrepancies due to
extrapolation of the QLF with an uncertain faint-end slope.
\citet{2018PASJ...70S..34A} obtained a very flat faint-end slope
$\beta=-1.30\pm 0.05$ that is inconsistent with our determinations at
all redshifts. Since they selected only point sources, they might be
missing a significant fraction of AGN at $M_{1450}>-23.5$.  In
addition, only $4.6$ per cent of their AGN have spectroscopic
redshifts, such that their correction for contamination is very rough.
The large difference in the inferred $M_{1450}<-18$ emissivity at
$z\simeq 5$ with respect to \citet{2018AJ....155..131M} is due to
their shallower faint-end slope $\beta=-1.97\pm 0.09$ compared to our
result at this redshift ($\beta=-2.30^{+0.11}_{-0.08}$). We attribute
the difference in the faint-end slope to the fixed bright-end slope
$\alpha=-4$ in \citet{2018AJ....155..131M} that is larger than our
measurements at $z>3.5$ (Figure~\ref{fig:evoln}).  Similarly,
\citet{2017ApJ...847L..15O} fixed the bright-end slope to
$\alpha=-2.8$ which had been determined by \citet{2016ApJ...833..222J}
from a single power-law fit to the bright end of the QLF.  Our
bright-end slopes are inconsistent with such high values at all
redshifts, indicating that single power-law fits to a limited range in
magnitude yield only approximate estimates of $\alpha$, biasing the
other derived QLF parameters, quasar number densities and
emissivities.  Full double-power law fits to unbinned data over a wide
magnitude range are required.

\begin{figure*}
  \begin{center}
    % rtg2.draw_g_paper() 
    \includegraphics[scale=0.65]{g.pdf}
  \end{center}
  \caption{AGN contribution to the hydrogen photoionisation rate,
    assuming unit escape fraction, when the AGN luminosity function is
    integrated down to $M_{1450}=-21$ (blue curve and shaded region)
    and $M_{1450}=-18$ (red curve and shaded region).  The shaded
    regions show the one-sigma (68.26\%) uncertainty.  Also shown are
    the photoionization rate measurements by \citet[filled
      circles]{2013MNRAS.436.1023B}, \citet[inverted
      triangles]{2011MNRAS.412.2543C}, and
    \citet[pentagons]{2017MNRAS.467.3172G}, and models of
    \citet[dotted brown curve]{2012ApJ...746..125H}, the QSO
    contribution in this model (dashed grey), \citet[dashed
      brown]{2015ApJ...813L...8M}, the QSO contribution from the model
    of \citet[dashed orange]{2015MNRAS.451L..30K}, \citet[dotted
      grey]{2017ApJ...837..106O}, and \citet[dashed
      grey]{2018arXiv180104931P}.}
  \label{fig:gammapi}
\end{figure*}

\subsection{Hydrogen photoionization rate}
\label{sec:gammahi}

Our calculation of the AGN contribution to the UV background follows
previous work on UV background synthesis models
\citep[e.g.][]{1996ApJ...461...20H,2012ApJ...746..125H}.  The main
quantity of interest is the \ion{H}{i} photoionization rate of the UV
background
\begin{equation}
  \Gamma_\ion{H}{i}\left(z\right)=\int_{\nu_{912}}^\infty\mathrm{d}\nu
  \frac{4\pi J_\nu\left(\nu,z\right)}{h\nu} \sigma_\ion{H}{i}\left(\nu\right)\quad,
\end{equation}
where $h$ is Planck's constant, $\sigma_\ion{H}{i}(\nu)$ is the \ion{H}{i} photoionization cross-section \textbf{Citation}, and
\begin{multline}
  J_\nu(\nu, z)=\frac{c}{4\pi}\int_{z}^\infty\mathrm{d}z^\prime\frac{\left(1+z\right)^3}{H\left(z^\prime\right)\left(1+z^\prime\right)}\epsilon_\nu\left(\nu_\mathrm{em},z^\prime\right)\\
  \times\exp{\left[-\tau_\mathrm{eff}\left(\nu, z, z^\prime\right)\right]}
  \label{eqn:flux}
\end{multline}
is the angle- and space-averaged specific intensity of the UV
background. In the above equation
$H\left(z^\prime\right)=H_0\sqrt{\Omega_\mathrm{m}\left(1+z^\prime\right)^3+\Omega_\Lambda}$
is the Hubble parameter, and
$\epsilon_\nu\left(\nu_\mathrm{em},z^\prime\right)$ is the comoving
emissivity of all \ion{H}{i} Lyman continuum sources at redshift
$z^\prime>z$ and emitted frequency
$\nu_\mathrm{em}=\nu\left(1+z^\prime\right)/\left(1+z\right)>\nu_{912}$.
In practice, we adopt an upper limit $z^\prime_\mathrm{max}=15$ when
integrating Equation~\eqref{eqn:flux}.  Assuming that quasars are the
only ionizing sources and considering Equation~\eqref{eqn:sed} we have
\begin{equation}
  \epsilon_\nu\left(\nu_\mathrm{em},z^\prime\right) = \epsilon_{912}\left(z^\prime\right)\left(\frac{\nu_\mathrm{em}}{\nu_{912}}\right)^{-1.70}\quad,
  \label{eqn:epsilon_freq}
\end{equation}
with $\epsilon_{912}\left(z^\prime\right)$ being represented by Equations~\eqref{eqn:e912_18} and \eqref{eqn:e912_21} for our two magnitude limits, respectively.
For Poisson-distributed absorbers with an \ion{H}{i} column density distribution $f\left(N_\ion{H}{i},z^{\prime\prime}\right)=\partial^2n/\left(\partial N_\ion{H}{i}\partial z^{\prime\prime}\right)$, the effective optical depth to \ion{H}{i} Lyman continuum photons travelling between redshifts $z^\prime$ and $z$ is \citep{1980ApJ...240..387P} 
\begin{equation}
  \tau_\mathrm{eff}\left(\nu,z,z^\prime\right) = \int_{z}^{z^\prime}\mathrm{d}z^{\prime\prime}\int_0^\infty
  \mathrm{d}N_\ion{H}{i} f\left(N_\ion{H}{i},z^{\prime\prime}\right)\left[1-e^{-\tau_1}\right]\quad,
   \label{eqn:taueff}
\end{equation}
where $\tau_1\approx
N_\ion{H}{i}\sigma_\ion{H}{i}\left(\nu\left[\frac{1+z^{\prime\prime}}{1+z}\right]\right)$
is the Lyman continuum optical depth through an individual
absorber\footnote{The helium content of the absorber can be ignored
  due to the small \ion{H}{i} photoionization cross section at
  $\lambda<304$\,\AA.}.  For the \ion{H}{i} column density
distribution $f\left(N_\ion{H}{i},z\right)$ we adopt the piecewise
power-law parametrization by \citet{2012ApJ...746..125H} that is
consistent with $f\left(N_\ion{H}{i},z\right)$ measurements at
$z<3.5$, and roughly reproduces both the \ion{H}{i} Ly$\alpha$
effective optical depth \citep[but not in detail --
  see][]{2015MNRAS.450.4081P,2017MNRAS.464..897B,2017ApJ...837..106O}
and the measured mean free path to \ion{H}{i} Lyman limit photons at
$z<5.5$ \citep{2009ApJ...705L.113P,2014MNRAS.445.1745W}.
\citet{2018arXiv180109693K} suggest that variations in
$f\left(N_\ion{H}{i},z\right)$ result in modest (10--40 per cent)
changes in $\Gamma_\ion{H}{i}$ at $z<3$. We note, however, that at
$z\ga 3.5$ all UV background synthesis models are based on brazen
extrapolations of $f\left(N_\ion{H}{i},z\right)$, whose detailed shape
for (partial) Lyman limit systems is not well constrained at these
redshifts \citep{2010ApJ...718..392P}.

Equations~\eqref{eqn:flux} and \eqref{eqn:taueff} assume that sources
and absorbers are uncorrelated, and that $J_\nu(\nu, z)$ is spatially
uniform, i.e.\ that the mean free path to \ion{H}{i} Lyman continuum
photons is much larger than the average distance between the sources
\citep[e.g.][]{1999ApJ...514..648M,2004MNRAS.350.1107M,2009ApJ...703.1416F,2012ApJ...746..125H}.
Obviously, these assumptions do not hold for rare sources and during
\ion{H}{i} reionization.  At $z\simeq 5$ our (extrapolated) AGN number
densities suggest an average distance of $\approx 70$\,cMpc between
$M_{1450}<-21$ AGN (Figure~\ref{fig:rhoqso}), which is comparable to
the mean free path \citep[$83\pm 10$\,cMpc,][]{2014MNRAS.445.1745W}.
Consequently, if only $M_{1450}<-21$ AGN contribute to the emissivity
then the UV radiation field at $z\simeq 5$ must fluctuate.  If the QLF
reaches to fainter magnitudes, or if star-forming galaxies contribute
to the emissivity, the UV background remains uniform to higher
redshifts, and the assumptions of standard UV background synthesis
models remain valid.

The red and blue curves in Figure~\ref{fig:gammapi} show the inferred
AGN \ion{H}{i} photoionization rate as a function of redshift for our
integration limits $M_{1450}=-18$ and $-21$, respectively. For
comparison we also plot the predictions of recent UV background
synthesis models for AGN
\citep{2012ApJ...746..125H,2015ApJ...813L...8M,2015MNRAS.451L..30K}
and AGN$+$galaxies \citep{2012ApJ...746..125H,2018arXiv180104931P}, as
well as inferences from the \ion{H}{i} Ly$\alpha$ forest
\citep{2011MNRAS.412.1926W,2013MNRAS.436.1023B,2017MNRAS.467.3172G,2018MNRAS.473..560D},
the Ly$\alpha+\beta$ forest \citep{2018ApJ...855..106D}, and the
quasar proximity effect \citep{2011MNRAS.412.2543C}.  Where necessary,
literature values have been rescaled by a few per cent to adjust to
our cosmology.

Differences in the \ion{H}{i} photoionization rates inferred from UV
background synthesis models mostly arise from obvious differences in
the emissivities of AGN and galaxies, but also due to differences in
the parametrization of the IGM and the AGN SED.

\textbf{Comments from GW:\\
1. Do we plot HM12, MH15 and KS15 as published or recalculated from the emissivities?\\
2. New forest measurements need discussion.\\
3. High z: Assumption of uniform UVB is violated.
}





Davies: Table 1, uniform UVB results plotted
novel statistical approach to infer Gamma by matching the observed and simulated joint distribution of transmission spikes in the $z\sim 6$ Ly$\alpha+\beta$ forest
GammaHI for one $z_em=6.54$QSO assuming uniform or fluctuating UVB, proof of concept,
report median of posterior PDFs with 68 percent credible interval, uncertainties do not include systematics (thermal state of IGM, continuum fit)

Onorbe:
correction to post-reionization GammaHI from HM12 to ensure that simulations match best observations of HI mean flux, goal is to reduce effect of post-process rescaling approach of simulations that aim to reproduce Lya statistics
plot late reion model $z_\mathrm{reion}=6.55$


We assume an
upper limit of $z=15$ while integrating Equation~(\ref{eqn:flux}).
The result is shown in Figure~\ref{fig:gammapi} for the luminosity
function integration limits of $M_{1450}=-21$ and $-18$.  Also shown
for comparison are the measurements of \citet{2013MNRAS.436.1023B},
derived from the Lyman-$\alpha$ forest, and the measurements by
\citet{2011MNRAS.412.2543C} from quasar proximity zones.
%% JFH Please add the data from this paper by Davies:
%% http://adsabs.harvard.edu/cgi-bin/nph-ref_query?bibcode=2018ApJ...855..106D&amp;refs=CITATIONS&amp;db_key=AST
For an
integration limit of $M_{1450}=-21$ the AGN contribution to the
hydrogen photoionization rate falls short of 100\% across the redshift
range.  It is marginally consistent with the measured photoionization
rate at $z=2.4$.  The photoionization rate in our model for
$M_{1450}=-21$ has the same evolution but a higher amplitude by a
factor of $\sim 2$ as the QSO contribution to the \HI\ photoionization
rate in the model of \citet{2012ApJ...746..125H}.  For both of our
integration limits, the photoionisation rate peaks at $z\sim 2$.  For
the integration limit of $M_{1450}=-18$, AGN can provide all the flux
necessary to explain the observed \lya forest between $z=2.4$ and
$3.2$, with a contribution from other sources necessary only at higher
redshifts.  The photoionization rate in the $M_{1450}<-18$ case also
agrees with the inference of \citet{2017MNRAS.467.3172G} from the
low-redshift ($z<0.6$) \lya\ data.  We discuss this low-redshift
evolution in greater detail in the next section.

An important conclusion from Figure~\ref{fig:gammapi} is that the AGN
contribution to hydrogen reionization is likely subdominant, although
it can be non-negligible if faint AGN down to $M_{1450}=-18$ emit
hydrogen-ionizing photons with our assumed SED and a unit escape
fraction.  At $z=6.1$, AGN with $M_{1450}<-18$ contribute about $10\%$
of the required \HI\ ionizing flux.  The contribution of
$M_{1450}<-21$ at this redshift is $\sim 3\%$.  At $z=6$, our
determinations are lower than those by \citet{2015AA...578A..83G} by
almost an order of magnitude.  This difference arises from the
difference in the inferred emissivities in our models relative to
\citet{2015AA...578A..83G}, as discussed in the previous section.  The
photoionization rate evolution in the model of
\citet{2015ApJ...813L...8M} agrees with our determination at low
redshifts ($z<0.5$) for $M_{1450}<-18$ but is much higher at $z>4$, as
expected from the higher emissivities assumed by these authors.  At
$3<z<6$ our model photoionization rates are understandably lower than
those in the models of \citet{2017ApJ...837..106O} and
\citet{2018arXiv180104931P} as these authors include contribution to
the photoionization rate from galaxies in their models.
\textbf{GW: KS15 is different due to their integration limits.}
The differences in our model from that of \citet{2015MNRAS.451L..30K} are
a result of our differences from the reported emissivities of
\citet{2009MNRAS.392...19C} and \citet{2013A&A...551A..29P}, which are
used by \citet{2015MNRAS.451L..30K} to derive an emissivity evolution
model.

\subsection{Photon underproduction at $z=0$?}

It is instructive to closely examine if the corresponding hydrogen
photoionization rate is consistent with the \HI\ column density
distribution function measured from the \lya\ forest
\citep{2016ApJ...817..111D} at low redshifts ($z<0.5$).
\citet{2014ApJ...789L..32K} argued that in order to match the
\HI\ column density distribution function observed by
\citet{2016ApJ...817..111D} at these redshifts, hydrodynamical
cosmological simulations require a hydrogen photoionization rate that
is a factor of five larger than that in the UV background model of
\citet{2012ApJ...746..125H}.  Several recent studies have addressed
this `photon underproduction crisis' \citep{2015MNRAS.451L..30K,
  2015ApJ...811....3S, 2017MNRAS.467.3172G, 2017MNRAS.467.4802F,
  2017MNRAS.466..838G, 2017MNRAS.467L..86V}.  On the one hand, these
studies emphasised the uncertainty in the \citet{2012ApJ...746..125H}
UVB model at these redshifts due to the lack of certainty in the UV
photon emissivities of galaxies and AGN \citep{2015MNRAS.451L..30K,
  2015ApJ...811....3S}.  On the other hand, they noted the uncertainty
in the results of the cosmological simulations at these redshifts, due
to effects such as AGN feedback and limited numerical resolution
\citep{2015ApJ...811....3S, 2017MNRAS.467L..86V, 2017MNRAS.471.1056N,
  2017ApJ...837..106O, 2017MNRAS.466..838G, 2017MNRAS.467.3172G,
  2017ApJ...835..175G}.  A general conclusion of these studies was
that the discrepancy between the photoionization rate required by the
observed \HI\ column density distribution and predicted by the UVB
model of \citet{2012ApJ...746..125H} is likely to be smaller than that
found by \citet{2014ApJ...789L..32K}.

\begin{figure}
  \begin{center}
    % rtg2.draw_g_puc()
    \includegraphics[width=\columnwidth,keepaspectratio]{g_puc.pdf}
  \end{center}
  \caption{Evolution of the hydrogen photoionisation rate at low
    redshifts, when the AGN luminosity function is integrated down to
    $M_{1450}=-21$ (blue curve and shaded region) and $M_{1450}=-18$
    (red curve and shaded region).  Shaded regions show the one-sigma
    (68.26\%) uncertainty.  Also shown are the models and inferences
    from \citet[dotted brown curve]{2012ApJ...746..125H}, the QSO
    contribution in this model (dashed grey), \citet[dotted
      green]{2015ApJ...813L...8M}, \citet[dashed
      black]{2015ApJ...811....3S}, the QSO contribution from the model
    of \citet[dashed orange]{2015MNRAS.451L..30K}, \citet[dotted
      grey]{2017ApJ...837..106O}, \citet[dashed
      brown]{2018arXiv180104931P}, \citet[yellow
      box]{2017MNRAS.467.4802F}, \citet[black
      box]{2017MNRAS.467L..86V}, \citet[inverted
      triangle]{2013MNRAS.436.1023B},
    \citet[pentagon]{2014ApJ...789L..32K}, and
    \citet[circle]{2017MNRAS.467.3172G}. Note that we use the the
    \HI\ column density distribution model from
    \citet{2012ApJ...746..125H} to derive the photoionization rate.
    \label{fig:puc}}
\end{figure}

Figure~\ref{fig:puc} shows the evolution of the \HI\ photoionization
rate due to AGN in our model for luminosity function integration
limits of $M_{1450}=-21$ and $-18$ at redshifts $z<3$.  Note that we
use the 912\,\AA\ emissivity from Equation~(\ref{eqn:e912_18}) and
(\ref{eqn:e912_21}), which are derived by fitting the model from
Equation~(\ref{eqn:e912fit}) to the emissivities obtained from
luminosity functions in various redshift bins.  While doing this, as
discussed above, redshift bins that were interpreted as being affected
by systematic errors were ignored.  As a result, the emissivity model
used in calculating the photoionization rate is an extrapolation at
$z<0.6$.  However, it is interesting to note that the extrapolated
emissivities at these redshifts are consistent with the emissivities
calculated from our luminosity function fits to the $z<0.6$ data for
either of our chosen integration limits.  This can be seen in
Figure~\ref{fig:e912_2}.  At $z\sim 0$, the comoving
912\,\AA\ emissivity in our model is lower by almost a factor of 2
than that derived from the luminosity function inferred at this
redshift by \citet{2009A&A...507..781S} for the $M_{1450}<-18$ case.
For the $M_{1450}<-21$ integration, our comoving emissivity is higher
than the emissivity in the model of \citet{2009A&A...507..781S} by
about 50\%.  This results from the very faint $M_*$ ($\sim -19$) and
steep faint-end slope ($\beta=-2$) obtained by
\citet{2009A&A...507..781S}.  (See Appendix~\ref{sec:qlfliterature}.)
We use a spectral slope of $\alpha=-1.7$ for
$\lambda<912$\,\AA\ (Equation~\ref{eqn:sed}).

We find that the low-redshift \HI\ photoionization rate in our model
is approximately equal to that that in the \citet{2012ApJ...746..125H}
UVB model for both of our integration limits, as shown in
Figure~\ref{fig:puc}.  At $z\sim 0$, the photoionisation rate for our
$M_{1450}<-21$ integration is lower than \citet{2012ApJ...746..125H}
photoionisation rate by about 10\%.  For the $M_{1450}<-18$ case, the
photoionisation rate is higher than that in the
\citet{2012ApJ...746..125H} model by about 10\%.  These differences
are small compared to the factor of 2 to 3 differences that we find
between our model and several recent estimates.  At $z=0.1$, this rate
is smaller by a factor of 2 than the photoionization rate derived by
\citet{2017MNRAS.467.3172G} from the \HI\ column density distribution
measurements of \citet{2016ApJ...817..111D}.  At $z<0.5$, this
photoionization rate evolution in our model appears lower than also
the rate derived by \citet{2015MNRAS.451L..30K} using the quasar
luminosity function measurements of \citet{2009MNRAS.392...19C} and
\citet{2013A&A...551A..29P}.  The model of \citet{2015ApJ...813L...8M}
also results in a higher ionisation rate than our $M_{1450}<-18$
inference at $z<0.7$.  \citet{2015ApJ...811....3S} compared the
\HI\ column density distribution measurements to cosmological
simulations with an enhanced photoionization rate relative to the
\citet{2012ApJ...746..125H} model to find that
$\Gamma_\mathrm{HI}=4.6\times 10^{-14}(1+z)^{4.4}\,\mathrm{s}^{-1}$
produces the observed \HI\ column densities.  This photoionization
rate was achieved in the simulations of \citet{2015ApJ...811....3S} by
a combination of quasars and galaxies (with an escape fraction of
hydrogen-ionizing photons assumed to be $f_\mathrm{esc}=0.05$).  As
seen in Figure~\ref{fig:puc}, this photoionization rate is in closer
agreement with the \citet{2015ApJ...813L...8M} model.  The
photoionization rate estimate by \citet{2017MNRAS.467.4802F} from the
H$\alpha$ surface brightness of a $z\sim 0$ galaxy observed by
VLT/MUSE is somewhat higher than most other models shown in
Figure~\ref{fig:puc}.  However, it is possible that the
\citet{2017MNRAS.467.4802F} estimate is an upper limit, as the
contribution of local sources to the photoionization rate is neglected
in their modelling.  The requirement of an enhanced photoionization
rate at $z\sim 0$ relative to our inference is also confirmed by the
simulations presented by \citet{2017MNRAS.467L..86V},
\citet{2017ApJ...837..106O}, and \citet{2018arXiv180104931P}.  But
note that, as discussed above in Section~\ref{sec:e912}, a direct
comparison of our results with the models of
\citet{2015ApJ...813L...8M}, \citet{2015MNRAS.451L..30K}, and
\citet{2018arXiv180104931P} is difficult because of the inhomogenous
redshift-dependent integration limits adopted by these authors.

Before considering the deficit in the photoionisation rate obtained
from AGN at low redshifts relative to measurements from the
\lya\ forest as a photon underproduction crisis, it is worthwhile to
recall various assumptions entering our derivation.  With a spectral
index of $-1.7$, our ionisation rate estimate falls short of the
\lya\ forest measurements by a factor of $\sim 2$ with an integration
limit of $M_{1450}=-18$ on the luminosity function.
\citet{2015ApJ...813L...8M} implicitly assume an integration limit of
$M_{1450}=-14$ at $z\sim 0$ (see also \citealt{2015MNRAS.451L..30K,
  2018arXiv180104931P}) as this limit corresponds to $M_*+5$ for the
\citet{2009A&A...507..781S} luminosity function measurement.  It is
unclear if the assumption of unit LyC escape fraction is valid for
such faint AGN, but we find that even with this extremely faint
integration limit, the photoionisation rate deficit is reduced by only
about 25\% for a spectral index of $-1.7$.  While this result does not
change significantly if we assume a spectral index of $-1.4$ as
suggested by \citet{2014ApJ...794...75S}, assuming a much harder
spectrum with a spectral index of $-0.56$ \citep{2004ApJ...615..135S}
for faint ($M_{1450}>-23$) AGN completely alleviates the deficit for
an integration limit of $M_{1450}=-18$.  One should also note that our
ionization rate model ignores the large uncertainties in the
\HI\ column density distribution at the Lyman limit
\citep{2011ApJ...736...42R, 2017ApJ...849..106S, 2017MNRAS.466..838G}.  Alternatively, it
may be possible to balance the ionization rate deficit by the
contribution from low-redshift LyC-leaking galaxies
\citep{2016Natur.529..178I, 2018MNRAS.474.4514I, 2018MNRAS.478.4851I}

\subsection{Helium reionization}

\begin{figure}
  \begin{center}
    % qhe.py 
    \includegraphics[width=\columnwidth,keepaspectratio]{q.pdf}
  \end{center}
  \caption{ Redshift evolution of the volume-averaged \ion{He}{iii}
    fraction $Q_\ion{He}{iii}$ considering AGN with $M_{1450}<-21$
    (blue) and $M_{1450}<-18$ (red) in our calculation of the emission
    rate of ionizing photons.  The shaded regions show the one-sigma
    (68.26\%) confidence interval resulting from the uncertainty in
    the emissivity.  The other curves show previous determinations of
    $Q_\ion{He}{iii}\left(z\right)$ based on
    Equation~\eqref{eqn:qHedot} \citep{2012ApJ...746..125H,
      2015ApJ...813L...8M, 2016ApJ...828...90L, 2018arXiv180104931P}
    and on cosmological radiative transfer simulations of
    quasar-driven \ion{He}{ii} reionization
    \citep{2009ApJ...694..842M, 2014MNRAS.445.4186C}.  Note that
    Equation~\eqref{eqn:qHedot} ignores the presence of \ion{He}{ii}
    Lyman limit systems impeding reionization at
    $Q_\ion{He}{iii}\rightarrow 1$, requiring further analytic
    modelling \citep{2017ApJ...851...50M} or numerical simulation
    \citep{2018arXiv180104931P}.
  \label{fig:qhe}}  
\end{figure}

We now consider the implications of our AGN luminosity function models
for \ion{He}{ii} reionization. The time evolution of the
volume-averaged \ion{He}{iii} fraction $Q_\ion{He}{iii}$ is given by
\citep[e.g.][]{2012ApJ...746..125H}
\begin{equation}
  \frac{\mathrm{d}Q_\ion{He}{iii}}{\mathrm{d}t}=\frac{\dot{n}_\mathrm{ion,4}}{\langle n_\mathrm{He}\rangle}-\frac{Q_\ion{He}{iii}}{\langle t_\mathrm{rec,He}\rangle}\quad,
  \label{eqn:qHedot}
\end{equation}
where $\dot n_\mathrm{ion,4}$ is the emission rate of $\ge 4$\,Ry
photons per unit proper volume, $\langle n_\mathrm{He}\rangle$ is the
average proper helium number density, and $\langle
t_\mathrm{rec,He}\rangle$ is the average recombination time scale for
\ion{He}{iii}.  With our quasar emissivity model from
Section~\ref{sec:e912} the emission rate can be written as
\begin{equation}
\dot
n_\mathrm{ion,4}\left(z\right)=-\frac{4^{\alpha_\nu}}{h\alpha_\nu}\left(1+z\right)^3\epsilon_{912}\left(z\right)\quad,
\end{equation}
with the quasar spectral index $\alpha_\nu=-1.7$ and the comoving 912\,\AA\ emissivity
$\epsilon_{912}\left(z\right)$ given by Equations~\eqref{eqn:e912_18} and \eqref{eqn:e912_21}
for $M_{1450}<-18$ and $M_{1450}<-21$, respectively.
The recombination time scale in Equation~\eqref{eqn:qHedot} is given by
\begin{equation}
  \langle t_{\rm rec}\rangle=[(1+2\chi) \langle n_\mathrm{H}\rangle \alpha_\mathrm{B}\,C]^{-1}\quad,
  \label{eq:trec}
\end{equation}
where $\langle n_\mathrm{H}\rangle=1.881\times
10^{-7}(1+z)^3$~cm$^{-3}$ is the average proper hydrogen number
density, $\alpha_\mathrm{B}$ is the Case~B \ion{He}{iii} recombination
coefficient \citep{1997MNRAS.292...27H}, and $\chi=0.079$ is the
cosmic number fraction of helium for a cosmic helium mass fraction of
$Y_\mathrm{He}=0.24$.  We assume that the clumping factor $C$ for
helium is the same as for hydrogen, for which
\citet{2012ApJ...747..100S} obtained
\begin{equation}
  C = 2.9\left(\frac{1+z}{6}\right)^{-1.1}
\end{equation}
over the redshift range $5<z<9$. 

Equation~\eqref{eqn:qHedot} neglects \ion{He}{ii} Lyman limit systems
that considerably delay the end of \ion{He}{ii} reionization
\citep{2009MNRAS.395..736B, 2017ApJ...851...50M}.  In the absence of
these self-shielded systems, $Q_\ion{He}{iii}$ can continue to
increase beyond unity and the mean free path of \ion{He}{ii}-ionizing
photons diverges.  We set $Q_\ion{He}{iii}=1$ when this happens.  This
can be corrected by accounting for the \ion{He}{ii} column density
distribution and filtering the \ion{He}{ii}-ionizing radiation field
through it. Unfortunately, the \ion{He}{ii} column density
distribution is itself uncertain, as it depends on the relative
contributions of quasars and galaxies to the UV background
\citep[e.g.][]{2012ApJ...746..125H,2018arXiv180104931P}.  We consider
the simplified treatment of $Q_\ion{He}{iii}$ accurate enough for the
purpose of this work, while noting that the redshift of \ion{He}{ii}
reionization is likely overestimated in any such model.

The resultant \ion{He}{ii} reionization histories are shown in
Figure~\ref{fig:qhe} for our two considered magnitude limits, with the
shaded regions showing the one-sigma uncertainty in $Q_\ion{He}{iii}$
resulting from the uncertainty in the quasar emissivity alone,
i.e.\ using a fixed quasar spectral energy distribution and a fixed
redshift evolution of the clumping factor.  In our model, \ion{He}{ii}
reionization starts around the earliest quasars at $z>6$ and finishes
at $3.2\la z\la 3.5$ depending on the extent of the faint-end QLF.
Considering the limitations of the modelling discussed above, both
\ion{He}{ii} reionization histories are consistent with recent
\ion{He}{ii} Ly$\alpha$ effective optical depth measurements
supporting substantial progression of \ion{He}{ii} reionization by
$z\simeq 3.4$ \citep{2016ApJ...825..144W}, and an end of the process
at $z\simeq 2.7$ \citep{2011ApJ...733L..24W,2016ApJ...825..144W} with
the build-up of a quasi-homogeneous \ion{He}{ii}-ionizing background
\citep{2014MNRAS.437.1141D,2017MNRAS.465.2886D}.

Figure~\ref{fig:qhe} also shows previous solutions of
Equation~\eqref{eqn:qHedot} with different parameterizations of the
quasar emissivity, spectral energy distribution and clumping factor
\citep{2012ApJ...746..125H, 2015ApJ...813L...8M, 2016ApJ...828...90L,
  2018arXiv180104931P}, as well as the results from cosmological
radiative transfer simulations of \ion{He}{ii} reionization
\citep{2009ApJ...694..842M,2014MNRAS.445.4186C}.  The differences of
our results to the ones by \citet{2012ApJ...746..125H} and
\citet{2015ApJ...813L...8M} mostly result from differences in the
adopted quasar emissivity (Figure~\ref{fig:e912_2}).  The large quasar
emissivity adopted by \citet{2015ApJ...813L...8M} results in an early
completion of \ion{He}{ii} reionization at $z\approx 4$, which is
inconsistent with the measured strong \ion{He}{ii} absorption at
$2.7<z<3$ \citep{2016ApJ...825..144W, 2018MNRAS.473.1416M,
  2018arXiv180104931P} if the \ion{He}{ii}-ionizing background is not
fluctuating on large scales due to rare clustered and/or short-lived
quasars that result in a spatially varying mean free path of
\ion{He}{ii}-ionizing photons \citep{2010ApJ...714..355F,
  2014MNRAS.440.2406M, 2014MNRAS.437.1141D, 2017MNRAS.465.2886D}.  The
quasar emissivity considered by \citet{2012ApJ...746..125H} falls
below our determinations at the relevant redshifts
(Figure~\ref{fig:e912_2}), which results in a delayed completion of
\ion{He}{ii} reionization at $z\simeq 2.8$.

\citet{2016ApJ...828...90L} varied the parameters of
Equation~\eqref{eqn:qHedot} to estimate the uncertainty in
$Q_\ion{He}{iii}\left(z\right)$, in particular the
QLF\footnote{\citet{2016ApJ...828...90L} erroneously varied the
  $z>3.5$ QLF parameters independently, neglecting their covariance
  and the survey data.}, shown in Figure~\ref{fig:e912_2}.
Considering the limitations of Equation~\eqref{eqn:qHedot},
\ion{He}{ii} reionization finishes too late ($z\sim 2.5$) in their
fiducial model, which is probably due to their considered QLFs and
varying magnitude limits their emissivity calculations.

\citet{2018arXiv180104931P} adopted a somewhat higher quasar
emissivity than \citet{2012ApJ...746..125H} that is comparable to our
fit for $M_{1450}<-21$ at the redshifts of interest.  Instead of their
solution to Equation~\eqref{eqn:qHedot} we show their results of
one-cell simulations with gas at cosmic mean density (see Section~3.3
and Appendix~C of \citealt{2018arXiv180104931P} for details).  In
these simulations the \ion{He}{ii} number density is correctly
calculated as \ion{He}{ii} is gradually ionized by a radiation field
resulting from one-dimensional radiative transfer calculations through
an inhomogeneous IGM.  This accounts for \ion{He}{ii} Lyman limit
systems in a simple way, resulting in a smooth transition of
$Q_\ion{He}{iii}$ to unity.

In Figure~\ref{fig:qhe} we also plot $Q_\ion{He}{iii}\left(z\right)$
from two numerical simulations of quasar-driven \ion{He}{ii}
reionization \citep{2009ApJ...694..842M,2014MNRAS.445.4186C} that
broadly reproduce the measured \ion{He}{ii} effective optical depths
\citep{2016ApJ...825..144W}.  In these simulations the outputs of
$N$-body or hydrodynamic simulations are post-processed with radiative
transfer around quasars sourced according to a specific model, and the
effects of \ion{He}{ii} Lyman limit systems are approximately captured
with sub-grid filtering methods \citep{2009ApJ...694..842M} or with
adaptive mesh refinement \citep{2014MNRAS.445.4186C}.  Their different
timing of \ion{He}{ii} reionization is mostly due to the adopted
quasar model (QLF, spectral energy distribution, quasar lifetime,
anisotropic emission, halo mass range) and stochasticity in the $z\ga
3.5$ quasar number density in the limited simulation volumes.  Thus,
while current simulations do not reproduce the measured scatter in the
\ion{He}{ii} effective optical depths at $2.7<z<3.5$ in detail
\citep{2016ApJ...825..144W}, this may reflect limitations of the
modelling other than the QLF \citep{2017MNRAS.468.4691D}.  Our
$Q_\ion{He}{iii}\left(z\right)$ for $M_{1450}<-18$ agrees well with
the simulation by \citet{2014MNRAS.445.4186C}, in which \ion{He}{ii}
reionization is accomplished by the $z<5$ quasar population evolved
assuming pure density or pure luminosity evolution to match the QLF by
\citet{2011ApJ...728L..26G} for $M_{1450}<-19.5$ \citep[see
  also][]{2013MNRAS.435.3169C}.  Future efforts in numerical modelling
of \ion{He}{ii} reionization should investigate quasar models which
follow the QLF evolution in detail, and which are consistent with the
measured redshift evolution and scatter of the \ion{He}{ii} effective
optical depth, as well as the measured redshift evolution of the IGM
temperature-density relation \citep[e.g.][]{2011MNRAS.410.1096B,
  2014MNRAS.441.1916B, 2018MNRAS.474.2871R, 2017arXiv171000700H}.


\section{Summary and Conclusions}
\label{sec:conc}

We have analysed the evolution of the AGN UV luminosity function from redshift $z=0$ to $7.5$
using a combined sample of 83,488 mostly UV-optical colour-selected AGN from 12 data sets,
homogenised with respect to the assumed cosmology, magnitude system and bandpass correction (if possible).
The vast majority of them (83,469 AGN from 11/12 samples) have spectroscopic redshifts,
and for all but 22 AGN the selection functions have been characterised.
After restricting the sample due to persisting incompleteness at the faint end we arrive at 77,659 spectroscopically
confirmed $0<z<7.5$ AGN extending to absolute AB magnitudes $M_{1450}\simeq -19$ at $z\simeq 0.5$
and to $M_{1450}\sim -22$ at $z>3.5$, respectively. 
To facilitate comparisons to future data sets we make the homogenised AGN sample and
the homogenised selection functions publicly available.

Binning the $1450$\,\AA\ AGN luminosity function in narrow redshift ranges we find that it is
excellently described by the customary double power law in magnitude at all redshifts (Figure~\ref{fig:mosaic}).
Its four parameters significantly evolve with redshift (Figure~\ref{fig:evoln}):
(i) The break magnitude $M_*$ of the AGN luminosity function shows a steep brightening from $M_*\simeq -24$ at $z\simeq 0.7$ to $M_*\sim -29$ at $z\simeq 6$.
(ii) The corresponding amplitude $\phi_*$ drops by a factor $\sim 20,000$ from $\phi_*\simeq 4\times 10^{-7}$~mag$^{-1}$cMpc$^{-3}$ at $z<2.2$ to $\phi_*\sim 2\times 10^{-11}$~mag$^{-1}$cMpc$^{-3}$ at $z\simeq 6$.
(iii) The faint-end slope $\beta$ significantly decreases from $\simeq -1.7$ at $z<2.2$ to $\simeq -2.4$ at $z\simeq 6$, resulting in a steepening of the luminosity function.
(iv) The bright-end slope $\alpha$ also shows a moderate decrease with redshift, but is less constrained at the highest redshifts because of the bright break magnitude.
In contrast to several previous studies, our fits are based on unbinned homogenised AGN data
and their selection functions, and the confidence intervals fully account for covariance in the luminosity function parameters. 
From the continuity of the luminosity function at lower redshifts we argue that its apparent
single power law description at $z\simeq 6$ can be interpreted as the faint end of the double power law
whose break magnitude $M_*$ is at the bright end of the $z\simeq 6$ quasar population.

Similar continuity arguments let us scrutinize the value of several
photometrically-selected AGN samples in determinations of the
luminosity function.  Our analysis has revealed systematic errors in
the survey selection functions caused by their fixed and simplified
assumptions, i.e.\ regarding the parameterisation of the IGM and the
AGN SED, and the treatment of photometric errors at the survey
magnitude limit.  Several of these systematic errors are easily
identified from artificial faint-end drops of the luminosity function
(Figure~\ref{fig:mosaic}), an apparently zero accessible volume for
discovered AGN, or a mismatch between the observed and the simulated
AGN color distribution.  In large samples the precision of the AGN
luminosity function is limited by unaccounted systematic errors in the
survey selection functions, which lead to (i) unphysical variations in
the luminosity function parameters (i.e.\ for BOSS in
Figure~\ref{fig:evoln}) and (ii) inter-survey systematics in combined
samples that are further amplified by heterogeneous selection function
parameter choices amongst the surveys.

With only partially credible data it is challenging to describe the
redshift evolution of the AGN luminosity function with a viable
parametric model.  We have developed three such models
(Figure~\ref{fig:evoln_global}), finding that our fourteen-parameter
Model 1 describes the redshift evolution of the luminosity function
rather well.  However, this model prefers a break in the faint-end
slope at $z\simeq 3.5$, which causes an unphysical discontinuity in
the cumulative AGN number density (Figure~\ref{fig:rhoqso}).  Our
other two models do not have these features, but they do not match the
measured faint-end slope at $z>4$ as well as Model 1 does.

With our determinations of the luminosity function we have revisited
the question of the contribution of AGN to reionization and the UV
background.  We have made the first determinations of the AGN
912\,\AA\ emissivity with homogeneous faint-end limits at all
redshifts, whose statistical uncertainties have been properly
calculated from the posterior distributions of the luminosity function
(Figure~\ref{fig:e912_2}). Our parametric fits yield a peak in the
912\,\AA\ emissivity at $z\approx 2.4$, with a decline by $\simeq 2$
orders of magnitude to $z\simeq 0$ and $z\simeq 7$, respectively. At
$z>4$ our determined emissivities are lower by a factor 2--10 than
recently claimed by \citet{2015AA...578A..83G}. At the same time, the
912\,\AA\ emissivity of $M_{1450}<-18$ AGN exceeds the model
considered by \citet{2012ApJ...746..125H} by a factor $>2$ at $z>3$,
suggesting a somewhat higher contribution of AGN to the high-$z$ UV
background if faint AGN have unit escape fractions of ionizing
photons.

Having derived the AGN \ion{H}{i} photoionization rate by filtering
the \ion{H}{i}-ionizing AGN emissivity through the IGM \ion{H}{i}
column density distribution, we find that while at $z=2$--$3$
$M_{1450}<-18$ AGN are almost sufficient to explain measurements from
the Ly$\alpha$ forest, additional UV sources are required at higher
redshifts (Figure~\ref{fig:gammapi}).  Boldly extrapolating the
\ion{H}{i} column density distribution to $z=6$, we estimate that
$M_{1450}<-18$ AGN contribute to the \ion{H}{i} photoionization rate
only at the $\sim 5$ per cent level.  This indicates a minor
contribution of such AGN to \ion{H}{i} reionization.  At $z<0.5$
$M_{1450}<-18$ AGN fall short by a factor of $\sim 2$ to explain UV
background measurements (Figure~\ref{fig:puc}), but we hesitate to
claim a `photon underproduction crisis' \citep{2014ApJ...789L..32K},
because (a) the apparent tension may be alleviated either by harder
extreme-UV SEDs in faint ($M_{1450}>-23$) AGN
\citep{2004ApJ...615..135S} or by known $-18\la M_{1450}\la -14$ AGN
at these redshifts \citep{2009A&A...507..781S} albeit their Lyman
continuum escape fraction is unknown, and (b) current UV background
synthesis models do not account for the large uncertainty in the
low-redshift column density distribution of (partial) \ion{H}{i} Lyman
limit systems \citep{2011ApJ...736...42R, 2017ApJ...849..106S}.  Helium
reionization is accomplished at $z\simeq 3.5$ by $M_{1450}<-18$ AGN
(Figure~\ref{fig:qhe}), but should be delayed by \ion{He}{ii} Lyman
limit systems \citep{2009MNRAS.395..736B, 2017ApJ...851...50M},
requiring further modelling and detailed numerical simulations of the
\ion{He}{ii} reionization process.

There are several promising paths forward in the characterization of
the AGN luminosity function, some of which may substantially advance
on current UV-optical broadband color-selected samples that have
dominated the field for the past 20 years.  These include, but are not
limited to, comprehensive surveys for faint high-$z$ quasars selected
by broadband photometry \citep{2016ApJ...828...26M,
  2017ApJ...839...27W} or variability \citep{2017MNRAS.464.1693H},
infrared-selected highly complete surveys at $z\sim 3$
\citep{2017ApJ...851...13S, 2018AJ....155..110Y}, multi-band
narrow-band surveys targeting AGN at all epochs
\citep{2014arXiv1403.5237B}, slitless spectroscopic surveys with
upcoming space telescopes \citep{2011arXiv1110.3193L,
  2013arXiv1305.5422S}, and next-generation X-ray surveys
\citep[e.g.][]{2013A&A...558A..89K}. However, we caution that future
progress based on massive surveys requires a detailed quantification
of their selection functions and the incorporation of systematic
errors into AGN luminosity function measurements, in a similar manner
to recent surveys for lensed high-redshift galaxies that are limited
by the accuracy of the lensing mass models \citep{2017ApJ...843..129B,
  2018ApJ...854...73I, 2018arXiv180309747A}.  For a better
characterization of the AGN contribution to the UV background one will
need to (a) perform spectroscopic surveys for faint AGN at all
redshifts and determine their characteristic UV SED, (b) carefully
assess the impact of the host galaxy on the survey selection and the
ionizing power, and (c) measure precisely the amplitude and shape of
the column density distribution of (partial) \ion{H}{i} Lyman limit
systems to inform UV background synthesis models at $z<2$ and $z>4$.

\textbf{A short final paragraph on Hopkins here...}

\section*{Acknowledgements}

We thank Eilat Glikman, Linhua Jiang, Nobunari Kashi\-kawa, Ian
McGreer, Nick Ross, Chris Willott, and Jinyi Yang for sharing data and
especially for sharing their quasar selection functions.  It is a
pleasure to acknowledge useful discussions with James Aird, Eduardo
Ba\~nados, Manda Banerji, Tirthankar Roy Choudhury, George Efstathiou,
Xiaohui Fan, Andrea Ferrara, Prakash Gaikwad, Francesco Haardt, Martin
Haehnelt, Paul Hewett, David Hogg, Vikram Khaire, Sergey Koposov,
Donald Lynden-Bell, Roberto Maiolino, Richard McMahon, Daniel
Mortlock, Ewald Puchwein, Gordon Richards, Alberto Rorai, Bram
Venemans and Stephen Warren.  GK acknowledges support from ERC
Advanced Grant 320596 `The Emergence of Structure During the Epoch of
Reionization'.
%% Joe, do you need to add anything to acknowledgements? 

\appendix

\section{Posterior distributions}

Figure~\ref{fig:corner} shows marginalised one-dimensional and
two-dimensional posterior probability distribution functions (PDFs) of
the four parameters, $\phi_*$, $M_*$, $\alpha$ and $\beta$, of the
double-power-law luminosity function in the $3.7\leq z < 4.1$ redshift
bin.  The procedure adopted for fitting this model is described in
Section~\ref{sec:bins}.  Figure~\ref{fig:corner} illustrates that the
four parameters are well-constrainted.  This figure also shows the
degeneracies between the parameters.  There is a relatively strong
correlation between the amplitude of the luminosity function $\phi_*$
and the break magnitude $M_*$.  The faint-end slope $\beta$ is
positively correlated with the other three parameters.  Similar
behaviour of the posterior distributions is seen in all the redshift
bins defined in Section~\ref{sec:bins}.  In our highest-redshift bin
($5.5\leq z < 6.5$), we impose a prior $\alpha < -4$, which changes
the posterior distributions.  However, various parameter correlations
remain qualitatively unchanged.

\begin{figure*}
  \begin{center}
    % corner.corner() with bins.lfs[-4]
    \includegraphics[width=\textwidth]{corner_z3p9.pdf}
  \end{center}
  \caption{Posterior distributions of the four double-power-law
    parameters in the $3.7\leq z < 4.1$ redshift bin.  The blue
    squares indicate median values.  Similar behaviour of the
    posterior distributions is seen in all other redshift bins defined
    in Section~\ref{sec:bins}.
    \label{fig:corner}}
\end{figure*}


\section{Comparison with other luminosity function determinations}
\label{sec:qlfliterature}
Figure~\ref{fig:params_grand} compares the parameters of the double
power law luminosity function from our analysis of
Section~\ref{sec:bins} to values reported in the literature.  In
general, our break luminosity is brighter and the faint-end slope is
steeper than other determinations.  Note however that several results
from the literature shown in Figure~\ref{fig:params_grand} make
restrictive assumptions while fitting double power law models to data.
For example, \citet{2013ApJ...768..105M} fix the bright-end slope
$\alpha$ to $-4$ at $z=4.7$--$5.1$.  \citet{2016ApJ...833..222J} fix
the faint-end slope $\beta$ to $-2.8$ at $z=5.7$--$6.4$.
\citet{2017ApJ...847L..15O} fix the bright-end slope $\alpha$ to
$-2.8$ at $z=5.5$--$6.5$.  \citet{2015AA...578A..83G} fix the
faint-end slope and the break luminosity in their highest-redshift bin
($z=5.0$--$6.5$).  The choice of data sets is also often different.
For instance, \citet{2015AA...578A..83G} do not include the SDSS
Stripe 82 data from \citet{2013ApJ...768..105M} and the AGN samples of
\citet{2010AJ....139..906W} and \citet{2015ApJ...798...28K} in their
analysis.  (We discuss the results of \citet{2015AA...578A..83G} in
greater detail in Appendix~\ref{sec:conv}.)  Of the remaining
determinations, our parameter values are closest to those reported by
\citet{2016ApJ...829...33Y} at $z=4.7$--$5.4$.

\begin{figure*}
  \begin{center}
    % summary_grand.py
    \includegraphics[width=0.7\textwidth]{evolution_grand.pdf}
  \end{center}
  \caption{A comparison of our inferred double power law parameter
    values with those reported in the literature.  Black points show
    our determinations from Figure~\ref{fig:evoln}.  Blue points show
    other values, from \citet[triangles]{2012ApJ...755..169M},
    \citet[squares]{2016ApJ...833..222J},
    \citet[diamonds]{2011ApJ...728L..26G},
    \citet[crosses]{2013ApJ...768..105M}, \citet[downward
      triangles]{2015AA...578A..83G},
    \citet[pentagons]{2018PASJ...70S..34A}, \citet[leftward
      triangles]{2017ApJ...847L..15O},
    \citet[asterisks]{2016ApJ...829...33Y},
    \citet[circles]{2013ApJ...773...14R}, and \citet[rightward
      triangles]{2009A&A...507..781S}.  Where necessary, values from
    the literature have been converted to our cosmology.
    \label{fig:params_grand}}
\end{figure*}

\section{Comparison with G15}
\label{sec:conv}
%% Define 'G15' somewhere above.
In the double power law luminosity function models presented in
distinct redshift bins in Section~\ref{sec:bins}, we did not include
the 19 low-luminosity ($M_{1450}>-22.6$) AGN between redshifts $z=4.1$
and $6.3$ reported by G15.  While this was done in order to restrict
our sample to quasars with spectroscopic redshift determinations, it
is instructive to consider how our results are affected if the G15 AGN
are added to the analysis.  In their work, G15 found a shallower
faint-end slope ($\beta\sim -1.5$ to $-1.8$) for the luminosity
function at $4.1 < z < 6.3$, relative to our result from other AGN
samples at these redshifts ($\beta\sim -2.0$ to $-2.5$).  Still, G15
derived a higher 912\,\AA\ emissivity than our estimates
(cf.\ Figure~\ref{fig:e912_2}), so that in their analysis AGN can
produce all ionizing photons necessary to keep hydrogen ionized and
explain the \lya data.  The black points in
Figure~\ref{fig:params_giallongo} show the parameters of the double
power law luminosity function from our analysis of
Section~\ref{sec:bins}.  The red open circles show the parameter
values obtained when the G15 sample is added to the analysis.  We find
that the two results are highly consistent, showing that the G15
sample is consistent with our double power law fit obtained from other
AGN samples are comparable redshifts.  This is surprising as the
integrated 912\,\AA\ emissivities in our model are smaller than those
derived by G15.  Figure~\ref{fig:lf_giallongo} provides an
explanation.  As seen in this figure, the double power law fits
favoured by G15 (red dashed curves) are quite different from our fits
(black curves).  The characteristic luminosity $M_*$ obtained by G15
is much fainter ($\sim -23$ at $z = 5$) than that resulting out of our
analysis ($\sim -29$ at $z = 5$).  Thus the 912\,\AA\ emissivities are
enhanced in G15 because of the increase contribution from
intermediate-luminosity AGN in their model.
Figure~\ref{fig:lf_giallongo} suggests that this is possibly because
of the inclusion of SDSS Stripe 82 data from
\citet{2013ApJ...768..105M} and the AGN samples of
\citet{2010AJ....139..906W} and \citet{2015ApJ...798...28K} in our
analysis.  Additionally, the homogenisation of data in our work may
also cause part of the difference.

\begin{figure*}
  \begin{center}
    % summary_giallongo.py
    \includegraphics[width=0.7\textwidth]{evolution_g.pdf}
  \end{center}
  \caption{Effect of the 19 AGN reported by \citet{2015AA...578A..83G}
    on the double power law luminosity function parameters in
    redshift bins from $z=0$ to $7$.  Black points show parameter
    values from Figure~\ref{fig:evoln}.  Red open circles show the
    parameter values obtained when the sample of G15 is added to the
    analysis.  In both cases, vertical error bars show one-sigma
    (68.26\%) uncertainties, and horizontal error bars show widths of
    the redshift bins. \label{fig:params_giallongo}}
\end{figure*}

\begin{figure*}
  \begin{center}
    % bins_withg.py
    % drawlf_giallongocompare.py 
    % giallongo_compare.py 
    \includegraphics[width=\textwidth]{giallongo_compare.pdf}
  \end{center}
  \caption{Luminosity functions in three redshift bins at $z>4.1$.
    Black curves in each panel show the double power law,
    with the corresponding one-sigma (68.26\%) uncertainty shown by
    the grey shaded area.  There are 451, 270, and 69 AGN in each
    redshift bin from left to right, respectively.  These numbers are
    higher than those in Figure~\ref{fig:mosaic} because they include,
    respectively, 9, 7, and 3 AGN from G15.  The magnitude bins
    containing these AGN are shown in purple.  The red dashed curves
    show the double power law fits reported by G15 at $z=4.25, 4.75,$
    and $5.75$. \label{fig:lf_giallongo}}
\end{figure*}

\section{Tables of emissivities and photoionisation rates}
\label{sec:tables}

Table~\ref{tab:emissivity_bins} shows the 912\,\AA\ and
1450\,\AA\ comoving emissivities obtained in various redshift bins
with the one-sigma (68.26\%) uncertainties.  Redshift bins severely
affected by systematic errors are also shown.  The result of the model
presented in Equation~(\ref{eqn:e912fit}), which describes the
emissivity evolution using a smooth function, is tabulated in
Table~\ref{tab:gamma2}, along with the hydrogen photoionization rate
computed in Section~\ref{sec:gammahi}.  In both tables, we show
results for our two integration limits of $M_{1450}<-18$ and
$M_{1450}<-21$.  These tables describe the curves shown in
Figures~\ref{fig:e912_2} and \ref{fig:gammapi}.  Note that the
photoionisation rate calculation assumes an \HI\ column density
distribution given by \citet{2012ApJ...746..125H} and extrapolates
this to high redshifts.

\begin{table*}
  % gammapi.py and tabulate_emissivities.py. 
  \caption{
    \textbf{GW: Delete the zbin column that we do not use. Indicate the credible (fitted) redshift bins!}
    Comoving emissivities at 912\,\AA\ and 1450\,\AA\ derived from
    our double power law luminosity function models in redshift bins
    (Table~\ref{tab:bins}) for two magnitude limits. The units are
    $10^{24}$\ erg\ s$^{-1}$\ Hz$^{-1}$\ cMpc$^{-3}$.
    Statistical uncertainties are one-sigma (68.26\%).
    }
  \label{tab:emissivity_bins}
  \begin{tabular}{cccc....}
    \hline
    $\langle z\rangle$ &
    $z_\mathrm{bin}$ &
    $z_\mathrm{min}$ &
    $z_\mathrm{max}$ &
    \multicolumn{1}{c}{$\epsilon_{912}$} &
    \multicolumn{1}{c}{$\epsilon_{1450}$} &
    \multicolumn{1}{c}{$\epsilon_{912}$} &
    \multicolumn{1}{c}{$\epsilon_{1450}$} \\
    &
    &
    &
    &
    \multicolumn{1}{c}{$(M_{1450}<-18)$} &
    \multicolumn{1}{c}{$(M_{1450}<-18)$} &
    \multicolumn{1}{c}{$(M_{1450}<-21)$} &
    \multicolumn{1}{c}{$(M_{1450}<-21)$} \\
    \hline
    0.31 & 0.25 & 0.10 & 0.40 & 0.53^{+0.02}_{-0.02} & 0.71^{+0.03}_{-0.03} & 0.30^{+0.01}_{-0.01} & 0.40^{+0.01}_{-0.01} \\
    0.50 & 0.50 & 0.40 & 0.60 & 0.75^{+0.01}_{-0.01} & 1.00^{+0.02}_{-0.02} & 0.58^{+0.01}_{-0.01} & 0.78^{+0.01}_{-0.01} \\
    0.72 & 0.70 & 0.60 & 0.80 & 1.73^{+0.06}_{-0.05} & 2.31^{+0.07}_{-0.06} & 1.19^{+0.02}_{-0.02} & 1.57^{+0.02}_{-0.02} \\
    0.91 & 0.90 & 0.80 & 1.00 & 2.85^{+0.16}_{-0.16} & 3.77^{+0.20}_{-0.19} & 2.10^{+0.05}_{-0.05} & 2.79^{+0.06}_{-0.06} \\
    1.10 & 1.10 & 1.00 & 1.20 & 3.71^{+0.11}_{-0.10} & 4.94^{+0.15}_{-0.13} & 2.91^{+0.05}_{-0.05} & 3.87^{+0.06}_{-0.06} \\
    1.30 & 1.30 & 1.20 & 1.40 & 5.69^{+0.23}_{-0.24} & 7.52^{+0.32}_{-0.30} & 4.46^{+0.10}_{-0.09} & 5.91^{+0.11}_{-0.12} \\
    1.50 & 1.50 & 1.40 & 1.60 & 7.04^{+0.22}_{-0.22} & 9.34^{+0.26}_{-0.31} & 5.59^{+0.09}_{-0.11} & 7.41^{+0.13}_{-0.13} \\
    1.71 & 1.70 & 1.60 & 1.80 & 7.69^{+0.18}_{-0.18} & 10.23^{+0.24}_{-0.26} & 6.71^{+0.10}_{-0.11} & 8.90^{+0.13}_{-0.14} \\
    1.98 & 2.00 & 1.80 & 2.20 & 10.59^{+0.36}_{-0.36} & 14.08^{+0.43}_{-0.42} & 7.93^{+0.13}_{-0.14} & 10.52^{+0.19}_{-0.17} \\
    2.25 & 2.25 & 2.20 & 2.30 & 11.34^{+0.35}_{-0.37} & 14.97^{+0.48}_{-0.50} & 10.06^{+0.20}_{-0.20} & 13.37^{+0.26}_{-0.23} \\
    2.35 & 2.35 & 2.30 & 2.40 & 9.30^{+0.30}_{-0.30} & 12.29^{+0.36}_{-0.35} & 8.02^{+0.15}_{-0.15} & 10.65^{+0.22}_{-0.20} \\
    2.45 & 2.45 & 2.40 & 2.50 & 8.06^{+0.24}_{-0.24} & 10.67^{+0.30}_{-0.26} & 7.16^{+0.13}_{-0.15} & 9.49^{+0.17}_{-0.18} \\
    2.65 & 2.65 & 2.60 & 2.70 & 6.55^{+0.16}_{-0.17} & 8.70^{+0.21}_{-0.20} & 6.46^{+0.15}_{-0.15} & 8.55^{+0.20}_{-0.20} \\
    2.75 & 2.75 & 2.70 & 2.80 & 7.44^{+0.24}_{-0.22} & 9.93^{+0.30}_{-0.31} & 7.17^{+0.20}_{-0.20} & 9.51^{+0.27}_{-0.25} \\
    2.85 & 2.85 & 2.80 & 2.90 & 7.94^{+0.28}_{-0.29} & 10.45^{+0.42}_{-0.41} & 7.48^{+0.29}_{-0.28} & 9.95^{+0.36}_{-0.35} \\
    2.95 & 2.95 & 2.90 & 3.00 & 7.93^{+0.34}_{-0.34} & 10.51^{+0.45}_{-0.46} & 6.83^{+0.20}_{-0.21} & 9.09^{+0.28}_{-0.28} \\
    3.05 & 3.05 & 3.00 & 3.10 & 7.18^{+0.38}_{-0.36} & 9.59^{+0.47}_{-0.50} & 6.24^{+0.20}_{-0.21} & 8.29^{+0.34}_{-0.30} \\
    3.15 & 3.15 & 3.10 & 3.20 & 9.99^{+0.95}_{-1.03} & 13.14^{+1.38}_{-1.56} & 7.01^{+0.35}_{-0.35} & 9.34^{+0.41}_{-0.47} \\
    3.25 & 3.25 & 3.20 & 3.30 & 8.27^{+0.94}_{-1.01} & 11.05^{+1.27}_{-1.17} & 6.06^{+0.35}_{-0.36} & 8.07^{+0.49}_{-0.50} \\
    3.34 & 3.35 & 3.30 & 3.40 & 12.00^{+2.57}_{-2.55} & 16.10^{+3.44}_{-3.31} & 7.21^{+0.61}_{-0.74} & 9.59^{+0.91}_{-0.93} \\
    3.44 & 3.45 & 3.40 & 3.50 & 4.47^{+0.52}_{-0.51} & 5.85^{+0.61}_{-0.64} & 4.37^{+0.45}_{-0.47} & 5.73^{+0.56}_{-0.67} \\
    3.88 & 3.90 & 3.70 & 4.10 & 4.36^{+1.44}_{-1.39} & 5.76^{+1.98}_{-1.84} & 2.54^{+0.44}_{-0.44} & 3.45^{+0.72}_{-0.66} \\
    4.35 & 4.40 & 4.10 & 4.70 & 4.13^{+2.15}_{-1.95} & 5.40^{+2.60}_{-2.57} & 1.81^{+0.58}_{-0.52} & 2.42^{+0.64}_{-0.68} \\
    4.92 & 5.10 & 4.70 & 5.50 & 2.51^{+0.89}_{-0.86} & 3.45^{+1.28}_{-1.31} & 0.97^{+0.20}_{-0.18} & 1.29^{+0.25}_{-0.26} \\
    6.00 & 6.00 & 5.50 & 6.50 & 0.65^{+0.30}_{-0.24} & 0.96^{+0.36}_{-0.37} & 0.21^{+0.04}_{-0.04} & 0.27^{+0.06}_{-0.06} \\
    \hline
  \end{tabular}
\end{table*}

\begin{table*}
  % rtg2.py and tabulate_emissivities_global.py.
  % Check these values once again.
  \caption{
    Comoving emissivities at 912\,\AA\ and 1450\,\AA\ obtained
    by fitting Equation~\eqref{eqn:e912fit} to the emissivities in the
    selected redshift bins from Table~\ref{tab:emissivity_bins}.
    Emissivities at $z<0.6$ and at $z>6.5$ are extrapolated assuming our best fits.
    The derived \ion{H}{i} photoionization rates (\textbf{Cite Eq.})
    are also given. Emissivity units are
    erg\ s$^{-1}$\ Hz$^{-1}$\ cMpc$^{-3}$, and photoionization rate
    units are s$^{-1}$. Statistical uncertainties are one-sigma (68.26\%).
    These values are shown in Figures~\ref{fig:e912_2} and \ref{fig:gammapi}.
    See Sections~\ref{sec:e912} and \ref{sec:gammahi} for more details.
    }
  \label{tab:gamma2}
  \begin{tabular}{d......}
    \hline
    \multicolumn{1}{c}{$z$} &
    \multicolumn{1}{c}{$\log_{10}\epsilon_{1450}$} &
    \multicolumn{1}{c}{$\log_{10}\epsilon_{1450}$} &
    \multicolumn{1}{c}{$\log_{10}\epsilon_{912}$} &
    \multicolumn{1}{c}{$\log_{10}\epsilon_{912}$} & 
    \multicolumn{1}{c}{$\log_{10}\Gamma_\mathrm{HI}$} &
    \multicolumn{1}{c}{$\log_{10}\Gamma_\mathrm{HI}$} \\ 
    &
    \multicolumn{1}{c}{$(M_{1450}<-18)$} &
    \multicolumn{1}{c}{$(M_{1450}<-21)$} &
    \multicolumn{1}{c}{$(M_{1450}<-18)$} &
    \multicolumn{1}{c}{$(M_{1450}<-21)$} &
    \multicolumn{1}{c}{$(M_{1450}<-18)$} &
    \multicolumn{1}{c}{$(M_{1450}<-21)$} \\
    \hline
    0.0 & 23.36^{+0.38}_{-0.20} & 23.18^{+0.25}_{-0.22} & 23.01^{+0.23}_{-0.16} & 22.83^{+0.12}_{-0.15} & -13.38^{+0.10}_{-0.06} & -13.57^{+0.06}_{-0.05} \\
    0.1 & 23.54^{+0.29}_{-0.16} & 23.35^{+0.18}_{-0.17} & 23.24^{+0.18}_{-0.13} & 23.06^{+0.09}_{-0.11} & -13.22^{+0.08}_{-0.06} & -13.41^{+0.05}_{-0.04} \\
    0.2 & 23.71^{+0.21}_{-0.14} & 23.51^{+0.13}_{-0.12} & 23.45^{+0.14}_{-0.10} & 23.27^{+0.07}_{-0.09} & -13.07^{+0.06}_{-0.05} & -13.26^{+0.04}_{-0.03} \\
    0.3 & 23.85^{+0.16}_{-0.10} & 23.66^{+0.10}_{-0.08} & 23.64^{+0.10}_{-0.07} & 23.46^{+0.05}_{-0.06} & -12.94^{+0.05}_{-0.04} & -13.12^{+0.03}_{-0.03} \\
    0.4 & 23.99^{+0.11}_{-0.08} & 23.80^{+0.07}_{-0.06} & 23.81^{+0.07}_{-0.06} & 23.63^{+0.04}_{-0.04} & -12.82^{+0.04}_{-0.04} & -12.99^{+0.03}_{-0.02} \\
    0.5 & 24.11^{+0.08}_{-0.06} & 23.93^{+0.05}_{-0.04} & 23.95^{+0.05}_{-0.04} & 23.78^{+0.03}_{-0.03} & -12.71^{+0.03}_{-0.03} & -12.87^{+0.02}_{-0.02} \\
    0.6 & 24.23^{+0.05}_{-0.04} & 24.06^{+0.03}_{-0.02} & 24.09^{+0.04}_{-0.03} & 23.93^{+0.02}_{-0.02} & -12.60^{+0.03}_{-0.03} & -12.76^{+0.02}_{-0.02} \\
    0.7 & 24.34^{+0.03}_{-0.03} & 24.18^{+0.02}_{-0.02} & 24.21^{+0.02}_{-0.03} & 24.06^{+0.01}_{-0.01} & -12.51^{+0.03}_{-0.03} & -12.66^{+0.02}_{-0.02} \\
    0.8 & 24.44^{+0.03}_{-0.02} & 24.30^{+0.01}_{-0.01} & 24.32^{+0.02}_{-0.02} & 24.18^{+0.01}_{-0.01} & -12.42^{+0.03}_{-0.03} & -12.57^{+0.02}_{-0.02} \\
    0.9 & 24.54^{+0.02}_{-0.02} & 24.41^{+0.01}_{-0.01} & 24.42^{+0.02}_{-0.01} & 24.29^{+0.01}_{-0.01} & -12.35^{+0.02}_{-0.03} & -12.48^{+0.02}_{-0.02} \\
    1.0 & 24.63^{+0.02}_{-0.02} & 24.51^{+0.01}_{-0.01} & 24.51^{+0.02}_{-0.01} & 24.38^{+0.01}_{-0.01} & -12.28^{+0.02}_{-0.03} & -12.40^{+0.02}_{-0.02} \\
    1.1 & 24.71^{+0.02}_{-0.02} & 24.60^{+0.01}_{-0.01} & 24.59^{+0.01}_{-0.01} & 24.47^{+0.01}_{-0.01} & -12.21^{+0.02}_{-0.03} & -12.33^{+0.02}_{-0.02} \\
    1.2 & 24.78^{+0.02}_{-0.02} & 24.68^{+0.01}_{-0.01} & 24.66^{+0.01}_{-0.01} & 24.55^{+0.01}_{-0.01} & -12.16^{+0.03}_{-0.03} & -12.27^{+0.02}_{-0.02} \\
    1.3 & 24.85^{+0.01}_{-0.02} & 24.75^{+0.01}_{-0.01} & 24.72^{+0.01}_{-0.01} & 24.63^{+0.01}_{-0.01} & -12.11^{+0.03}_{-0.03} & -12.22^{+0.02}_{-0.02} \\
    1.4 & 24.91^{+0.01}_{-0.02} & 24.82^{+0.01}_{-0.01} & 24.78^{+0.01}_{-0.01} & 24.69^{+0.01}_{-0.01} & -12.07^{+0.03}_{-0.03} & -12.18^{+0.02}_{-0.02} \\
    1.5 & 24.95^{+0.01}_{-0.01} & 24.87^{+0.01}_{-0.01} & 24.83^{+0.01}_{-0.01} & 24.74^{+0.01}_{-0.01} & -12.04^{+0.03}_{-0.03} & -12.15^{+0.02}_{-0.02} \\
    1.6 & 24.99^{+0.02}_{-0.01} & 24.91^{+0.01}_{-0.01} & 24.87^{+0.01}_{-0.01} & 24.79^{+0.01}_{-0.01} & -12.02^{+0.03}_{-0.03} & -12.12^{+0.02}_{-0.02} \\
    1.7 & 25.03^{+0.02}_{-0.02} & 24.95^{+0.01}_{-0.01} & 24.91^{+0.01}_{-0.01} & 24.83^{+0.01}_{-0.01} & -12.00^{+0.03}_{-0.03} & -12.11^{+0.02}_{-0.02} \\
    1.8 & 25.05^{+0.02}_{-0.02} & 24.98^{+0.01}_{-0.01} & 24.94^{+0.02}_{-0.01} & 24.86^{+0.01}_{-0.01} & -11.99^{+0.04}_{-0.04} & -12.10^{+0.02}_{-0.02} \\
    1.9 & 25.07^{+0.02}_{-0.02} & 24.99^{+0.01}_{-0.01} & 24.96^{+0.02}_{-0.02} & 24.88^{+0.01}_{-0.01} & -11.99^{+0.04}_{-0.04} & -12.09^{+0.03}_{-0.03} \\
    2.0 & 25.08^{+0.02}_{-0.02} & 25.01^{+0.01}_{-0.01} & 24.98^{+0.02}_{-0.02} & 24.90^{+0.01}_{-0.01} & -11.99^{+0.04}_{-0.05} & -12.09^{+0.03}_{-0.03} \\
    2.1 & 25.09^{+0.03}_{-0.02} & 25.01^{+0.02}_{-0.02} & 25.00^{+0.03}_{-0.02} & 24.91^{+0.02}_{-0.02} & -11.99^{+0.05}_{-0.05} & -12.10^{+0.03}_{-0.03} \\
    2.2 & 25.09^{+0.03}_{-0.03} & 25.01^{+0.02}_{-0.02} & 25.01^{+0.03}_{-0.03} & 24.91^{+0.02}_{-0.02} & -12.00^{+0.05}_{-0.06} & -12.11^{+0.04}_{-0.03} \\
    2.3 & 25.09^{+0.04}_{-0.04} & 25.01^{+0.02}_{-0.02} & 25.01^{+0.03}_{-0.03} & 24.91^{+0.02}_{-0.03} & -12.01^{+0.05}_{-0.07} & -12.12^{+0.04}_{-0.04} \\
    2.4 & 25.08^{+0.04}_{-0.04} & 25.00^{+0.03}_{-0.03} & 25.02^{+0.04}_{-0.04} & 24.91^{+0.03}_{-0.03} & -12.03^{+0.06}_{-0.08} & -12.14^{+0.04}_{-0.04} \\
    2.5 & 25.07^{+0.04}_{-0.05} & 24.99^{+0.03}_{-0.03} & 25.01^{+0.04}_{-0.04} & 24.89^{+0.03}_{-0.04} & -12.04^{+0.06}_{-0.08} & -12.16^{+0.05}_{-0.05} \\
    2.6 & 25.06^{+0.05}_{-0.06} & 24.97^{+0.04}_{-0.04} & 25.01^{+0.05}_{-0.05} & 24.88^{+0.04}_{-0.04} & -12.06^{+0.07}_{-0.09} & -12.18^{+0.05}_{-0.05} \\
    2.7 & 25.04^{+0.05}_{-0.07} & 24.96^{+0.04}_{-0.04} & 25.00^{+0.05}_{-0.05} & 24.86^{+0.04}_{-0.04} & -12.09^{+0.07}_{-0.10} & -12.21^{+0.05}_{-0.05} \\
    2.8 & 25.02^{+0.06}_{-0.07} & 24.93^{+0.05}_{-0.04} & 24.99^{+0.06}_{-0.06} & 24.84^{+0.05}_{-0.04} & -12.11^{+0.07}_{-0.11} & -12.24^{+0.06}_{-0.06} \\
    2.9 & 25.01^{+0.06}_{-0.08} & 24.91^{+0.05}_{-0.05} & 24.98^{+0.06}_{-0.07} & 24.81^{+0.05}_{-0.05} & -12.14^{+0.08}_{-0.12} & -12.27^{+0.06}_{-0.06} \\
    3.0 & 24.98^{+0.06}_{-0.09} & 24.88^{+0.05}_{-0.05} & 24.96^{+0.06}_{-0.07} & 24.78^{+0.05}_{-0.05} & -12.17^{+0.08}_{-0.12} & -12.31^{+0.06}_{-0.07} \\
    3.1 & 24.96^{+0.07}_{-0.10} & 24.85^{+0.05}_{-0.06} & 24.94^{+0.07}_{-0.08} & 24.75^{+0.06}_{-0.05} & -12.20^{+0.09}_{-0.13} & -12.35^{+0.07}_{-0.07} \\
    3.2 & 24.93^{+0.07}_{-0.10} & 24.82^{+0.06}_{-0.06} & 24.91^{+0.07}_{-0.09} & 24.71^{+0.06}_{-0.05} & -12.23^{+0.09}_{-0.14} & -12.39^{+0.07}_{-0.08} \\
    3.3 & 24.91^{+0.08}_{-0.11} & 24.79^{+0.06}_{-0.06} & 24.89^{+0.08}_{-0.09} & 24.68^{+0.06}_{-0.06} & -12.26^{+0.10}_{-0.15} & -12.43^{+0.07}_{-0.08} \\
    3.4 & 24.88^{+0.08}_{-0.12} & 24.75^{+0.06}_{-0.07} & 24.86^{+0.08}_{-0.09} & 24.64^{+0.07}_{-0.06} & -12.30^{+0.10}_{-0.17} & -12.47^{+0.07}_{-0.09} \\
    3.5 & 24.85^{+0.09}_{-0.14} & 24.72^{+0.07}_{-0.07} & 24.83^{+0.08}_{-0.10} & 24.59^{+0.07}_{-0.06} & -12.34^{+0.11}_{-0.18} & -12.52^{+0.08}_{-0.09} \\
    3.6 & 24.82^{+0.10}_{-0.15} & 24.68^{+0.07}_{-0.08} & 24.80^{+0.09}_{-0.11} & 24.55^{+0.07}_{-0.06} & -12.38^{+0.11}_{-0.19} & -12.56^{+0.08}_{-0.10} \\
    3.7 & 24.79^{+0.10}_{-0.16} & 24.64^{+0.07}_{-0.08} & 24.77^{+0.09}_{-0.11} & 24.51^{+0.07}_{-0.06} & -12.42^{+0.12}_{-0.20} & -12.61^{+0.08}_{-0.10} \\
    3.8 & 24.75^{+0.10}_{-0.16} & 24.59^{+0.07}_{-0.09} & 24.73^{+0.10}_{-0.12} & 24.47^{+0.07}_{-0.07} & -12.46^{+0.12}_{-0.21} & -12.66^{+0.08}_{-0.11} \\
    3.9 & 24.72^{+0.11}_{-0.17} & 24.55^{+0.07}_{-0.09} & 24.70^{+0.10}_{-0.13} & 24.42^{+0.07}_{-0.07} & -12.50^{+0.13}_{-0.22} & -12.72^{+0.08}_{-0.11} \\
    4.0 & 24.68^{+0.11}_{-0.18} & 24.51^{+0.07}_{-0.09} & 24.66^{+0.10}_{-0.14} & 24.37^{+0.07}_{-0.08} & -12.54^{+0.13}_{-0.24} & -12.77^{+0.08}_{-0.12} \\
    4.1 & 24.65^{+0.11}_{-0.20} & 24.46^{+0.07}_{-0.10} & 24.63^{+0.11}_{-0.16} & 24.33^{+0.06}_{-0.09} & -12.58^{+0.13}_{-0.25} & -12.82^{+0.08}_{-0.12} \\
    4.2 & 24.61^{+0.12}_{-0.21} & 24.42^{+0.07}_{-0.10} & 24.58^{+0.12}_{-0.16} & 24.28^{+0.06}_{-0.10} & -12.63^{+0.14}_{-0.27} & -12.88^{+0.08}_{-0.12} \\
    4.3 & 24.58^{+0.13}_{-0.23} & 24.37^{+0.07}_{-0.11} & 24.54^{+0.12}_{-0.17} & 24.23^{+0.07}_{-0.10} & -12.67^{+0.14}_{-0.29} & -12.94^{+0.09}_{-0.13} \\
    4.4 & 24.54^{+0.12}_{-0.24} & 24.32^{+0.08}_{-0.11} & 24.50^{+0.13}_{-0.18} & 24.17^{+0.07}_{-0.11} & -12.72^{+0.14}_{-0.30} & -13.00^{+0.09}_{-0.13} \\
    4.5 & 24.50^{+0.13}_{-0.26} & 24.27^{+0.08}_{-0.11} & 24.45^{+0.14}_{-0.19} & 24.12^{+0.07}_{-0.12} & -12.76^{+0.15}_{-0.32} & -13.06^{+0.09}_{-0.14} \\
    4.6 & 24.46^{+0.13}_{-0.28} & 24.22^{+0.08}_{-0.12} & 24.40^{+0.15}_{-0.19} & 24.07^{+0.08}_{-0.12} & -12.81^{+0.15}_{-0.34} & -13.12^{+0.10}_{-0.14} \\
    4.7 & 24.42^{+0.14}_{-0.30} & 24.16^{+0.09}_{-0.12} & 24.36^{+0.16}_{-0.21} & 24.01^{+0.08}_{-0.13} & -12.86^{+0.16}_{-0.36} & -13.18^{+0.11}_{-0.14} \\
    4.8 & 24.38^{+0.15}_{-0.31} & 24.11^{+0.09}_{-0.13} & 24.31^{+0.17}_{-0.22} & 23.96^{+0.09}_{-0.14} & -12.91^{+0.17}_{-0.38} & -13.25^{+0.11}_{-0.15} \\
    4.9 & 24.34^{+0.15}_{-0.33} & 24.05^{+0.10}_{-0.13} & 24.26^{+0.17}_{-0.24} & 23.90^{+0.09}_{-0.15} & -12.96^{+0.18}_{-0.39} & -13.32^{+0.12}_{-0.15} \\
    5.0 & 24.30^{+0.16}_{-0.35} & 23.99^{+0.11}_{-0.13} & 24.21^{+0.17}_{-0.25} & 23.84^{+0.10}_{-0.17} & -13.01^{+0.19}_{-0.41} & -13.38^{+0.13}_{-0.16} \\
    \hline
  \end{tabular}
\end{table*}

\begin{table*}
  \contcaption{}
  \begin{tabular}{d......}
    \hline
    \multicolumn{1}{c}{$z$} &    
    \multicolumn{1}{c}{$\log_{10}\epsilon_{1450}$} &
    \multicolumn{1}{c}{$\log_{10}\epsilon_{1450}$} &
    \multicolumn{1}{c}{$\log_{10}\epsilon_{912}$} &
    \multicolumn{1}{c}{$\log_{10}\epsilon_{912}$} & 
    \multicolumn{1}{c}{$\log_{10}\Gamma_\mathrm{HI}$} &
    \multicolumn{1}{c}{$\log_{10}\Gamma_\mathrm{HI}$} \\ 
    &
    \multicolumn{1}{c}{$(M_{1450}<-18)$} &
    \multicolumn{1}{c}{$(M_{1450}<-21)$} &
    \multicolumn{1}{c}{$(M_{1450}<-18)$} &
    \multicolumn{1}{c}{$(M_{1450}<-21)$} &
    \multicolumn{1}{c}{$(M_{1450}<-18)$} &
    \multicolumn{1}{c}{$(M_{1450}<-21)$} \\
    \hline
    5.1 & 24.26^{+0.17}_{-0.36} & 23.93^{+0.11}_{-0.14} & 24.16^{+0.18}_{-0.26} & 23.78^{+0.10}_{-0.19} & -13.06^{+0.20}_{-0.42} & -13.45^{+0.13}_{-0.16} \\
    5.2 & 24.21^{+0.19}_{-0.37} & 23.88^{+0.12}_{-0.14} & 24.10^{+0.19}_{-0.27} & 23.72^{+0.11}_{-0.20} & -13.12^{+0.21}_{-0.44} & -13.52^{+0.14}_{-0.17} \\
    5.3 & 24.17^{+0.20}_{-0.39} & 23.82^{+0.13}_{-0.15} & 24.05^{+0.19}_{-0.29} & 23.66^{+0.12}_{-0.22} & -13.17^{+0.22}_{-0.46} & -13.59^{+0.15}_{-0.18} \\
    5.4 & 24.13^{+0.21}_{-0.41} & 23.76^{+0.13}_{-0.16} & 24.00^{+0.20}_{-0.30} & 23.60^{+0.12}_{-0.23} & -13.22^{+0.23}_{-0.47} & -13.67^{+0.16}_{-0.19} \\
    5.5 & 24.09^{+0.22}_{-0.43} & 23.70^{+0.14}_{-0.17} & 23.94^{+0.21}_{-0.31} & 23.54^{+0.12}_{-0.24} & -13.29^{+0.24}_{-0.49} & -13.75^{+0.17}_{-0.20} \\
    5.6 & 24.04^{+0.23}_{-0.45} & 23.63^{+0.16}_{-0.17} & 23.89^{+0.22}_{-0.33} & 23.48^{+0.12}_{-0.25} & -13.37^{+0.25}_{-0.51} & -13.85^{+0.18}_{-0.21} \\
    5.7 & 24.00^{+0.24}_{-0.46} & 23.57^{+0.16}_{-0.18} & 23.84^{+0.23}_{-0.34} & 23.42^{+0.13}_{-0.26} & -13.45^{+0.26}_{-0.53} & -13.94^{+0.18}_{-0.22} \\
    5.8 & 23.95^{+0.25}_{-0.48} & 23.51^{+0.16}_{-0.20} & 23.78^{+0.24}_{-0.35} & 23.35^{+0.14}_{-0.28} & -13.53^{+0.27}_{-0.55} & -14.04^{+0.19}_{-0.23} \\
    5.9 & 23.91^{+0.26}_{-0.50} & 23.45^{+0.17}_{-0.20} & 23.73^{+0.24}_{-0.38} & 23.29^{+0.15}_{-0.30} & -13.62^{+0.28}_{-0.57} & -14.15^{+0.20}_{-0.24} \\
    6.0 & 23.87^{+0.27}_{-0.52} & 23.38^{+0.18}_{-0.20} & 23.67^{+0.25}_{-0.40} & 23.23^{+0.15}_{-0.32} & -13.70^{+0.29}_{-0.58} & -14.26^{+0.21}_{-0.24} \\
    6.1 & 23.82^{+0.29}_{-0.54} & 23.32^{+0.19}_{-0.21} & 23.61^{+0.27}_{-0.42} & 23.16^{+0.16}_{-0.34} & -13.80^{+0.31}_{-0.60} & -14.37^{+0.22}_{-0.25} \\
    6.2 & 23.76^{+0.30}_{-0.56} & 23.25^{+0.20}_{-0.22} & 23.55^{+0.28}_{-0.42} & 23.10^{+0.16}_{-0.36} & -13.90^{+0.32}_{-0.62} & -14.48^{+0.23}_{-0.26} \\
    6.3 & 23.71^{+0.32}_{-0.57} & 23.19^{+0.21}_{-0.23} & 23.49^{+0.29}_{-0.43} & 23.03^{+0.17}_{-0.38} & -14.00^{+0.34}_{-0.64} & -14.59^{+0.24}_{-0.28} \\
    6.4 & 23.66^{+0.33}_{-0.59} & 23.13^{+0.22}_{-0.24} & 23.42^{+0.30}_{-0.44} & 22.96^{+0.18}_{-0.39} & -14.10^{+0.35}_{-0.66} & -14.71^{+0.25}_{-0.29} \\
    6.5 & 23.62^{+0.34}_{-0.61} & 23.06^{+0.23}_{-0.25} & 23.36^{+0.31}_{-0.46} & 22.90^{+0.19}_{-0.40} & -14.20^{+0.36}_{-0.68} & -14.83^{+0.26}_{-0.31} \\
    6.6 & 23.56^{+0.36}_{-0.62} & 23.00^{+0.24}_{-0.26} & 23.30^{+0.33}_{-0.48} & 22.83^{+0.19}_{-0.42} & -14.31^{+0.38}_{-0.69} & -14.96^{+0.27}_{-0.32} \\
    6.7 & 23.52^{+0.37}_{-0.64} & 22.93^{+0.25}_{-0.28} & 23.23^{+0.34}_{-0.50} & 22.76^{+0.20}_{-0.44} & -14.42^{+0.39}_{-0.72} & -15.08^{+0.28}_{-0.34} \\
    6.8 & 23.47^{+0.38}_{-0.67} & 22.87^{+0.26}_{-0.29} & 23.15^{+0.36}_{-0.51} & 22.70^{+0.20}_{-0.45} & -14.52^{+0.40}_{-0.74} & -15.22^{+0.29}_{-0.36} \\
    6.9 & 23.42^{+0.40}_{-0.68} & 22.80^{+0.27}_{-0.30} & 23.08^{+0.39}_{-0.53} & 22.63^{+0.21}_{-0.47} & -14.64^{+0.41}_{-0.77} & -15.35^{+0.30}_{-0.37} \\
    7.0 & 23.37^{+0.41}_{-0.71} & 22.73^{+0.28}_{-0.31} & 23.01^{+0.41}_{-0.54} & 22.56^{+0.22}_{-0.49} & -14.74^{+0.42}_{-0.79} & -15.48^{+0.32}_{-0.40} \\
    7.1 & 23.33^{+0.41}_{-0.73} & 22.66^{+0.29}_{-0.33} & 22.94^{+0.42}_{-0.56} & 22.49^{+0.23}_{-0.52} & -14.86^{+0.43}_{-0.82} & -15.62^{+0.33}_{-0.42} \\
    7.2 & 23.28^{+0.43}_{-0.76} & 22.60^{+0.29}_{-0.35} & 22.87^{+0.44}_{-0.57} & 22.42^{+0.23}_{-0.54} & -14.97^{+0.45}_{-0.85} & -15.75^{+0.34}_{-0.45} \\
    7.3 & 23.22^{+0.45}_{-0.77} & 22.53^{+0.31}_{-0.36} & 22.80^{+0.47}_{-0.59} & 22.35^{+0.24}_{-0.57} & -15.09^{+0.47}_{-0.87} & -15.88^{+0.36}_{-0.47} \\
    7.4 & 23.17^{+0.46}_{-0.80} & 22.46^{+0.32}_{-0.38} & 22.73^{+0.49}_{-0.60} & 22.29^{+0.25}_{-0.59} & -15.20^{+0.48}_{-0.90} & -16.02^{+0.37}_{-0.50} \\
    7.5 & 23.13^{+0.47}_{-0.83} & 22.39^{+0.34}_{-0.39} & 22.67^{+0.51}_{-0.61} & 22.21^{+0.25}_{-0.62} & -15.30^{+0.49}_{-0.93} & -16.15^{+0.39}_{-0.52} \\
    7.6 & 23.09^{+0.47}_{-0.87} & 22.32^{+0.35}_{-0.41} & 22.60^{+0.52}_{-0.63} & 22.14^{+0.26}_{-0.64} & -15.41^{+0.49}_{-0.97} & -16.29^{+0.41}_{-0.55} \\
    7.7 & 23.04^{+0.48}_{-0.90} & 22.26^{+0.37}_{-0.42} & 22.53^{+0.54}_{-0.64} & 22.07^{+0.27}_{-0.66} & -15.52^{+0.50}_{-1.00} & -16.42^{+0.43}_{-0.57} \\
    7.8 & 22.99^{+0.50}_{-0.92} & 22.19^{+0.38}_{-0.44} & 22.46^{+0.57}_{-0.65} & 22.00^{+0.28}_{-0.68} & -15.63^{+0.52}_{-1.02} & -16.55^{+0.45}_{-0.60} \\
    7.9 & 22.94^{+0.51}_{-0.94} & 22.12^{+0.41}_{-0.46} & 22.38^{+0.58}_{-0.65} & 21.92^{+0.29}_{-0.70} & -15.74^{+0.53}_{-1.05} & -16.68^{+0.47}_{-0.63} \\
    8.0 & 22.88^{+0.53}_{-0.97} & 22.05^{+0.43}_{-0.48} & 22.31^{+0.60}_{-0.66} & 21.85^{+0.30}_{-0.73} & -15.84^{+0.54}_{-1.07} & -16.80^{+0.50}_{-0.66} \\
    8.1 & 22.83^{+0.54}_{-0.99} & 21.97^{+0.44}_{-0.49} & 22.24^{+0.61}_{-0.68} & 21.78^{+0.31}_{-0.76} & -15.95^{+0.56}_{-1.10} & -16.93^{+0.52}_{-0.68} \\
    8.2 & 22.78^{+0.56}_{-1.02} & 21.90^{+0.46}_{-0.51} & 22.17^{+0.63}_{-0.70} & 21.71^{+0.32}_{-0.78} & -16.05^{+0.57}_{-1.12} & -17.06^{+0.54}_{-0.71} \\
    8.3 & 22.73^{+0.57}_{-1.04} & 21.82^{+0.48}_{-0.52} & 22.10^{+0.65}_{-0.71} & 21.63^{+0.33}_{-0.81} & -16.15^{+0.58}_{-1.15} & -17.20^{+0.56}_{-0.72} \\
    8.4 & 22.68^{+0.58}_{-1.07} & 21.74^{+0.50}_{-0.53} & 22.02^{+0.68}_{-0.72} & 21.56^{+0.33}_{-0.84} & -16.25^{+0.59}_{-1.18} & -17.33^{+0.58}_{-0.74} \\
    8.5 & 22.63^{+0.59}_{-1.10} & 21.66^{+0.52}_{-0.54} & 21.94^{+0.71}_{-0.73} & 21.49^{+0.34}_{-0.86} & -16.35^{+0.61}_{-1.21} & -17.45^{+0.60}_{-0.76} \\
    8.6 & 22.59^{+0.60}_{-1.13} & 21.59^{+0.53}_{-0.55} & 21.86^{+0.75}_{-0.73} & 21.41^{+0.35}_{-0.89} & -16.45^{+0.61}_{-1.24} & -17.57^{+0.61}_{-0.78} \\
    8.7 & 22.54^{+0.61}_{-1.16} & 21.52^{+0.54}_{-0.57} & 21.78^{+0.78}_{-0.73} & 21.34^{+0.36}_{-0.92} & -16.55^{+0.63}_{-1.26} & -17.69^{+0.63}_{-0.80} \\
    8.8 & 22.48^{+0.63}_{-1.18} & 21.45^{+0.56}_{-0.58} & 21.69^{+0.81}_{-0.73} & 21.26^{+0.37}_{-0.94} & -16.65^{+0.64}_{-1.28} & -17.82^{+0.64}_{-0.82} \\
    8.9 & 22.42^{+0.65}_{-1.20} & 21.37^{+0.58}_{-0.59} & 21.61^{+0.84}_{-0.74} & 21.19^{+0.38}_{-0.97} & -16.75^{+0.66}_{-1.31} & -17.94^{+0.67}_{-0.84} \\
    9.0 & 22.37^{+0.66}_{-1.23} & 21.30^{+0.60}_{-0.61} & 21.53^{+0.87}_{-0.75} & 21.11^{+0.39}_{-0.99} & -16.85^{+0.67}_{-1.33} & -18.06^{+0.69}_{-0.86} \\
    9.1 & 22.33^{+0.67}_{-1.26} & 21.22^{+0.62}_{-0.62} & 21.45^{+0.90}_{-0.76} & 21.04^{+0.40}_{-1.02} & -16.94^{+0.68}_{-1.36} & -18.18^{+0.72}_{-0.89} \\
    9.2 & 22.27^{+0.68}_{-1.28} & 21.15^{+0.65}_{-0.64} & 21.36^{+0.93}_{-0.77} & 20.96^{+0.41}_{-1.05} & -17.04^{+0.69}_{-1.38} & -18.30^{+0.74}_{-0.91} \\
    9.3 & 22.22^{+0.69}_{-1.31} & 21.07^{+0.67}_{-0.66} & 21.29^{+0.95}_{-0.80} & 20.89^{+0.43}_{-1.08} & -17.14^{+0.71}_{-1.41} & -18.41^{+0.77}_{-0.93} \\
    9.4 & 22.16^{+0.70}_{-1.33} & 21.00^{+0.69}_{-0.67} & 21.21^{+0.97}_{-0.82} & 20.81^{+0.44}_{-1.10} & -17.23^{+0.72}_{-1.43} & -18.53^{+0.79}_{-0.95} \\
    9.5 & 22.11^{+0.72}_{-1.36} & 20.92^{+0.71}_{-0.68} & 21.13^{+1.00}_{-0.83} & 20.73^{+0.45}_{-1.13} & -17.33^{+0.74}_{-1.45} & -18.65^{+0.81}_{-0.96} \\
    9.6 & 22.06^{+0.73}_{-1.38} & 20.84^{+0.73}_{-0.70} & 21.05^{+1.03}_{-0.85} & 20.65^{+0.46}_{-1.16} & -17.42^{+0.75}_{-1.47} & -18.77^{+0.83}_{-0.98} \\
    9.7 & 22.00^{+0.74}_{-1.40} & 20.76^{+0.75}_{-0.72} & 20.97^{+1.06}_{-0.86} & 20.57^{+0.48}_{-1.18} & -17.52^{+0.76}_{-1.49} & -18.89^{+0.85}_{-1.01} \\
    9.8 & 21.95^{+0.76}_{-1.42} & 20.69^{+0.77}_{-0.73} & 20.88^{+1.10}_{-0.88} & 20.49^{+0.49}_{-1.21} & -17.61^{+0.78}_{-1.51} & -19.01^{+0.87}_{-1.03} \\
    9.9 & 21.90^{+0.76}_{-1.45} & 20.61^{+0.79}_{-0.75} & 20.80^{+1.13}_{-0.89} & 20.42^{+0.50}_{-1.24} & -17.70^{+0.79}_{-1.54} & -19.12^{+0.89}_{-1.06} \\
    10.0 & 21.85^{+0.77}_{-1.48} & 20.53^{+0.81}_{-0.77} & 20.72^{+1.16}_{-0.90} & 20.34^{+0.51}_{-1.26} & -17.79^{+0.80}_{-1.57} & -19.23^{+0.91}_{-1.09} \\
    \hline
  \end{tabular}
\end{table*}

\begin{table*}
  \contcaption{}
  \begin{tabular}{d......}
    \hline
    \multicolumn{1}{c}{$z$} &    
    \multicolumn{1}{c}{$\log_{10}\epsilon_{1450}$} &
    \multicolumn{1}{c}{$\log_{10}\epsilon_{1450}$} &
    \multicolumn{1}{c}{$\log_{10}\epsilon_{912}$} &
    \multicolumn{1}{c}{$\log_{10}\epsilon_{912}$} & 
    \multicolumn{1}{c}{$\log_{10}\Gamma_\mathrm{HI}$} &
    \multicolumn{1}{c}{$\log_{10}\Gamma_\mathrm{HI}$} \\ 
    &
    \multicolumn{1}{c}{$(M_{1450}<-18)$} &
    \multicolumn{1}{c}{$(M_{1450}<-21)$} &
    \multicolumn{1}{c}{$(M_{1450}<-18)$} &
    \multicolumn{1}{c}{$(M_{1450}<-21)$} &
    \multicolumn{1}{c}{$(M_{1450}<-18)$} &
    \multicolumn{1}{c}{$(M_{1450}<-21)$} \\
    \hline
    10.0 & 21.85^{+0.77}_{-1.48} & 20.53^{+0.81}_{-0.77} & 20.72^{+1.16}_{-0.90} & 20.34^{+0.51}_{-1.26} & -17.79^{+0.80}_{-1.57} & -19.23^{+0.91}_{-1.09} \\
    10.1 & 21.80^{+0.80}_{-1.50} & 20.46^{+0.83}_{-0.79} & 20.63^{+1.19}_{-0.91} & 20.26^{+0.52}_{-1.29} & -17.88^{+0.83}_{-1.58} & -19.35^{+0.93}_{-1.11} \\
    10.2 & 21.74^{+0.83}_{-1.52} & 20.38^{+0.85}_{-0.81} & 20.55^{+1.23}_{-0.93} & 20.18^{+0.53}_{-1.31} & -17.97^{+0.85}_{-1.60} & -19.46^{+0.96}_{-1.13} \\
    10.3 & 21.69^{+0.85}_{-1.54} & 20.31^{+0.88}_{-0.83} & 20.47^{+1.26}_{-0.94} & 20.10^{+0.54}_{-1.34} & -18.07^{+0.87}_{-1.62} & -19.57^{+0.98}_{-1.16} \\
    10.4 & 21.63^{+0.87}_{-1.56} & 20.23^{+0.90}_{-0.85} & 20.39^{+1.29}_{-0.95} & 20.03^{+0.55}_{-1.36} & -18.16^{+0.89}_{-1.64} & -19.69^{+1.01}_{-1.18} \\
    10.5 & 21.58^{+0.89}_{-1.59} & 20.15^{+0.93}_{-0.87} & 20.31^{+1.31}_{-0.96} & 19.95^{+0.56}_{-1.39} & -18.25^{+0.90}_{-1.67} & -19.80^{+1.03}_{-1.21} \\
    10.6 & 21.52^{+0.90}_{-1.62} & 20.07^{+0.95}_{-0.89} & 20.22^{+1.35}_{-0.97} & 19.87^{+0.58}_{-1.42} & -18.34^{+0.91}_{-1.69} & -19.91^{+1.05}_{-1.23} \\
    10.7 & 21.47^{+0.90}_{-1.65} & 19.99^{+0.97}_{-0.90} & 20.13^{+1.38}_{-0.98} & 19.79^{+0.59}_{-1.44} & -18.43^{+0.92}_{-1.72} & -20.03^{+1.08}_{-1.25} \\
    10.8 & 21.42^{+0.91}_{-1.67} & 19.91^{+0.99}_{-0.92} & 20.04^{+1.42}_{-0.99} & 19.71^{+0.60}_{-1.47} & -18.52^{+0.93}_{-1.74} & -20.15^{+1.10}_{-1.27} \\
    10.9 & 21.36^{+0.93}_{-1.70} & 19.83^{+1.01}_{-0.94} & 19.95^{+1.46}_{-1.00} & 19.63^{+0.61}_{-1.50} & -18.62^{+0.95}_{-1.76} & -20.26^{+1.13}_{-1.29} \\
    11.0 & 21.30^{+0.94}_{-1.72} & 19.75^{+1.04}_{-0.96} & 19.86^{+1.50}_{-1.01} & 19.55^{+0.62}_{-1.53} & -18.71^{+0.96}_{-1.78} & -20.38^{+1.16}_{-1.32} \\
    11.1 & 21.24^{+0.96}_{-1.74} & 19.67^{+1.08}_{-0.98} & 19.78^{+1.53}_{-1.03} & 19.47^{+0.63}_{-1.56} & -18.81^{+0.98}_{-1.81} & -20.49^{+1.20}_{-1.34} \\
    11.2 & 21.18^{+0.97}_{-1.76} & 19.59^{+1.12}_{-1.00} & 19.69^{+1.56}_{-1.04} & 19.39^{+0.64}_{-1.59} & -18.90^{+0.99}_{-1.82} & -20.60^{+1.24}_{-1.36} \\
    11.3 & 21.12^{+0.99}_{-1.78} & 19.51^{+1.15}_{-1.01} & 19.60^{+1.60}_{-1.05} & 19.31^{+0.66}_{-1.62} & -18.99^{+1.01}_{-1.84} & -20.72^{+1.28}_{-1.39} \\
    11.4 & 21.06^{+1.00}_{-1.79} & 19.43^{+1.20}_{-1.03} & 19.51^{+1.63}_{-1.06} & 19.23^{+0.67}_{-1.65} & -19.09^{+1.02}_{-1.85} & -20.83^{+1.32}_{-1.40} \\
    11.5 & 21.00^{+1.02}_{-1.82} & 19.35^{+1.23}_{-1.05} & 19.42^{+1.67}_{-1.07} & 19.15^{+0.68}_{-1.68} & -19.18^{+1.04}_{-1.88} & -20.95^{+1.35}_{-1.42} \\
    11.6 & 20.94^{+1.03}_{-1.84} & 19.27^{+1.27}_{-1.06} & 19.32^{+1.71}_{-1.08} & 19.07^{+0.69}_{-1.71} & -19.27^{+1.05}_{-1.90} & -21.07^{+1.39}_{-1.44} \\
    11.7 & 20.88^{+1.05}_{-1.86} & 19.18^{+1.31}_{-1.08} & 19.23^{+1.75}_{-1.09} & 18.99^{+0.70}_{-1.74} & -19.37^{+1.07}_{-1.92} & -21.18^{+1.43}_{-1.45} \\
    11.8 & 20.82^{+1.06}_{-1.88} & 19.10^{+1.35}_{-1.09} & 19.14^{+1.79}_{-1.10} & 18.91^{+0.71}_{-1.77} & -19.46^{+1.09}_{-1.94} & -21.30^{+1.46}_{-1.47} \\
    11.9 & 20.76^{+1.08}_{-1.90} & 19.01^{+1.38}_{-1.10} & 19.04^{+1.83}_{-1.11} & 18.83^{+0.72}_{-1.80} & -19.56^{+1.10}_{-1.96} & -21.41^{+1.49}_{-1.48} \\
    12.0 & 20.70^{+1.10}_{-1.93} & 18.93^{+1.41}_{-1.11} & 18.95^{+1.86}_{-1.13} & 18.75^{+0.73}_{-1.82} & -19.65^{+1.12}_{-1.98} & -21.52^{+1.52}_{-1.50} \\
    12.1 & 20.64^{+1.12}_{-1.95} & 18.86^{+1.43}_{-1.14} & 18.87^{+1.90}_{-1.14} & 18.67^{+0.74}_{-1.85} & -19.74^{+1.14}_{-2.00} & -21.63^{+1.55}_{-1.52} \\
    12.2 & 20.58^{+1.14}_{-1.97} & 18.78^{+1.46}_{-1.16} & 18.78^{+1.93}_{-1.16} & 18.59^{+0.76}_{-1.88} & -19.83^{+1.16}_{-2.03} & -21.74^{+1.57}_{-1.54} \\
    12.3 & 20.52^{+1.16}_{-1.99} & 18.70^{+1.50}_{-1.18} & 18.69^{+1.96}_{-1.18} & 18.51^{+0.77}_{-1.91} & -19.92^{+1.18}_{-2.05} & -21.85^{+1.61}_{-1.56} \\
    12.4 & 20.46^{+1.16}_{-2.02} & 18.62^{+1.53}_{-1.20} & 18.60^{+1.99}_{-1.20} & 18.42^{+0.78}_{-1.93} & -20.01^{+1.19}_{-2.08} & -21.96^{+1.64}_{-1.58} \\
    12.5 & 20.41^{+1.17}_{-2.05} & 18.54^{+1.56}_{-1.22} & 18.52^{+2.02}_{-1.22} & 18.34^{+0.80}_{-1.96} & -20.10^{+1.20}_{-2.10} & -22.07^{+1.67}_{-1.60} \\
    12.6 & 20.35^{+1.19}_{-2.07} & 18.46^{+1.59}_{-1.24} & 18.43^{+2.05}_{-1.24} & 18.25^{+0.81}_{-1.98} & -20.19^{+1.22}_{-2.12} & -22.18^{+1.70}_{-1.63} \\
    12.7 & 20.29^{+1.21}_{-2.09} & 18.38^{+1.62}_{-1.26} & 18.35^{+2.07}_{-1.27} & 18.17^{+0.82}_{-2.01} & -20.28^{+1.24}_{-2.14} & -22.29^{+1.72}_{-1.65} \\
    12.8 & 20.23^{+1.23}_{-2.12} & 18.29^{+1.65}_{-1.28} & 18.27^{+2.10}_{-1.29} & 18.09^{+0.83}_{-2.04} & -20.37^{+1.26}_{-2.16} & -22.41^{+1.76}_{-1.66} \\
    12.9 & 20.17^{+1.25}_{-2.14} & 18.21^{+1.68}_{-1.29} & 18.19^{+2.12}_{-1.32} & 18.01^{+0.84}_{-2.07} & -20.46^{+1.28}_{-2.19} & -22.52^{+1.79}_{-1.68} \\
    13.0 & 20.11^{+1.28}_{-2.16} & 18.12^{+1.71}_{-1.31} & 18.11^{+2.15}_{-1.35} & 17.92^{+0.85}_{-2.09} & -20.55^{+1.30}_{-2.21} & -22.64^{+1.82}_{-1.69} \\
    13.1 & 20.05^{+1.30}_{-2.18} & 18.04^{+1.75}_{-1.32} & 18.02^{+2.18}_{-1.37} & 17.84^{+0.86}_{-2.12} & -20.64^{+1.31}_{-2.23} & -22.75^{+1.86}_{-1.71} \\
    13.2 & 19.99^{+1.31}_{-2.21} & 17.95^{+1.79}_{-1.34} & 17.93^{+2.21}_{-1.39} & 17.76^{+0.88}_{-2.15} & -20.73^{+1.33}_{-2.26} & -22.86^{+1.89}_{-1.72} \\
    13.3 & 19.93^{+1.32}_{-2.24} & 17.87^{+1.83}_{-1.36} & 17.84^{+2.24}_{-1.41} & 17.67^{+0.89}_{-2.17} & -20.81^{+1.34}_{-2.29} & -22.98^{+1.93}_{-1.74} \\
    13.4 & 19.88^{+1.33}_{-2.27} & 17.79^{+1.86}_{-1.37} & 17.76^{+2.27}_{-1.43} & 17.59^{+0.91}_{-2.20} & -20.90^{+1.34}_{-2.32} & -23.09^{+1.96}_{-1.75} \\
    13.5 & 19.83^{+1.34}_{-2.30} & 17.70^{+1.90}_{-1.39} & 17.66^{+2.30}_{-1.45} & 17.51^{+0.92}_{-2.23} & -20.98^{+1.35}_{-2.35} & -23.21^{+2.00}_{-1.77} \\
    13.6 & 19.78^{+1.34}_{-2.33} & 17.62^{+1.93}_{-1.41} & 17.57^{+2.33}_{-1.46} & 17.42^{+0.93}_{-2.26} & -21.06^{+1.36}_{-2.38} & -23.32^{+2.03}_{-1.78} \\
    13.7 & 19.72^{+1.35}_{-2.36} & 17.53^{+1.97}_{-1.42} & 17.48^{+2.36}_{-1.48} & 17.34^{+0.95}_{-2.29} & -21.15^{+1.37}_{-2.40} & -23.43^{+2.06}_{-1.79} \\
    13.8 & 19.67^{+1.37}_{-2.38} & 17.45^{+2.00}_{-1.44} & 17.39^{+2.39}_{-1.50} & 17.25^{+0.96}_{-2.32} & -21.24^{+1.38}_{-2.43} & -23.54^{+2.09}_{-1.81} \\
    13.9 & 19.61^{+1.38}_{-2.41} & 17.37^{+2.03}_{-1.46} & 17.30^{+2.42}_{-1.52} & 17.17^{+0.97}_{-2.35} & -21.32^{+1.39}_{-2.46} & -23.65^{+2.12}_{-1.83} \\
    14.0 & 19.55^{+1.39}_{-2.44} & 17.29^{+2.06}_{-1.48} & 17.21^{+2.45}_{-1.54} & 17.09^{+0.99}_{-2.38} & -21.41^{+1.40}_{-2.48} & -23.76^{+2.15}_{-1.84} \\
    14.1 & 19.50^{+1.40}_{-2.46} & 17.20^{+2.10}_{-1.50} & 17.12^{+2.48}_{-1.55} & 17.00^{+1.00}_{-2.41} & -21.49^{+1.41}_{-2.51} & -23.87^{+2.18}_{-1.86} \\
    14.2 & 19.44^{+1.41}_{-2.50} & 17.12^{+2.13}_{-1.52} & 17.03^{+2.51}_{-1.56} & 16.92^{+1.01}_{-2.44} & -21.58^{+1.42}_{-2.54} & -23.98^{+2.21}_{-1.87} \\
    14.3 & 19.39^{+1.42}_{-2.53} & 17.04^{+2.16}_{-1.54} & 16.94^{+2.54}_{-1.57} & 16.83^{+1.02}_{-2.47} & -21.66^{+1.43}_{-2.57} & -24.09^{+2.24}_{-1.89} \\
    14.4 & 19.34^{+1.43}_{-2.56} & 16.96^{+2.19}_{-1.56} & 16.85^{+2.57}_{-1.58} & 16.75^{+1.04}_{-2.50} & -21.75^{+1.44}_{-2.60} & -24.20^{+2.26}_{-1.91} \\
    14.5 & 19.29^{+1.44}_{-2.60} & 16.88^{+2.22}_{-1.58} & 16.76^{+2.60}_{-1.59} & 16.66^{+1.05}_{-2.53} & -21.84^{+1.45}_{-2.63} & -24.32^{+2.29}_{-1.93} \\
    14.6 & 19.24^{+1.46}_{-2.63} & 16.80^{+2.25}_{-1.60} & 16.67^{+2.63}_{-1.60} & 16.58^{+1.06}_{-2.56} & -21.93^{+1.47}_{-2.66} & -24.43^{+2.32}_{-1.94} \\
    14.7 & 19.18^{+1.47}_{-2.65} & 16.72^{+2.28}_{-1.62} & 16.57^{+2.66}_{-1.61} & 16.50^{+1.07}_{-2.59} & -22.03^{+1.48}_{-2.68} & -24.56^{+2.34}_{-1.96} \\
    14.8 & 19.13^{+1.48}_{-2.69} & 16.64^{+2.31}_{-1.64} & 16.48^{+2.69}_{-1.63} & 16.41^{+1.09}_{-2.61} & -22.15^{+1.48}_{-2.71} & -24.69^{+2.37}_{-1.97} \\
    14.9 & 19.08^{+1.48}_{-2.72} & 16.56^{+2.34}_{-1.66} & 16.39^{+2.72}_{-1.64} & 16.33^{+1.10}_{-2.64} & -22.28^{+1.48}_{-2.74} & -24.85^{+2.39}_{-1.99} \\
    15.0 & 19.02^{+1.49}_{-2.75} & 16.48^{+2.37}_{-1.68} & 16.30^{+2.75}_{-1.66} & 16.24^{+1.11}_{-2.67} & -22.50^{+1.49}_{-2.76} & -25.08^{+2.41}_{-2.00} \\
    \hline
  \end{tabular}
\end{table*}

\section{Code and data}
\label{sec:code}

We make the code and data used in this work publicly available at
\url{https://github.com/gkulkarni/QLF}.  This includes homogenised AGN
catalogues and selection functions and the code used for developing
and analysing luminosity function models.
%% Make URL public. Set appropriate license.  Take permission from
%% other authors.  Set up a DOI.

\bibliographystyle{mnras}
\bibliography{refs}

\bsp
\label{lastpage}
\end{document}


