\documentclass[a4paper,fleqn,usenatbib]{mnras}
\usepackage[T1]{fontenc}
\usepackage{ae,aecompl}
\usepackage{graphicx}	
\usepackage{amsmath}	
\usepackage{amssymb}
\usepackage{pdflscape}
\usepackage{siunitx}
\def\lya{Ly$\alpha$~}

\title[AGN luminosity function]{Evolution of the AGN UV luminosity function}

\author[Kulkarni et al.]{{Girish Kulkarni$^{1}$\thanks{Email:
      kulkarni@ast.cam.ac.uk}, G\'abor Worseck$^{2}$ and Joseph
    F.~Hennawi$^{3}$} \\ $^1$Institute of Astronomy and Kavli
  Institute of Cosmology, University of Cambridge, Madingley Road,
  Cambridge CB3 0HA, UK \\ $^2$Max Planck Institute for Astronomy,
  K\"onigstuhl 17, D-69117 Heidelberg, Germany\\ $^3$Department of
  Physics, Broida Hall, UC Santa Barbara, Santa Barbara, CA 93106-9530
  USA}

\date{Accepted ---. Received ---; in original form ---}

\pubyear{2016}

\begin{document}
\label{firstpage}
\pagerange{\pageref{firstpage}--\pageref{lastpage}}
\maketitle

\begin{abstract}
  Lorem ipsum dolor sit amet, consectetur adipiscing elit. Curabitur
  eget nisi augue. Vivamus quis purus quis massa tempor posuere non
  quis magna. Aenean eleifend, metus eget facilisis faucibus, turpis
  erat suscipit tortor, ut dapibus nulla neque ac sapien. Suspendisse
  luctus eros eu quam laoreet, vel dapibus sem porttitor. Mauris nec
  massa ultrices, porttitor nulla at, euismod diam. In ultricies
  malesuada mauris ac facilisis. Nulla quis suscipit diam, vitae
  semper tortor. Proin nec nulla at massa egestas porta. Cras ac arcu
  in velit placerat facilisis fringilla in sem. Mauris finibus, mauris
  ut pretium malesuada, massa urna mattis nibh, at placerat ligula
  elit ac odio.  Nunc vel leo arcu. Nullam sagittis tincidunt
  interdum. Mauris sed lacus cursus, dapibus enim non, sagittis
  leo. Aliquam sit amet ex ut quam iaculis sollicitudin. Vivamus ut
  pharetra dolor. Vivamus id nibh leo. Vestibulum consectetur eu lorem
  eu tincidunt. Sed eget nibh orci. Pellentesque sit amet blandit
  massa, vel tempor neque.
\end{abstract}

\begin{keywords}
quasars
\end{keywords}

%% * Possible outline of paper:
%% 
%% Introduction 
%% Data
%% Method
%%   Individual fits
%%   Global fits
%% Contribution to reionizatin 
%% Conclusions
%% 
%% * Optional possibilities:
%%
%% Bolometric LF
%% Public code description 
%% More global models: PLE, PDE, Trenti, White-Conroy
%% Goodness of fit 
%% BH growth 
%% Comparison with galaxy LF
%% Comparison with previous work 

\section{Introduction}

The luminosity function of active galactic nuclei (AGN) and its
evolution over cosmological time scales has been a matter of central
interest of a large body of work over the last four decades
\citep[e.g.,][]{1978A&A....68...17M, 1983ApJ...269..352S,
  1988ApJ...325...92K, 1988MNRAS.235..935B, 1993ApJ...406L..43H,
  1994ApJ...421..412W, 1995AJ....110...68S, 1995AJ....110.2553K,
  1995ApJ...438..623P, 2000MNRAS.317.1014B, 2001AJ....121...54F,
  2004AJ....128..515F, 2006AJ....131.2766R, 2007ApJ...654..731H,
  2009MNRAS.392...19C, 2010AJ....139..906W, 2011ApJ...728L..26G,
  2013ApJ...773...14R, 2013ApJ...768..105M, 2015AA...578A..83G,
  2015ApJ...798...28K, 2016ApJ...829...33Y, 2016ApJ...833..222J,
  2017MNRAS.466.1160M}.  Determination of the AGN luminosity function
constrains models of the build-up of supermassive black holes
\citep{2015MNRAS.452..575S, 2016MNRAS.462..190R}.  Due to the
incidence of supermassive black holes in most galaxies, the tight
scaling relations observed to exist between the mass of these black
holes and properties of their host galaxies
\citep{2013ARA&A..51..511K, 2013ApJ...764..184M}, and the increasing
consensus that AGN activity feeds back on the host galaxy evolution,
the AGN luminosity function also constrains models of galaxy
formation.  Finally, thanks to their relative brightness and high
Lyman continuum escape fraction, the luminosity function of AGN
determines their influence on the temperature and ionization state of
the intergalactic medium (IGM) over large scales, possibly even
primarily driving hydrogen and helium reionization.

The claimed discovery of 19 low-luminosity ($M_{1450}>-22.6$) AGN
between redshifts $z=4.1$ and $6.3$ by \citet{2015AA...578A..83G}
using a novel X-ray/NIR selection criterion, has renewed the interest
in the evolution of the UV luminosity function of AGN.  This finding
suggested that the faint end of the quasar UV luminosity function is
steeper at these redshifts than previously thought
\citep{2007ApJ...654..731H, 2012ApJ...746..125H}.  Using far-UV
spectral slopes from composite spectra of low-redshift quasars, and
assuming a Lyman continuum escape fraction of 100\%,
\citet{2015AA...578A..83G} argued that AGN brighter than
$M_{1450}=-18$ can potentially produce all of the metagalactic
hydrogen photoionization rate inferred from the \lya forest at
$4<z<6$.  Additional indications of a significant presence of AGN at
high redshift ($z\sim 6$) have emerged.  First,
\citet{2015MNRAS.447.3402B} reported a large scatter in the \lya
opacity between different sightlines close to redshift
$z=6$. \cite{2015MNRAS.453.2943C} showed that these opacity
fluctuations extend to substantially larger scales ($\gtrsim 50\,
h^{-1}$cMpc) than expected in reionization histories dominated by
low-luminosity galaxies (see also \citealt{2016MNRAS.460.1328D}).
\citet{2017MNRAS.465.3429C} further demonstrated that opacity
fluctuations on such large scales arise naturally if there is a
significant contribution ($\gtrsim 50\%$) of AGN to the ionising
emissivity at the redshift of the observed opacity fluctuations ($z
\sim 5.5$--$6$) as would be expected for a QSO luminosity that is
consistent with the measurements of \citet{2015AA...578A..83G}.
Second, measurements of the Lyman continuum escape fraction from
high-redshift galaxies are still elusive.  Although high-redshift
galaxies as faint as rest-frame UV magnitude $M_\mathrm{UV}=-12.5$
($L\sim 10^{-3}L^*$) at $z=6$ \citep{2017ApJ...835..113L} and
redshifts as high as $z=11.1$ \citep{2016ApJ...819..129O} have now
been reported, the escape of Lyman continuum photons has been detected
in only a small number of comparatively bright ($L>0.5L^*$)
low-redshift ($z < 4$) galaxies.  In these galaxies, the escape
fraction is typically found to be 2--20\% \citep{2010ApJ...725.1011V,
  2011ApJ...736...41B, 2015ApJ...804...17S, 2015ApJ...810..107M,
  2016A&A...585A..48G, 2017MNRAS.468..389J, 2017MNRAS.465..316M} but
reionization would require escape fraction of about 20\% in galaxies
down to $M_\mathrm{UV}=-13$ \citep{2016PASA...33...37F,
  2015ApJ...802L..19R, 2016MNRAS.457.4051K}.  Finally, incidence of
high-redshift AGN is also consistent with the shallow bright-end
slopes of the high-redshift ($z\sim 7$) UV luminosity function of
galaxies relative to a Schechter-function representation
\citep{2012MNRAS.426.2772B, 2014MNRAS.440.2810B, 2014ApJ...792...76B,
  2015MNRAS.452.1817B} and the hard spectra of these bright galaxies
\citep{2015MNRAS.450.1846S, 2015MNRAS.454.1393S, 2017MNRAS.464..469S}.

However, these arguements in favour of a higher incidence of
UV-emitting AGN at high redshit, have been contested by several
observations.  \citet{2016arXiv160706467D} considered the effect of
AGN-dominated reionization on the Ly$\alpha$ opacity at $z>5$,
He~\textsc{ii} Ly$\alpha$ opacity at $z\sim 3.1$--$3.3$, and the
thermal history of the IGM.  In agreement with
\citet{2015MNRAS.453.2943C}, these authors found that AGN did provide
a plausible explanation for the large fluctuations in the Ly$\alpha$
opacity at $z>5$.  However, they found that reionization of
He~\textsc{ii} occurs much earlier in these AGN-dominated models (see
also \citealt{2016arXiv160602719M}).  For instance, in the model of
\citet{2015ApJ...813L...8M}, He~\textsc{ii} reionization is complete
at $z=4.5$, compared to $z=3$ in the standard scenario
\citep{2012ApJ...746..125H}.  This early Helium reionization could
result in higher IGM temperatures due to the associated photoheating.
The temperature of the IGM at mean density is enhanced in
AGN-dominated models by factors of $\sim 2$ relative to the standard
models for $z=3.5$--$5$, in conflict with measurements.  This
inconsistency could be avoided by reducing the escape fraction of 4~Ry
photons in AGN, but it is not clear if this can be achieved while
requiring a 100\% escape fraction of 1~Ry photons in order to explain
the Ly$\alpha$ opacity fluctuations.  Further evidence against
AGN-dominated reionization models has emerged from metal-line
absorbers at $z\sim 6$.  In their cosmological radiation
hydrodynamical simulations, \citet{2016MNRAS.459.2299F} find that the
hard spectral slopes of UV backgrounds in AGN-only reionization models
produce too many C~\textsc{iv} absorption systems relative to
Si~\textsc{iv} and C~\textsc{ii} at $z\sim 6$.  However, these
simulations assume an $L_\nu\propto\nu^{-1.57}$ AGN SED at extreme UV
\citep{2001AJ....122..549V, 2002ApJ...579..500T, 2012ApJ...746..125H}.
This slope is marginally harder than recent measurements
($L_\nu\propto\nu^{-1.7}$) from a stack of $z\sim 2.4$ quasars
\citep{2015MNRAS.449.4204L}.  \citet{2016MNRAS.459.2299F} also find
that the N(Si~\textsc{iv})/N(C~\textsc{iv}) column density ratio
measurements prefer a somewhat harder and more intense $>4$~Ry
background than the standard model of \citet{2012ApJ...746..125H}.
Using a large sample of X-ray-selected quasars in the redshift range
$z=0$--$6$, \citet{2017MNRAS.465.1915R} find that the faint end of the
AGN UV luminosity function at $z\sim 6$ is likely to be much shallower
that that reported by \citet{2015AA...578A..83G}.  In their analysis,
\citet{2017MNRAS.465.1915R} use an AGN obscuration optical depth
($\log N_\mathrm{H}$) cut-off that reproduces low-redshift AGN UV
luminosity functions and an X-ray-to-optical/UV luminosity ratio
calibrated at redshifts $z=0.05$--$4$ \citep{2010A&A...512A..34L}.
These authors argue that the apparent contradiction with the results
of \citet{2015AA...578A..83G} could be explained by contamination
from the AGN host galaxies.  It has also been recently argued that the
Lyman continuum escape fraction of AGN might not be 100\% as is
usually assumed \citep{2017MNRAS.465..302M}.  This may further reduce
the contribution of AGN to reionization.

We assume a flat cosmology with density parameters
$\left(\Omega_\mathrm{m},\Omega_\Lambda\right)=\left(0.3,0.7\right)$
and a Hubble constant $H_0=70$\,km\,s$^{-1}$\,Mpc$^{-1}$. Magnitudes
are reported in the AB system \citep{1983ApJ...266..713O}, and
observed magnitudes are point spread function (PSF) magnitudes
\citep{2002AJ....123..485S} corrected for Galactic extinction
\citep{1998ApJ...500..525S} unless otherwise noted.

\section{Homogenised Quasar Sample}

\subsection{Sample Selection}

We started with compiling the samples of recent UV-optical quasar
surveys. The restriction to UV-optical surveys was mainly driven by
our science goal to characterise the UV luminosity function of Type~I
quasars, but also by the still uncertain conversion from X-ray flux to
UV flux, which adds unnecessary systematics for large
samples. \textbf{References!} The individual surveys and their main
characteristics are listed in
Table~\ref{tab:samples}. Figure~\ref{fig:qsos} presents a redshift
histogram of the contributing surveys.

We included surveys based on a set of simple criteria:
\begin{enumerate}
\item Spectroscopic redshifts for the majority of targets.
\item Accurate rest-frame UV-optical CCD photometry.
\item Statistical power (sample size, coverage in $z$ and/or $M_{1450}$).
\item A characterised selection function.
\end{enumerate}
As a prerequisite for a joint analysis of the QLF we obtained the
survey selection functions in electronic form, either from the
publication or from the authors. As a reference for future surveys we
publish them here in modified and homogenised form
(Section~\ref{sect:datahom}). \textbf{Maybe get the authors' consent.}

Due to their selection criteria and their statistical power specific
surveys contribute to distinct redshift ranges. At $z<2.2$ we
considered quasars from the SDSS DR7 quasar catalogue
\citep{2010AJ....139.2360S} and the 2SLAQ catalogue
\citep{2009MNRAS.392...19C}. We restricted the SDSS DR7 sample to XXX
$z<2.2$ quasars selected with the final SDSS quasar selection
algorithm \citep{2002AJ....123.2945R, 2006AJ....131.2766R} from a
survey area of 6248\,deg$^2$ \citep{2012ApJ...746..169S}. We adopted
the SDSS targeting photometry corrected for Galactic extinction
\citep{2010AJ....139.2360S}. To limit systematic uncertainties in the
correction for host galaxy light \citep{2009MNRAS.392...19C} we
restricted the 2SLAQ sample to XXX $g<21.85$ $0.4<z<2.2$ quasars from
its spectroscopic survey footprint near the North Galactic Pole (NGP,
XX quasars in $127.7$\,deg$^2$) and the South Galactic Pole (SGP, XX
quasars in $64.2$\,deg$^2$). The small sample overlap between SDSS and
2SLAQ (112 quasars) has negligible impact on the QLF evaluation.

At $2.2<z<3.5$ we used a single sample of 23,301 uniformly
colour-selected quasars from 2236\,deg$^2$ in BOSS DR9
\citep{2013ApJ...773...14R} due to several improvements compared to
previous surveys. First, it covers a similar magnitude range as 2SLAQ
but with $>20$ times as many quasars. Second, although the SDSS DR7
sample provides better coverage of the bright end of the QLF at these
redshifts, its selection function is highly dependent on the assumed
incidence of (partial) Lyman limit systems in the IGM
\citep{2009ApJ...705L.113P, 2011ApJ...728...23W}. While the BOSS DR9
selection function considers these improvements, the uncertainty in
the QLF remains dominated by assumptions in the selection function
given the large sample size
\citep{2013ApJ...773...14R}. Variability-selected quasar samples
circumvent this issue \citep{2013ApJ...773...14R, 2013A&A...551A..29P,
  2016A&A...587A..41P}, but may be affected by (i) single-epoch
imaging incompleteness at the faint end \citep{2013ApJ...773...14R},
and (ii) uncertainties in the selection function caused by the limited
number of known $z\ga 3$ quasars not selected by variability in the
same footprint \citep{2013A&A...551A..29P, 2016A&A...587A..41P}.

At $3.7<z<4.1$ we used a combination of SDSS DR7 \citep[XX uniformly
  selected quasars from][]{2010AJ....139.2360S} and the NDWFS$+$DLS
survey \citep{2010ApJ...710.1498G,2011ApJ...728L..26G}. The lower cut
$z>3.7$ limits the impact of systematic uncertainties in the
\citet{2006AJ....131.2766R} SDSS selection function
\citep{2009ApJ...705L.113P, 2011ApJ...728...23W}. We did not consider
the results from surveys for faint $z\sim 4$ quasars in the COSMOS
field \citep{2011ApJ...728L..25I, 2012ApJ...755..169M} due to
systematic errors in their selection functions\footnote{Both studies
  simulated quasar colours with a mean IGM attenuation curve
  \citep{1995ApJ...441...18M} that cannot account for stochastic Lyman
  continuum absorption, and therefore underpredicts the variance in
  quasar colours \citep{1999ApJ...518..103B, 2008MNRAS.387.1681I,
    2011ApJ...728...23W}. Modelling the colour variance in these
  surveys is essential, as most of the \citet{2011ApJ...728L..25I}
  quasars are near the edge of their colour selection region (see
  their Figure~1), and \citet{2012ApJ...755..169M} require modest
  attenuation of the $U$ band flux relative to the mid-infrared
  flux.}. Furthermore, 30 per cent of the \citet{2012ApJ...755..169M}
COSMOS sample have visually estimated photometric redshifts, and the
spectroscopic subsample reveals that 40 per cent of the visually
estimated redshifts are biased low
($z_\mathrm{spec}>z_\mathrm{est}+0.3$, see their Figure~9). These
unaccounted systematic redshift errors at least partly explain the
discrepancy in the $z\sim 4$ QLF between \citet{2011ApJ...728L..26G}
and \citet{2012ApJ...755..169M}, which justifies our preference for
the former sample.

Between $z=4.1$ and $z=4.7$ our combined sample has XX uniformly
selected quasars from the SDSS DR7 quasar catalogue
\citep{2010AJ....139.2360S} and XX quasars from
\citet{2011ApJ...728L..26G}. Although we do not include the highly
debated \citet{2015AA...578A..83G} sample in our main analysis due to
its rough selection function and lacking spectroscopy for 17 of the 22
quasar candidates, we use it in Section~XX to constrain the faint end
($M_{1450}>-23$) of the QLF at $z>4.1$. We restricted the
\citet{2015AA...578A..83G} sample to the 19/22 sources considered in
their QLF.

At $4.7\le z<5.1$ we combined several recent surveys, accounting for
sample overlap and updated selection functions. At the bright end of
the QLF we used the 71 $z<5.1$ quasars from the SDSS+WISE survey
\citep{2016ApJ...829...33Y} together with its selection function.
%Although \citet{yang} are sensitive to $z>5.1$ quasars, we limit
%their sample to $z<5.1$ to limit differential evolution of the QLF
%across the redshift range.
To avoid double-counting quasars contained both in
\citet{2016ApJ...829...33Y} and the SDSS DR7 sample from
\citet{2013ApJ...768..105M}, we used the latter only at
$M_{1450}>-26.72$ (in our adopted cosmology), yielding 94 additional
$4.7\le z<5.1$ quasars selected in 6248\,deg$^2$. We used the updated
$z\sim 5$ SDSS DR7 selection function \citep{2013ApJ...768..105M}
instead of the one from \citet{2006AJ....131.2766R}.
%\footnote{We note that the major difference in the selection
%functions must be due to different input parameters, since the single
%color selection criterion in \citet{mcgreer} selects the vast
%majority of the SDSS DR7 quasars.}.
To these two bright-end samples we added the faint-end sample from the
\citet{2013ApJ...768..105M} SDSS Stripe~82 survey (52 uniformly
selected $4.7\le z<5.1$ quasars in 235\,deg$^2$) and two quasars from
\citet{2011ApJ...728L..26G}, adopting the respective selection
functions. We did not consider the limit on the $z\sim 5$ QLF by
\citet{2012ApJ...756..160I} due to systematic errors in their
selection function\footnote{\citet{2012ApJ...756..160I} underestimated
  the dispersion in rest-frame UV quasar colours with respect to SDSS
  at all redshifts (their Figure~4). Contrary to their claim,
  photometric errors have a small effect on the colour distribution of
  SDSS quasars given the statistical errors of $<0.03$\,mag in $gri$
  for 90 per cent of the SDSS DR7 bright quasar sample ($i<19.1$) and
  a relative calibration error of $\sim 1$ per cent
  \citep{2008ApJ...674.1217P}.}.

The SDSS colours of $5.1\le z<5.5$ quasars are similar to those of M
and L dwarf stars, resulting in a low and uncertain selection
efficiency \citep{2013ApJ...768..105M}. WISE mid-infrared selection
performs better \citep{2016ApJ...829...33Y}, but is restricted to the
bright end of the quasar population. Our combined sample contains the
remaining 28 quasars from \citet{2016ApJ...829...33Y} and 10 quasars
from the \citet{2013ApJ...768..105M} Stripe~82 survey. As we will show
in Section~XX, the resulting QLF is consistent with those at lower and
higher redshifts, indicating that the \citet{2013ApJ...768..105M}
selection functions are quite reliable. \textbf{Check this!}

Finally, we combined the samples from all spectroscopic $z\sim 6$
quasar surveys with a determined selection function as of June
2017. \citet{2016ApJ...833..222J} recently compiled all quasars
discovered in several SDSS $z\sim 6$ surveys together with
consistently derived selection functions. Their uniform sample
consists of 24 quasars from the SDSS main survey (11240\,deg$^2$), 10
additional quasars in regions with two or more SDSS imaging scans
(so-called overlap regions, 4223\,deg$^2$), and 13 faint quasars from
SDSS Stripe~82 (277\,deg$^2$). The CFHQS \citep{2010AJ....139..906W}
provided a uniform sample of 16 quasars in the Very Wide Survey
(494\,deg$^2$) and a single quasar in the Deep Survey
($4.47$\,deg$^2$). The one quasar detected in both SDSS and CFHQS does
not lead to underestimated statistical errors in the QLF. Lastly, we
included the two objects from \citet{2015ApJ...798...28K}. One might
be a Lyman break galaxy due to its narrow Ly$\alpha$ emission line
(half width at half maximum 427\,km\,s$^{-1}$), but
\citet{2015ApJ...798...28K} note that 10/17 candidates do not have
follow-up spectroscopy, hence their QLF may be underestimated. Since
very small samples preclude a statistical correction via the selection
function, complete spectroscopic follow-up of $z\sim 6$ candidates is
essential. Although we will account for the slightly different
redshift sensitivities for the different surveys, we will quote a
nominal redshift range $5.7<z<6.5$ for the combined $z\sim 6$ sample.

\begin{figure*}
  \begin{center}
    \includegraphics[width=\textwidth]{qsos.pdf}
  \end{center}
  \caption{Quasar samples.}
  \label{fig:qsos}
\end{figure*}

\begin{table*}
  % In published version of this table, we could remove file names and add data homogenisation in Notes.
  % Add z ~ 7 qso when you add it to analysis
  % See data.tex for a version of this table that includes file names
  \caption{Quasar data sets}
  \label{tab:samples}
  \begin{tabular}{crcllrSl}
    \hline
    & ID & $z$ & Survey & Reference & Number & {Area} & Notes \\
    & & & & & of quasars & {(deg$^2$)} & \\
    \hline
    1. & 13 & 0.1--2.2 & SDSS DR7 & \citet{2006AJ....131.2766R} & 48664 & 6248.0 & \\
    2. & 15 & 0.4--2.6 & 2SLAQ SGP & \citet{2009MNRAS.392...19C} & 3663 & 64.2 & \\
    3. & 15  & 0.4--2.6 & --- NGP & \citet{2009MNRAS.392...19C} & 8153 & 127.7 & \\
    4. &  1 & 2.2--3.5 & BOSS DR9 & \citet{2013ApJ...773...14R} & 23301 & 2236.0 & \\
    5. & 13 & 3.7--4.7 & SDSS DR7 & \citet{2006AJ....131.2766R} & 1785 & 6248.0 & Restricted to $z<4.7$ \\
    6. & 17 & 4.7--5.4 & SDSS+WISE & \citet{2016ApJ...829...33Y} & 99 & 14555.0 & Overlaps with 8 for $M_{1450}<-26.73$ \\
    7. &  8 & 4.7--5.1 & SDSS DR7 & \citet{2013ApJ...768..105M} & 148 & 6248.0 & \\
    8. &  8 & 4.7--5.1 & --- Stripe 82 & \citet{2013ApJ...768..105M} & 52 & 235.0 & \\
    9. &  8 & 5.1--5.5 & --- DR7 Extended & \citet{2013ApJ...768..105M} & 28 & 6248.0 & \\
    10. & 8 & 5.1--5.4 & ---  Stripe 82 Extended & \citet{2013ApJ...768..105M} & 10 & 235.0 & \\
    11. & 6 & 3.7--4.8 & NDWFS & \citet{2011ApJ...728L..26G} & 12 & 1.71 & \\
    12. & 6 & 3.8--5.1 & DLS & \citet{2011ApJ...728L..26G} & 12 & 2.05 & \\
    13. & 7 & 4.1--6.3 & CANDELS GOODS-S & \citet{2015AA...578A..83G} & 19 & 0.047 & Rescaled selection probabilities \\
    14. & 18 & 5.8--6.4 & --- Main & \citet{2016ApJ...833..222J} & 24 & 11240.0 & \\
    15. & 18 & 5.9--6.1 & --- Overlap & \citet{2016ApJ...833..222J} & 10 & 4223.0 & \\
    16. & 18 & 5.7--6.1 & --- Stripe 82 & \citet{2016ApJ...833..222J} & 13 & 277.0 & \\
    17. & 10 & 6.0 & CFHQS Deep & \citet{2010AJ....139..906W} & 1 & 4.47 & \\
    18. & 10 & 5.9--6.4 & --- Very Wide & \citet{2010AJ....139..906W} & 16 & 494.0 & \\
    19. & 11 & 6.0--6.2 & Subaru High-$z$ Quasar & \citet{2015ApJ...798...28K} & 2 & 6.5 & \\
    20. & 19 & 6.5--7.4 & UKIDSS & \citet{2011Natur.474..616M} & 1 & 3370.0 & \\
    21. & 19 & 6.5--7.4 & UKIDSS & \citet{2007MNRAS.376L..76V} & 1 & 3370.0 & \\
  \end{tabular}
\end{table*}

%\clearpage
\subsection{Sample Homogenisation}
\label{sect:datahom}

\begin{figure}
    \includegraphics[width=\columnwidth]{kcorr.pdf}
  \caption{Bandpass corrections $K_{m,1450}$ from a broadband magnitude $m=\{g,i,z_\mathrm{AB}\}$
           to the monochromatic AB magnitude at 1450\,\AA\ as a function of redshift $z$ for the \citet{2015MNRAS.449.4204L}
           quasar SED used in this work, and for two quasar composite spectra \citep{2001AJ....122..549V, 2002ApJ...565..773T}.
           The redshift range has been restricted to exclude the Ly$\alpha$ forest and to account for the different
           rest frame wavelength coverage of the spectra.}
  \label{fig:kcorr}
\end{figure}

For a joint fit of the QLF it is necessary to homogenise the different
survey samples in absolute magnitude, and to convert their selection
functions to the same absolute magnitude system. For the analysis of
the quasar UV emissivity and to be consistent with published work at
$z>3$ we chose to convert all samples and selection functions to the
absolute AB magnitude at a rest frame wavelength $\lambda=1450$\,\AA
\begin{equation}\label{eq:absmag}
M_{1450}\left(z\right) = m-5\log{\left(\frac{d_L\left(z\right)}{\mathrm{Mpc}}\right)}-25-K_{m,1450}\left(z\right),
\end{equation}
with the luminosity distance $d_L\left(z\right)$ to a quasar at
redshift $z$, the apparent magnitude $m$ in a filter used in the
survey, and the bandpass correction $K_{m,1450}\left(z\right)$
\citep{1956AJ.....61...97H, 1968ApJ...154...21O, 2000A&A...353..861W,
  2002astro.ph.10394H}. For the bandpass correction we used a
combination of the \citet{2015MNRAS.449.4204L} quasar SED at
$\lambda<2500$\,\AA, and the \citet{2001AJ....122..549V} quasar
composite spectrum at longer wavelengths to cover the lowest
redshifts.  The samples from SDSS and BOSS are defined in the SDSS $i$
band, while 2SLAQ is defined in the $g$ band. At $z>4.7$ we adopted
the SDSS $z$ band magnitude (in the following denoted $z_\mathrm{AB}$)
for SDSS DR7 quasars to avoid additional corrections due to the
Ly$\alpha$ forest. Figure~\ref{fig:kcorr} shows our bandpass
corrections for SDSS, BOSS and 2SLAQ as a function of redshift. We
ignored the luminosity dependence of the bandpass correction due to
the known anticorrelation of emission line equivalent width and
luminosity \citep{1977ApJ...214..679B}.  While the
\citet{2015MNRAS.449.4204L} SED is for luminous ($M_{1450}\simeq
-27.2$) quasars, UV composite spectra including fainter quasars
\citep{2002ApJ...565..773T, 2012ApJ...752..162S, 2014ApJ...794...75S}
give similar values, such that our bandpass corrections remain
applicable at $M_{1450}\la -24$. Empirical luminosity-dependent
bandpass corrections show a $\la 0.2$\,mag variation over $\sim 4$
orders of absolute magnitude depending on redshift and the filter, and
with a $\sim 0.2$\,mag intrinsic scatter due to individual SED
variations \citep{2013ApJ...773...14R, 2013ApJ...768..105M,
  2013A&A...551A..29P}.

For the $z<2.2$ sample we corrected the SDSS $i$ and 2SLAQ $g$ band
magnitudes for host galaxy contamination following
\citet{2009MNRAS.392...19C}. Considering the different magnitude
limits of 2SLAQ and SDSS, the modelled host galaxy contamination is
small for $z>0.5$ quasars ($<0.1$\,mag in $g$, $<0.2$\,mag in $i$),
and is negligible at $z>0.8$.  In case the band defining the magnitude
limit of the survey undesirably overlaps with the Ly$\alpha$ forest
\citep{2010ApJ...710.1498G, 2011ApJ...728L..26G, 2013ApJ...768..105M}
we adopted their respective bandpass corrections to $M_{1450}$. In
particular, for the \citet{2010ApJ...710.1498G, 2011ApJ...728L..26G}
sample we recomputed $M_{1450}$ from the $R$ band photometry to be
consistent with the selection function defined in $R$, and to avoid
uncertainties in their spectrophotometry due to incomplete spectral
coverage. Since the \citet{2010ApJ...710.1498G, 2011ApJ...728L..26G}
$R$ band traces the rest frame UV, we assumed negligible host galaxy
contamination for their faint quasars. For the remaining high-redshift
surveys reporting $M_{1450}$ obtained by various methods
\citep{2010AJ....139..906W, 2015ApJ...798...28K, 2015AA...578A..83G,
  2016ApJ...829...33Y, 2016ApJ...833..222J} we did not re-compute
$M_{1450}$, but applied appropriate shifts to correct to our adopted
cosmology.

The selection functions were treated similarly, i.e.\ the photometric
selection function of survey $j$ given in observed magnitudes
$f_{\mathrm{p},j}\left(m,z\right)$ \citep{2006AJ....131.2766R,
  2009MNRAS.392...19C, 2010ApJ...710.1498G, 2013ApJ...773...14R} was
transformed to our absolute magnitudes
$f_{\mathrm{p},j}\left(M_{1450},z\right)$ with
Equation~\ref{eq:absmag}, while the ones given in $M_{1450}$ were
adjusted to our cosmology. Note, however, that many surveys report
additional sources of incompleteness that preclude a straightforward
re-evaluation of the QLF.

For 2SLAQ we corrected for magnitude-dependent spectroscopic coverage
in the two survey areas \citep[$f_\mathrm{c,NGP}\left(g\right)$ and
  $f_\mathrm{c,SGP}\left(g\right)$; Figure~4
  in][]{2009MNRAS.392...19C} and spectroscopic redshift success
\citep[$f_\mathrm{s,2SLAQ}\left(g\right)$; Figure~6b
  in][]{2009MNRAS.392...19C} by multiplying them with the photometric
selection function, resulting in two area-specific 2SLAQ selection
functions
$f_\mathrm{NGP}\left(M_{1450},z\right)=f_\mathrm{p,2SLAQ}f_\mathrm{c,NGP}f_\mathrm{s,2SLAQ}$
and
$f_\mathrm{SGP}\left(M_{1450},z\right)=f_\mathrm{p,2SLAQ}f_\mathrm{c,SGP}f_\mathrm{s,2SLAQ}$
that are relevant for the QLF.  The $z<4.7$ SDSS photometric selection
function was modified to include known imaging incompleteness to
$f_{\mathrm{SDSS},z<4.7}=0.95f_{\mathrm{p,SDSS},z<4.7}$
\citep{2006AJ....131.2766R}. The BOSS colour-selected sample contains
quasars with $f_\mathrm{c,BOSS}f_\mathrm{s,BOSS}\ge 0.85$
\citep{2013ApJ...773...14R}, and we adopted
$f_\mathrm{BOSS}=\overline{f_\mathrm{c,BOSS}f_\mathrm{s,BOSS}}f_\mathrm{p,BOSS}=0.962f_\mathrm{p,BOSS}$. \citet{2010ApJ...710.1498G}
presented two area-specific photometric selection functions due to
different filters employed, and more follow-up spectroscopy was
reported in \citet{2011ApJ...728L..26G}. We accounted for remaining
spectroscopic incompleteness at $R>23$, yielding the final selection
functions $f_\mathrm{NDWFS}$ and $f_\mathrm{DLS}$. The updated $z\sim
5$ SDSS photometric selection function \citep{2013ApJ...768..105M} was
modified to include imaging and spectroscopic incompleteness, yielding
$f_{\mathrm{SDSS},z\sim 5}=0.95^2f_{\mathrm{p,SDSS},z\sim 5}$. In the
deeper $z\sim 5$ SDSS Stripe~82 survey the spectroscopic
incompleteness is larger and magnitude-dependent \citep[Figure~14
  in][]{2013ApJ...768..105M}, resulting in $f_{\mathrm{S82},z\sim
  5}=0.95f_{\mathrm{s,S82},z\sim
  5}\left(i\right)f_{\mathrm{p,S82},z\sim 5}$. Likewise, imaging and
magnitude-dependent spectroscopic incompleteness was factored into the
\citet{2016ApJ...829...33Y} photometric selection function (their
Figures~5 and 7), resulting in
$f_\mathrm{SDSS+WISE}=0.97f_\mathrm{s,SDSS+WISE}\left(z_\mathrm{AB}\right)f_\mathrm{p,SDSS+WISE}$. Finally,
we obtained a rough estimate of the \citet{2015AA...578A..83G}
selection function by comparing the corrected and observed QLFs,
i.e.\ taking $f_\mathrm{GOODS-S}=\phi_\mathrm{obs}/\phi_\mathrm{corr}$
(see their Table~3).


% BOSS:
% - color selection function in Table 5 of Ross et al. (imag, z), sample contains QSOs with spectrocoverage completeness >0.85, mean spectrocoverage completeness 0.962 multiplied onto color selection function
% - interpolate Beta's K correction (imag->1450A) and compute M1450 for our cosmology (sample and map)

% 2SLAQ:
% - low-luminosity AGN -> model and subtract host galaxy contribution, magnitudes only for nuclear component, host galaxy contribution <20% at z>0.4 for faintest sources in 2SLAQ (Croom et al. 2009, MNRAS, 392, 19)
% - several selection functions (morphological, color, spectroscopic coverage, spectroscopic success)
% - Table 12: color selection function (available online), gmag is total PSF mag including host galaxy contribution, subtract gmag host galaxy contribution listed in Table 12, selection function extrapolated at bright end (lacking coverage of whole catalog)
% - coverage completeness for NGP and SGP given in Fig 4 as a function of gmag, interpolated and multiplied into color selection function
% - spectroscopic success as function of gmag in Fig 6b (only quality=1 redshifts considered), interpolated and multiplied onto color selection function
% - convert g_agn to M1450 with L15 SED (valid for bright quasars, check with K correction for Telfer/Shull)
% - sample: correct PSF gmag for Galactic extinction, subtract host galaxy light (Table 12), interpolate Beta's K correction (gmag->1450A) and compute M1450 for our cosmology
% SDSS:
% - Richards et al. selection function given in (z,imag) Fig 6 and Table 1, corrected for edge effects at magnitude limits (steep gradient) and NaN values, selection function corrected for imaging incompleteness (factor 0.95)
% - imag in selection function converted to M1450 with L15 SED, conversion uncertain at z>4.7 due to Lya forest
% - sample: targeting psf imag corrected for Galactic extinction (Schlegel) and converted to M1450 with L15 SED
% - z<2.2: sample limited to i_dered<19.1, imag of sample and selection function corrected for host galaxy light (Table 12 in Croom et al 2009, given for imag and gmag)
% Glikman:
% - different selection functions for 2 sub-surveys, selection functions corrected for interpolation errors (values >1 reset to 1), selection functions given in (z,Rmag) Fig 7 of G10, converted to M1450 with their Rmag K correction (Fig 8 in G10)
% - spectroscopic incompleteness (f_spec, given in G10 and updated in G11) incorporated into selection functions via linear fit to f_spec(Rmag) at Rmag>23 and constant otherwise
% - sample M1450 recomputed from Rmag and K correction to be consistent with completeness map, M1450=Rmag-Kcorr-5*log10(d_L/Mpc)-25
% McGreer:
% - Stripe 82: selection function (Fig. 11) modified to include photometric incompleteness (factor 0.95) and spectroscopic incompleteness (imag>20.8, Fig. 14), M1450 in selection map and sample M1450 converted to our cosmology (adding 0.07)
% - DR7: sample not listed in paper, M1450 computed from zmag (QSO continuum) and L15 SED, same procedure applied to S82 reveals L15 magnitudes are slightly brighter than M13 magnitudes (median 0.12mag) -> small change,
%        M13 selection function corrected for photometric (factor 0.95) and spectroscopic (factor 0.95) incompleteness
% Yang: selection function (Fig. 5) modified to include image selection incompleteness (factor 0.97) and spectroscopic incompleteness (factor ~0.84 at zmag>19, Fig 7), M1450 in selection map and sample M1450 converted to our cosmology
% Willott: selection map M1450 adjusted to our cosmology adding 0.05, slightly different cosmological parameters in Willott et al. (2007,2009,2010) -> M1450 converted
% Jiang: selection functions unchanged, M1450 as given in paper
% Kashikawa: map not corrected for spectroscopic incompleteness (sample size), M1450 as given in paper
% Giallongo: selection function estimated from phi_obs and phi_corr (Table 3 of Giallongo et al.), M1450 as given in paper


\section{Luminosity function}

Given quasar luminosities and estimates for survey completeness, we
begin our analysis by first deriving luminosity functions binned both
in redshift and magnitude.  In this discussion all magnitudes are
absolute magnitudes at rest-frame 1450~{\AA}.

% Put that sentence about magnitudes in the introduction.

\subsection{Binned luminosity function estimates}

In a magnitude bin $[M_\mathrm{min}, M_\mathrm{max})$, and redshift
  bin $[z_\mathrm{min}, z_\mathrm{max})$, we define the luminosity
    function as \citep{2000MNRAS.311..433P}
  \begin{equation}
    \phi \equiv \frac{N}{V_\mathrm{bin}},
  \end{equation}
  where $N$ is the number of quasars with magnitude
  $M_\mathrm{min}\leq M<M_\mathrm{max}$ and redshift
  $z_\mathrm{min}\leq z<z_\mathrm{max}$, and
  \begin{equation}
    V_\mathrm{bin} = \int_{M_\mathrm{min}}^{M_\mathrm{max}}dM\int_{z_\mathrm{min}}^{z_\mathrm{max}}dz\, f(M, z)\,\frac{dV}{dz},
    \label{eqn:vi}
  \end{equation}
  is the effective volume of the bin.  Here $f(M,z)$ is the quasar
  selection probability, which includes incompleteness.  Inclusion of
  the selection probability in Equation~(\ref{eqn:vi}) accounts for
  what are sometimes called ``incomplete bins''
  \citep{2006AJ....131.2766R}.  The comoving volume element $dV/dz$ is
  given by
  \begin{equation}
    \frac{dV}{dz}=\frac{dV}{dz\,d\Omega}\cdot A\cdot\frac{4\pi}{41253},
  \end{equation}
  where $A$ is the survey area in deg$^2$, and 
  \begin{equation}
    \frac{dV}{dz\,d\Omega}=\frac{c}{H_0}\frac{d_L(z)^2}{(1+z)^2\sqrt{\Omega_m(1+z)^3+\Omega_\Lambda}},
    \label{eqn:dvdzdo}
  \end{equation}
  denotes the comoving volume element per unit solid angle
  \citep{1999astro.ph..5116H}.  The luminosity distance $d_L$ is given
  by
  \begin{equation}
    d_L(z)=(1+z)\frac{c}{H_0}\int_0^z\frac{dz}{\sqrt{\Omega_m(1+z)^3+\Omega_\Lambda}}.
    \label{eqn:dl}
  \end{equation}
  Equations~(\ref{eqn:dvdzdo}) and (\ref{eqn:dl}) assume a flat
  Universe ($\Omega_k=0$).  The luminosity function $\phi$ has units
  of $\mathrm{cMpc}^{-3}\mathrm{mag}^{-1}$.  We evaluate the double
  integral in Equation~(\ref{eqn:vi}) by the Euler method, i.e., by
  simply summing over the ``tiles'' of the selection function map,
  without interpolating between the redshift and luminosity values of
  neighbouring tiles.  This may result in $V_i=0$ for some quasars, in
  which case we remove them from our analysis.

  In each bin, we estimate the uncertainty in the luminosity function
  by assuming Poisson statistics \citep{1986ApJ...303..336G}.  The
  resultant binned luminosity function estimates are shown by the
  points in Figure~\ref{fig:mosaic}.  These agree quite well with the
  published luminosity functions by these surveys.

  % Do we want to give the details of how good/bad this agreement
  % actually is?
  
  %% Figure~\ref{fig:boss}
  %% shows the luminosity function derived in this way for the BOSS DR9
  %% colour-selected sample of 23,301 quasars, in comparison with the
  %% published luminosity function of \citet{2013ApJ...773...14R}.
  %% Figure~\ref{fig:sdss} shows the luminosity function of a sample of
  %% 15,179 quasars from the SDSS DR3 in comparison with the published
  %% luminosity function of \citet{2006AJ....131.2766R}.


\begin{figure*}
  \begin{center}
    % \includegraphics[width=\textwidth,keepaspectratio]{mosaic_small.pdf}
    \includegraphics[height=0.88\textheight,keepaspectratio]{mosaic_small.pdf}
  \end{center}
  \caption{Luminosity function estimates from $z=0.3$ to $2.6$.  The
    symbols show our inferred binned luminosity functions.  In each
    redshift bin, the black curve shows the best-fit double power law
    model, which is represented by the median of the posterior
    probability distribution function.  Yellow curves show a random
    draw from the posterior probability distribution function.  As
    discussed in the text, luminosity bins that show a decrease in the
    luminosity function at the faint end are deemed to be incomplete
    are are removed from our analysis while fitting the double power
    law model.  These bins are represented by open symbols in this
    figure.  Each panel shows the number of quasars in the respective
    redshift bin.  The number in parenthesis indicates the total
    number of quasars in the bin, before excluding incomplete
    bins.}
  \label{fig:mosaic}
\end{figure*}

\begin{figure*}
  \begin{center}
    \includegraphics[width=\textwidth]{evolution_alldata.pdf}
  \end{center}
  \caption{Parameter evolution from individual fits in
    Figures~\ref{fig:mosaic}.  Parameters from BOSS quasars are seen
    to be systematically offset.}
  \label{fig:evoln}
\end{figure*}

\subsection{Luminosity function estimates in redshift bins}

As our next step, we stop binning the quasars in luminosity, and only
bin them in redshifts.  In each redshift bin, we model the quasar
luminosity function as a double power law given by
\citep{1988MNRAS.235..935B, 1995ApJ...438..623P, 2000MNRAS.317.1014B}
\begin{equation}
  \phi(M) =
  \frac{\phi_*}{10^{0.4(\alpha+1)(M-M_*)}+10^{0.4(\beta+1)(M-M_*)}}.
  \label{eqn:dpl}
\end{equation}
Equation~(\ref{eqn:dpl}) has four free parameters: the amplitude
$\phi_*$, the ``break'' $M_*$, the faint-end slope $\beta$, and the
bright-end slope $\alpha$.  We obtain posterior probability
distributions for these parameters using MCMC \citep[e.g.,][]{jaynes}.
The joint posterior probability distribution of the model parameters
is then written as
\begin{multline}
  p(\phi_*, M_*, \alpha, \beta | \{M_i, z_i\}) \propto \\ p(\phi_*, M_*, \alpha, \beta)p(\{M_i, z_i\} | \phi_*, M_*, \alpha, \beta),
\end{multline}
where the constant of proportionality is independent of the luminosity
function parameters, and $\{M_i, z_i\}$ denotes the magnitudes and
redshifts of quasars falling in a redshift bin $[z_\mathrm{min},
  z_\mathrm{max})$.  We use a uniform prior distribution $p(\phi_*,
  M_*, \alpha, \beta)$ and assume that the likelihood
\begin{equation}
  \mathcal{L}\equiv p(\{M_i, z_i\} | \phi_*, M_*, \alpha, \beta)
\end{equation}
is given by $\phi(M)$ with suitable normalisation.  The negative
logarithm of the likelihood $S\equiv -2\ln\mathcal{L}$ can then be
written as
\begin{multline}
  S = -2\sum_{i=1}^N\ln\phi(M_i, z_i)\\+2\int_{M_\mathrm{min}}^{M_\mathrm{max}}dM\int_{z_\mathrm{min}}^{z_\mathrm{max}}dz\, \phi(M,z) f(M, z)\,\frac{dV}{dz}.
  \label{eqn:S}
\end{multline}
where $N$ is the total number of quasars in the redshift bin and the
luminosity integral in the second term on the right hand side is on
the surveyed range of $M$.

The above likelihood can also be understood as the limit of the
Poisson likelihood in luminosity and redshift bins
\citep{1983ApJ...269...35M, 2001AJ....121...54F}.  We can write the
probability of observing $n_{ij}$ quasars in the $(M_i, z_j)$ bin as
the Poisson distribution
\begin{equation}
  \mathcal{L}=\prod_{i,j}\frac{e^{-\mu_{ij}}\mu_{ij}^{n_{ij}}}{n_{ij}!},
  \label{eqn:lhood}
\end{equation}
where 
\begin{equation}
  \mu_{ij}= \int_{M_i}^{M_{i+1}}dM\int_{z_j}^{z_{j+1}}dz\, \phi(M,z) f(M, z)\,\frac{dV}{dz},
\end{equation}
is the average number of quasars expected in the $(M_i, z_j)$ bin
given the luminosity function $\phi(M,z)$.  In the limit of
infinitesimal bins, $n_{ij}=0$ or $1$, Equation~(\ref{eqn:lhood}) can
be simplified and we get Equation~(\ref{eqn:S}).

Our estimates for the luminosity function in several redshift bins are
shown in Figure~\ref{fig:mosaic}.  The median of the posterior is
shown as the best fit model.  Yellow curves show a random draw from
the posterior.  Consistent with previous studies, the double power law
model provides an excellent description of the luminosity function
model over almost the complete range of redshifts spanned by the data.
It is only in the highest redshift bin ($z=5.5$--$6.5$) that the data
seem to favour a single power law.  In this bin, we the resultant
posterior distribution of the break magnitude $M_*$ is bimodal with
favoured values at the faint ($M_*>-18$) and bright end of the data
($M_*<-30$).  We consider only the latter possibility as consistent
with the data at lower redshifts ($z<5.5$).  In light of the low
redshift data, we use restricted priors in this redshift bins: we
restrict the bright-end slope $\alpha$ to values less than $-4$.
Other paramters continue to have wide uniform priors.

As seen in Figure~\ref{fig:mosaic}, in several redshift bins the
binned quasar luminosity function seems to show a decline at the faint
end.  This is clearly seen, for example, in the 2SLAQ sample at $z<2$.
We take this to be a systematic error: the decrease in the luminosity
function is an indication of incompleteness in the quasar surveys that
has not been accounted for by the reported completeness map.  We
choose to discard such quasars from our analysis.  This behaviour is
also seen in the SDSS quasar sample at $z<1.8$ and $z\sim 4$.  The
discarded luminosity bins are shown in Figure~\ref{fig:mosaic} by open
circles.  We can see that the surviving quasar sample is in excellent
agreement with the double power law model at all redshifts.

The values of the four double power law parameters in various redshift
bins are shown in Figure~\ref{fig:evoln}.  We find interesting
evolutionary trends in each of the four parameters.  The break
luminosity $M_*$ evolves by more than eight magnitudes from redshift
$z=0$ to $7$.  The amplitude of the luminosity function $\phi_*$
evolves moderately from $z=0$ to $z\sim 3$ and then drops by six
orders of magnitude to about $10^{-12}$ at $z\sim 7$.  The bright end
slope $\alpha$ has significant scatter, but still shows a trend
towards larger negative values---that is, steeper bright-end
luminosity functions---at high redshifts.  Finally, the faint end of
the luminosity function also shows signs of increasing steepness
towards high redshifts.  Somewhat similar to the amplitude $\phi_*$,
the faint-end slope $\beta$ also shows a break at $z\sim 3$.  It stays
roughly constant up to this redshift, and the drops to increasingly
negative values at high redshifts, indicating a steeper faint end.

The behaviour of BOSS quasars in Figure~\ref{fig:evoln} is striking.
In the redshift range covered by BOSS the values of the four double
power law parameters deviate from the general trends.  The faint end
of the luminosity function of the BOSS quasars is much shallower and
the break luminosity is fainter by a magnitude.  The bright end of the
luminosity function of quasars in this sample is also much flatter
than that for other samples.  The conspicuousness of the deviation of
the BOSS sample from the general trends makes it unlikely that the
luminosity function evolution indicated by BOSS in the redshift range
$z=2$--$3$ is physical.  Indeed, in this redshift range BOSS quasar
selection is known to be affected by a colour bias
\citep{2011ApJ...728...23W}, as the quasar colour selection is
contaminated by a population of blue stars.  We therefore deem the
BOSS sample as affected by this systematic error.

The open circles in Figure~\ref{fig:evoln} show the results of a
similar analysis of the quasar UV luminosity function by
\citep{2017MNRAS.466.1160M}.  Evidently, our results are quite
different from the results of these authors.  While their analysis
generally also reveals a steep faint end slop, the break luminosity
evolution is dramatically different.  In their analysis the break
luminosity tends to remain roughly constant $M_*\sim -24$ from $z=0$
to $7$, but this does not seem to be indicated by the data shown in
Figure~\ref{fig:mosaic}.  A possible reason behind their significanly
different results could be a bias introduced by binning in
luminosities and redshifts while fitting the luminosity function model.

\subsection{Luminosity function estimate without binning}

The smooth evolution of the luminosity function parameters apparant in
Figure~\ref{fig:evoln} suggests that it should be possible to describe
the quasar UV luminosity function evolution from $z\sim 7$ to $0$
using far fewer number of parameters.  Such descriptions have been put
forward in the literature for the X-ray and bolometric luminosity
functions \citep[e.g.,][]{2007ApJ...654..731H, 2015MNRAS.451.1892A}.
Such ``global'' descriptions of the luminosity function evolution are
useful as they give a continuous description of the luminosity
function.  This makes them valuable while developing an understanding
of the physics behind the luminosity function. We therefore obtain
estimates of the luminosity function in which the evolution of the
four parameters of the double power law models in
Equation~(\ref{eqn:dpl}) is parameterised to write
\begin{align}
  &\phi_*(z) = f_0(\{c_{0,i}\}, z)\nonumber\\
  &M_*(z) = f_1(\{c_{1,i}\}, z)\nonumber\\
  &\alpha(z) = f_2(\{c_{2,i}\}, z)\nonumber\\
  &\beta(z) = f_3(\{c_{3,i}\}, z),
\end{align}
where the $c_i$'s are the new model parameters.  The joint posterior
probability distribution of these parameters can be now written as
\begin{equation}
  p(\{c_i\} | \{M_i, z_i\}) \propto p(\{c_i\})p(\{M_i, z_i\} | \{c_i\}),
\end{equation}
where the likelihood 
\begin{equation}
  \mathcal{L}\equiv p(\{M_i, z_i\} | \{c_i\}),
\end{equation}
is now given by $\phi(M,z)$ with suitable normalisation.  Note that
$\phi(M,z)$ is simply 
\begin{multline}
  \phi(M,z) = \phi_*(z) \left[10^{0.4(\alpha(z)+1)(M-M_*(z))}\right. \\ \left.+ 10^{0.4(\beta(z)+1)(M-M_*(z))}\right]^{-1}.
\end{multline}
The ngative logarithm of the likelihood $S\equiv -2\ln\mathcal{L}$ is
straightforward generalisation of Equation~(\ref{eqn:S}), so that
\begin{multline}
  S = -2\sum_{i=1}^N\ln\phi(M_i, z_i)\\+2\int_{M_\mathrm{min}}^{M_\mathrm{max}}dM\int_{z_\mathrm{min}}^{z_\mathrm{max}}dz\, \phi(M,z) f(M, z)\,\frac{dV}{dz},
  \label{eqn:S2}
\end{multline}
where now $N$ is the total number of quasars included in the analysis,
and the integrals in the second term on the right-hand side are over
the complete surveyed range of luminosity and redshift.

\section{Hydrogen photoionization rate}

We assume that all quasars have a universal SED, which can be
parameterised in some fashion.  \citet{2015MNRAS.449.4204L} have
parameterised the quasar SED as a power law with a break at 912~{\AA},
\begin{equation}
f_\nu\propto\begin{cases}
               \nu^{-0.61\pm 0.01} & \text{if}~\lambda\geq 912~\text{\AA},\\
               \nu^{-1.70\pm 0.61} & \text{if}~600~\text{\AA}<\lambda<912~\text{\AA}.\\                
               \end{cases}
\label{eqn:sed}
\end{equation}
Here $f_\nu$ is the flux density, with units of ergs s$^{-1}$
cm$^{-2}$ Hz$^{-1}$, although it is often simply called flux.  Note
that although $f_\nu$ is often the quantity of interest, e.g., when
calculating the hydrogen ionizing emissivity of quasars, what is shown
in spectra is often $f_\lambda$, the flux density
in \emph{wavelength}, which has units of ergs s$^{-1}$
cm$^{-2}$ \AA$^{-1}$.  If the flux density in frequency is power law,
then the flux density in wavelength is also power law.  The power law
index of the flux density in wavelength, $\alpha_\lambda$, is related
to the power law index of the flux density in frequency, $\alpha_\nu$,
by $\alpha_\lambda=-(\alpha_\nu+2)$.  From Equation~(\ref{eqn:sed}),
we can now see that $\alpha_\lambda=-0.3$ in the extreme UV (EUV;
600~{\AA}$<\lambda<$912~{\AA}) and $\alpha_\lambda=-1.39$ in the far
UV (FUV; $\lambda>$912~{\AA}).  The quasar SED that we have adopted is
steep in FUV and shallow in EUV.

How sensible is it to assume that the quasar SED of
Equation~(\ref{eqn:sed}) is universal?  Of the four published
composite quasar SEDs (see references in
\citealt{2015MNRAS.449.4204L}), three agree with
Equation~(\ref{eqn:sed}).  There is one published composite quasar SED
that does not agree with Equation~(\ref{eqn:sed}).  This is the
composite of \citet{2004ApJ...615..135S}, which uses more than 100
quasars at $z<0.1$.  The EUV slope of this SED is $-0.56$.  (This
means that the power law index of the flux density in wavelength is
$-1.44$, so that the spectrum actually becomes even steeper in the
EUV.  See Figure 8 of \citealt{2015MNRAS.449.4204L}.)  The quasars
considered by \citet{2004ApJ...615..135S} are much fainter that those
considered by \citet{2015MNRAS.449.4204L}, which means that it is
possible that faint quasars have a steep EUV spectrum.  So we could
say that for faint quasars,
\begin{equation}
f_\nu\propto\begin{cases}
               \nu^{-0.61\pm 0.01} & \text{if}~\lambda\geq 912~\text{\AA},\\
               \nu^{-0.56\pm 0.61} & \text{if}~600~\text{\AA}<\lambda<912~\text{\AA}.\\                
               \end{cases}
\label{eqn:sed_faint}
\end{equation}
Here, ``faint'' is defined as $M_i(z=2)>-26$.  $M_i(z=2)$ can be
related to the bolometric luminosity of the quasar (see Figure 9 of
\citealt{2015MNRAS.449.4204L}.)

The rate of emission of photons of frequency $\nu$ by quasars can be
written as
\begin{equation}
\dot n_\nu = \int dM_\nu \phi(M_\nu) \frac{L_\nu(M_\nu)}{h\nu}.
\end{equation}
The units of $\dot n_\nu$ are s$^{-1}$~cMpc$^{-3}$~Hz$^{-1}$.  Here
the integral is over some range of magnitudes, the monochromatic
luminosity $L_\nu(M)$ is related to the (absolute AB) magnitude
by \citep{1983ApJ...266..713O}
\begin{equation}
M_\nu = -2.5\log_{10}L_\nu+51.60,
\end{equation}
and $\phi$ is the luminosity function from Equation~(\ref{eqn:dpl}).
The luminosity at 1~Ry is related to that at 1450~\AA\ by
Equation~(\ref{eqn:sed}) or (\ref{eqn:sed_faint})
\begin{equation}
  L_{912}=L_{1450}\left(\frac{\nu_{912}}{\nu_{1450}}\right)^{-0.61}=L_{1450}\left(\frac{912}{1450}\right)^{0.61}
\end{equation}
The corresponding volume emissivity is conventionally written
as 
\begin{equation}
\epsilon_\nu = \dot n_\nu h\nu (1+z)^3.
\label{eqn:epsilon}
\end{equation}
This is a \emph{physical} density.  The units of $\epsilon_\nu$ are
erg~s$^{-1}$~pMpc$^{-3}$~Hz$^{-1}$.  This emissivity can now be used
to calculate the ionizing flux and photoionization rate contributed by
quasars.  The frequency dependence above 1~Ry is given by
Equation~(\ref{eqn:sed}) or (\ref{eqn:sed_faint}), so that for $\nu >
\nu_{912}$
\begin{equation}
  \epsilon_\nu = \epsilon_{912}\left(\frac{\nu}{\nu_{912}}\right)^\alpha,
  \label{eqn:epsilon_freq}
\end{equation}
where $\alpha=-1.70$ (Equation~\ref{eqn:sed}) or $-0.56$ (Equation~\ref{eqn:sed_faint}). 

The emissivity calculated in Equation~(\ref{eqn:epsilon}) is a source
property.  We are now interested in what flux this emissivity produces
in the surrounding intergalactic medium.  This brings in the medium's
properties also, via the radiative transfer equation.  The flux is
\citep{2012ApJ...746..125H}
\begin{multline}
  j(\nu_0, z_0)=\frac{1}{4\pi}\int_{z_0}^\infty dz\frac{dl}{dz}\frac{(1+z_0)^3}{(1+z)^3}\epsilon(\nu,z)\\
  \times\exp{(-\tau_\mathrm{eff}(\nu_0, z_0, z))},
  \label{eqn:flux}
\end{multline}
where
\begin{equation}
  \nu = \nu_0\left(\frac{1+z}{1+z_0}\right).
\end{equation}
We estimate the effective optical depth as
\begin{equation}
  \tau_\mathrm{eff}(\nu_0, z_0, z) = \int_{z_0}^z dz^\prime\int_0^\infty dN_\mathrm{HI} f(N_\mathrm{HI}, z^\prime) (1-e^{-\tau_\nu}),
\end{equation}
where $\tau_\nu=\sigma_\nu N_\mathrm{HI}$ and the column density
distribution $f(N_\mathrm{HI}, z)$ is taken to be a power law \citep{2013MNRAS.436.1023B}
\begin{equation}
  f(N_\mathrm{HI}, z) = \frac{A}{N_\mathrm{LL}}\left(\frac{N_\mathrm{HI}}{N_\mathrm{LL}}\right)^{-\beta_N}\left(\frac{1+z}{4.5}\right)^{-\beta_z},
\end{equation}
with $A=0.93$, $\beta_N=1.33$, $\beta_z=1.92$, and
$N_\mathrm{LL}=10^{17.2}$~cm$^{-2}$.  The hydrogen photoionization
rate is then given by
\begin{equation}
  \Gamma_\mathrm{HI}=4\pi\int_{\nu_{912}}^\infty d\nu \frac{j(\nu,z)}{h\nu} \sigma(\nu),
\end{equation}

\subsection{Local source approximation}

In the local source approximation, Equation~(\ref{eqn:flux}) becomes
\begin{equation}
  j(\nu_0, z_0) = \frac{1}{4\pi}\lambda_\mathrm{mfp}(\nu_0, z_0)\epsilon(\nu_0, z_0),
\end{equation}
where $\lambda_\mathrm{mfp}$ is the mean free path.  The hydrogen
photoionization rate is given by
\begin{equation}
  \Gamma_\mathrm{HI}=4\pi\int_{\nu_{912}}^\infty d\nu \frac{j_\nu}{h\nu} \sigma(\nu),
  \label{eqn:gammapi}
\end{equation}
where $\sigma$ is the ionization cross-section
\begin{equation}
  \sigma(\nu) = \sigma_0\left(\frac{\nu}{\nu_{912}}\right)^{-3},
  \label{eqn:sigma}
\end{equation}
with $\sigma_0=6.3\times 10^{-18}$~cm$^2$ \citep{2006agna.book.....O}.
In the redshift range $z=2.3$--$5.5$, the mean free path is measured
to be \citep{2014MNRAS.445.1745W}
\begin{equation}
  \lambda_\mathrm{mfp}(\nu, z)= \lambda_0\left(\frac{1+z}{5}\right)^{-5.4}\left(\frac{\nu}{\nu_{912}}\right)^{1.5},
  \label{eqn:mfp}
\end{equation}
where the frequency dependence comes from the assumed column density
distribution $f(N_\mathrm{HI})\propto N_\mathrm{HI}^{-1.5}$.  We
extrapolate the redshift dependence in Equation~(\ref{eqn:mfp}) at
$z<2.3$ and $z>5.5$.  Combining Equations~(\ref{eqn:epsilon_freq}),
(\ref{eqn:sigma}), and (\ref{eqn:mfp}), Equation~(\ref{eqn:gammapi})
gives
\begin{multline}
  \Gamma_\mathrm{HI}=4.6\times 10^{-13} \mathrm{s}^{-1} \left(\frac{\epsilon_{912}}{10^{24}\mathrm{erg s^{-1} Hz^{-1} cMpc^{-3}}}\right)\\
  \times\left(\frac{1}{1.5-\alpha}\right)\left(\frac{1+z}{5}\right)^{-2.4}.
\end{multline}
Note that this equation refers to the \emph{comoving} emissivity.  The
local source approximation works well for $z>3$ \citep{2013MNRAS.436.1023B}.




%% \begin{figure*}
%%   \begin{center}
%%     \includegraphics[width=\textwidth,keepaspectratio]{boss.pdf}
%%   \end{center}
%%   \caption{Comparison of LF derived by our method with the published
%%     LF of \citet{2013ApJ...773...14R} for the BOSS DR9 color-selected
%%     sample. There are 23,301 quasars in this sample ($2.2<z<3.5$).  Of
%%     these, 231 quasars have $V_i=0$ and are dropped from the
%%     analysis.}
%%   \label{fig:boss}
%% \end{figure*}

%% \begin{figure*}
%%   \begin{center}
%%     \includegraphics[width=\textwidth,keepaspectratio]{sdss.pdf}
%%   \end{center}
%%   \caption{Comparison with the published LF of
%%     \citet{2006AJ....131.2766R} for SDSS DR3.  There are 15,343
%%     quasars in this sample, of which 155 quasars have $z<0.3$ and 9
%%     quasars have $z>5$.  Of the remaining 15179 quasars with
%%     $0.3<z<5$, 60 quasars have $V_i=0$ and are dropped from the
%%     analysis.  As a result, the green points above contain 15,119
%%     quasars.  The red circles contain 14,953 quasars.}
%%   \label{fig:sdss}
%% \end{figure*}

%% \begin{figure*}
%%   \begin{center}
%%     \includegraphics[width=0.6\textwidth,keepaspectratio]{mcgreer.pdf}
%%   \end{center}
%%   \caption{Comparison with the published LF of
%%     \citet{2013ApJ...768..105M} for $4.7\leq z<5.1$.  There are 148
%%     quasars in their DR7 sample and 52 quasars in the Stripe 82
%%     sample. Of these 1 quasar from the Stripe 82 sample has $V_i=0$.
%%     The published samples of \citet{2013ApJ...768..105M} have 146 and
%%     51 quasars.}
%%   \label{fig:mcgreer}
%% \end{figure*}

%% \begin{figure*}
%%   \begin{center}
%%     \includegraphics[width=0.6\textwidth,keepaspectratio]{glikman.pdf}
%%   \end{center}
%%   \caption{Comparison with the published LF of
%%     \citet{2011ApJ...728L..26G} for $3.5<z<5.2$.  There are 24 
%%     quasars in their two samples (NDWFS and DLS).}
%%   \label{fig:mcgreer}
%% \end{figure*}

%% \begin{figure*}
%%   \begin{center}
%%     \includegraphics[width=\textwidth]{glikman3.pdf}
%%   \end{center}
%%   \caption{Attempts at reproducing Glikman's published LF.  Panels (a)
%%     and (b) use her volume calculations.  Panels (c)--(e) use our
%%     volume calculations.  Various corrections incorporated by Gabor
%%     are in panels (c)--(e).  Red open circles in all panels are
%%     Glikman's published values; I cannot reproduce them in any of the
%%     panels.  All luminosity function in this version of the draft use
%%     Gabor's October data from panel (d).}
%%   \label{fig:glikman}
%% \end{figure*}

%% \begin{figure*}
%%   \begin{center}
%%     \includegraphics[width=\textwidth]{giallongo2.pdf}
%%   \end{center}
%%   \caption{Giallongo's LF when selection function is taken to be
%%     $\phi_\mathrm{corr}/\phi$ from their paper.  This plots shows that
%%     additional correction is required, which has been incorporated in
%%     our LFs.}
%%   \label{fig:giallongo}
%% \end{figure*}

%% \begin{figure*}
%%   \begin{center}
%%     \includegraphics[width=\textwidth,keepaspectratio]{mosaic2.pdf}
%%   \end{center}
%%   \caption{Individual luminosity function fits for $z=2.6$ to $6.5$.}
%%   \label{fig:mosaic2}
%% \end{figure*}

\begin{figure*}
  \begin{center}
    \includegraphics[width=0.6\textwidth,keepaspectratio]{rhoqso_allz.pdf}
  \end{center}
  \caption{AGN number density evolution, showing the systematic offset in BOSS quasars.}
  \label{fig:rhoqso_allz}
\end{figure*}


\begin{figure*}
  \begin{center}
    \includegraphics[width=0.6\textwidth,keepaspectratio]{rhoqso.pdf}
  \end{center}
  \caption{AGN number density evolution with some BOSS quasars
    removed.}
  \label{fig:rhoqso}
\end{figure*}

\begin{figure*}
  \begin{center}
    \includegraphics[width=\textwidth]{evolution_seldata.pdf}
  \end{center}
  \caption{Parameter evolution with some BOSS quasars removed.  We (1)
    use \emph{only} BOSS quasars for $z=2.2$--$3.5$, and (2) use only
    $p>0.9$ quasars in the $z>3.7$ sample of Richards.}
\end{figure*}

\begin{figure*}
  \begin{center}
    \includegraphics[width=\textwidth,keepaspectratio]{mosaic_withGlobal.pdf}
  \end{center}
  \caption{Individual and composite luminosity function fits for
    $z=0.3$ to $2.6$.  They do not agree for $2.2 < z < 2.6$ because
    composite fit does not consider quasars at those redshifts.}
  \label{fig:mosaic_withGlobal}
\end{figure*}

\begin{figure*}
  \begin{center}
    \includegraphics[width=\textwidth,keepaspectratio]{mosaic2_withGlobal.pdf}
  \end{center}
  \caption{Individual and composite luminosity function fits for
    $z=2.6$ to $6.5$.  They do not agree for $2.6 < z < 2.8$ because
    composite fit does not consider quasars at those redshifts.}
  \label{fig:mosaic2_withGlobal}
\end{figure*}

\begin{figure*}
  \begin{center}
    \includegraphics[width=\textwidth]{evolution_global.pdf}
  \end{center}
  \caption{Luminosity function parameter evolution in the global model.}
\end{figure*}

\begin{figure*}
  \begin{center}
    \includegraphics[width=0.6\textwidth,keepaspectratio]{rhoqso_withGlobal.pdf}
  \end{center}
  \caption{AGN number density evolution in the global model.}
  \label{fig:rhoqso}
\end{figure*}

\begin{figure*}
  \begin{center}
    \begin{tabular}{cc}
    \includegraphics[width=0.5\textwidth]{emissivity_18.pdf}
    \includegraphics[width=0.5\textwidth]{emissivity_20.pdf}
    \end{tabular}
  \end{center}
  \caption{LyC emissivity of AGN assuming 100\% escape fraction.
    Model luminosity functions are integrated down to $M_{1450}=-18$
    in the left panel and $-20$ in the right panel.}
  \label{fig:gammapi}
\end{figure*}

\begin{figure*}
  \begin{center}
    \begin{tabular}{cc}
    \includegraphics[width=0.5\textwidth]{g_18.pdf}
    \includegraphics[width=0.5\textwidth]{g_20.pdf}
    \end{tabular}
  \end{center}
  \caption{AGN contribution to the hydrogen photoionisation rate,
    assuming 100\% escape fraction.  Model luminosity functions are
    integrated down to $M_{1450}=-18$ in the left panel and $-20$ in
    the right panel.}
  \label{fig:gammapi}
\end{figure*}

\begin{figure*}
  \begin{center}
    \includegraphics[width=\textwidth,keepaspectratio]{evolution.pdf}
  \end{center}
  \caption{Parameter evolution from individual fits.  Coloured points
    show results when Giallongo quasars are not included.  Black
    points show results when Giallongo quasars are included.}
\end{figure*}

\begin{figure*}
  \begin{center}
    \includegraphics[width=0.7\textwidth,keepaspectratio]{q.pdf}
  \end{center}
  \caption{Helium reionization.}
\end{figure*}

\section*{Acknowledgements}

Lorem ipsum dolor sit amet, consectetur adipiscing elit. Curabitur
eget nisi augue. Vivamus quis purus quis massa tempor posuere non quis
magna. Aenean eleifend, metus eget facilisis faucibus, turpis erat
suscipit tortor, ut dapibus nulla neque ac sapien. Suspendisse luctus
eros eu quam laoreet, vel dapibus sem porttitor. Mauris nec massa
ultrices, porttitor nulla at, euismod diam. In ultricies

\bibliographystyle{mnras}
\bibliography{refs}

\bsp
\label{lastpage}
\end{document}


