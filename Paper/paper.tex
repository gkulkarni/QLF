\documentclass[a4paper,fleqn,usenatbib]{mnras}
\usepackage[T1]{fontenc}
\usepackage{ae,aecompl}
\usepackage{graphicx}	
\usepackage{amsmath}	
\usepackage{amssymb}
\usepackage{pdflscape}
\usepackage{siunitx}
\def\lya{Ly$\alpha$~}
\usepackage{color}
\definecolor{notecolor}{rgb}{0.8,0,0}
\newcommand{\gk}[1]{{\bf \color{notecolor} [#1]}}
\def\HI{\hbox{H~$\scriptstyle\rm I$}}
\def\HII{\hbox{H~$\scriptstyle\rm II$}}
\def\nHI{{\rm HI}}
\def\nH{{\rm H}}
\def\nHII{{\rm HII}}
\def\nHe{{\rm He}}
\def\nHeI{{\rm HeI}}
\def\nHeII{{\rm HeII}}
\def\nHeIII{{\rm HeIII}}
\def\bHII{\hbox{\bf H~$\scriptstyle\bf II$}}
\def\HeI{\hbox{He~$\scriptstyle\rm I$}}
\def\HeII{\hbox{He~$\scriptstyle\rm II$}}
\def\HeIII{\hbox{He~$\scriptstyle\rm III$}}
\def\bHeII{\hbox{\bf He~$\scriptstyle\bf II$}}
\def\HeIII{\hbox{He~$\scriptstyle\rm III$}}
\usepackage{dcolumn}
\newcolumntype{.}{D{.}{.}{2.6}}
\newcolumntype{d}{D{.}{.}{2.3}}

\title[AGN luminosity function]{Evolution of the AGN UV luminosity function from redshift 7.5}

\author[Kulkarni et al.]{{Girish Kulkarni$^{1}$\thanks{Email:
      kulkarni@ast.cam.ac.uk}, G\'abor Worseck$^{2}$ and Joseph
    F.~Hennawi$^{3}$} \\ $^1$Institute of Astronomy and Kavli
  Institute of Cosmology, University of Cambridge, Madingley Road,
  Cambridge CB3 0HA, UK \\ $^2$Institute f\"ur Physik und Astronomie,
  Universit\"at Potsdam, Karl-Liebknecht-Str.\ 24/25, 14476 Potsdam,
  Germany\\ $^3$Department of Physics, Broida Hall, UC Santa Barbara,
  Santa Barbara, CA 93106-9530 USA}
% Update Gabor's address to Potsdam
\date{Accepted ---. Received ---; in original form ---}

\pubyear{2016}

\begin{document}
\label{firstpage}
\pagerange{\pageref{firstpage}--\pageref{lastpage}}
\maketitle

\begin{abstract}
We present the evolution UV/optical luminosity function of quasars
from a homogeneous reanalysis of more than 85,000 spectroscopically
selected quasars from redshift $z=0$ to $7.5$.  We derived luminosity
functions when the observed quasar population is binned in luminosity
and redshift, and then construct a continuous evolutionary model of
the luminosity function spanning a range of redshifts.  We find that
quasars selected from the BOSS survey at $z=2$--$3$ show a significant
systematic offset from the general trends, possiblity due to the
colour bias suffered by BOSS.  Some systematic errors also persist in
the lowest-redshift quasar sample due to host galaxy effects. However,
we find that it is possible to describe the luminosity function
evolution from redshift $z=0$ to $7.5$ using a simple 14-parameter
evolving double power law model.  The evolution of the number density
of qsos shows the familiar downsizing effect, although the evolution
in the faint quasar population is quite flat at high redshifts such
that the the drop in their number density at high redshifts is much
slower than that for bright quasars.  When integrated down to
$M_{1450}=-20$ quasars can produce all of the hydrogen-ionizing
photons in the intergalactic medium upto redshifts 3.5.  They fall
short of the emissivity required to explain the Ly$\alpha$ data at
higher redshifts, and are therefore unlikely to have singlehandedly
driven hydrogen reionization.  This conclusion can be altered,
however, if lower luminosity qsos (down to $M_{1450}=-18$) are assumed
to exist and to have a unity escape fraction, in which case qsos can
potentially produce all the hydrogen-ionizing flux at $z<7.5$.
He~\textsc{ii} reionization in this model occurs between $z=3.2$ and
$3.7$.
\end{abstract}

\begin{keywords}
dark ages, reionization, first stars -- intergalactic medium --
quasars: general -- galaxies: active
\end{keywords}

%% * Possible outline of paper:
%% 
%% Introduction 
%% Data
%% Method
%%   Individual fits
%%   Global fits
%% Contribution to reionizatin 
%% Conclusions
%% 
%% * Optional possibilities:
%%
%% Low-z photon under-production
%% Bolometric LF
%% Public code description 
%% More global models: PLE, PDE, Trenti, White-Conroy
%% Goodness of fit 
%% BH growth 
%% Comparison with galaxy LF
%% Comparison with previous work 

\section{Introduction}

The luminosity function of active galactic nuclei (AGN) and its
evolution over cosmological time scales has been a matter of central
interest of a large body of work over the last four decades
\citep[e.g.,][]{1978A&A....68...17M, 1983ApJ...269..352S,
  1988ApJ...325...92K, 1988MNRAS.235..935B, 1993ApJ...406L..43H,
  1994ApJ...421..412W, 1995AJ....110...68S, 1995AJ....110.2553K,
  1995ApJ...438..623P, 2000MNRAS.317.1014B, 2001AJ....121...54F,
  2004AJ....128..515F, 2006AJ....131.2766R, 2007ApJ...654..731H,
  2009MNRAS.392...19C, 2010AJ....139..906W, 2011ApJ...728L..26G,
  2013ApJ...773...14R, 2013ApJ...768..105M, 2015AA...578A..83G,
  2015ApJ...798...28K, 2016ApJ...829...33Y, 2016ApJ...833..222J,
  2017MNRAS.466.1160M}.  Determination of the AGN luminosity function
constrains models of the build-up of supermassive black holes
\citep{2015MNRAS.452..575S, 2016MNRAS.462..190R}.  Due to the
incidence of supermassive black holes in most galaxies, the tight
scaling relations observed to exist between the mass of these black
holes and properties of their host galaxies
\citep{2013ARA&A..51..511K, 2013ApJ...764..184M}, and the increasing
consensus that AGN activity feeds back on the host galaxy evolution,
the AGN luminosity function also constrains models of galaxy
formation.  Finally, thanks to their relative brightness and high
Lyman continuum escape fraction, the luminosity function of AGN
determines their influence on the temperature and ionization state of
the intergalactic medium (IGM) over large scales, possibly even
primarily driving hydrogen and helium reionization.

The claimed discovery of 19 low-luminosity ($M_{1450}>-22.6$) AGN
between redshifts $z=4.1$ and $6.3$ by \citet{2015AA...578A..83G}
using a novel X-ray/NIR selection criterion, has renewed the interest
in the evolution of the UV luminosity function of AGN.  This finding
suggested that the faint end of the quasar UV luminosity function is
steeper at these redshifts than previously thought
\citep{2007ApJ...654..731H, 2012ApJ...746..125H}.  Using far-UV
spectral slopes from composite spectra of low-redshift quasars, and
assuming a Lyman continuum escape fraction of 100\%,
\citet{2015AA...578A..83G} argued that AGN brighter than
$M_{1450}=-18$ can potentially produce all of the metagalactic
hydrogen photoionization rate inferred from the \lya forest at
$4<z<6$.  Additional indications of a significant presence of AGN at
high redshift ($z\sim 6$) have emerged.  First,
\citet{2015MNRAS.447.3402B} reported a large scatter in the \lya
opacity between different sightlines close to redshift
$z=6$. \cite{2015MNRAS.453.2943C} showed that these opacity
fluctuations extend to substantially larger scales ($\gtrsim 50\,
h^{-1}$cMpc) than expected in reionization histories dominated by
low-luminosity galaxies (see also \citealt{2016MNRAS.460.1328D}).
\citet{2017MNRAS.465.3429C} further demonstrated that opacity
fluctuations on such large scales arise naturally if there is a
significant contribution ($\gtrsim 50\%$) of AGN to the ionising
emissivity at the redshift of the observed opacity fluctuations ($z
\sim 5.5$--$6$) as would be expected for a QSO luminosity that is
consistent with the measurements of \citet{2015AA...578A..83G}.
Second, measurements of the Lyman continuum escape fraction from
high-redshift galaxies are still elusive.  Although high-redshift
galaxies as faint as rest-frame UV magnitude $M_\mathrm{UV}=-12.5$
($L\sim 10^{-3}L^*$) at $z=6$ \citep{2017ApJ...835..113L} and
redshifts as high as $z=11.1$ \citep{2016ApJ...819..129O} have now
been reported, the escape of Lyman continuum photons has been detected
in only a small number of comparatively bright ($L>0.5L^*$)
low-redshift ($z < 4$) galaxies.  In these galaxies, the escape
fraction is typically found to be 2--20\% \citep{2010ApJ...725.1011V,
  2011ApJ...736...41B, 2015ApJ...804...17S, 2015ApJ...810..107M,
  2016A&A...585A..48G, 2017MNRAS.468..389J, 2017MNRAS.465..316M} but
reionization would require escape fraction of about 20\% in galaxies
down to $M_\mathrm{UV}=-13$ \citep{2016PASA...33...37F,
  2015ApJ...802L..19R, 2016MNRAS.457.4051K}.  Finally, incidence of
high-redshift AGN is also consistent with the shallow bright-end
slopes of the high-redshift ($z\sim 7$) UV luminosity function of
galaxies relative to a Schechter-function representation
\citep{2012MNRAS.426.2772B, 2014MNRAS.440.2810B, 2014ApJ...792...76B,
  2015MNRAS.452.1817B} and the hard spectra of these bright galaxies
\citep{2015MNRAS.450.1846S, 2015MNRAS.454.1393S, 2017MNRAS.464..469S}.

However, these arguements in favour of a higher incidence of
UV-emitting AGN at high redshit, have been contested by several
observations.  \citet{2016arXiv160706467D} considered the effect of
AGN-dominated reionization on the Ly$\alpha$ opacity at $z>5$,
He~\textsc{ii} Ly$\alpha$ opacity at $z\sim 3.1$--$3.3$, and the
thermal history of the IGM.  In agreement with
\citet{2015MNRAS.453.2943C}, these authors found that AGN did provide
a plausible explanation for the large fluctuations in the Ly$\alpha$
opacity at $z>5$.  However, they found that reionization of
He~\textsc{ii} occurs much earlier in these AGN-dominated models (see
also \citealt{2016arXiv160602719M}).  For instance, in the model of
\citet{2015ApJ...813L...8M}, He~\textsc{ii} reionization is complete
at $z=4.5$, compared to $z=3$ in the standard scenario
\citep{2012ApJ...746..125H}.  This early Helium reionization could
result in higher IGM temperatures due to the associated photoheating.
The temperature of the IGM at mean density is enhanced in
AGN-dominated models by factors of $\sim 2$ relative to the standard
models for $z=3.5$--$5$, in conflict with measurements.  This
inconsistency could be avoided by reducing the escape fraction of 4~Ry
photons in AGN, but it is not clear if this can be achieved while
requiring a 100\% escape fraction of 1~Ry photons in order to explain
the Ly$\alpha$ opacity fluctuations.  Further evidence against
AGN-dominated reionization models has emerged from metal-line
absorbers at $z\sim 6$.  In their cosmological radiation
hydrodynamical simulations, \citet{2016MNRAS.459.2299F} find that the
hard spectral slopes of UV backgrounds in AGN-only reionization models
produce too many C~\textsc{iv} absorption systems relative to
Si~\textsc{iv} and C~\textsc{ii} at $z\sim 6$.  However, these
simulations assume an $L_\nu\propto\nu^{-1.57}$ AGN SED at extreme UV
\citep{2001AJ....122..549V, 2002ApJ...579..500T, 2012ApJ...746..125H}.
This slope is marginally harder than recent measurements
($L_\nu\propto\nu^{-1.7}$) from a stack of $z\sim 2.4$ quasars
\citep{2015MNRAS.449.4204L}.  \citet{2016MNRAS.459.2299F} also find
that the N(Si~\textsc{iv})/N(C~\textsc{iv}) column density ratio
measurements prefer a somewhat harder and more intense $>4$~Ry
background than the standard model of \citet{2012ApJ...746..125H}.
Using a large sample of X-ray-selected quasars in the redshift range
$z=0$--$6$, \citet{2017MNRAS.465.1915R} find that the faint end of the
AGN UV luminosity function at $z\sim 6$ is likely to be much shallower
that that reported by \citet{2015AA...578A..83G}.  In their analysis,
\citet{2017MNRAS.465.1915R} use an AGN obscuration optical depth
($\log N_\mathrm{H}$) cut-off that reproduces low-redshift AGN UV
luminosity functions and an X-ray-to-optical/UV luminosity ratio
calibrated at redshifts $z=0.05$--$4$ \citep{2010A&A...512A..34L}.
These authors argue that the apparent contradiction with the results
of \citet{2015AA...578A..83G} could be explained by contamination
from the AGN host galaxies.  It has also been recently argued that the
Lyman continuum escape fraction of AGN might not be 100\% as is
usually assumed \citep{2017MNRAS.465..302M}.  This may further reduce
the contribution of AGN to reionization.

We assume a flat cosmology with density parameters
$\left(\Omega_\mathrm{m},\Omega_\Lambda\right)=\left(0.3,0.7\right)$
and a Hubble constant $H_0=70$\,km\,s$^{-1}$\,Mpc$^{-1}$. Magnitudes
are reported in the AB system \citep{1983ApJ...266..713O}, and
observed magnitudes are point spread function (PSF) magnitudes
\citep{2002AJ....123..485S} corrected for Galactic extinction
\citep{1998ApJ...500..525S} unless otherwise noted.

\begin{figure*}
  \begin{center}
    \includegraphics[width=\textwidth]{qsos.pdf}
    % data.py 
  \end{center}
  \caption{Redshift distribution of the XX quasars considered in this
    analysis.  Shown here are the observed quasar numbers, without
    correcting for incompleteness.  Further details on each of these
    data sets are in Table~\ref{tab:samples} and
    Section~\ref{sec:sample}.}
  \label{fig:qsos}
\end{figure*}

\begin{table*}
  % In published version of this table, we could remove file names and add data homogenisation in Notes.
  % Add z ~ 7 qso when you add it to analysis
  % See data.tex for a version of this table that includes file names
  % Should we mark Giallongo, Mortlock, and Banados entries to say that these are sometimes excluded from analysis?
  % Remove notes?
  % Do we really need a sample ID column?
  % Say in caption or text that sample 13 is restricted to $z<4.7$
  % Also that sample 17 overlaps with 8 for $M_{1450}<-26.73$
  % Also that in sample 7 we rescaled selection probabilities 
  \caption{Quasar data sets}
  \label{tab:samples}
  \begin{tabular}{crcllrS}
    \hline
    & ID & $z$ & Survey & Reference & Number & {Area} \\
    & & & & & of quasars & {(deg$^2$)} \\
    \hline
    1. & 13 & 0.1--2.2 & SDSS DR7 & \citet{2006AJ....131.2766R} & 48664 & 6248.0 \\
    2. & 15 & 0.4--2.6 & 2SLAQ SGP & \citet{2009MNRAS.392...19C} & 3663 & 64.2 \\
    3. & 15  & 0.4--2.6 & --- NGP & \citet{2009MNRAS.392...19C} & 8153 & 127.7 \\
    4. &  1 & 2.2--3.5 & BOSS DR9 & \citet{2013ApJ...773...14R} & 23301 & 2236.0 \\
    5. & 13 & 3.7--4.7 & SDSS DR7 & \citet{2006AJ....131.2766R} & 1785 & 6248.0 \\
    6. & 17 & 4.7--5.4 & SDSS+WISE & \citet{2016ApJ...829...33Y} & 99 & 14555.0 \\
    7. &  8 & 4.7--5.1 & SDSS DR7 & \citet{2013ApJ...768..105M} & 148 & 6248.0 \\
    8. &  8 & 4.7--5.1 & --- Stripe 82 & \citet{2013ApJ...768..105M} & 52 & 235.0 \\
    9. &  8 & 5.1--5.5 & --- DR7 Extended & \citet{2013ApJ...768..105M} & 28 & 6248.0 \\
    10. & 8 & 5.1--5.4 & ---  Stripe 82 Extended & \citet{2013ApJ...768..105M} & 10 & 235.0 \\
    11. & 6 & 3.7--4.8 & NDWFS & \citet{2011ApJ...728L..26G} & 12 & 1.71 \\
    12. & 6 & 3.8--5.1 & DLS & \citet{2011ApJ...728L..26G} & 12 & 2.05 \\
    13. & 7 & 4.1--6.3 & CANDELS GOODS-S & \citet{2015AA...578A..83G} & 19 & 0.047 \\
    14. & 18 & 5.8--6.4 & --- Main & \citet{2016ApJ...833..222J} & 24 & 11240.0 \\
    15. & 18 & 5.9--6.1 & --- Overlap & \citet{2016ApJ...833..222J} & 10 & 4223.0 \\
    16. & 18 & 5.7--6.1 & --- Stripe 82 & \citet{2016ApJ...833..222J} & 13 & 277.0 \\
    17. & 10 & 6.0 & CFHQS Deep & \citet{2010AJ....139..906W} & 1 & 4.47 \\
    18. & 10 & 5.9--6.4 & --- Very Wide & \citet{2010AJ....139..906W} & 16 & 494.0 \\
    19. & 11 & 6.0--6.2 & Subaru High-$z$ Quasar & \citet{2015ApJ...798...28K} & 2 & 6.5 \\
    20. & 19 & 6.5--7.4 & UKIDSS & \citet{2011Natur.474..616M} & 1 & 3370.0 \\
    21. & 19 & 6.5--7.4 & UKIDSS & \citet{2007MNRAS.376L..76V} & 1 & 3370.0 \\
    22. & 20 & 6.5--7.4 & ALLWISE+UKIDSS+DECaLS & \citet{2018Natur.553..473B} & 1 & 2500.0 \\
    \hline
  \end{tabular}
\end{table*}

\section{Homogenised Quasar Sample}
\label{sec:sample}

\subsection{Sample Selection}

We started with compiling the samples of recent UV-optical quasar
surveys. The restriction to UV-optical surveys was mainly driven by
our science goal to characterise the UV luminosity function of Type~I
quasars, but also by the still uncertain conversion from X-ray flux to
UV flux, which adds unnecessary systematics for large
samples. \textbf{References!} The individual surveys and their main
characteristics are listed in
Table~\ref{tab:samples}. Figure~\ref{fig:qsos} presents a redshift
histogram of the contributing surveys.

We included surveys based on a set of simple criteria:
\begin{enumerate}
\item Spectroscopic redshifts for the majority of targets.
\item Accurate rest-frame UV-optical CCD photometry.
\item Statistical power (sample size, coverage in $z$ and/or $M_{1450}$).
\item A characterised selection function.
\end{enumerate}
As a prerequisite for a joint analysis of the QLF we obtained the
survey selection functions in electronic form, either from the
publication or from the authors. As a reference for future surveys we
publish them here in modified and homogenised form
(Section~\ref{sect:datahom}). \textbf{Maybe get the authors' consent.}

Due to their selection criteria and their statistical power specific
surveys contribute to distinct redshift ranges. At $z<2.2$ we
considered quasars from the SDSS DR7 quasar catalogue
\citep{2010AJ....139.2360S} and the 2SLAQ catalogue
\citep{2009MNRAS.392...19C}. We restricted the SDSS DR7 sample to XXX
$z<2.2$ quasars selected with the final SDSS quasar selection
algorithm \citep{2002AJ....123.2945R, 2006AJ....131.2766R} from a
survey area of 6248\,deg$^2$ \citep{2012ApJ...746..169S}. We adopted
the SDSS targeting photometry corrected for Galactic extinction
\citep{2010AJ....139.2360S}. To limit systematic uncertainties in the
correction for host galaxy light \citep{2009MNRAS.392...19C} we
restricted the 2SLAQ sample to XXX $g<21.85$ $0.4<z<2.2$ quasars from
its spectroscopic survey footprint near the North Galactic Pole (NGP,
XX quasars in $127.7$\,deg$^2$) and the South Galactic Pole (SGP, XX
quasars in $64.2$\,deg$^2$). The small sample overlap between SDSS and
2SLAQ (112 quasars) has negligible impact on the QLF evaluation.

At $2.2<z<3.5$ we used a single sample of 23,301 uniformly
colour-selected quasars from 2236\,deg$^2$ in BOSS DR9
\citep{2013ApJ...773...14R} due to several improvements compared to
previous surveys. First, it covers a similar magnitude range as 2SLAQ
but with $>20$ times as many quasars. Second, although the SDSS DR7
sample provides better coverage of the bright end of the QLF at these
redshifts, its selection function is highly dependent on the assumed
incidence of (partial) Lyman limit systems in the IGM
\citep{2009ApJ...705L.113P, 2011ApJ...728...23W}. While the BOSS DR9
selection function considers these improvements, the uncertainty in
the QLF remains dominated by assumptions in the selection function
given the large sample size
\citep{2013ApJ...773...14R}. Variability-selected quasar samples
circumvent this issue \citep{2013ApJ...773...14R, 2013A&A...551A..29P,
  2016A&A...587A..41P}, but may be affected by (i) single-epoch
imaging incompleteness at the faint end \citep{2013ApJ...773...14R},
and (ii) uncertainties in the selection function caused by the limited
number of known $z\ga 3$ quasars not selected by variability in the
same footprint \citep{2013A&A...551A..29P, 2016A&A...587A..41P}.

At $3.7<z<4.1$ we used a combination of SDSS DR7 \citep[XX uniformly
  selected quasars from][]{2010AJ....139.2360S} and the NDWFS$+$DLS
survey \citep{2010ApJ...710.1498G,2011ApJ...728L..26G}. The lower cut
$z>3.7$ limits the impact of systematic uncertainties in the
\citet{2006AJ....131.2766R} SDSS selection function
\citep{2009ApJ...705L.113P, 2011ApJ...728...23W}. We did not consider
the results from surveys for faint $z\sim 4$ quasars in the COSMOS
field \citep{2011ApJ...728L..25I, 2012ApJ...755..169M} due to
systematic errors in their selection functions\footnote{Both studies
  simulated quasar colours with a mean IGM attenuation curve
  \citep{1995ApJ...441...18M} that cannot account for stochastic Lyman
  continuum absorption, and therefore underpredicts the variance in
  quasar colours \citep{1999ApJ...518..103B, 2008MNRAS.387.1681I,
    2011ApJ...728...23W}. Modelling the colour variance in these
  surveys is essential, as most of the \citet{2011ApJ...728L..25I}
  quasars are near the edge of their colour selection region (see
  their Figure~1), and \citet{2012ApJ...755..169M} require modest
  attenuation of the $U$ band flux relative to the mid-infrared
  flux.}. Furthermore, 30 per cent of the \citet{2012ApJ...755..169M}
COSMOS sample have visually estimated photometric redshifts, and the
spectroscopic subsample reveals that 40 per cent of the visually
estimated redshifts are biased low
($z_\mathrm{spec}>z_\mathrm{est}+0.3$, see their Figure~9). These
unaccounted systematic redshift errors at least partly explain the
discrepancy in the $z\sim 4$ QLF between \citet{2011ApJ...728L..26G}
and \citet{2012ApJ...755..169M}, which justifies our preference for
the former sample.

Between $z=4.1$ and $z=4.7$ our combined sample has XX uniformly
selected quasars from the SDSS DR7 quasar catalogue
\citep{2010AJ....139.2360S} and XX quasars from
\citet{2011ApJ...728L..26G}. Although we do not include the highly
debated \citet{2015AA...578A..83G} sample in our main analysis due to
its rough selection function and lacking spectroscopy for 17 of the 22
quasar candidates, we use it in Section~XX to constrain the faint end
($M_{1450}>-23$) of the QLF at $z>4.1$. We restricted the
\citet{2015AA...578A..83G} sample to the 19/22 sources considered in
their QLF.

At $4.7\le z<5.1$ we combined several recent surveys, accounting for
sample overlap and updated selection functions. At the bright end of
the QLF we used the 71 $z<5.1$ quasars from the SDSS+WISE survey
\citep{2016ApJ...829...33Y} together with its selection function.
%Although \citet{yang} are sensitive to $z>5.1$ quasars, we limit
%their sample to $z<5.1$ to limit differential evolution of the QLF
%across the redshift range.
To avoid double-counting quasars contained both in
\citet{2016ApJ...829...33Y} and the SDSS DR7 sample from
\citet{2013ApJ...768..105M}, we used the latter only at
$M_{1450}>-26.72$ (in our adopted cosmology), yielding 94 additional
$4.7\le z<5.1$ quasars selected in 6248\,deg$^2$. We used the updated
$z\sim 5$ SDSS DR7 selection function \citep{2013ApJ...768..105M}
instead of the one from \citet{2006AJ....131.2766R}.
%\footnote{We note that the major difference in the selection
%functions must be due to different input parameters, since the single
%color selection criterion in \citet{mcgreer} selects the vast
%majority of the SDSS DR7 quasars.}.
To these two bright-end samples we added the faint-end sample from the
\citet{2013ApJ...768..105M} SDSS Stripe~82 survey (52 uniformly
selected $4.7\le z<5.1$ quasars in 235\,deg$^2$) and two quasars from
\citet{2011ApJ...728L..26G}, adopting the respective selection
functions. We did not consider the limit on the $z\sim 5$ QLF by
\citet{2012ApJ...756..160I} due to systematic errors in their
selection function\footnote{\citet{2012ApJ...756..160I} underestimated
  the dispersion in rest-frame UV quasar colours with respect to SDSS
  at all redshifts (their Figure~4). Contrary to their claim,
  photometric errors have a small effect on the colour distribution of
  SDSS quasars given the statistical errors of $<0.03$\,mag in $gri$
  for 90 per cent of the SDSS DR7 bright quasar sample ($i<19.1$) and
  a relative calibration error of $\sim 1$ per cent
  \citep{2008ApJ...674.1217P}.}.

The SDSS colours of $5.1\le z<5.5$ quasars are similar to those of M
and L dwarf stars, resulting in a low and uncertain selection
efficiency \citep{2013ApJ...768..105M}. WISE mid-infrared selection
performs better \citep{2016ApJ...829...33Y}, but is restricted to the
bright end of the quasar population. Our combined sample contains the
remaining 28 quasars from \citet{2016ApJ...829...33Y} and 10 quasars
from the \citet{2013ApJ...768..105M} Stripe~82 survey. As we will show
in Section~XX, the resulting QLF is consistent with those at lower and
higher redshifts, indicating that the \citet{2013ApJ...768..105M}
selection functions are quite reliable. \textbf{Check this!}

Finally, we combined the samples from all spectroscopic $z\sim 6$
quasar surveys with a determined selection function as of June
2017. \citet{2016ApJ...833..222J} recently compiled all quasars
discovered in several SDSS $z\sim 6$ surveys together with
consistently derived selection functions. Their uniform sample
consists of 24 quasars from the SDSS main survey (11240\,deg$^2$), 10
additional quasars in regions with two or more SDSS imaging scans
(so-called overlap regions, 4223\,deg$^2$), and 13 faint quasars from
SDSS Stripe~82 (277\,deg$^2$). The CFHQS \citep{2010AJ....139..906W}
provided a uniform sample of 16 quasars in the Very Wide Survey
(494\,deg$^2$) and a single quasar in the Deep Survey
($4.47$\,deg$^2$). The one quasar detected in both SDSS and CFHQS does
not lead to underestimated statistical errors in the QLF. Lastly, we
included the two objects from \citet{2015ApJ...798...28K}. One might
be a Lyman break galaxy due to its narrow Ly$\alpha$ emission line
(half width at half maximum 427\,km\,s$^{-1}$), but
\citet{2015ApJ...798...28K} note that 10/17 candidates do not have
follow-up spectroscopy, hence their QLF may be underestimated. Since
very small samples preclude a statistical correction via the selection
function, complete spectroscopic follow-up of $z\sim 6$ candidates is
essential. Although we will account for the slightly different
redshift sensitivities for the different surveys, we will quote a
nominal redshift range $5.7<z<6.5$ for the combined $z\sim 6$ sample.

\subsection{Sample Homogenisation}
\label{sect:datahom}

\begin{figure}
    \includegraphics[width=\columnwidth]{kcorr.pdf}
  \caption{Bandpass corrections $K_{m,1450}$ from a broadband magnitude $m=\{g,i,z_\mathrm{AB}\}$
           to the monochromatic AB magnitude at 1450\,\AA\ as a function of redshift $z$ for the \citet{2015MNRAS.449.4204L}
           quasar SED used in this work, and for two quasar composite spectra \citep{2001AJ....122..549V, 2002ApJ...565..773T}.
           The redshift range has been restricted to exclude the Ly$\alpha$ forest and to account for the different
           rest frame wavelength coverage of the spectra.}
  \label{fig:kcorr}
\end{figure}

For a joint fit of the QLF it is necessary to homogenise the different
survey samples in absolute magnitude, and to convert their selection
functions to the same absolute magnitude system. For the analysis of
the quasar UV emissivity and to be consistent with published work at
$z>3$ we chose to convert all samples and selection functions to the
absolute AB magnitude at a rest frame wavelength $\lambda=1450$\,\AA
\begin{equation}\label{eq:absmag}
M_{1450}\left(z\right) = m-5\log{\left(\frac{d_L\left(z\right)}{\mathrm{Mpc}}\right)}-25-K_{m,1450}\left(z\right),
\end{equation}
with the luminosity distance $d_L\left(z\right)$ to a quasar at
redshift $z$, the apparent magnitude $m$ in a filter used in the
survey, and the bandpass correction $K_{m,1450}\left(z\right)$
\citep{1956AJ.....61...97H, 1968ApJ...154...21O, 2000A&A...353..861W,
  2002astro.ph.10394H}. For the bandpass correction we used a
combination of the \citet{2015MNRAS.449.4204L} quasar SED at
$\lambda<2500$\,\AA, and the \citet{2001AJ....122..549V} quasar
composite spectrum at longer wavelengths to cover the lowest
redshifts.  The samples from SDSS and BOSS are defined in the SDSS $i$
band, while 2SLAQ is defined in the $g$ band. At $z>4.7$ we adopted
the SDSS $z$ band magnitude (in the following denoted $z_\mathrm{AB}$)
for SDSS DR7 quasars to avoid additional corrections due to the
Ly$\alpha$ forest. Figure~\ref{fig:kcorr} shows our bandpass
corrections for SDSS, BOSS and 2SLAQ as a function of redshift. We
ignored the luminosity dependence of the bandpass correction due to
the known anticorrelation of emission line equivalent width and
luminosity \citep{1977ApJ...214..679B}.  While the
\citet{2015MNRAS.449.4204L} SED is for luminous ($M_{1450}\simeq
-27.2$) quasars, UV composite spectra including fainter quasars
\citep{2002ApJ...565..773T, 2012ApJ...752..162S, 2014ApJ...794...75S}
give similar values, such that our bandpass corrections remain
applicable at $M_{1450}\la -24$. Empirical luminosity-dependent
bandpass corrections show a $\la 0.2$\,mag variation over $\sim 4$
orders of absolute magnitude depending on redshift and the filter, and
with a $\sim 0.2$\,mag intrinsic scatter due to individual SED
variations \citep{2013ApJ...773...14R, 2013ApJ...768..105M,
  2013A&A...551A..29P}.

For the $z<2.2$ sample we corrected the SDSS $i$ and 2SLAQ $g$ band
magnitudes for host galaxy contamination following
\citet{2009MNRAS.392...19C}. Considering the different magnitude
limits of 2SLAQ and SDSS, the modelled host galaxy contamination is
small for $z>0.5$ quasars ($<0.1$\,mag in $g$, $<0.2$\,mag in $i$),
and is negligible at $z>0.8$.  In case the band defining the magnitude
limit of the survey undesirably overlaps with the Ly$\alpha$ forest
\citep{2010ApJ...710.1498G, 2011ApJ...728L..26G, 2013ApJ...768..105M}
we adopted their respective bandpass corrections to $M_{1450}$. In
particular, for the \citet{2010ApJ...710.1498G, 2011ApJ...728L..26G}
sample we recomputed $M_{1450}$ from the $R$ band photometry to be
consistent with the selection function defined in $R$, and to avoid
uncertainties in their spectrophotometry due to incomplete spectral
coverage. Since the \citet{2010ApJ...710.1498G, 2011ApJ...728L..26G}
$R$ band traces the rest frame UV, we assumed negligible host galaxy
contamination for their faint quasars. For the remaining high-redshift
surveys reporting $M_{1450}$ obtained by various methods
\citep{2010AJ....139..906W, 2015ApJ...798...28K, 2015AA...578A..83G,
  2016ApJ...829...33Y, 2016ApJ...833..222J} we did not re-compute
$M_{1450}$, but applied appropriate shifts to correct to our adopted
cosmology.

The selection functions were treated similarly, i.e.\ the photometric
selection function of survey $j$ given in observed magnitudes
$f_{\mathrm{p},j}\left(m,z\right)$ \citep{2006AJ....131.2766R,
  2009MNRAS.392...19C, 2010ApJ...710.1498G, 2013ApJ...773...14R} was
transformed to our absolute magnitudes
$f_{\mathrm{p},j}\left(M_{1450},z\right)$ with
Equation~\ref{eq:absmag}, while the ones given in $M_{1450}$ were
adjusted to our cosmology. Note, however, that many surveys report
additional sources of incompleteness that preclude a straightforward
re-evaluation of the QLF.

For 2SLAQ we corrected for magnitude-dependent spectroscopic coverage
in the two survey areas \citep[$f_\mathrm{c,NGP}\left(g\right)$ and
  $f_\mathrm{c,SGP}\left(g\right)$; Figure~4
  in][]{2009MNRAS.392...19C} and spectroscopic redshift success
\citep[$f_\mathrm{s,2SLAQ}\left(g\right)$; Figure~6b
  in][]{2009MNRAS.392...19C} by multiplying them with the photometric
selection function, resulting in two area-specific 2SLAQ selection
functions
$f_\mathrm{NGP}\left(M_{1450},z\right)=f_\mathrm{p,2SLAQ}f_\mathrm{c,NGP}f_\mathrm{s,2SLAQ}$
and
$f_\mathrm{SGP}\left(M_{1450},z\right)=f_\mathrm{p,2SLAQ}f_\mathrm{c,SGP}f_\mathrm{s,2SLAQ}$
that are relevant for the QLF.  The $z<4.7$ SDSS photometric selection
function was modified to include known imaging incompleteness to
$f_{\mathrm{SDSS},z<4.7}=0.95f_{\mathrm{p,SDSS},z<4.7}$
\citep{2006AJ....131.2766R}. The BOSS colour-selected sample contains
quasars with $f_\mathrm{c,BOSS}f_\mathrm{s,BOSS}\ge 0.85$
\citep{2013ApJ...773...14R}, and we adopted
$f_\mathrm{BOSS}=\overline{f_\mathrm{c,BOSS}f_\mathrm{s,BOSS}}f_\mathrm{p,BOSS}=0.962f_\mathrm{p,BOSS}$. \citet{2010ApJ...710.1498G}
presented two area-specific photometric selection functions due to
different filters employed, and more follow-up spectroscopy was
reported in \citet{2011ApJ...728L..26G}. We accounted for remaining
spectroscopic incompleteness at $R>23$, yielding the final selection
functions $f_\mathrm{NDWFS}$ and $f_\mathrm{DLS}$. The updated $z\sim
5$ SDSS photometric selection function \citep{2013ApJ...768..105M} was
modified to include imaging and spectroscopic incompleteness, yielding
$f_{\mathrm{SDSS},z\sim 5}=0.95^2f_{\mathrm{p,SDSS},z\sim 5}$. In the
deeper $z\sim 5$ SDSS Stripe~82 survey the spectroscopic
incompleteness is larger and magnitude-dependent \citep[Figure~14
  in][]{2013ApJ...768..105M}, resulting in $f_{\mathrm{S82},z\sim
  5}=0.95f_{\mathrm{s,S82},z\sim
  5}\left(i\right)f_{\mathrm{p,S82},z\sim 5}$. Likewise, imaging and
magnitude-dependent spectroscopic incompleteness was factored into the
\citet{2016ApJ...829...33Y} photometric selection function (their
Figures~5 and 7), resulting in
$f_\mathrm{SDSS+WISE}=0.97f_\mathrm{s,SDSS+WISE}\left(z_\mathrm{AB}\right)f_\mathrm{p,SDSS+WISE}$. Finally,
we obtained a rough estimate of the \citet{2015AA...578A..83G}
selection function by comparing the corrected and observed QLFs,
i.e.\ taking $f_\mathrm{GOODS-S}=\phi_\mathrm{obs}/\phi_\mathrm{corr}$
(see their Table~3).


% BOSS:
% - color selection function in Table 5 of Ross et al. (imag, z), sample contains QSOs with spectrocoverage completeness >0.85, mean spectrocoverage completeness 0.962 multiplied onto color selection function
% - interpolate Beta's K correction (imag->1450A) and compute M1450 for our cosmology (sample and map)

% 2SLAQ:
% - low-luminosity AGN -> model and subtract host galaxy contribution, magnitudes only for nuclear component, host galaxy contribution <20% at z>0.4 for faintest sources in 2SLAQ (Croom et al. 2009, MNRAS, 392, 19)
% - several selection functions (morphological, color, spectroscopic coverage, spectroscopic success)
% - Table 12: color selection function (available online), gmag is total PSF mag including host galaxy contribution, subtract gmag host galaxy contribution listed in Table 12, selection function extrapolated at bright end (lacking coverage of whole catalog)
% - coverage completeness for NGP and SGP given in Fig 4 as a function of gmag, interpolated and multiplied into color selection function
% - spectroscopic success as function of gmag in Fig 6b (only quality=1 redshifts considered), interpolated and multiplied onto color selection function
% - convert g_agn to M1450 with L15 SED (valid for bright quasars, check with K correction for Telfer/Shull)
% - sample: correct PSF gmag for Galactic extinction, subtract host galaxy light (Table 12), interpolate Beta's K correction (gmag->1450A) and compute M1450 for our cosmology
% SDSS:
% - Richards et al. selection function given in (z,imag) Fig 6 and Table 1, corrected for edge effects at magnitude limits (steep gradient) and NaN values, selection function corrected for imaging incompleteness (factor 0.95)
% - imag in selection function converted to M1450 with L15 SED, conversion uncertain at z>4.7 due to Lya forest
% - sample: targeting psf imag corrected for Galactic extinction (Schlegel) and converted to M1450 with L15 SED
% - z<2.2: sample limited to i_dered<19.1, imag of sample and selection function corrected for host galaxy light (Table 12 in Croom et al 2009, given for imag and gmag)
% Glikman:
% - different selection functions for 2 sub-surveys, selection functions corrected for interpolation errors (values >1 reset to 1), selection functions given in (z,Rmag) Fig 7 of G10, converted to M1450 with their Rmag K correction (Fig 8 in G10)
% - spectroscopic incompleteness (f_spec, given in G10 and updated in G11) incorporated into selection functions via linear fit to f_spec(Rmag) at Rmag>23 and constant otherwise
% - sample M1450 recomputed from Rmag and K correction to be consistent with completeness map, M1450=Rmag-Kcorr-5*log10(d_L/Mpc)-25
% McGreer:
% - Stripe 82: selection function (Fig. 11) modified to include photometric incompleteness (factor 0.95) and spectroscopic incompleteness (imag>20.8, Fig. 14), M1450 in selection map and sample M1450 converted to our cosmology (adding 0.07)
% - DR7: sample not listed in paper, M1450 computed from zmag (QSO continuum) and L15 SED, same procedure applied to S82 reveals L15 magnitudes are slightly brighter than M13 magnitudes (median 0.12mag) -> small change,
%        M13 selection function corrected for photometric (factor 0.95) and spectroscopic (factor 0.95) incompleteness
% Yang: selection function (Fig. 5) modified to include image selection incompleteness (factor 0.97) and spectroscopic incompleteness (factor ~0.84 at zmag>19, Fig 7), M1450 in selection map and sample M1450 converted to our cosmology
% Willott: selection map M1450 adjusted to our cosmology adding 0.05, slightly different cosmological parameters in Willott et al. (2007,2009,2010) -> M1450 converted
% Jiang: selection functions unchanged, M1450 as given in paper
% Kashikawa: map not corrected for spectroscopic incompleteness (sample size), M1450 as given in paper
% Giallongo: selection function estimated from phi_obs and phi_corr (Table 3 of Giallongo et al.), M1450 as given in paper

\section{Luminosity function}

Given quasar luminosities and estimates for survey completeness, we
are in a position to estimate the UV luminosity function of AGN.  We
begin our analysis by first deriving luminosity functions in bins of
redshift and magnitude.  In this discussion all magnitudes are
absolute magnitudes at rest-frame 1450~{\AA}.

% Put that sentence about magnitudes in the introduction.

\subsection{Binned luminosity function estimates}
\label{sec:binnedlf}

In a magnitude bin $[M_\mathrm{min}, M_\mathrm{max})$, and redshift
  bin $[z_\mathrm{min}, z_\mathrm{max})$, we define the luminosity
    function as \citep{2000MNRAS.311..433P}
  \begin{equation}
    \phi \equiv \frac{N}{V_\mathrm{bin}},
  \end{equation}
  where $N$ is the number of quasars with magnitude
  $M_\mathrm{min}\leq M<M_\mathrm{max}$ and redshift
  $z_\mathrm{min}\leq z<z_\mathrm{max}$, and
  \begin{equation}
    V_\mathrm{bin} = \int_{M_\mathrm{min}}^{M_\mathrm{max}}dM\int_{z_\mathrm{min}}^{z_\mathrm{max}}dz\, f(M, z)\,\frac{dV}{dz},
    \label{eqn:vi}
  \end{equation}
  is the effective volume of the bin.  Here $f(M,z)$ is the quasar
  selection probability, which includes incompleteness.  Inclusion of
  the selection probability in Equation~(\ref{eqn:vi}) accounts for
  what are sometimes called ``incomplete bins''
  \citep{2006AJ....131.2766R}.  The comoving volume element $dV/dz$ is
  given by
  \begin{equation}
    \frac{dV}{dz}=\frac{dV}{dz\,d\Omega}\cdot A\cdot\frac{4\pi}{41253},
  \end{equation}
  where $A$ is the survey area in deg$^2$, and 
  \begin{equation}
    \frac{dV}{dz\,d\Omega}=\frac{c}{H_0}\frac{d_L(z)^2}{(1+z)^2\sqrt{\Omega_m(1+z)^3+\Omega_\Lambda}},
    \label{eqn:dvdzdo}
  \end{equation}
  denotes the comoving volume element per unit solid angle
  \citep{1999astro.ph..5116H}.  The luminosity distance $d_L$ is given
  by
  \begin{equation}
    d_L(z)=(1+z)\frac{c}{H_0}\int_0^z\frac{dz}{\sqrt{\Omega_m(1+z)^3+\Omega_\Lambda}}.
    \label{eqn:dl}
  \end{equation}
  Equations~(\ref{eqn:dvdzdo}) and (\ref{eqn:dl}) assume a flat
  Universe ($\Omega_k=0$).  The luminosity function $\phi$ has units
  of $\mathrm{cMpc}^{-3}\mathrm{mag}^{-1}$.  We evaluate the double
  integral in Equation~(\ref{eqn:vi}) by the Euler method, i.e., by
  simply summing over the ``tiles'' of the selection function map,
  without interpolating between the redshift and luminosity values of
  neighbouring tiles.  This may result in $V_i=0$ for some quasars, in
  which case we remove them from our analysis.

  In each bin, we estimate the uncertainty in the luminosity function
  by assuming Poisson statistics \citep{1986ApJ...303..336G} for the
  number of quasars in each bin.  The resultant binned luminosity
  function estimates are shown by the points in
  Figure~\ref{fig:mosaic}.

  As seen in Figure~\ref{fig:mosaic}, the distribution of luminosity
  function values in each redshift bin are suggestive of a double
  power law form.  We will fit such a form below.  However, we also
  see in several redshift bins of Figure~\ref{fig:mosaic} that the
  luminosity function seems to show a decline at the faint end.  This
  is clearly seen, for example, in the 2SLAQ sample at $z<2$ and in
  the SDSS quasar sample at $z<1.8$ and $z\sim 4$.  We take this to be
  a systematic error.  The decrease in the luminosity function is an
  indication of further low-level incompleteness in the quasar surveys
  that has not been captured by the reported completeness values.  We
  choose to discard such bins from our analysis.  The discarded
  luminosity bins are shown in Figure~\ref{fig:mosaic} by open
  circles.

\begin{figure*}
  \begin{center}
    \includegraphics[width=\textwidth,keepaspectratio]{mosaic_small.pdf}
    % mosaic.py 
  \end{center}
  \caption{Luminosity function estimates in redshift bins from $z=0.1$
    to $6.5$.  The symbols show our inferred binned luminosity
    functions from various data sets: sample ID 13 (red), 15 (green),
    1 (blue), 17 (yellow), 8 (brown), 6 (light blue), 18 (pink), 10
    (orange), and 11 (grey).  See Table~\ref{tab:samples} for
    references.  Open circles in corresponding colours indicate
    excluded incomplete bins for the respective data sets.  In each
    redshift bin, the black curve shows the best-fit double power law
    model, which is represented by the median of the posterior
    probability distribution function.  The grey shaded area shows the
    one-sigma (68.26\%) uncertainty.  Each panel shows the number of
    quasars in the respective redshift bin.  The numbers in
    parentheses indicate the total number of reported quasars in the
    respective redshift bin, before excluding incomplete bins.  See
    Sections~\ref{sec:binnedlf} and \ref{sec:bins} for further
    details.}
  \label{fig:mosaic}
\end{figure*}

\begin{figure*}
  \begin{center}
    \includegraphics[width=0.7\textwidth]{evolution_individuals.pdf}
    % summary_fromFile.py 
  \end{center}
  \caption{Evolution of the four double power law parameters from the
    redshift bins shown in Figure~\ref{fig:mosaic}.  Vertical error
    bars show one-sigma (68.26\%) uncertainties, whereas horizontal
    error bars show widths of the redshift bins.  We identify the
    general evolutionary trends of each of these parameters from the
    bins shown by the filled symbols.  The open symbols show bins that
    appear to be offset from these trends, likely due to unknown
    systematic errors.  The open circles at $z\sim 2$--$4$ show the
    BOSS sample, while the bins at $z < 0.6$ contain AGN from the SDSS
    and 2SLAQ data sets.  See Section~\ref{sec:bins} for further
    details.}
  \label{fig:evoln}
\end{figure*}

\begin{table*}
  % bins_tabulate.py; Nqso added by hand
  \caption{Best-fit double power law luminosity function parameters in
    various redshift bins.  The luminosity function parameters
    $\phi_*$, $M_*$, $\alpha$, and $\beta$ are defined in
    Equation~(\ref{eqn:dpl}), with $\beta$ denoting the faint-end
    slope.  Quasars in each bin have redshifts $z_\mathrm{min}\leq z <
    z_\mathrm{max}$.  The number of quasars in each bin is given by
    $N_\mathrm{qso}$.  The bin centre and sample mean redshift are
    given by $z_\mathrm{bin}$ and $\langle z\rangle$, respectively.
    Errors indicate one-sigma (68.26\%) uncertainties.  These values
    are shown in Figure~\ref{fig:evoln}.  The corresponding luminosity
    functions are shown in Figure~\ref{fig:mosaic}.  See
    Section~\ref{sec:bins} for further details.}
  \label{tab:bins}
  \begin{tabular}{ccccrcccc}
    \hline
    $\langle z\rangle$ & $z_\mathrm{bin}$ & $z_\mathrm{min}$ & $z_\mathrm{max}$ & $N_\mathrm{qso}$ & $\log_{10}(\phi_*/$ & $M_*$ & $\alpha$ & $\beta$ \\
    & & & & & cMpc$^{-3}$mag$^{-1}$) & & & \\
    \hline
    $0.31$ & $0.25$ & $0.10$ & $0.40$ & 3520 & $-5.62^{+0.09}_{-0.10}$ & $-21.37^{+0.21}_{-0.22}$ & $-2.94^{+0.09}_{-0.10}$ & $-1.22^{+0.18}_{-0.16}$ \\
    $0.50$ & $0.50$ & $0.40$ & $0.60$ & 4555 & $-6.12^{+0.04}_{-0.04}$ & $-23.00^{+0.08}_{-0.08}$ & $-3.37^{+0.07}_{-0.08}$ & $-1.36^{+0.04}_{-0.04}$ \\
    $0.72$ & $0.70$ & $0.60$ & $0.80$ & 3308 & $-6.42^{+0.09}_{-0.08}$ & $-24.03^{+0.15}_{-0.13}$ & $-3.55^{+0.12}_{-0.13}$ & $-1.75^{+0.05}_{-0.04}$ \\
    $0.91$ & $0.90$ & $0.80$ & $1.00$ & 3922 & $-6.39^{+0.08}_{-0.07}$ & $-24.51^{+0.12}_{-0.11}$ & $-3.69^{+0.10}_{-0.11}$ & $-1.73^{+0.06}_{-0.05}$ \\
    $1.10$ & $1.10$ & $1.00$ & $1.20$ & 6367 & $-6.53^{+0.03}_{-0.03}$ & $-25.14^{+0.04}_{-0.04}$ & $-4.17^{+0.08}_{-0.09}$ & $-1.73^{+0.02}_{-0.02}$ \\
    $1.30$ & $1.30$ & $1.20$ & $1.40$ & 6479 & $-6.43^{+0.04}_{-0.04}$ & $-25.29^{+0.06}_{-0.06}$ & $-3.97^{+0.08}_{-0.08}$ & $-1.74^{+0.03}_{-0.03}$ \\
    $1.50$ & $1.50$ & $1.40$ & $1.60$ & 7539 & $-6.51^{+0.03}_{-0.03}$ & $-25.68^{+0.04}_{-0.04}$ & $-4.28^{+0.08}_{-0.08}$ & $-1.75^{+0.02}_{-0.02}$ \\
    $1.71$ & $1.70$ & $1.60$ & $1.80$ & 6788 & $-6.28^{+0.04}_{-0.04}$ & $-25.55^{+0.05}_{-0.05}$ & $-3.96^{+0.07}_{-0.07}$ & $-1.60^{+0.03}_{-0.03}$ \\
    $1.98$ & $2.00$ & $1.80$ & $2.20$ & 10478 & $-6.72^{+0.03}_{-0.03}$ & $-26.26^{+0.04}_{-0.04}$ & $-4.20^{+0.07}_{-0.07}$ & $-1.87^{+0.02}_{-0.02}$ \\
    $2.25$ & $2.25$ & $2.20$ & $2.30$ & 4102 & $-5.98^{+0.10}_{-0.09}$ & $-25.26^{+0.19}_{-0.18}$ & $-3.21^{+0.15}_{-0.16}$ & $-1.54^{+0.07}_{-0.06}$ \\
    $2.35$ & $2.35$ & $2.30$ & $2.40$ & 4317 & $-6.28^{+0.09}_{-0.09}$ & $-25.63^{+0.17}_{-0.15}$ & $-3.39^{+0.17}_{-0.19}$ & $-1.64^{+0.06}_{-0.05}$ \\
    $2.45$ & $2.45$ & $2.40$ & $2.50$ & 3403 & $-6.40^{+0.08}_{-0.07}$ & $-25.86^{+0.14}_{-0.13}$ & $-3.60^{+0.20}_{-0.21}$ & $-1.60^{+0.05}_{-0.05}$ \\
    $2.65$ & $2.65$ & $2.60$ & $2.70$ & 1883 & $-5.98^{+0.06}_{-0.06}$ & $-25.15^{+0.14}_{-0.13}$ & $-3.13^{+0.12}_{-0.13}$ & $-1.05^{+0.09}_{-0.08}$ \\
    $2.75$ & $2.75$ & $2.70$ & $2.80$ & 1135 & $-6.30^{+0.08}_{-0.07}$ & $-25.96^{+0.15}_{-0.13}$ & $-3.81^{+0.28}_{-0.31}$ & $-1.34^{+0.07}_{-0.06}$ \\
    $2.85$ & $2.85$ & $2.80$ & $2.90$ & 1069 & $-6.46^{+0.12}_{-0.10}$ & $-26.23^{+0.23}_{-0.18}$ & $-3.60^{+0.36}_{-0.42}$ & $-1.46^{+0.08}_{-0.07}$ \\
    $2.95$ & $2.95$ & $2.90$ & $3.00$ & 1104 & $-6.77^{+0.07}_{-0.06}$ & $-26.53^{+0.10}_{-0.09}$ & $-5.05^{+0.60}_{-0.64}$ & $-1.71^{+0.05}_{-0.04}$ \\
    $3.05$ & $3.05$ & $3.00$ & $3.10$ & 1127 & $-6.77^{+0.08}_{-0.07}$ & $-26.48^{+0.12}_{-0.10}$ & $-4.71^{+0.46}_{-0.52}$ & $-1.70^{+0.06}_{-0.05}$ \\
    $3.15$ & $3.15$ & $3.10$ & $3.20$ & 1041 & $-7.25^{+0.17}_{-0.12}$ & $-27.09^{+0.25}_{-0.18}$ & $-4.42^{+0.81}_{-1.23}$ & $-1.96^{+0.08}_{-0.06}$ \\
    $3.25$ & $3.25$ & $3.20$ & $3.30$ & 815 & $-7.33^{+0.13}_{-0.13}$ & $-27.19^{+0.19}_{-0.23}$ & $-4.36^{+0.65}_{-0.76}$ & $-1.93^{+0.06}_{-0.06}$ \\
    $3.34$ & $3.35$ & $3.30$ & $3.40$ & 510 & $-7.54^{+0.19}_{-0.21}$ & $-27.39^{+0.25}_{-0.37}$ & $-4.90^{+1.36}_{-1.35}$ & $-2.08^{+0.09}_{-0.07}$ \\
    $3.44$ & $3.45$ & $3.40$ & $3.50$ & 155 & $-6.77^{+0.19}_{-0.20}$ & $-26.64^{+0.42}_{-0.35}$ & $-3.70^{+0.62}_{-0.80}$ & $-1.24^{+0.27}_{-0.22}$ \\
    $3.88$ & $3.90$ & $3.70$ & $4.10$ & 628 & $-7.92^{+0.12}_{-0.10}$ & $-27.26^{+0.14}_{-0.12}$ & $-4.81^{+0.34}_{-0.39}$ & $-2.07^{+0.10}_{-0.09}$ \\
    $4.35$ & $4.40$ & $4.10$ & $4.70$ & 442 & $-8.32^{+0.29}_{-0.26}$ & $-27.38^{+0.38}_{-0.31}$ & $-4.19^{+0.41}_{-0.47}$ & $-2.20^{+0.16}_{-0.13}$ \\
    $4.92$ & $5.10$ & $4.70$ & $5.50$ & 263 & $-9.02^{+0.29}_{-0.22}$ & $-27.88^{+0.36}_{-0.27}$ & $-4.51^{+0.66}_{-0.83}$ & $-2.31^{+0.11}_{-0.08}$ \\
    $6.00$ & $6.00$ & $5.50$ & $6.50$ & 66 & $-10.60^{+0.71}_{-1.06}$ & $-29.12^{+1.04}_{-1.78}$ & $-4.99^{+0.72}_{-1.22}$ & $-2.40^{+0.10}_{-0.08}$ \\
    \hline
  \end{tabular}
\end{table*}

\subsection{Double power law fits}
\label{sec:bins}

As our next step, we combine magnitude bins by binning the quasars
only in redshifts.  This allows us to estimate the luminosity function
at each redshift.  In each redshift bin, we model the quasar
luminosity function as a double power law given by
\citep{1988MNRAS.235..935B, 1995ApJ...438..623P, 2000MNRAS.317.1014B}
\begin{equation}
  \phi(M) =
  \frac{\phi_*}{10^{0.4(\alpha+1)(M-M_*)}+10^{0.4(\beta+1)(M-M_*)}}.
  \label{eqn:dpl}
\end{equation}
Equation~(\ref{eqn:dpl}) has four free parameters: the amplitude
$\phi_*$, the break magnitude $M_*$, the faint-end slope $\beta$, and
the bright-end slope $\alpha$.  By assuming broad, uniform priors, we
obtain posterior probability distributions for these parameters using
MCMC \citep[e.g.,][]{jaynes}.  The joint posterior probability
distribution of the model parameters is then written as
\begin{multline}
  p(\phi_*, M_*, \alpha, \beta | \{M_i, z_i\}) \propto \\ p(\phi_*, M_*,
  \alpha, \beta)p(\{M_i, z_i\} | \phi_*, M_*, \alpha, \beta),
\end{multline}
where the constant of proportionality is independent of the luminosity
function parameters, and $\{M_i, z_i\}$ denotes the magnitudes and
redshifts of quasars falling in a redshift bin $[z_\mathrm{min},
  z_\mathrm{max})$.  We use a uniform prior distribution $p(\phi_*,
  M_*, \alpha, \beta)$ and assume that the likelihood
\begin{equation}
  \mathcal{L}\equiv p(\{M_i, z_i\} | \phi_*, M_*, \alpha, \beta)
\end{equation}
is given by $\phi(M)$ with suitable normalisation.  The negative
logarithm of the likelihood $S\equiv -2\ln\mathcal{L}$ can then be
written as
\begin{multline}
  S = -2\sum_{i=1}^N\ln\phi(M_i, z_i)\\+2\int_{M_\mathrm{min}}^{M_\mathrm{max}}dM
  \int_{z_\mathrm{min}}^{z_\mathrm{max}}dz\, \phi(M,z) f(M, z)\,\frac{dV}{dz}.
  \label{eqn:S}
\end{multline}
where $N$ is the total number of quasars in the redshift bin and the
luminosity integral in the second term on the right hand side is on
the surveyed range of $M$.  We use the \texttt{emcee} code
\citep{2013PASP..125..306F} for MCMC.

The above likelihood can also be understood as the limit of the
Poisson likelihood in luminosity and redshift bins
\citep{1983ApJ...269...35M, 2001AJ....121...54F}.  We can write the
probability of observing $n_{ij}$ quasars in the $(M_i, z_j)$ bin as
the Poisson distribution
\begin{equation}
  \mathcal{L}=\prod_{i,j}\frac{e^{-\mu_{ij}}\mu_{ij}^{n_{ij}}}{n_{ij}!},
  \label{eqn:lhood}
\end{equation}
where 
\begin{equation}
  \mu_{ij}= \int_{M_i}^{M_{i+1}}dM\int_{z_j}^{z_{j+1}}dz\, \phi(M,z) f(M, z)
  \,\frac{dV}{dz},
\end{equation}
is the average number of quasars expected in the $(M_i, z_j)$ bin
given the luminosity function $\phi(M,z)$.  In the limit of
infinitesimal bins, $n_{ij}=0$ or $1$, and Equation~(\ref{eqn:lhood})
can be simplified to obtain Equation~(\ref{eqn:S}).

Our estimates for the double power law luminosity function are shown
in Figure~\ref{fig:mosaic} for 25 redshift bins.  The posterior median
is shown as the best-fit model at each redshift, while the grey shaded
region shows the one-sigma (68.26\%) uncertainty.  Consistent with
previous studies, the double power law model provides an excellent
description of the luminosity function model over almost the complete
range of redshifts spanned by the data.  It is only in the highest
redshift bin ($z=5.5$--$6.5$) that the data seem to favour a single
power law.  In this bin, the resultant posterior distribution of the
break magnitude $M_*$ is bimodal with favoured values at the faint
($M_*>-18$) and bright end of the data ($M_*<-30$).  While in the
literature the $z\sim 6$ AGN have been assumed to lie on the bright
end of the luminosity function \citep[e.g.,][]{2016ApJ...833..222J}, a
comparison with the luminosity function at lower redshifts ($z<5.5$)
suggests that these AGN should instead be understood to describe the
faint-end of a double power law.  Therefore, in light of the
low-redshift data, we use restricted priors in this redshift bin: we
restrict the bright-end slope $\alpha$ to values less than $-4$, which
is equivalent to forcing $M_*$ to be at the bright end of the data.
Other parameters continue to have wide uniform priors.  This also
illustrates the importance of analysing the high-redshift AGN
population in a wider context of AGN data.

The redshift evolution of the four double power law parameters is
shown in Figure~\ref{fig:evoln}.  These values have been tabulated in
Table~\ref{tab:bins}.  We find interesting evolutionary trends in each
of the four parameters.  The break luminosity $M_*$ evolves by more
than eight magnitudes from redshift $z=0$ to $7$.  The amplitude of
the luminosity function $\phi_*$ evolves moderately from $z=0$ to
$z\sim 3$ and then drops by six orders of magnitude to about
$10^{-12}$ at $z\sim 7$.  The bright end slope $\alpha$ has
significant scatter, but still shows a trend towards larger negative
values, i.e., steeper luminosity function bright ends, at high
redshifts.  Finally, the faint end of the luminosity function also
shows signs of increasing steepness towards high redshifts.  Somewhat
similar to the amplitude $\phi_*$, the faint-end slope $\beta$ also
shows a break at $z\sim 3$.  It stays roughly constant up to this
redshift, and the drops to increasingly negative values of close to
$-2.5$ at high redshifts, indicating a steeper faint end.

Figure~\ref{fig:evoln} reveals further potentially systematic errors.
The behaviour of BOSS quasars, shown by open circles in
Figure~\ref{fig:evoln} at redshifts $z=2$--$4$ is striking.  In this
redshift range covered by BOSS the values of the four double power law
parameters deviate from the general trends.  The faint end of the
luminosity function of the BOSS quasars is much shallower and the
break luminosity is fainter by a magnitude compared to the general
trend expected from a smooth evolution.  The bright end of the
luminosity function of quasars in this sample is also much flatter
than that for other samples.  The conspicuousness of the deviation of
the BOSS sample from the general trends makes it unlikely that the
luminosity function evolution indicated by BOSS in the redshift range
$z=2$--$3$ is physical.  Indeed, in this redshift range BOSS quasar
selection is known to be affected by a colour bias
\citep{2011ApJ...728...23W}, which can make estimating completeness
difficult \citep{2006AJ....131.2766R}.  \gk{Gabor, could you please
  check if I have written this paragraph OK?}

\begin{figure*}
  \begin{center}
    \includegraphics[width=\textwidth,keepaspectratio]{mosaic_small_global.pdf}
    % mosaic.py 
  \end{center}
  \caption{Luminosity function estimates from $z=0.3$ to $2.6$.
    Similar to Figure~\ref{fig:mosaic}, the symbols show our inferred
    binned luminosity functions.  In each redshift bin, the yellow
    curves show the best-fit double power law luminosity function in
    that redshift bin.  The green curves show the posterior
    distribution of the global model.  The global model agrees very
    well with the data in all redshift bins, except those spanned by
    BOSS, as BOSS quasars are excluded from the global
    analysis. \gk{Show all three models here.}}
  \label{fig:mosaic_global}
\end{figure*}

\begin{figure*}
  \begin{center}
    \includegraphics[width=0.7\textwidth]{evolution_global.pdf}
    % summary_fromFile.py 
  \end{center}
  \caption{Luminosity function parameter evolution in the global
    model.  The symbols show the best fit parameters with 68\%
    uncertainties from the double power law fits to the data in
    redshift bins.  These points are identical to those in
    Figure~\ref{fig:evoln}, except that bins with BOSS qsos ($z\sim
    2$--$3$) are not shown.  In each panel, the black solid curve
    shows the best fit global model, and the coloured curves show a
    random sample drawn from the global posterior. \gk{Cannot see
      best-fit curves.  Add open and closed circles to legend.}}
  \label{fig:evoln_global}
\end{figure*}

\begin{figure}
  \begin{center}
    \includegraphics[width=\columnwidth,keepaspectratio]{rhoqso_withGlobal.pdf}
    % rhoqso.py -- draw_withGlobal()
  \end{center}
  \caption{AGN number density evolution in the global model.  Similar
    to Figure~\ref{fig:evoln_global}, symbols show the best fit
    parameters with 68\% uncertainties from the double power law fits
    to the data in redshift bins.  Black solid curve shows the best
    fit global model, and the coloured curves show a random sample
    drawn from the global posterior. \gk{I still think we should show
      the $-18$ case here.  Mention magnitude limits.  Reduce
      transarency.  Extend plot to at least redshift 7.5.  Increase
      legend handle length to match other figure. Should we add French
      curve fits to this?}}
  \label{fig:rhoqso}
\end{figure}

\begin{figure*}
  \begin{center}
    \includegraphics[scale=0.65]{emissivity.pdf}
    % gammapi.py -- draw_emissivity() 
  \end{center}
  \caption{The 912~\AA\ emissivity of AGN assuming 100\% escape
    fraction down to a limiting magnitude.  Brown points show the
    emissivity when the best-fit double power law luminosity functions
    at respective redshifts are integrated down to $M_{1450}=-18$.
    Black points show the integration limit when the limit is assumed
    to be $M_{1450}=-21$.  For comparison, various other published
    models of the 1~Ry emissivity of AGN are also shown.  The blue
    curve in these panels shows the AGN 912~\AA\ emissivity model of
    \citet{2012ApJ...746..125H}, which is very similar to the model of
    \citet{2007ApJ...654..731H}.  Red circles with error bars show the
    determination from the fits of \citet{2015AA...578A..83G}.  The
    maroon curve is the published fit to the 1~Ry emissivity evolution
    in the fits of \citet{2017MNRAS.466.1160M}.  The green curve is
    the model of \citet{2015ApJ...813L...8M}.  The grey and brown
    curves show the best-fit form of Equation XX. \gk{Mention MH15 and
      HM12 Lstar limits.  Mention that Akiyama open and filled circles
      are very close.}}
  \label{fig:e912}
\end{figure*}

\begin{figure*}
  \begin{center}
    \begin{tabular}{cc}
      \includegraphics[width=0.47\textwidth]{emissivity_18.pdf} & 
      \includegraphics[width=0.47\textwidth]{emissivity_21.pdf} \\
      % gammapi.py -- draw_emissivity()
    \end{tabular}
  \end{center}
  \caption{The 912~\AA\ emissivity of AGN assuming 100\% escape
    fraction down to a limiting magnitude.  Brown points show the
    emissivity when the best-fit double power law luminosity functions
    at respective redshifts are integrated down to $M_{1450}=-18$.
    Black points show the integration limit when the limit is assumed
    to be $M_{1450}=-21$.  For comparison, various other published
    models of the 1~Ry emissivity of AGN are also shown.  The blue
    curve in these panels shows the AGN 912~\AA\ emissivity model of
    \citet{2012ApJ...746..125H}, which is very similar to the model of
    \citet{2007ApJ...654..731H}.  Red circles with error bars show the
    determination from the fits of \citet{2015AA...578A..83G}.  The
    maroon curve is the published fit to the 1~Ry emissivity evolution
    in the fits of \citet{2017MNRAS.466.1160M}.  The green curve is
    the model of \citet{2015ApJ...813L...8M}.  The grey and brown
    curves show the best-fit form of Equation XX. \gk{Mention MH15 and
      HM12 Lstar limits.  Mention that Akiyama open and filled circles
      are very close.}}
  \label{fig:e912_2}
\end{figure*}

\subsection{Evolution of the luminosity function}

The smooth evolution of the luminosity function parameters apparant in
Figure~\ref{fig:evoln} suggests that it should be possible to describe
the quasar UV luminosity function evolution out to $z\sim 7$ using far
fewer number of parameters.  Such descriptions have been developed in
the literature for the X-ray \citep[e.g.,][]{2015MNRAS.451.1892A} and
bolometric luminosity functions \citep[e.g.,][]{2007ApJ...654..731H}.
Such ``global'' descriptions of the luminosity function evolution are
useful as they give a continuous description of the luminosity
function.  This allows one to reduce the bias introduced by binning
the data in arbitrary redshift bins.  By potentially allowing for
extrapolations beyond the redshift range spanned by the data, such
models are valuable for understanding of the physics behind the
luminosity function.  Ideally, one would want to use physically
meaningful parameters that govern the formation and evolution of the
AGN population.  Unfortunately, such physical parameterisation is yet
to be developed.  We therefore set up an empirical parameterisation to
describe the evolution of the four parameters of the double power law
model in Equation~(\ref{eqn:dpl}) as 
\begin{align}
  &\phi_*(z) = f_0(\{c_{0,i}\}, z)\nonumber\\
  &M_*(z) = f_1(\{c_{1,i}\}, z)\nonumber\\
  &\alpha(z) = f_2(\{c_{2,i}\}, z)\nonumber\\
  &\beta(z) = f_3(\{c_{3,i}\}, z),
\label{eqn:global}
\end{align}
where the $c_{j,i}$'s are the new model parameters, and the $f_j$'s
are functions that vary smoothly with redshift $z$.  The joint
posterior probability distribution of these parameters can be now
written as
\begin{equation}
  p(\{c_i\} | \{M_i, z_i\}) \propto p(\{c_i\})p(\{M_i, z_i\} | \{c_i\}),
\end{equation}
where the likelihood 
\begin{equation}
  \mathcal{L}\equiv p(\{M_i, z_i\} | \{c_i\}),
\end{equation}
is now given by $\phi(M,z)$ with suitable normalisation.  Note that
$\phi(M,z)$ is given by Equation~(\ref{eqn:dpl}), but now the four
parameters in that equation are redshift-dependent, so that
\begin{multline}
  \phi(M,z) = \phi_*(z) \left[10^{0.4(\alpha(z)+1)(M-M_*(z))}\right. \\ \left.+ 10^{0.4(\beta(z)+1)(M-M_*(z))}\right]^{-1}.
\end{multline}
The negative logarithm of the likelihood $S\equiv -2\ln\mathcal{L}$ is
straightforward generalisation of Equation~(\ref{eqn:S}), given by 
\begin{multline}
  S = -2\sum_{i=1}^N\ln\phi(M_i, z_i)\\+2\int_{M_\mathrm{min}}^{M_\mathrm{max}}dM\int_{z_\mathrm{min}}^{z_\mathrm{max}}dz\, \phi(M,z) f(M, z)\,\frac{dV}{dz},
  \label{eqn:S2}
\end{multline}
where now $N$ is the total number of quasars included in the analysis,
and the integrals in the second term on the right-hand side are over
the complete surveyed range of luminosity and redshift, instead of
redshift and luminosity bins.  We consider models in which the
evolution of the four double power law parameter is modelled
independently as in Equations~(\ref{eqn:global}).  This approach has
been termed as ``flexible double power law'' by
\citet{2015MNRAS.451.1892A}.  We present three such models in this
paper.  These are shown in Figure~\ref{fig:evoln_global}.  The three
models differ in the way they describe the evolution of the faint-end
slope $\beta$ and in the selection of the AGN data.  The models are
defined as follows.

\begin{itemize}

\item In Model 1, the parameters $\phi_*$, $M_*$ and $\alpha$ we
  assume that the functions $f_i$'s in Equations~(\ref{eqn:global})
  are Chebyshev polynomials in $(1+z)$.  Thus the functions $f_i$s are
  written as
  \begin{equation}
    f_i=\sum_{j=0}^{n_i}c_{i,j}T_j(1+z),
    \label{eqn:cbs}
  \end{equation}
  where $c_{i,j}$ are the parameters that appear in
  Equations~(\ref{eqn:global}) and $T_j(1+z)$ are Chebyshev
  polynomials of the first kind.  We try successively higher orders of
  Chebyshev polynomials in order to arrive at a good fit with the
  data.  As we discuss below, we find that $\phi_*$, $M_*$ and
  $\alpha$ prefer quadratic, cubic, and linear evolutions in $(1+z)$.
  The faint-end slope $\beta$ seems to require a break in its
  evolution at $z\sim 3$, as discussed in the previous section.  As a
  result, it is difficult to model the evolution of $\beta$ using the
  Chebyshev expansion of Equation~(\ref{eqn:cbs}).  Instead, we
  consider a double power law model for $\beta$ and write \gk{Is this
    really only four parameters?}
  \begin{equation}
    f_3(1+z)=c_{3,0}+\frac{c_{3,1}}{10^{c_{3,3}\zeta}+10^{c_{3,4}\zeta}},
  \end{equation}
  where
  \begin{equation}
    \zeta = \log_{10}\left(\frac{1+z}{1+c_{3,2}}\right),
  \end{equation}
  thus resulting in a five-parameter model with parameters $c_{3,i}$.
  The parameters $c_{3,3}$ and $c_{3,4}$ thus determine the low and
  high redshift slopes of this evolution, with a break at redshift
  $c_{3,2}$.  This is similar to the model of
  \citet{2007ApJ...654..731H}, who also favoured a broken power law
  model for the evolution of the faint-end slope of the bolometric
  luminosity function of quasars.  Model 1 thus has 14 parameters.
  Excluding those deemed to be dominated by systematic errors, as
  discussed in the previous section, all of the remaining AGN from
  Table~\ref{tab:samples} are included while fitting this model.

\item Model 2 parameterises the luminosity function evolution in the
  same way as Model 1, so that the faint-end slope evolution is
  described by a double power law while the evolution of the other
  parameters $\phi_*$, $M_*$ and $\alpha$ is described by,
  respectively, quadratic, cubic, and linear polynomials in $(1+z)$.
  The total number of parameter is 14.  However, while fitting this
  model, we drop samples 7, 19, and 20 from the analysis.  The
  selection function for these samples are not yet well-characterised.
  Removing them allows us to understand the effect this has on the
  favoured evolution model.

\item In Model 3, we again exclude samples 7, 19, and 20 from the
  analysis.  We also continue to describe the evolution of $\phi_*$,
  $M_*$ and $\alpha$ by quadratic, cubic, and linear polynomials in
  $(1+z)$, respectively.  But in this model, the evolution of the
  faint-end slope $\beta$ is also assumed to be linear in $(1+z)$.
  This model thus has just 11 parameters.
\end{itemize}

Figure~\ref{fig:mosaic_global} shows the three models in comparison
with the binned fits from the previous section.  The shaded regions
show the one-sigma (68.26\%) uncertainty.  The symbols show the
luminosity function binned in luminosity and redshift, and the yellow
shaded regions show the posterior distribution of the double power law
luminosity functions in various redshift bins, as in
Figure~\ref{fig:mosaic}.  Bins containing data with large systematic
error are excluded from Figure~\ref{fig:mosaic_global}.  All three
global models are in excellent agreement with the binned models,
although Model~1 performs better.  

Figure~\ref{fig:evoln_global} shows the parameter evolution in the
global models, by comparing it with the results from the fits in
individual redshift bins shown in Figure~\ref{fig:evoln}.  All three
models capture the steepening of the faint end of the luminosity
function towards higher redshifts.  The best-fit form of Model~1 is
shown by the black curves in Figure~\ref{fig:evoln_global}.  The
accompanying grey shaded area depicts the one-sigma (68.26\%)
uncertainty.  The model is in excellent agreement with the results of
the fits in redshift bins discussed in the previous section.  The
deviation of the BOSS quasars at $z=2$--$4$ from the smooth evolution
is again strikingly apparent.  So is the deviation of the SDSS and
2SLAQ quasars at $z<0.6$.  This is an indirect justification of the
data selection discussion previously in Section~\ref{sec:bins}.
Unfortunately, this model suffers with a remarkably sharp break in the
evolution of the faint-end slope $\beta$ at about $z\sim 3.5$.  As
seen in Figure~\ref{fig:evoln_global}, the data seem to require this
break, although it seems unlikely that such a sharp break at this
redshift would be physical.  Model 1 thus serves to emphasize the
necessity of better quality data at these redshifts.  Models 2 and 3
are shown in Figure~\ref{fig:evoln_global} by the green and orange
curves, respectively.  The corresponding best-fit parameter values are
tabulated in Table~\ref{tab:global}.

The evolution of the comoving number density of quasars is shown in
Figure~\ref{fig:rhoqso} when the luminosity function is integrated
down to different limits.  Similar to Figure~\ref{fig:evoln_global},
symbols show the best-fit values with 68\% uncertainties from the
double power law fits to the data in redshift bins.  Black solid curve
shows the best fit global model, and the coloured curves show a random
sample drawn from the global posterior.  This number density evolution
again highlights the systematic error in data at $z\sim 3$.  The
number density of AGN down to the limit of spectroscopic data
($M_{1450}<-21$) is about $10^{-5}$ cMpc$^{-3}$ at its peak.  This
density rapidly increases at low redshifts and then drops gradually at
high redshifts.  Figure~\ref{fig:rhoqso} also shows the familiar
downsizing feature in which the number density of faint qsos peaks are
lower redshifts than that of the bright qsos.  While the number
density of qsos with $M_*<-27$ peaks at $z\sim 2.5$, the number
density of qsos with $M_*<-24$ peaks at $z\sim 2$.  At the faintest
luminosity at which spectroscopic data exist, $M_*<-21$, the number
density of qsos peaks at $z\sim 1$.  However, the difference between
our three models is dramatically evidencet in Figure~\ref{fig:rhoqso}.
Model~1 prefers a decrease in the number density of faint quasars at
$z\sim 3$ followed by an increase at higher redshift.  This is caused
by the rapid steepening of the faint-end slope in this model at this
redshift.  Figure~\ref{fig:rhoqso} reveals another property of these
models: when extrapolated, the AGN number density diverges in all
three models are high redshifts.  This is result of the steep
faint-end slope at high redshifts combined with the rapid brightening
of the break luminosity at $z\sim 3$.  While no data exist at redshift
$z>7.5$, this divergent behaviour is shared by previous models in the
literature.  Figure~\ref{fig:rhoqso} also shows that although Models 2
and 3 exhibit regular behaviour in the evolution of the AGN number
density at $z\sim 3$, they do not fit the $z\sim 4$--$5$ data as well
as Model 1.  \gk{Discuss comparison of expected number of quasars at
  redshift 7 with UKIDSS.  Also discuss our new fit?}

\begin{table}
  \caption{Best-fit luminosity function evolution models.  These
    parameters are defined in Equations XX.  See Section XX for
    further details and the redshift range of validity of these
    models.  Errors indicate one-sigma (68.26\%) uncertainties.  Model
    2 is our preferred model.}
  \label{tab:global}
  \begin{tabular}{lrrr}
    \hline 
    Parameter & Model 1 & Model 2 & Model 3 \\
    \hline
    $c_{0,0}$ & $-7.559^{+0.131}_{-0.139}$ & $-7.084^{+0.136}_{-0.142}$ & $-6.842^{+0.077}_{-0.076}$ \\     
    $c_{0,1}$ & $1.013^{+0.079}_{-0.072}$ & $0.753^{+0.080}_{-0.073}$ & $0.590^{+0.039}_{-0.041}$ \\        
    $c_{0,2}$ & $-0.113^{+0.005}_{-0.005}$ & $-0.096^{+0.004}_{-0.005}$ & $-0.083^{+0.003}_{-0.003}$ \\
    \\
    $c_{1,0}$ & $-17.006^{+0.230}_{-0.243}$ & $-15.423^{+0.263}_{-0.261}$ & $-15.140^{+0.144}_{-0.136}$ \\  
    $c_{1,1}$ & $-5.548^{+0.156}_{-0.143}$ & $-6.725^{+0.156}_{-0.171}$ & $-6.910^{+0.091}_{-0.093}$ \\     
    $c_{1,2}$ & $0.588^{+0.016}_{-0.019}$ & $0.737^{+0.018}_{-0.015}$ & $0.750^{+0.012}_{-0.013}$ \\        
    $c_{1,3}$ & $-0.023^{+0.001}_{-0.001}$ & $-0.029^{+0.001}_{-0.001}$ & $-0.029^{+0.001}_{-0.001}$ \\
    \\
    $c_{2,0}$ & $-3.246^{+0.121}_{-0.123}$ & $-2.973^{+0.117}_{-0.133}$ & $-2.950^{+0.105}_{-0.097}$ \\     
    $c_{2,1}$ & $-0.250^{+0.048}_{-0.050}$ & $-0.347^{+0.050}_{-0.050}$ & $-0.363^{+0.040}_{-0.043}$ \\
    \\
    $c_{3,0}$ & $-2.350^{+0.051}_{-0.060}$ & $-2.545^{+0.123}_{-0.290}$ & $-1.424^{+0.031}_{-0.031}$ \\     
    $c_{3,1}$ & $0.647^{+0.072}_{-0.060}$ & $1.581^{+0.520}_{-0.264}$ & $-0.111^{+0.010}_{-0.010}$ \\       
    $c_{3,2}$ & $3.857^{+0.092}_{-0.073}$ & $2.102^{+0.450}_{-0.283}$ & --- \\        
    $c_{3,3}$ & $27.534^{+41.364}_{-9.405}$ & $1.965^{+0.461}_{-0.464}$ & --- \\      
    $c_{3,4}$ & $-0.002^{+0.074}_{-0.082}$ & $-0.641^{+0.169}_{-0.154}$ & --- \\      
    \hline 
  \end{tabular}
\end{table}

\section{Contribution of AGN to reionization}

\subsection{Hydrogen-ionizing emissivity}
\label{sec:e912}

We can now calculate the contribution of quasars to the hydrogen
photoionization rate in the intergalactic medium (IGM).  We assume
that all quasars have a universal UV SED, parameterised by
\citet{2015MNRAS.449.4204L} as a power law with a break at 912~{\AA},
\begin{equation}
f_\nu\propto\begin{cases}
               \nu^{-0.61\pm 0.01} & \text{if}~\lambda\geq 912~\text{\AA},\\
               \nu^{-1.70\pm 0.61} & \text{if}~600~\text{\AA}<\lambda<912~\text{\AA}.\\                
               \end{cases}
\label{eqn:sed}
\end{equation}
This fit was derived by \citet{2015MNRAS.449.4204L} from a composite
spectrum of 53 quasars at $z\sim 2.4$.

While four published composite quasar UV SEDs
\citep{2002ApJ...565..773T, 2001AJ....122..549V, 2012ApJ...752..162S,
  2014ApJ...794...75S} agree with Equation~(\ref{eqn:sed}), the
composite of \citet{2004ApJ...615..135S}, which uses more than 100
quasars at $z<0.1$, is somewhat steeper, with a EUV slope of $-0.56$.
The quasars considered by \citet{2004ApJ...615..135S} are fainter
($M_i(z=2)>-26$) than those considered by \citet{2015MNRAS.449.4204L},
which suggest that faint quasars may have a steeper EUV spectrum.  We
therefore assume an SED of the form
\begin{equation}
f_\nu\propto\begin{cases}
               \nu^{-0.61\pm 0.01} & \text{if}~\lambda\geq 912~\text{\AA},\\
               \nu^{-0.56\pm 0.61} & \text{if}~600~\text{\AA}<\lambda<912~\text{\AA}.\\                
               \end{cases}
\label{eqn:sed_faint}
\end{equation}
for AGN with $M_{1450}<-23$.

The rate of emission of photons of frequency $\nu$ by quasars can be
written as
\begin{equation}
\dot n_\nu = \int dM_\nu \phi(M_\nu) \frac{L_\nu(M_\nu)}{h\nu}.
\end{equation}
Here the integral is over a suitable range of magnitudes, the
monochromatic luminosity $L_\nu(M)$ is related to the (absolute AB)
magnitude by \citep{1983ApJ...266..713O}
\begin{equation}
M_\nu = -2.5\log_{10}L_\nu+51.60,
\end{equation}
and $\phi$ is the luminosity function from Equation~(\ref{eqn:dpl}).
The luminosity at 1~Ry is related to that at 1450~\AA\ by
Equation~(\ref{eqn:sed}) or (\ref{eqn:sed_faint})
\begin{equation}
  L_{912}=L_{1450}\left(\frac{\nu_{912}}{\nu_{1450}}\right)^{-0.61}=L_{1450}\left(\frac{912}{1450}\right)^{0.61},
\end{equation}
depending on the AGN luminosity.  The corresponding physical volume
emissivity is 
\begin{equation}
\epsilon_\nu = \dot n_\nu h\nu (1+z)^3.
\label{eqn:epsilon}
\end{equation}
The open and filled black circles with errorbars in
Figure~\ref{fig:e912} show the resultant 1~Ry emissivity when the
luminosity function is integrated down to $M_{1450}<-21$ (open
circles) and when it is integrated down to $M_{1450}<-18$ (filled
circles).  Redshift bins that were removed from analysis due to large
systematic errors are not shown.  Similar to the previous figures, the
black open circles in this figure show the result form individual
redshift bins, while the black curve shown the result from best-fit
global model.  Due to the problems with the global luminosity function
models discussed above, we avoid using them to describe the emissivity
evolution.  Instead, we describe the emissivity evolution by the
5-parameter functional form \citep{2012ApJ...746..125H}
\begin{equation}
  \epsilon_{912}=\epsilon_0(1+z)^a\frac{\exp(-bz)}{\exp(cz)+d}.
  \label{eqn:e912fit}
\end{equation}
We fit this to the points shown in Figure~\ref{fig:e912} assuming a
Gaussian likelihood in each bin.  The corresponding best-fit curves
and the corresponding one-sigma uncertainty is shown in red
($M_{1450}<-18$) and blue ($M_{1450}<-21$).  The best-fit form of
Equation~\ref{eqn:e912fit} is given by 
\begin{multline}
  \epsilon_{912}=(10^{24.49}\mathrm{erg\, s^{-1}\, Hz^{-1}\, cMpc^{-3}})(1+z)^{7.57}\\\times\frac{\exp(-1.78z)}{\exp(1.01z)+29.20}
\end{multline}
for $M_{1450}<-18$, and by 
\begin{multline}
  \epsilon_{912}=(10^{24.11}\mathrm{erg\, s^{-1}\, Hz^{-1}\, cMpc^{-3}})(1+z)^{6.64}\\\times\frac{\exp(-0.65z)}{\exp(1.82z)+20.45}
\end{multline}
for $M_{1450}<-21$.  The 1~Ry emissivity increases rapidly from $z=0$
up to $z=2$ and then stays roughly constant until up to $z=3.5$.  At
higher redshifts, due to the relatively steeper faint end slope of the
luminosity function, the integration limit makes a large difference in
the resultant emissivity.  For the conservative case, in which we
integrate under the luminosity function down to $M_{1450}=-20$, the
emissivity drops rapidly at $z>3.5$For comparison, various other
published determinations of the 1~Ry emissivity of quasars are also
shown in both panels of Figure~\ref{fig:e912}.  We now comment on the
comparison of our results with these measurements.
\citet{2009A&A...507..781S} combined data from the Hamburg/ESO survey
and the SDSS to measure a quasar luminosity function at $z=0$ in their
$B_J$ band to minimize host galaxy contribution.  We convert their
$B_J$ magnitudes (Vega system) to $M_{1450}$ (AB) as
\begin{equation}
  M_{1450,\mathrm{AB}}=M_{B_J, \mathrm{Vega}}+0.59.  
\end{equation}
In this magnitude system, the luminosity function reported by
\citet{2009A&A...507..781S} has a very faint break luminosity of
$M_*=-18.87$.  The faint-end slope is also steep ($\beta=-2$).  We
integrate this luminosity function down to our limiting magnitudes of
$M_{1450}=-18$ and $-21$.  The resultant values are shown by the open
and closed brown circles at $z=0$ in Figure~\ref{fig:e912}.  Our
emissivity values at $z=0$ lie in between the
\citet{2009A&A...507..781S} measurements.  The convergence in the
emissivity is slower in their measurements than in our models because
of the shallower faint-end slope at this redshift in our model
($\beta\sim -1.22$).  We discuss the 1~Ry emissivity and the
corresponding hydrogen photoionization rate at $z=0$ below.

The brown dashed curve in Figure~\ref{fig:e912} shows the AGN 1~Ry
emissivity model of \citet{2012ApJ...746..125H}, which based on the
bolometric luminosity function model of \citet{2007ApJ...654..731H}.
Their assumed emissivity agrees reasonably well with our
$M_{1450}<-21$ case.  The green dotten curve in Figure~\ref{fig:e912}
shows the model presented by \citet{2015ApJ...813L...8M}.  This model
disagrees with ours rather severely at almost all redshifts in the
range considered here.  Note that a direct comparison with
\citet{2012ApJ...746..125H} and \citet{2015ApJ...813L...8M} is
difficult because these models integrate the luminosity function to a
fixed fraction of the break luminosity $L_*$ instead of a fixed
absolute magnitude.  The chosen limite by \citet{2012ApJ...746..125H}
and \citet{2015ApJ...813L...8M} is $0.01L_*$.  Using a limit that
depends on the break luminosity can be problematic as different quasar
surveys used to develop these emissivity models often report
strikingly different break luminosity values.  Even when these samples
are homogeneized appropriately, have such an integration limit is only
justified if the faint-end slope of the luminosity function is
sufficiently shallow.

The grey dashed curve shows the emissivity model of
\citet{2017MNRAS.466.1160M}.  These authors integrate the luminosity
function down to $M_{1450}=-19$.  However, their model strongly
disagrees with our inference, potentially due to biases resulting from
combining binned luminosity function estimates from different sources
inhomogeneously.  The green curves in Figure~\ref{fig:e912} show the
luminosity function derived from a variability-selected sample
observed using the Sloan telescope and the Multiple Mirror Telescope
(MMT) by \citet{2013A&A...551A..29P}.  This sample of 1877 quasars
does not overlap with the variability-selected sample reported by the
BOSS survey \citep{2013ApJ...773...14R}.  \citet{2013A&A...551A..29P}
assum constant luminosity function slopes with a break at $z=2.2$ and
assume a pure luminosity evolution.  This leads to a discontinuity in
the emissivity evolution in their model at $z=2.2$.  We convert their
$g$-band magnitudes to $M_{1450}$ and integrate the resultant
luminosity function down to our two chosen integration limits.  The
result is shown in Figure~\ref{fig:e912} by the solid and dashed
orange curves, which correspond to $M_{1450}<-21$ and $-18$,
respectively.  For $M_{1450}<-21$, our models are in reasonable
agreement at $z<2.2$.  At other redshifts and for $M_{1450}<-21$, the
models disagree by 10--30\%.  Figure~\ref{fig:e912} also shows the
model presented by \citet{2016A&A...587A..41P}.  We consider their
PLE$+$LEDE model, in which the faint-end slope is assumed to stay
constant with redshift.  Although the emissivity evolution in this
model is continuous at $z=2.2$, it is not smooth.  This model agrees
with our $M_{1450}<-18$ result to a better degree relative to the
\citet{2013A&A...551A..29P} model.  We also compare our result with
the PLE$+$LEDE luminosity function model presented by
\citet{2017A&A...608A..64C} by reanalysing the
\citet{2016A&A...587A..41P} sample with an improved $K$-correction
implementation.  \citet{2017ApJ...847L..15O} recently fit a double
power law luminosity function model to a combined binned samples of
\citet{2016ApJ...833..222J}, \citet{2010AJ....139..906W}, and
\citet{2015ApJ...798...28K}, and an X-ray-selected AGN candidate with
photometric redshift $z\sim 6$ reported by
\citet{2018MNRAS.474.2904P}.  We show the resultant 1~Ry emissivity by
orange rectangle in Figure~\ref{fig:e912}.  The lower side of the
rectangle corresponds to the emissivity for the
\citet{2017ApJ...847L..15O} model excluding the
\citet{2018MNRAS.474.2904P} AGN.  When this AGN candidate is included,
the emissivity is higher; this is shown by the upper end of the
rectangle.  The open rectangle corresponds to $M_{1450}<-18$ and the
closed rectangle corresponds to $M_{1450}<-21$.

The red open cirles with error bars in Figure~\ref{fig:e912} show the
determination from the fits of \citet{2015AA...578A..83G}.  These
authors also choose a break luminosity-dependent integration limit of
$0.01L_*$, which corresponds to $M_{1450}\sim -18$ in their model.
The resultant emissivities are factors of 2 to 3 higher than our
determinations for $z=4$--$6$.  This is puzzling at first sight,
because the AGN candidates reported by \citet{2015AA...578A..83G} are
consistent with our luminosity function estimates obtained without
these candidates (see Appendix~\ref{sec:conv}).  Our luminosity
function are also somewhat steeper than those reported by
\citet{2015AA...578A..83G}.  The discrepancy is explained by the
difference in the two luminosity function models at intermediate
magnitudes.  We find that the data prefer a much brighter break
luminosity than that infered by \citet{2015AA...578A..83G}.  This
reduces our emissivities relative to their estimates.  We go into this
issue in further detail in Appendix~\ref{sec:conv}.

\citet{2018AJ....155..131M} recently extended their previous Stripe~82
AGN sample by 1.5 magnitude and fit a double power law luminosity
function model to a combined SDSS $+$ Stripe~82 $+$ CFHTLS sample at
$4.7 < z < 5.1$.  These authors fixed the bright-end slope to
$\alpha=-4$ and obtained a shallower faint-end slope ($\beta=-1.97$)
than our models at this redshift ($\beta\sim -2.3$).  Rescaling their
LF to our cosmology and assumed our SED obtain the emissivities shown
by the green circles in Figure~\ref{fig:e912}.  Again the open and
filled circles corresponds to our two integration limits.  Our
emissivity agrees with the measurement of \citet{2018AJ....155..131M}
for $M_{1450}<-21$ but is higher than their measurement for
$M_{1450}<-18$, which is not surprising due to the difference in the
luminosity function slopes.  The emissivity derived from the
luminosity function of \citet{2018PASJ...70S..34A} is shown in yellow
in Figure~\ref{fig:e912}.  These authors report 1668 AGN candidates at
$3.5<z<4.3$, of which 76 have spectroscopic redshifts.  Their reported
faint-end slope is very flat ($\beta=-1.3\pm 0.05$), inconsistent with
our determination at all redshifts.  Their inferred emissivity is
consistent with our estimate for $M_{1450}\sim -21$.  Their emissivity
does not change significantly with the integration limit thanks to the
shallow slope.  We also compare our emissivities with the
determinations reported by \citet{2012ApJ...755..169M} at $z\sim 3.2$
and $z\sim 4$.  Their emissivity in reasonably agreement with our
model for $M_{1450}<-21$ at $z\sim 3.2$.  But their $M_{1450}<-18$
emissivity disagrees with all other high-redshift estimates, possibly
indicating a systematic error.  Our emissivity estimates are in good
agreement with those of \citet{2018MNRAS.474.2904P}.  Mention
Bongiorno.

\begin{figure*}
  \begin{center}
    % rtg2.draw_g_paper() 
    \includegraphics[scale=0.65]{g.pdf}
  \end{center}
  \caption{AGN contribution to the hydrogen photoionisation rate,
    assuming 100\% escape fraction.  Model luminosity functions are
    integrated down to $M_{1450}=-18$ in the left panel and $-20$ in
    the right panel.  The solid black curve shows the hydrogen
    photoionisate rate in our best-fit global model, and the yellow
    curves show a random sample from the global posterior.  The red
    points with error bars in this figure show the measurements of
    \citet{2013MNRAS.436.1023B}, derived from the Lyman-$\alpha$
    forest, and the orange points show the measurements by
    \citet{2011MNRAS.412.2543C} from quasar proximity zones.  The
    dashed green curve shows the hydrogen photoionisation rate
    evolution in the model of \citet{2012ApJ...746..125H}, while the
    solid green curve shows the contribution of quasars to it. }
  \label{fig:gammapi}
\end{figure*}

\subsection{Hydrogen photoionization rate}
\label{sec:gammahi}

The 1~Ry emissivity can now be used to calculate the ionizing flux and
photoionization rate contributed by quasars.  The frequency dependence
above 1~Ry is given by Equation~(\ref{eqn:sed}) or
(\ref{eqn:sed_faint}), so that for $\nu > \nu_{912}$
\begin{equation}
  \epsilon_\nu = \epsilon_{912}\left(\frac{\nu}{\nu_{912}}\right)^\alpha,
  \label{eqn:epsilon_freq}
\end{equation}
where $\alpha=-1.70$ (Equation~\ref{eqn:sed}) or $-0.56$
(Equation~\ref{eqn:sed_faint}).  The flux in the IGM is
written as a solution of the cosmological radiative transfer equation
as \citep{2012ApJ...746..125H}
\begin{multline}
  j(\nu_0, z_0)=\frac{1}{4\pi}\int_{z_0}^\infty dz\frac{dl}{dz}\frac{(1+z_0)^3}{(1+z)^3}\epsilon(\nu,z)\\
  \times\exp{(-\tau_\mathrm{eff}(\nu_0, z_0, z))},
  \label{eqn:flux}
\end{multline}
where
\begin{equation}
  \nu = \nu_0\left(\frac{1+z}{1+z_0}\right).
\end{equation}
Here the effective optical depth is estimated as
\begin{equation}
  \tau_\mathrm{eff}(\nu_0, z_0, z) = \int_{z_0}^z dz^\prime\int_0^\infty dN_\mathrm{HI} f(N_\mathrm{HI}, z^\prime) (1-e^{-\tau_\nu}),
\end{equation}
where $\tau_\nu=\sigma_\nu N_\mathrm{HI}$ and the column density
distribution $f(N_\mathrm{HI}, z)$ is taken to be a power law \citep{2013MNRAS.436.1023B}
\begin{equation}
  f(N_\mathrm{HI}, z) = \frac{A}{N_\mathrm{LL}}\left(\frac{N_\mathrm{HI}}{N_\mathrm{LL}}\right)^{-\beta_N}\left(\frac{1+z}{4.5}\right)^{-\beta_z},
\end{equation}
with $A=0.93$, $\beta_N=1.33$, $\beta_z=1.92$, and
$N_\mathrm{LL}=10^{17.2}$~cm$^{-2}$.  The hydrogen photoionization
rate is then given by
\begin{equation}
  \Gamma_\mathrm{HI}=4\pi\int_{\nu_{912}}^\infty d\nu \frac{j(\nu,z)}{h\nu} \sigma(\nu),
\end{equation}
where $\sigma$ is the photoionization cross-section.  The result is
shown in Figure~\ref{fig:gammapi}, in which as before the solid black
curve shows the hydrogen photoionisate rate in our best-fit global
model, and the yellow curves show a random sample from the global
posterior.  Also, as before, the right panel of
Figure~\ref{fig:gammapi} shows the result when we only account for
quasars brighter than $M_{1450}=-20$, while the left panel shows the
result when the integration limit is $M_{1450}=-18$.  The red points
with error bars in this figure show the measurements of
\citet{2013MNRAS.436.1023B}, derived from the Lyman-$\alpha$ forest,
and the orange points show the measurements by
\citet{2011MNRAS.412.2543C} from quasar proximity zones.  The dashed
green curve shows the hydrogen photoionisation rate evolution in the
model of \citet{2012ApJ...746..125H}, while the solid green curve
shows the contribution of quasars to it.

The black open circles show the photoionisation rate values derived
from our fits in the individual redshift bins.  However, note that
unlike in the global model it is not possible to do the integral in
Equation~(\ref{eqn:flux}) for these individual fits.  We can only
derive the photoionisatoin rate by assuming what is known as the local
source approximation, that is by assuming that the photons emitted at
redshift $z$ are absorbed at the same redshift without propagation to
large distances.

In the local source approximation, Equation~(\ref{eqn:flux}) becomes
\begin{equation}
  j(\nu_0, z_0) = \frac{1}{4\pi}\lambda_\mathrm{mfp}(\nu_0, z_0)\epsilon(\nu_0, z_0),
\end{equation}
where $\lambda_\mathrm{mfp}$ is the mean free path.  The hydrogen
photoionization rate is given by
\begin{equation}
  \Gamma_\mathrm{HI}=4\pi\int_{\nu_{912}}^\infty d\nu \frac{j_\nu}{h\nu} \sigma(\nu),
  \label{eqn:gammapi}
\end{equation}
where $\sigma$ is the ionization cross-section
\begin{equation}
  \sigma(\nu) = \sigma_0\left(\frac{\nu}{\nu_{912}}\right)^{-3},
  \label{eqn:sigma}
\end{equation}
with $\sigma_0=6.3\times 10^{-18}$~cm$^2$ \citep{2006agna.book.....O}.
In the redshift range $z=2.3$--$5.5$, the mean free path is measured
to be \citep{2014MNRAS.445.1745W}
\begin{equation}
  \lambda_\mathrm{mfp}(\nu, z)= \lambda_0\left(\frac{1+z}{5}\right)^{-5.4}\left(\frac{\nu}{\nu_{912}}\right)^{1.5},
  \label{eqn:mfp}
\end{equation}
where the frequency dependence comes from the assumed column density
distribution $f(N_\mathrm{HI})\propto N_\mathrm{HI}^{-1.5}$.  We
extrapolate the redshift dependence in Equation~(\ref{eqn:mfp}) at
$z<2.3$ and $z>5.5$.  Combining Equations~(\ref{eqn:epsilon_freq}),
(\ref{eqn:sigma}), and (\ref{eqn:mfp}), Equation~(\ref{eqn:gammapi})
gives
\begin{multline}
  \Gamma_\mathrm{HI}=4.6\times 10^{-13} \mathrm{s}^{-1} \left(\frac{\epsilon_{912}}{10^{24}\mathrm{erg\, s^{-1}\, Hz^{-1}\, cMpc^{-3}}}\right)\\
  \times\left(\frac{1}{1.5-\alpha}\right)\left(\frac{1+z}{5}\right)^{-2.4}.
\end{multline}
Note that this equation refers to the \emph{comoving} emissivity.  The
local source approximation works well for $z>3$, as can be seen in
Figure~\ref{fig:gammapi}, where the points from the individual fits
are in much better agreement with the curves from the global model.
At lower redshift, the photoionisation rate in the individual models
is much higher than that in the global model, because the individual
models overestimate the flux at these redshifts. 

%% \begin{figure*}
%%   \begin{center}
%%     \includegraphics[width=\textwidth,keepaspectratio]{evolution.pdf}
%%   \end{center}
%%   \caption{Parameter evolution from individual fits.  Coloured points
%%     show results when Giallongo quasars are not included.  Black
%%     points show results when Giallongo quasars are included.}
%% \end{figure*}

%% \begin{figure}
%%   \begin{center}
%%     \includegraphics[width=\columnwidth,keepaspectratio]{rhoqso_withGlobal_dense.pdf}
%%   \end{center}
%%   \caption{Alternative presentation of Figure~\ref{fig:rhoqso}.}
%% \end{figure}

%% \begin{figure}
%%   \begin{center}
%%     \includegraphics[width=\columnwidth,keepaspectratio]{rhoqso_diff.pdf}
%%   \end{center}
%%   \caption{Alternative presentation of Figure~\ref{fig:rhoqso}.}
%% \end{figure}

%% \begin{figure}
%%   \begin{center}
%%     \includegraphics[width=\columnwidth,keepaspectratio]{rhoqso_diff_ind.pdf}
%%   \end{center}
%%   \caption{Alternative presentation of Figure~\ref{fig:rhoqso}.}
%% \end{figure}

\subsection{Photon underproduction at $z=0$?}

Our derivation of the LyC emissivity of AGN allows us to see if the
corresponding hydrogen photoionization rate is consistent with the
measured CDDF of the low-redshift ($z<0.5$) Ly$\alpha$ forest
\citep{2016ApJ...817..111D}.  \citet{2014ApJ...789L..32K} argued that
in order to match the CDDF observed by \citet{2016ApJ...817..111D} at
\gk{Define acronym CDDF.}  these redshifts, hydrodynamical
cosmological simulations require a hydrogen photoionization rate that
is a factor of five larger than that in the UV background (UVB) model
of \citet{2012ApJ...746..125H}.  \gk{This factor looks more than five
  in our plot!}  Several recent studies in the literature have
addressed this ``photon underproduction crisis''.  On the one hand,
these studies emphasised the uncertainty in the
\citet{2012ApJ...746..125H} UVB model at these redshifts due to the
lack of certainty in the UV photon emissivities of galaxies and AGN
\citep{2015MNRAS.451L..30K, 2015ApJ...811....3S}.  On the other hand,
they noted the uncertainty in the results of the cosmological
simulations at these redshifts, due to effects such as AGN feedback
and limited numerical resolution \citep{2015ApJ...811....3S,
  2017MNRAS.467L..86V, 2017MNRAS.471.1056N, 2017ApJ...837..106O,
  2017MNRAS.466..838G, 2017MNRAS.467.3172G, 2017ApJ...835..175G}.  A
general conclusion of these studies was that the discrepancy between
the photoionization rate required by the observed CDDF and predicted
by the UVB model of \citet{2012ApJ...746..125H} is likely to be
smaller than that found by \citet{2014ApJ...789L..32K}.  The
discrepancy is about a factor of 2 instead of 5.

\begin{figure}
  \begin{center}
    % rtg2.draw_g_puc()
    \includegraphics[width=\columnwidth,keepaspectratio]{g_puc.pdf}
  \end{center}
  \caption{Evolution of the hydrogen photoionisation rate at low
    redshifts (cf.\ Figure~\ref{fig:gammapi}).\label{fig:puc}}
\end{figure}

The black curve in Figure~\ref{fig:puc} shows the evolution of the
hydrogen photoionization rate due to AGN in our best-fit global
luminosity function model.  (Yellow shaded region around the black
curve shows the small 68\% uncertainty.)  We find that the
photoionisation rate in our model due to AGN alone is higher than that
in the \citet{2012ApJ...746..125H} UVB model by a factor of 2.  This
higher photoionisation rate is consistent with that derived by
\citet{2015MNRAS.451L..30K} using the quasar luminosity function
measurements of \citet{2009MNRAS.392...19C} and
\citet{2013A&A...551A..29P}.  It is also consistent with the
photoionization rate found by \citet{2017MNRAS.467.3172G}, who found
that this photoionisation rate is sufficient to explain the CDDF
measured by \citet{2016ApJ...817..111D}.

\gk{Add discussion about Schulze point from Gabor's 8 March email
  here.}

\subsection{Helium reionization}

%% Q notation is different on figure and in text.
%% MH15 curve seems wrong; it should reionize at z ~ 4 or so.
\begin{figure}
  \begin{center}
    \includegraphics[width=\columnwidth,keepaspectratio]{q.pdf}
  \end{center}
  \caption{Evolution of the \HeIII\ ionization fraction in our model,
    when the AGN luminosity function is integrated down to
    $M_{1450}=-21$ (blue curve) and $M_{1450}=-18$ (red curve).  The
    shaded regions show the one-sigma (68.26\%) uncertainty.  Other
    curves show models of \citet[solid grey]{2012ApJ...746..125H},
    \citet[dashed grey]{2015ApJ...813L...8M}, \citet[light blue and
      shaded region]{2016ApJ...828...90L},
    \citet[blue]{2018arXiv180104931P},
    \citet[brown]{2009ApJ...694..842M}, and
    \citet[green]{2014MNRAS.445.4186C}.  Note that we derive
    $Q_\nHeIII$ using Equation~(\ref{eqn:qHedot}), which ignores the
    presence of \HeII\ Lyman-limit systems and becomes inaccurate
    towards the end of \HeII\ reionization. \label{fig:qhe}}
\end{figure}

We now consider the implications of our AGN luminosity function models
for \HeII\ reionization.  We use the emissivity model developed in
Section~\ref{sec:e912} for this purpose.  The evolution of the
volume-averaged fraction $Q_\nHeIII$ of \HeIII\ is given by
\citep{2012ApJ...746..125H}
\begin{equation}
{dQ_\nHeIII\over dt}={\dot n_{\rm ion,4}\over \langle n_\nHe \rangle} -{Q_\nHeIII\over
\langle t_{\rm rec,He}\rangle},
\label{eqn:qHedot}
\end{equation}
where $\dot n_{\rm ion,4}$ is the number density of photons with
energy 4~Ry and above produced per unit time, $n_\nHe$ is the Helium
number density, and $\langle t_{\rm rec,He}\rangle$ is an average
recombination time scale for \HeIII.  We obtain the evolution of $\dot
n_{\rm ion,4}$ for AGN with $M_\mathrm{1450}<-21$ and
$M_\mathrm{1450}<-18$ by integrating the contribution of double power
law luminosity function obtained in distinct redshifts bins and
fitting Equation~(\ref{eqn:e912fit}) to the resultant \HeII-ionizing
emissivities.  We extrapolate the resultant fit to $z>7$, although the
contribution from this extrapolated emissivity to \HeII-reionization
is small.  As before, we assume the SEDs given by
Equations~(\ref{eqn:sed}) and (\ref{eqn:sed_faint}) for bright and
faint AGN, respectively.  The recombination time scale in
Equation~(\ref{eqn:qHedot}) is given by
\begin{equation}
\langle t_{\rm rec}\rangle=[(1+2\chi) \langle n_\nH\rangle \alpha_B\,C]^{-1},
\label{eq:trec}
\end{equation}
where $n_\nH$ is the average physical number density of hydrogen,
$\alpha_B$ is the case-B \HeIII\ recombination rate, and $\chi=0.083$
is the cosmic number fraction of Helium.  We assume that the helium
clumping factor $C$ is given by \citep{2015ApJ...813L...8M}
\begin{equation}
  C = 2.9\left(\frac{1+z}{6}\right)^{-1.1}.
\end{equation}
We use the fitting function provided by \citet{1997MNRAS.292...27H}
for the recombination coefficient.

Note that Equation~(\ref{eqn:qHedot}) neglects the \HeII\ Lyman-limit
systems that play an increasingly important role towards the end of
Helium reionization \citep{2009MNRAS.395..736B, 2017ApJ...851...50M}.
In the absence of these self-shielded systems of high-density \HeII,
$Q_\nHeIII$ can continue to increase beyond unity.  We set it to one
when this happens.  This can be corrected by accounting for the
\HeII\ column density distribution and filtering the \HeII-ionizing
radiation background through it.  Unfortunately, the \HeII\ column
density distribution is itself uncertain.  We therefore consider our
treatment of $Q_\nHeIII$ accurate enough for the purpose of this work.

The resultant reionization histories of \HeII\ are shown in
Figure~\ref{fig:qhe} for $M_\mathrm{1450}<-18$ and
$M_\mathrm{1450}<-21$ in red and blue, respectively.  We find that
Helium reionization is complete, respectively, at $z\sim 3.2$ and
$3.1$ in these models, with a one-sigma uncertainty of about $\Delta
z\sim 0.2$.  Observational constraints on the \HeII\ reionization
history are derived from measurements of the effective \HeII\ \lya
optical depth \citep{2001Sci...293.1112K, 2004ApJ...600..570S,
  2006A&A...455...91F, 2010ApJ...722.1312S, 2011ApJ...733L..24W,
  2016ApJ...825..144W}.  While we do not compute the effective
\HeII\ \lya optical depth in our model, the reionization histories
shown in Figure~\ref{fig:qhe} may likely be consistent with the
optical depth measurements of \citet{2016ApJ...825..144W}, who find
that the effective optical depth rises at $z>2.7$, but low optical
depth sightlines continue to be seen beyond this redshift.  The
incidence of these low opacity sightlines implies that the substantial
volume of \HeII\ was ionized already at $z=3.4$.  \HeII\ reionization
begins at $z\gtrsim 5$ in our models.  When we consider AGN with
$M_\mathrm{1450}<-18$, the ionization fraction $Q_\nHeIII$ has a
one-sigma spread of 0.8--1 at $z=3.4$.

In Figure~\ref{fig:qhe}, we also show the reionization histories from
other models of \HeII\ reionization in the literature.  Our model
curves are closest in agreement to the simulations of
\citet{2009ApJ...694..842M} and the analytical model of
\citet{2018arXiv180104931P}.  \citet{2018arXiv180104931P} model the
\HeII-ionizing emissivity of AGN by fitting the function form in
Equation~(\ref{eqn:e912fit}) to the emissivities reported by
\citet{2009A&A...507..781S}, \citet{2007A&A...472..443B}, and
\citet{2012ApJ...755..169M}.  The resultant emissivity in the model of
\citet{2018arXiv180104931P} is similar in shape but higher in
amplitude than the emissivity in the \citet{2012ApJ...746..125H}
model.  The \HeII-ionizing emissivity of \citet{2018arXiv180104931P}
is closer to our inference for $M_\mathrm{1450}<-21$ than the model of
\citet{2012ApJ...746..125H}.  (However, note that both these models
suffer from inhomogeneous luminosity function integration limits, as
the integration limit is set to a fixed fraction of the break
luminosity (cf.\ discussion in Section~\ref{sec:e912}.)  A spectral
index of $\alpha_\mathrm{UV}=-1.7$ is assumed.
\citet{2018arXiv180104931P} compute the evolution of $Q_\nHeIII$ by
evolving the \HeIII\ ionization fraction in a gas cloud with the mean
cosmic baryonic density exposed to the average \HeII-ionizing flux.
The ionizing flux is obtained by filtering the ionizing emissivity
through an inhomogeneous IGM in which the column density distribution
of \HeIII\ is derived from that of \HI\ in a manner similar to
\citet{2012ApJ...746..125H}.  As a result, this model incorporates the
effect of \HeII\ Lyman-limit systems, and the $Q_\nHeIII$ evolution in
this model deviates from our predictions towards the end of
reionization, as seen in Figure~\ref{fig:qhe}.  Comparing the
emissivity model in the cosmological radiative transfer simulations of
\citet{2009ApJ...694..842M} with our models is less straightforward.
Quasars are modelled in the simulations of \citet{2009ApJ...694..842M}
by relating quasar luminosities and lifetimes to halo masses following
the galaxy-merger-induced quasar activity model of
\citet{2005ApJ...630..705H}.  The resultant bolometric quasar
luminosity functions are somewhat shallower than those of
\citet{2007ApJ...654..731H}, but with a higher amplitude.  A spectral
index of $\alpha_\mathrm{UV}=-1.6$ is assumed.  Figure~\ref{fig:qhe}
shows that the resultant reionization history in these simulations
agrees with our model for $M_\mathrm{1450}<-21$, except towards the
end of reionization, where, as before, \HeII\ Lyman-limit systems
become important.

Helium reionization is delayed in the model of
\citet{2012ApJ...746..125H} due to their reduced emissivity at $z\sim
3$.  In the \citet{2015ApJ...813L...8M} model, a significant
\HeII\ fraction is ionized already at $z=6$ due to the very high
emissivity assumed in this model.  In the radiation-hydrodynamic
simulations by \citet{2016ApJ...828...90L}, the AGN population is
calibrated to the SDSS and COSMOS data at $z>3.5$
\citep{2012ApJ...755..169M, 2013ApJ...768..105M}, combined with a
model for quasar light curves.  The fiducial \HeII\ reionization
history obtained by these authors is shown in Figure~\ref{fig:qhe}
along with an estimate of the uncertainty.  Reionization ends
considerably later in this model ($z\sim 2.6$).  Our redshift of
reionization agrees more closely with that in the simulations
presented by \citet{2014MNRAS.445.4186C}.

Helium reionization has also been studied by
\citet{2018MNRAS.473.1416M}, who constrained the evolution of the
emissivity from AGN using the effect optical depth measurement of
\citet{2016ApJ...825..144W}, combined with an analytical model of the
IGM.  While a detailed comparison with their work is difficult because
of reasons, \citet{2018MNRAS.473.1416M} find that the AGN emissivity
has to drop considerably rapidly at $z>4$. \citet{2018arXiv180109693K}
have recently presented a model of Helium reionization.  Reionization
ends at redshift 2.8 in their model.  The AGN luminosity function used
is by a combination of deteminations from various data sets
\citep{2015MNRAS.451L..30K}.  The integration limits are inconsistent
however as they are $0.01L_*$.  Resulting emissivity is comparable to
our model at $z\sim 2$ but drops more rapidly towards higher
redshifts.  For a steep EUV spectral index of $\alpha=-1.8$ their
helium reionization redshift is $z=2.8$.  \citet{2016arXiv160706467D}
considered the effect of AGN-dominated reionization on the Ly$\alpha$
opacity at $z>5$, He~\textsc{ii} Ly$\alpha$ opacity at $z\sim
3.1$--$3.3$, and the thermal history of the IGM.  In agreement with
\citet{2015MNRAS.453.2943C}, these authors found that AGN did provide
a plausible explanation for the large fluctuations in the Ly$\alpha$
opacity at $z>5$.  However, they found that reionization of
He~\textsc{ii} occurs much earlier in these AGN-dominated models (see
also \citealt{2018MNRAS.473.1416M}).

\section{Conclusions}

We have presented an analysis of the evolution of the UV luminosity
function of AGN from redshift $z=0$ to $7.5$ using a combined
homogenised sample of close to 90,000 UV-selected AGN.  We only
consider quasars with spectroscopic redshifts and reliable
completeness estimates.  These data span magnitudes down to
$M_{1450}=-22$ at high redshifts ($z\gtrsim 3.5$).

\begin{itemize}
\item The AGN luminosity function prefers a double power law
  description at all redshift, except possibly at very high redshifts
  ($z\sim 6$) where only the faint-end of the luminosity function is
  observed.

\item The break magnitude $M_*$ of the double power law luminosity
  function shows a steep brightening towards the high redshift.  The
  break evolves from $M_*<-25$ at $z\sim 2$ to $M_*\sim -29$ at $z\sim
  6$.  Correspondingly, the amplitude $\phi_*$ of the luminosity
  function drops rapidly towards high redshifts from $\phi_*\sim
  10^{-6}$~mag$^{-1}$cMpc$^{-3}$ at $z\sim 2$ to $\phi_*<
  10^{-10}$~mag$^{-1}$cMpc$^{-3}$ at $z\sim 6$.  The break magnitude
  at $z=6$ is very bright, about $-30$, which is why the constraints
  on the bright end slope at these redshifts are weak.

\item The faint end slope $\beta$ of the luminosity function shows
  increased steepening at high redshifts, consistent with the findings
  of \citep{2015AA...578A..83G}.  The faint-end slope is $\beta\sim
  -1.8$ at $z\sim 2$ and steepens to $\beta\sim-2.5$ at $z\sim 6$.
  The bright-end slope $\alpha$ also shows moderate steepening towards
  high redshifts.  However, the constraints on $\alpha$ are weak,
  particularly at the high redshift because of the bright break
  luminosity.

\item A fourteen-parameter model accurately captures the redshift
  evolution of the luminosity function.  In this model, the evolution
  of the break magnitude $M_*$ is described by a cubic polynomial in
  $1+z$, while the evolution of the luminosity function amplitude
  follows a quadratic polynomial in $1+z$.  The evolution of the
  bright-end slope $\alpha$ evolves linearly in $1+z$.  The evolution
  of the faint-end slope $\beta$ is more complicated.  The faint-end
  slope stays roughly constant at lower redshifts $z<3$, steepens
  rapidly at $z\sim 3$--$4$, and then remains constant at higher
  redshifts.

\item An important finding of this work is that there are severe
  systematic errors in observed luminosity function.  We find that the
  most significant systematic bias is at redshift $z\sim 3$ in the
  BOSS sample of quasars.  There is also some evidence of a systematic
  bias in the lowest redshifts $z\lesssim 0.5$.  At $z<2.2$ there is
  some indication of a systematic error in completeness estimates at
  the faintest magnitudes $M_{1450}\sim -19$.  

\item In spite of the rapid steepening of the luminosity function at
  high redshifts, our derived LyC emissivity from AGN is quite low.
  This is because of the bright break luminosities, which results in a
  reduced contribution of intermediate brightness quasars to the Lyc
  emissivity as compared to some other determinations in the
  literature.  The LyC emissivity increases with time and peaks at
  $z\sim 2$.  At lower redshifts, the emissivity drops rapidly with
  time.  When our luminoisty function model is extrapolated and
  integrated down to $M_{1450}\sim -20$, the implied LyC emissivity is
  comparable to that in the model of \citet{2007ApJ...654..731H} and
  \citet{2012ApJ...746..125H}.  This indicates that hydrogen
  reionization is unlikely to have caused by AGN alone.

\item We derive photoionization rate by considering the HI column
  density distribution used by \citep{2012ApJ...746..125H}.  We find
  that the HI photoionization rate is significantly lower than that
  required by the Lyman-$\alpha$ forest data at $z>2$.  Our hydrogen
  photoionization rate estimates are also considerably lower (by a
  factor of at least two) than that of \citet{2015AA...578A..83G}.  At
  the lowest redshifts, we find that the hydrogen photoionization rate
  is about a factor of two higher than the estimate of
  \citet{2007ApJ...654..731H} and \citet{2012ApJ...746..125H}.  When
  compared with the recent analysis of \citet{2017MNRAS.467.3172G},
  this photoionisation rate appears to allevaite the photon
  underproduction crisis.  Cite Kollmeier here.  
  
\item Helium reionization in our model occurs at redshift $z\sim
  3$ or $z=3.5$ if the luminosity function is integrated down to
  $M_{1450}=-20$ and $-18$, respectively.
  
\item Our analysis allows us to analyse the X-ray selected quasars of
  \citet{2015AA...578A..83G} in the larger context of UV-selected
  quasars.  We find that the the faint-end luminosity function
  measurements of \citet{2015AA...578A..83G} at $z=4$--$6$ are
  consitent with the measurements of the luminosity function at
  brighter luminosities.  Stil, our inferred value of the hydrogen
  ionizing luminosity density from quasars at these redshifts is much
  lower that the value derived by \citet{2015AA...578A..83G}.  This
  difference arises due to the difference in the data used in the two
  analyses at intermediate to bright luminosities.  
\end{itemize}

Some next steps at $z>4$: 1. include extended objects at faint
magntiudes that could be impacted by host galaxy light, 2. take
follow-up spectroscopy to confirm a fration of these as high-z broad
line AGN, 3. carefully assess host galaxy contamination.  See Gabor's
email of 14 March 13:18.

\section*{Acknowledgements}

We thank Eilat Glikman, Linhua Jiang, Nobunari Kashi\-kawa, Ian
McGreer, Nick Ross, Chris Willott, and Jinyi Yang for sharing data and
especially for sharing their quasar selection functions.  It is a
pleasure to acknowledge useful discussions with James Aird, Eduardo
Ba\~nados, Manda Banerji, Tirthankar Roy Choudhury, George Efstathiou,
Xiaohui Fan, Andrea Ferrara, Prakash Gaikwad, Martin Haehnelt, Paul
Hewett, David Hogg, Vikram Khaire, Sergey Koposov, Donald Lynden-Bell,
Roberto Maiolino, Richard McMahon, Daniel Mortlock, Ewald Puchwein,
Gordon Richards, Alberto Rorai, Bram Venemans and Stephen Warren.  GK
acknowledges support from ERC Advanced Grant 320596 `The Emergence of
Structure During the Epoch of Reionization'.
%% Joe, do you need to add anything to acknowledgements? 

\appendix

\section{Comparison with G15}
\label{sec:conv}
%% Define 'G15' somewhere above.
In the double power law luminosity function models presented in
distinct redshift bins in Section~\ref{sec:bins}, we did not include
the 19 low-luminosity ($M_{1450}>-22.6$) AGN between redshifts $z=4.1$
and $6.3$ reported by G15.  While this was done in order to restrict
our sample to quasars with spectroscopic redshift determinations, it
is instructive to consider how our results are affected if the G15 AGN
are added to the analysis.  In their work, G15 found a shallower
faint-end slope ($\beta\sim -1.5$ to $-1.8$) for the luminosity
function at $4.1 < z < 6.3$, relative to our result from other AGN
samples at these redshifts ($\beta\sim -2.0$ to $-2.5$).  Still, G15
derived a higher 912\,\AA\ emissivity than our estimates
(cf.\ Figure~\ref{fig:e912}), so that in their analysis AGN can
produce all ionizing photons necessary to keep hydrogen ionized and
explain the \lya data.  The black points in
Figure~\ref{fig:params_giallongo} show the parameters of the double
power law luminosity function from our analysis of
Section~\ref{sec:bins}.  The red open circles show the parameter
values obtained when the G15 sample is added to the analysis.  We find
that the two results are highly consistent, showing that the G15
sample is consistent with our double power law fit obtained from other
AGN samples are comparable redshifts.  This is surprising as the
integrated 912\,\AA\ emissivities in our model are smaller than those
derived by G15.  Figure~\ref{fig:lf_giallongo} provides an
explanation.  As seen in this figure, the double power law fits
favoured by G15 (red dashed curves) are quite different from our fits
(black curves).  The characteristic luminosity $M_*$ obtained by G15
is much fainter ($\sim -23$ at $z = 5$) than that resulting out of our
analysis ($\sim -29$ at $z = 5$).  Thus the 912\,\AA\ emissivities are
enhanced in G15 because of the increase contribution from
intermediate-luminosity AGN in their model.
Figure~\ref{fig:lf_giallongo} suggests that this is possibly because
of the inclusion of SDSS Stripe 82 data from
\citet{2013ApJ...768..105M} and the AGN samples of
\citet{2010AJ....139..906W} and \citet{2015ApJ...798...28K} in our
analysis.  Additionally, the homogenisation of data in our work may
also cause part of the difference.

\begin{figure*}
  \begin{center}
    \includegraphics[width=0.7\textwidth]{evolution_g.pdf}
  \end{center}
  \caption{Effect of the 19 AGN reported by \citet{2015AA...578A..83G}
    on the best-fit double power law luminosity function parameters in
    redshift bins from $z=0$ to $7$.  Black points show parameter
    values from Figure~\ref{fig:evoln}.  Red open circles show the
    parameter values obtained when the sample of G15 is added to the
    analysis.  In both cases, vertical error bars show one-sigma
    (68.26\%) uncertainties, and horizontal error bars show widths of
    the redshift bins. \label{fig:params_giallongo}}
\end{figure*}

\begin{figure*}
  \begin{center}
    \includegraphics[width=\textwidth]{giallongo_compare.pdf}
  \end{center}
  \caption{Luminosity functions in three redshift bins at $z>4.1$.
    Black curves in each panel show the best-fit double power law,
    with the corresponding one-sigma (68.26\%) uncertainty shown by
    the grey shaded area.  There are 451, 270, and 69 AGN in each
    redshift bin from left to right, respectively.  These numbers are
    higher than those in Figure~\ref{fig:mosaic} because they include,
    respectively, 9, 7, and 3 AGN from G15.  The magnitude bins
    containing these AGN are shown in purple.  The red dashed curves
    show the double power law fits reported by G15 at $z=4.25, 4.75,$
    and $5.75$. \label{fig:lf_giallongo}}
\end{figure*}

\section{Tables of emissivities and photoionisation rates}
\label{sec:tables}

Table~\ref{tab:emissivity_bins} shows the 912\,\AA\ and
1450\,\AA\ comoving emissivities obtained in various redshift bins
with the one-sigma (68.26\%) uncertainties.  Redshift bins severely
affected by systematic errors are also shown.  The result of the model
presented in Equation~(\ref{eqn:e912fit}), which describes the
emissivity evolution using a smooth function, is tabulated in
Table~\ref{tab:gamma2}, along with the hydrogen photoionization rate
computed in Section~\ref{sec:gammahi}.  In both tables, we show
results for our two integration limits of $M_{1450}<-18$ and
$M_{1450}<-21$.  These tables describe the curves shown in
Figures~\ref{fig:e912} and \ref{fig:gammapi}.  Note that the
photoionisation rate calculation assumes an \HI\ column density
distribution given by \citet{2012ApJ...746..125H} and extrapolates
this to high redshifts.

\begin{table*}
  % gammapi.py and tabulate_emissivities.py. 
  \caption{The 912\,\AA\ and 1450\,\AA\ comoving emissivities
    corresponding to the double power law luminosity function models
    in redshift bins presented in Table~\ref{tab:bins} for two
    integration limits.  These emissivities are shown in
    Figure~\ref{fig:e912_2}.  The units are
    $10^{24}$\ erg\ s$^{-1}$\ Hz$^{-1}$\ cMpc$^{-3}$.  Uncertainties
    are one-sigma (68.26\%).}
  \label{tab:emissivity_bins}
  \begin{tabular}{cccc....}
    \hline
    $\langle z\rangle$ &
    $z_\mathrm{bin}$ &
    $z_\mathrm{min}$ &
    $z_\mathrm{max}$ &
    \multicolumn{1}{c}{$\epsilon_{912}$} &
    \multicolumn{1}{c}{$\epsilon_{1450}$} &
    \multicolumn{1}{c}{$\epsilon_{912}$} &
    \multicolumn{1}{c}{$\epsilon_{1450}$} \\
    &
    &
    &
    &
    \multicolumn{1}{c}{$(M_{1450}<-18)$} &
    \multicolumn{1}{c}{$(M_{1450}<-18)$} &
    \multicolumn{1}{c}{$(M_{1450}<-21)$} &
    \multicolumn{1}{c}{$(M_{1450}<-21)$} \\
    \hline
    0.31 & 0.25 & 0.10 & 0.40 & 0.53^{+0.02}_{-0.02} & 0.71^{+0.03}_{-0.03} & 0.30^{+0.01}_{-0.01} & 0.40^{+0.01}_{-0.01} \\
    0.50 & 0.50 & 0.40 & 0.60 & 0.75^{+0.01}_{-0.01} & 1.00^{+0.02}_{-0.02} & 0.58^{+0.01}_{-0.01} & 0.78^{+0.01}_{-0.01} \\
    0.72 & 0.70 & 0.60 & 0.80 & 1.73^{+0.06}_{-0.05} & 2.31^{+0.07}_{-0.06} & 1.19^{+0.02}_{-0.02} & 1.57^{+0.02}_{-0.02} \\
    0.91 & 0.90 & 0.80 & 1.00 & 2.85^{+0.16}_{-0.16} & 3.77^{+0.20}_{-0.19} & 2.10^{+0.05}_{-0.05} & 2.79^{+0.06}_{-0.06} \\
    1.10 & 1.10 & 1.00 & 1.20 & 3.71^{+0.11}_{-0.10} & 4.94^{+0.15}_{-0.13} & 2.91^{+0.05}_{-0.05} & 3.87^{+0.06}_{-0.06} \\
    1.30 & 1.30 & 1.20 & 1.40 & 5.69^{+0.23}_{-0.24} & 7.52^{+0.32}_{-0.30} & 4.46^{+0.10}_{-0.09} & 5.91^{+0.11}_{-0.12} \\
    1.50 & 1.50 & 1.40 & 1.60 & 7.04^{+0.22}_{-0.22} & 9.34^{+0.26}_{-0.31} & 5.59^{+0.09}_{-0.11} & 7.41^{+0.13}_{-0.13} \\
    1.71 & 1.70 & 1.60 & 1.80 & 7.69^{+0.18}_{-0.18} & 10.23^{+0.24}_{-0.26} & 6.71^{+0.10}_{-0.11} & 8.90^{+0.13}_{-0.14} \\
    1.98 & 2.00 & 1.80 & 2.20 & 10.59^{+0.36}_{-0.36} & 14.08^{+0.43}_{-0.42} & 7.93^{+0.13}_{-0.14} & 10.52^{+0.19}_{-0.17} \\
    2.25 & 2.25 & 2.20 & 2.30 & 11.34^{+0.35}_{-0.37} & 14.97^{+0.48}_{-0.50} & 10.06^{+0.20}_{-0.20} & 13.37^{+0.26}_{-0.23} \\
    2.35 & 2.35 & 2.30 & 2.40 & 9.30^{+0.30}_{-0.30} & 12.29^{+0.36}_{-0.35} & 8.02^{+0.15}_{-0.15} & 10.65^{+0.22}_{-0.20} \\
    2.45 & 2.45 & 2.40 & 2.50 & 8.06^{+0.24}_{-0.24} & 10.67^{+0.30}_{-0.26} & 7.16^{+0.13}_{-0.15} & 9.49^{+0.17}_{-0.18} \\
    2.65 & 2.65 & 2.60 & 2.70 & 6.55^{+0.16}_{-0.17} & 8.70^{+0.21}_{-0.20} & 6.46^{+0.15}_{-0.15} & 8.55^{+0.20}_{-0.20} \\
    2.75 & 2.75 & 2.70 & 2.80 & 7.44^{+0.24}_{-0.22} & 9.93^{+0.30}_{-0.31} & 7.17^{+0.20}_{-0.20} & 9.51^{+0.27}_{-0.25} \\
    2.85 & 2.85 & 2.80 & 2.90 & 7.94^{+0.28}_{-0.29} & 10.45^{+0.42}_{-0.41} & 7.48^{+0.29}_{-0.28} & 9.95^{+0.36}_{-0.35} \\
    2.95 & 2.95 & 2.90 & 3.00 & 7.93^{+0.34}_{-0.34} & 10.51^{+0.45}_{-0.46} & 6.83^{+0.20}_{-0.21} & 9.09^{+0.28}_{-0.28} \\
    3.05 & 3.05 & 3.00 & 3.10 & 7.18^{+0.38}_{-0.36} & 9.59^{+0.47}_{-0.50} & 6.24^{+0.20}_{-0.21} & 8.29^{+0.34}_{-0.30} \\
    3.15 & 3.15 & 3.10 & 3.20 & 9.99^{+0.95}_{-1.03} & 13.14^{+1.38}_{-1.56} & 7.01^{+0.35}_{-0.35} & 9.34^{+0.41}_{-0.47} \\
    3.25 & 3.25 & 3.20 & 3.30 & 8.27^{+0.94}_{-1.01} & 11.05^{+1.27}_{-1.17} & 6.06^{+0.35}_{-0.36} & 8.07^{+0.49}_{-0.50} \\
    3.34 & 3.35 & 3.30 & 3.40 & 12.00^{+2.57}_{-2.55} & 16.10^{+3.44}_{-3.31} & 7.21^{+0.61}_{-0.74} & 9.59^{+0.91}_{-0.93} \\
    3.44 & 3.45 & 3.40 & 3.50 & 4.47^{+0.52}_{-0.51} & 5.85^{+0.61}_{-0.64} & 4.37^{+0.45}_{-0.47} & 5.73^{+0.56}_{-0.67} \\
    3.88 & 3.90 & 3.70 & 4.10 & 4.36^{+1.44}_{-1.39} & 5.76^{+1.98}_{-1.84} & 2.54^{+0.44}_{-0.44} & 3.45^{+0.72}_{-0.66} \\
    4.35 & 4.40 & 4.10 & 4.70 & 4.13^{+2.15}_{-1.95} & 5.40^{+2.60}_{-2.57} & 1.81^{+0.58}_{-0.52} & 2.42^{+0.64}_{-0.68} \\
    4.92 & 5.10 & 4.70 & 5.50 & 2.51^{+0.89}_{-0.86} & 3.45^{+1.28}_{-1.31} & 0.97^{+0.20}_{-0.18} & 1.29^{+0.25}_{-0.26} \\
    6.00 & 6.00 & 5.50 & 6.50 & 0.65^{+0.30}_{-0.24} & 0.96^{+0.36}_{-0.37} & 0.21^{+0.04}_{-0.04} & 0.27^{+0.06}_{-0.06} \\
    \hline
  \end{tabular}
\end{table*}

\begin{table*}
  % rtg2.py and tabulate_emissivities_global.py.
  % Check these values once again.
  \caption{The 912\,\AA\ and 1450\,\AA\ comoving emissivities obtained
    by fitting Equation~(\ref{eqn:e912fit}) to the selected redshift
    bins from Table~\ref{tab:emissivity_bins}, and the corresponding
    hydrogen photoionisation rates with the one-sigma (68.26\%)
    uncertainties.  The emissivities are extrapolated here up to
    $z=15$.  Emissivity units are
    erg\ s$^{-1}$\ Hz$^{-1}$\ cMpc$^{-3}$, and photoionisation rate
    units are s$^{-1}$.  These values are shown in
    Figures~\ref{fig:e912} and \ref{fig:gammapi}.  See
    Sections~\ref{sec:e912} and \ref{sec:gammahi} for more details.}
  \label{tab:gamma2}
  \begin{tabular}{d......}
    \hline
    \multicolumn{1}{c}{$z$} &
    \multicolumn{1}{c}{$\log_{10}\epsilon_{1450}$} &
    \multicolumn{1}{c}{$\log_{10}\epsilon_{1450}$} &
    \multicolumn{1}{c}{$\log_{10}\epsilon_{912}$} &
    \multicolumn{1}{c}{$\log_{10}\epsilon_{912}$} & 
    \multicolumn{1}{c}{$\log_{10}\Gamma_\mathrm{HI}$} &
    \multicolumn{1}{c}{$\log_{10}\Gamma_\mathrm{HI}$} \\ 
    &
    \multicolumn{1}{c}{$(M_{1450}<-18)$} &
    \multicolumn{1}{c}{$(M_{1450}<-21)$} &
    \multicolumn{1}{c}{$(M_{1450}<-18)$} &
    \multicolumn{1}{c}{$(M_{1450}<-21)$} &
    \multicolumn{1}{c}{$(M_{1450}<-18)$} &
    \multicolumn{1}{c}{$(M_{1450}<-21)$} \\
    \hline
    0.0 & 23.36^{+0.38}_{-0.20} & 23.18^{+0.25}_{-0.22} & 23.01^{+0.23}_{-0.16} & 22.83^{+0.12}_{-0.15} & -13.38^{+0.10}_{-0.06} & -13.57^{+0.06}_{-0.05} \\
    0.1 & 23.54^{+0.29}_{-0.16} & 23.35^{+0.18}_{-0.17} & 23.24^{+0.18}_{-0.13} & 23.06^{+0.09}_{-0.11} & -13.22^{+0.08}_{-0.06} & -13.41^{+0.05}_{-0.04} \\
    0.2 & 23.71^{+0.21}_{-0.14} & 23.51^{+0.13}_{-0.12} & 23.45^{+0.14}_{-0.10} & 23.27^{+0.07}_{-0.09} & -13.07^{+0.06}_{-0.05} & -13.26^{+0.04}_{-0.03} \\
    0.3 & 23.85^{+0.16}_{-0.10} & 23.66^{+0.10}_{-0.08} & 23.64^{+0.10}_{-0.07} & 23.46^{+0.05}_{-0.06} & -12.94^{+0.05}_{-0.04} & -13.12^{+0.03}_{-0.03} \\
    0.4 & 23.99^{+0.11}_{-0.08} & 23.80^{+0.07}_{-0.06} & 23.81^{+0.07}_{-0.06} & 23.63^{+0.04}_{-0.04} & -12.82^{+0.04}_{-0.04} & -12.99^{+0.03}_{-0.02} \\
    0.5 & 24.11^{+0.08}_{-0.06} & 23.93^{+0.05}_{-0.04} & 23.95^{+0.05}_{-0.04} & 23.78^{+0.03}_{-0.03} & -12.71^{+0.03}_{-0.03} & -12.87^{+0.02}_{-0.02} \\
    0.6 & 24.23^{+0.05}_{-0.04} & 24.06^{+0.03}_{-0.02} & 24.09^{+0.04}_{-0.03} & 23.93^{+0.02}_{-0.02} & -12.60^{+0.03}_{-0.03} & -12.76^{+0.02}_{-0.02} \\
    0.7 & 24.34^{+0.03}_{-0.03} & 24.18^{+0.02}_{-0.02} & 24.21^{+0.02}_{-0.03} & 24.06^{+0.01}_{-0.01} & -12.51^{+0.03}_{-0.03} & -12.66^{+0.02}_{-0.02} \\
    0.8 & 24.44^{+0.03}_{-0.02} & 24.30^{+0.01}_{-0.01} & 24.32^{+0.02}_{-0.02} & 24.18^{+0.01}_{-0.01} & -12.42^{+0.03}_{-0.03} & -12.57^{+0.02}_{-0.02} \\
    0.9 & 24.54^{+0.02}_{-0.02} & 24.41^{+0.01}_{-0.01} & 24.42^{+0.02}_{-0.01} & 24.29^{+0.01}_{-0.01} & -12.35^{+0.02}_{-0.03} & -12.48^{+0.02}_{-0.02} \\
    1.0 & 24.63^{+0.02}_{-0.02} & 24.51^{+0.01}_{-0.01} & 24.51^{+0.02}_{-0.01} & 24.38^{+0.01}_{-0.01} & -12.28^{+0.02}_{-0.03} & -12.40^{+0.02}_{-0.02} \\
    1.1 & 24.71^{+0.02}_{-0.02} & 24.60^{+0.01}_{-0.01} & 24.59^{+0.01}_{-0.01} & 24.47^{+0.01}_{-0.01} & -12.21^{+0.02}_{-0.03} & -12.33^{+0.02}_{-0.02} \\
    1.2 & 24.78^{+0.02}_{-0.02} & 24.68^{+0.01}_{-0.01} & 24.66^{+0.01}_{-0.01} & 24.55^{+0.01}_{-0.01} & -12.16^{+0.03}_{-0.03} & -12.27^{+0.02}_{-0.02} \\
    1.3 & 24.85^{+0.01}_{-0.02} & 24.75^{+0.01}_{-0.01} & 24.72^{+0.01}_{-0.01} & 24.63^{+0.01}_{-0.01} & -12.11^{+0.03}_{-0.03} & -12.22^{+0.02}_{-0.02} \\
    1.4 & 24.91^{+0.01}_{-0.02} & 24.82^{+0.01}_{-0.01} & 24.78^{+0.01}_{-0.01} & 24.69^{+0.01}_{-0.01} & -12.07^{+0.03}_{-0.03} & -12.18^{+0.02}_{-0.02} \\
    1.5 & 24.95^{+0.01}_{-0.01} & 24.87^{+0.01}_{-0.01} & 24.83^{+0.01}_{-0.01} & 24.74^{+0.01}_{-0.01} & -12.04^{+0.03}_{-0.03} & -12.15^{+0.02}_{-0.02} \\
    1.6 & 24.99^{+0.02}_{-0.01} & 24.91^{+0.01}_{-0.01} & 24.87^{+0.01}_{-0.01} & 24.79^{+0.01}_{-0.01} & -12.02^{+0.03}_{-0.03} & -12.12^{+0.02}_{-0.02} \\
    1.7 & 25.03^{+0.02}_{-0.02} & 24.95^{+0.01}_{-0.01} & 24.91^{+0.01}_{-0.01} & 24.83^{+0.01}_{-0.01} & -12.00^{+0.03}_{-0.03} & -12.11^{+0.02}_{-0.02} \\
    1.8 & 25.05^{+0.02}_{-0.02} & 24.98^{+0.01}_{-0.01} & 24.94^{+0.02}_{-0.01} & 24.86^{+0.01}_{-0.01} & -11.99^{+0.04}_{-0.04} & -12.10^{+0.02}_{-0.02} \\
    1.9 & 25.07^{+0.02}_{-0.02} & 24.99^{+0.01}_{-0.01} & 24.96^{+0.02}_{-0.02} & 24.88^{+0.01}_{-0.01} & -11.99^{+0.04}_{-0.04} & -12.09^{+0.03}_{-0.03} \\
    2.0 & 25.08^{+0.02}_{-0.02} & 25.01^{+0.01}_{-0.01} & 24.98^{+0.02}_{-0.02} & 24.90^{+0.01}_{-0.01} & -11.99^{+0.04}_{-0.05} & -12.09^{+0.03}_{-0.03} \\
    2.1 & 25.09^{+0.03}_{-0.02} & 25.01^{+0.02}_{-0.02} & 25.00^{+0.03}_{-0.02} & 24.91^{+0.02}_{-0.02} & -11.99^{+0.05}_{-0.05} & -12.10^{+0.03}_{-0.03} \\
    2.2 & 25.09^{+0.03}_{-0.03} & 25.01^{+0.02}_{-0.02} & 25.01^{+0.03}_{-0.03} & 24.91^{+0.02}_{-0.02} & -12.00^{+0.05}_{-0.06} & -12.11^{+0.04}_{-0.03} \\
    2.3 & 25.09^{+0.04}_{-0.04} & 25.01^{+0.02}_{-0.02} & 25.01^{+0.03}_{-0.03} & 24.91^{+0.02}_{-0.03} & -12.01^{+0.05}_{-0.07} & -12.12^{+0.04}_{-0.04} \\
    2.4 & 25.08^{+0.04}_{-0.04} & 25.00^{+0.03}_{-0.03} & 25.02^{+0.04}_{-0.04} & 24.91^{+0.03}_{-0.03} & -12.03^{+0.06}_{-0.08} & -12.14^{+0.04}_{-0.04} \\
    2.5 & 25.07^{+0.04}_{-0.05} & 24.99^{+0.03}_{-0.03} & 25.01^{+0.04}_{-0.04} & 24.89^{+0.03}_{-0.04} & -12.04^{+0.06}_{-0.08} & -12.16^{+0.05}_{-0.05} \\
    2.6 & 25.06^{+0.05}_{-0.06} & 24.97^{+0.04}_{-0.04} & 25.01^{+0.05}_{-0.05} & 24.88^{+0.04}_{-0.04} & -12.06^{+0.07}_{-0.09} & -12.18^{+0.05}_{-0.05} \\
    2.7 & 25.04^{+0.05}_{-0.07} & 24.96^{+0.04}_{-0.04} & 25.00^{+0.05}_{-0.05} & 24.86^{+0.04}_{-0.04} & -12.09^{+0.07}_{-0.10} & -12.21^{+0.05}_{-0.05} \\
    2.8 & 25.02^{+0.06}_{-0.07} & 24.93^{+0.05}_{-0.04} & 24.99^{+0.06}_{-0.06} & 24.84^{+0.05}_{-0.04} & -12.11^{+0.07}_{-0.11} & -12.24^{+0.06}_{-0.06} \\
    2.9 & 25.01^{+0.06}_{-0.08} & 24.91^{+0.05}_{-0.05} & 24.98^{+0.06}_{-0.07} & 24.81^{+0.05}_{-0.05} & -12.14^{+0.08}_{-0.12} & -12.27^{+0.06}_{-0.06} \\
    3.0 & 24.98^{+0.06}_{-0.09} & 24.88^{+0.05}_{-0.05} & 24.96^{+0.06}_{-0.07} & 24.78^{+0.05}_{-0.05} & -12.17^{+0.08}_{-0.12} & -12.31^{+0.06}_{-0.07} \\
    3.1 & 24.96^{+0.07}_{-0.10} & 24.85^{+0.05}_{-0.06} & 24.94^{+0.07}_{-0.08} & 24.75^{+0.06}_{-0.05} & -12.20^{+0.09}_{-0.13} & -12.35^{+0.07}_{-0.07} \\
    3.2 & 24.93^{+0.07}_{-0.10} & 24.82^{+0.06}_{-0.06} & 24.91^{+0.07}_{-0.09} & 24.71^{+0.06}_{-0.05} & -12.23^{+0.09}_{-0.14} & -12.39^{+0.07}_{-0.08} \\
    3.3 & 24.91^{+0.08}_{-0.11} & 24.79^{+0.06}_{-0.06} & 24.89^{+0.08}_{-0.09} & 24.68^{+0.06}_{-0.06} & -12.26^{+0.10}_{-0.15} & -12.43^{+0.07}_{-0.08} \\
    3.4 & 24.88^{+0.08}_{-0.12} & 24.75^{+0.06}_{-0.07} & 24.86^{+0.08}_{-0.09} & 24.64^{+0.07}_{-0.06} & -12.30^{+0.10}_{-0.17} & -12.47^{+0.07}_{-0.09} \\
    3.5 & 24.85^{+0.09}_{-0.14} & 24.72^{+0.07}_{-0.07} & 24.83^{+0.08}_{-0.10} & 24.59^{+0.07}_{-0.06} & -12.34^{+0.11}_{-0.18} & -12.52^{+0.08}_{-0.09} \\
    3.6 & 24.82^{+0.10}_{-0.15} & 24.68^{+0.07}_{-0.08} & 24.80^{+0.09}_{-0.11} & 24.55^{+0.07}_{-0.06} & -12.38^{+0.11}_{-0.19} & -12.56^{+0.08}_{-0.10} \\
    3.7 & 24.79^{+0.10}_{-0.16} & 24.64^{+0.07}_{-0.08} & 24.77^{+0.09}_{-0.11} & 24.51^{+0.07}_{-0.06} & -12.42^{+0.12}_{-0.20} & -12.61^{+0.08}_{-0.10} \\
    3.8 & 24.75^{+0.10}_{-0.16} & 24.59^{+0.07}_{-0.09} & 24.73^{+0.10}_{-0.12} & 24.47^{+0.07}_{-0.07} & -12.46^{+0.12}_{-0.21} & -12.66^{+0.08}_{-0.11} \\
    3.9 & 24.72^{+0.11}_{-0.17} & 24.55^{+0.07}_{-0.09} & 24.70^{+0.10}_{-0.13} & 24.42^{+0.07}_{-0.07} & -12.50^{+0.13}_{-0.22} & -12.72^{+0.08}_{-0.11} \\
    4.0 & 24.68^{+0.11}_{-0.18} & 24.51^{+0.07}_{-0.09} & 24.66^{+0.10}_{-0.14} & 24.37^{+0.07}_{-0.08} & -12.54^{+0.13}_{-0.24} & -12.77^{+0.08}_{-0.12} \\
    4.1 & 24.65^{+0.11}_{-0.20} & 24.46^{+0.07}_{-0.10} & 24.63^{+0.11}_{-0.16} & 24.33^{+0.06}_{-0.09} & -12.58^{+0.13}_{-0.25} & -12.82^{+0.08}_{-0.12} \\
    4.2 & 24.61^{+0.12}_{-0.21} & 24.42^{+0.07}_{-0.10} & 24.58^{+0.12}_{-0.16} & 24.28^{+0.06}_{-0.10} & -12.63^{+0.14}_{-0.27} & -12.88^{+0.08}_{-0.12} \\
    4.3 & 24.58^{+0.13}_{-0.23} & 24.37^{+0.07}_{-0.11} & 24.54^{+0.12}_{-0.17} & 24.23^{+0.07}_{-0.10} & -12.67^{+0.14}_{-0.29} & -12.94^{+0.09}_{-0.13} \\
    4.4 & 24.54^{+0.12}_{-0.24} & 24.32^{+0.08}_{-0.11} & 24.50^{+0.13}_{-0.18} & 24.17^{+0.07}_{-0.11} & -12.72^{+0.14}_{-0.30} & -13.00^{+0.09}_{-0.13} \\
    4.5 & 24.50^{+0.13}_{-0.26} & 24.27^{+0.08}_{-0.11} & 24.45^{+0.14}_{-0.19} & 24.12^{+0.07}_{-0.12} & -12.76^{+0.15}_{-0.32} & -13.06^{+0.09}_{-0.14} \\
    4.6 & 24.46^{+0.13}_{-0.28} & 24.22^{+0.08}_{-0.12} & 24.40^{+0.15}_{-0.19} & 24.07^{+0.08}_{-0.12} & -12.81^{+0.15}_{-0.34} & -13.12^{+0.10}_{-0.14} \\
    4.7 & 24.42^{+0.14}_{-0.30} & 24.16^{+0.09}_{-0.12} & 24.36^{+0.16}_{-0.21} & 24.01^{+0.08}_{-0.13} & -12.86^{+0.16}_{-0.36} & -13.18^{+0.11}_{-0.14} \\
    4.8 & 24.38^{+0.15}_{-0.31} & 24.11^{+0.09}_{-0.13} & 24.31^{+0.17}_{-0.22} & 23.96^{+0.09}_{-0.14} & -12.91^{+0.17}_{-0.38} & -13.25^{+0.11}_{-0.15} \\
    4.9 & 24.34^{+0.15}_{-0.33} & 24.05^{+0.10}_{-0.13} & 24.26^{+0.17}_{-0.24} & 23.90^{+0.09}_{-0.15} & -12.96^{+0.18}_{-0.39} & -13.32^{+0.12}_{-0.15} \\
    5.0 & 24.30^{+0.16}_{-0.35} & 23.99^{+0.11}_{-0.13} & 24.21^{+0.17}_{-0.25} & 23.84^{+0.10}_{-0.17} & -13.01^{+0.19}_{-0.41} & -13.38^{+0.13}_{-0.16} \\
    \hline
  \end{tabular}
\end{table*}

\begin{table*}
  \contcaption{}
  \begin{tabular}{d......}
    \hline
    \multicolumn{1}{c}{$z$} &    
    \multicolumn{1}{c}{$\log_{10}\epsilon_{1450}$} &
    \multicolumn{1}{c}{$\log_{10}\epsilon_{1450}$} &
    \multicolumn{1}{c}{$\log_{10}\epsilon_{912}$} &
    \multicolumn{1}{c}{$\log_{10}\epsilon_{912}$} & 
    \multicolumn{1}{c}{$\log_{10}\Gamma_\mathrm{HI}$} &
    \multicolumn{1}{c}{$\log_{10}\Gamma_\mathrm{HI}$} \\ 
    &
    \multicolumn{1}{c}{$(M_{1450}<-18)$} &
    \multicolumn{1}{c}{$(M_{1450}<-21)$} &
    \multicolumn{1}{c}{$(M_{1450}<-18)$} &
    \multicolumn{1}{c}{$(M_{1450}<-21)$} &
    \multicolumn{1}{c}{$(M_{1450}<-18)$} &
    \multicolumn{1}{c}{$(M_{1450}<-21)$} \\
    \hline
    5.1 & 24.26^{+0.17}_{-0.36} & 23.93^{+0.11}_{-0.14} & 24.16^{+0.18}_{-0.26} & 23.78^{+0.10}_{-0.19} & -13.06^{+0.20}_{-0.42} & -13.45^{+0.13}_{-0.16} \\
    5.2 & 24.21^{+0.19}_{-0.37} & 23.88^{+0.12}_{-0.14} & 24.10^{+0.19}_{-0.27} & 23.72^{+0.11}_{-0.20} & -13.12^{+0.21}_{-0.44} & -13.52^{+0.14}_{-0.17} \\
    5.3 & 24.17^{+0.20}_{-0.39} & 23.82^{+0.13}_{-0.15} & 24.05^{+0.19}_{-0.29} & 23.66^{+0.12}_{-0.22} & -13.17^{+0.22}_{-0.46} & -13.59^{+0.15}_{-0.18} \\
    5.4 & 24.13^{+0.21}_{-0.41} & 23.76^{+0.13}_{-0.16} & 24.00^{+0.20}_{-0.30} & 23.60^{+0.12}_{-0.23} & -13.22^{+0.23}_{-0.47} & -13.67^{+0.16}_{-0.19} \\
    5.5 & 24.09^{+0.22}_{-0.43} & 23.70^{+0.14}_{-0.17} & 23.94^{+0.21}_{-0.31} & 23.54^{+0.12}_{-0.24} & -13.29^{+0.24}_{-0.49} & -13.75^{+0.17}_{-0.20} \\
    5.6 & 24.04^{+0.23}_{-0.45} & 23.63^{+0.16}_{-0.17} & 23.89^{+0.22}_{-0.33} & 23.48^{+0.12}_{-0.25} & -13.37^{+0.25}_{-0.51} & -13.85^{+0.18}_{-0.21} \\
    5.7 & 24.00^{+0.24}_{-0.46} & 23.57^{+0.16}_{-0.18} & 23.84^{+0.23}_{-0.34} & 23.42^{+0.13}_{-0.26} & -13.45^{+0.26}_{-0.53} & -13.94^{+0.18}_{-0.22} \\
    5.8 & 23.95^{+0.25}_{-0.48} & 23.51^{+0.16}_{-0.20} & 23.78^{+0.24}_{-0.35} & 23.35^{+0.14}_{-0.28} & -13.53^{+0.27}_{-0.55} & -14.04^{+0.19}_{-0.23} \\
    5.9 & 23.91^{+0.26}_{-0.50} & 23.45^{+0.17}_{-0.20} & 23.73^{+0.24}_{-0.38} & 23.29^{+0.15}_{-0.30} & -13.62^{+0.28}_{-0.57} & -14.15^{+0.20}_{-0.24} \\
    6.0 & 23.87^{+0.27}_{-0.52} & 23.38^{+0.18}_{-0.20} & 23.67^{+0.25}_{-0.40} & 23.23^{+0.15}_{-0.32} & -13.70^{+0.29}_{-0.58} & -14.26^{+0.21}_{-0.24} \\
    6.1 & 23.82^{+0.29}_{-0.54} & 23.32^{+0.19}_{-0.21} & 23.61^{+0.27}_{-0.42} & 23.16^{+0.16}_{-0.34} & -13.80^{+0.31}_{-0.60} & -14.37^{+0.22}_{-0.25} \\
    6.2 & 23.76^{+0.30}_{-0.56} & 23.25^{+0.20}_{-0.22} & 23.55^{+0.28}_{-0.42} & 23.10^{+0.16}_{-0.36} & -13.90^{+0.32}_{-0.62} & -14.48^{+0.23}_{-0.26} \\
    6.3 & 23.71^{+0.32}_{-0.57} & 23.19^{+0.21}_{-0.23} & 23.49^{+0.29}_{-0.43} & 23.03^{+0.17}_{-0.38} & -14.00^{+0.34}_{-0.64} & -14.59^{+0.24}_{-0.28} \\
    6.4 & 23.66^{+0.33}_{-0.59} & 23.13^{+0.22}_{-0.24} & 23.42^{+0.30}_{-0.44} & 22.96^{+0.18}_{-0.39} & -14.10^{+0.35}_{-0.66} & -14.71^{+0.25}_{-0.29} \\
    6.5 & 23.62^{+0.34}_{-0.61} & 23.06^{+0.23}_{-0.25} & 23.36^{+0.31}_{-0.46} & 22.90^{+0.19}_{-0.40} & -14.20^{+0.36}_{-0.68} & -14.83^{+0.26}_{-0.31} \\
    6.6 & 23.56^{+0.36}_{-0.62} & 23.00^{+0.24}_{-0.26} & 23.30^{+0.33}_{-0.48} & 22.83^{+0.19}_{-0.42} & -14.31^{+0.38}_{-0.69} & -14.96^{+0.27}_{-0.32} \\
    6.7 & 23.52^{+0.37}_{-0.64} & 22.93^{+0.25}_{-0.28} & 23.23^{+0.34}_{-0.50} & 22.76^{+0.20}_{-0.44} & -14.42^{+0.39}_{-0.72} & -15.08^{+0.28}_{-0.34} \\
    6.8 & 23.47^{+0.38}_{-0.67} & 22.87^{+0.26}_{-0.29} & 23.15^{+0.36}_{-0.51} & 22.70^{+0.20}_{-0.45} & -14.52^{+0.40}_{-0.74} & -15.22^{+0.29}_{-0.36} \\
    6.9 & 23.42^{+0.40}_{-0.68} & 22.80^{+0.27}_{-0.30} & 23.08^{+0.39}_{-0.53} & 22.63^{+0.21}_{-0.47} & -14.64^{+0.41}_{-0.77} & -15.35^{+0.30}_{-0.37} \\
    7.0 & 23.37^{+0.41}_{-0.71} & 22.73^{+0.28}_{-0.31} & 23.01^{+0.41}_{-0.54} & 22.56^{+0.22}_{-0.49} & -14.74^{+0.42}_{-0.79} & -15.48^{+0.32}_{-0.40} \\
    7.1 & 23.33^{+0.41}_{-0.73} & 22.66^{+0.29}_{-0.33} & 22.94^{+0.42}_{-0.56} & 22.49^{+0.23}_{-0.52} & -14.86^{+0.43}_{-0.82} & -15.62^{+0.33}_{-0.42} \\
    7.2 & 23.28^{+0.43}_{-0.76} & 22.60^{+0.29}_{-0.35} & 22.87^{+0.44}_{-0.57} & 22.42^{+0.23}_{-0.54} & -14.97^{+0.45}_{-0.85} & -15.75^{+0.34}_{-0.45} \\
    7.3 & 23.22^{+0.45}_{-0.77} & 22.53^{+0.31}_{-0.36} & 22.80^{+0.47}_{-0.59} & 22.35^{+0.24}_{-0.57} & -15.09^{+0.47}_{-0.87} & -15.88^{+0.36}_{-0.47} \\
    7.4 & 23.17^{+0.46}_{-0.80} & 22.46^{+0.32}_{-0.38} & 22.73^{+0.49}_{-0.60} & 22.29^{+0.25}_{-0.59} & -15.20^{+0.48}_{-0.90} & -16.02^{+0.37}_{-0.50} \\
    7.5 & 23.13^{+0.47}_{-0.83} & 22.39^{+0.34}_{-0.39} & 22.67^{+0.51}_{-0.61} & 22.21^{+0.25}_{-0.62} & -15.30^{+0.49}_{-0.93} & -16.15^{+0.39}_{-0.52} \\
    7.6 & 23.09^{+0.47}_{-0.87} & 22.32^{+0.35}_{-0.41} & 22.60^{+0.52}_{-0.63} & 22.14^{+0.26}_{-0.64} & -15.41^{+0.49}_{-0.97} & -16.29^{+0.41}_{-0.55} \\
    7.7 & 23.04^{+0.48}_{-0.90} & 22.26^{+0.37}_{-0.42} & 22.53^{+0.54}_{-0.64} & 22.07^{+0.27}_{-0.66} & -15.52^{+0.50}_{-1.00} & -16.42^{+0.43}_{-0.57} \\
    7.8 & 22.99^{+0.50}_{-0.92} & 22.19^{+0.38}_{-0.44} & 22.46^{+0.57}_{-0.65} & 22.00^{+0.28}_{-0.68} & -15.63^{+0.52}_{-1.02} & -16.55^{+0.45}_{-0.60} \\
    7.9 & 22.94^{+0.51}_{-0.94} & 22.12^{+0.41}_{-0.46} & 22.38^{+0.58}_{-0.65} & 21.92^{+0.29}_{-0.70} & -15.74^{+0.53}_{-1.05} & -16.68^{+0.47}_{-0.63} \\
    8.0 & 22.88^{+0.53}_{-0.97} & 22.05^{+0.43}_{-0.48} & 22.31^{+0.60}_{-0.66} & 21.85^{+0.30}_{-0.73} & -15.84^{+0.54}_{-1.07} & -16.80^{+0.50}_{-0.66} \\
    8.1 & 22.83^{+0.54}_{-0.99} & 21.97^{+0.44}_{-0.49} & 22.24^{+0.61}_{-0.68} & 21.78^{+0.31}_{-0.76} & -15.95^{+0.56}_{-1.10} & -16.93^{+0.52}_{-0.68} \\
    8.2 & 22.78^{+0.56}_{-1.02} & 21.90^{+0.46}_{-0.51} & 22.17^{+0.63}_{-0.70} & 21.71^{+0.32}_{-0.78} & -16.05^{+0.57}_{-1.12} & -17.06^{+0.54}_{-0.71} \\
    8.3 & 22.73^{+0.57}_{-1.04} & 21.82^{+0.48}_{-0.52} & 22.10^{+0.65}_{-0.71} & 21.63^{+0.33}_{-0.81} & -16.15^{+0.58}_{-1.15} & -17.20^{+0.56}_{-0.72} \\
    8.4 & 22.68^{+0.58}_{-1.07} & 21.74^{+0.50}_{-0.53} & 22.02^{+0.68}_{-0.72} & 21.56^{+0.33}_{-0.84} & -16.25^{+0.59}_{-1.18} & -17.33^{+0.58}_{-0.74} \\
    8.5 & 22.63^{+0.59}_{-1.10} & 21.66^{+0.52}_{-0.54} & 21.94^{+0.71}_{-0.73} & 21.49^{+0.34}_{-0.86} & -16.35^{+0.61}_{-1.21} & -17.45^{+0.60}_{-0.76} \\
    8.6 & 22.59^{+0.60}_{-1.13} & 21.59^{+0.53}_{-0.55} & 21.86^{+0.75}_{-0.73} & 21.41^{+0.35}_{-0.89} & -16.45^{+0.61}_{-1.24} & -17.57^{+0.61}_{-0.78} \\
    8.7 & 22.54^{+0.61}_{-1.16} & 21.52^{+0.54}_{-0.57} & 21.78^{+0.78}_{-0.73} & 21.34^{+0.36}_{-0.92} & -16.55^{+0.63}_{-1.26} & -17.69^{+0.63}_{-0.80} \\
    8.8 & 22.48^{+0.63}_{-1.18} & 21.45^{+0.56}_{-0.58} & 21.69^{+0.81}_{-0.73} & 21.26^{+0.37}_{-0.94} & -16.65^{+0.64}_{-1.28} & -17.82^{+0.64}_{-0.82} \\
    8.9 & 22.42^{+0.65}_{-1.20} & 21.37^{+0.58}_{-0.59} & 21.61^{+0.84}_{-0.74} & 21.19^{+0.38}_{-0.97} & -16.75^{+0.66}_{-1.31} & -17.94^{+0.67}_{-0.84} \\
    9.0 & 22.37^{+0.66}_{-1.23} & 21.30^{+0.60}_{-0.61} & 21.53^{+0.87}_{-0.75} & 21.11^{+0.39}_{-0.99} & -16.85^{+0.67}_{-1.33} & -18.06^{+0.69}_{-0.86} \\
    9.1 & 22.33^{+0.67}_{-1.26} & 21.22^{+0.62}_{-0.62} & 21.45^{+0.90}_{-0.76} & 21.04^{+0.40}_{-1.02} & -16.94^{+0.68}_{-1.36} & -18.18^{+0.72}_{-0.89} \\
    9.2 & 22.27^{+0.68}_{-1.28} & 21.15^{+0.65}_{-0.64} & 21.36^{+0.93}_{-0.77} & 20.96^{+0.41}_{-1.05} & -17.04^{+0.69}_{-1.38} & -18.30^{+0.74}_{-0.91} \\
    9.3 & 22.22^{+0.69}_{-1.31} & 21.07^{+0.67}_{-0.66} & 21.29^{+0.95}_{-0.80} & 20.89^{+0.43}_{-1.08} & -17.14^{+0.71}_{-1.41} & -18.41^{+0.77}_{-0.93} \\
    9.4 & 22.16^{+0.70}_{-1.33} & 21.00^{+0.69}_{-0.67} & 21.21^{+0.97}_{-0.82} & 20.81^{+0.44}_{-1.10} & -17.23^{+0.72}_{-1.43} & -18.53^{+0.79}_{-0.95} \\
    9.5 & 22.11^{+0.72}_{-1.36} & 20.92^{+0.71}_{-0.68} & 21.13^{+1.00}_{-0.83} & 20.73^{+0.45}_{-1.13} & -17.33^{+0.74}_{-1.45} & -18.65^{+0.81}_{-0.96} \\
    9.6 & 22.06^{+0.73}_{-1.38} & 20.84^{+0.73}_{-0.70} & 21.05^{+1.03}_{-0.85} & 20.65^{+0.46}_{-1.16} & -17.42^{+0.75}_{-1.47} & -18.77^{+0.83}_{-0.98} \\
    9.7 & 22.00^{+0.74}_{-1.40} & 20.76^{+0.75}_{-0.72} & 20.97^{+1.06}_{-0.86} & 20.57^{+0.48}_{-1.18} & -17.52^{+0.76}_{-1.49} & -18.89^{+0.85}_{-1.01} \\
    9.8 & 21.95^{+0.76}_{-1.42} & 20.69^{+0.77}_{-0.73} & 20.88^{+1.10}_{-0.88} & 20.49^{+0.49}_{-1.21} & -17.61^{+0.78}_{-1.51} & -19.01^{+0.87}_{-1.03} \\
    9.9 & 21.90^{+0.76}_{-1.45} & 20.61^{+0.79}_{-0.75} & 20.80^{+1.13}_{-0.89} & 20.42^{+0.50}_{-1.24} & -17.70^{+0.79}_{-1.54} & -19.12^{+0.89}_{-1.06} \\
    10.0 & 21.85^{+0.77}_{-1.48} & 20.53^{+0.81}_{-0.77} & 20.72^{+1.16}_{-0.90} & 20.34^{+0.51}_{-1.26} & -17.79^{+0.80}_{-1.57} & -19.23^{+0.91}_{-1.09} \\
    \hline
  \end{tabular}
\end{table*}

\begin{table*}
  \contcaption{}
  \begin{tabular}{d......}
    \hline
    \multicolumn{1}{c}{$z$} &    
    \multicolumn{1}{c}{$\log_{10}\epsilon_{1450}$} &
    \multicolumn{1}{c}{$\log_{10}\epsilon_{1450}$} &
    \multicolumn{1}{c}{$\log_{10}\epsilon_{912}$} &
    \multicolumn{1}{c}{$\log_{10}\epsilon_{912}$} & 
    \multicolumn{1}{c}{$\log_{10}\Gamma_\mathrm{HI}$} &
    \multicolumn{1}{c}{$\log_{10}\Gamma_\mathrm{HI}$} \\ 
    &
    \multicolumn{1}{c}{$(M_{1450}<-18)$} &
    \multicolumn{1}{c}{$(M_{1450}<-21)$} &
    \multicolumn{1}{c}{$(M_{1450}<-18)$} &
    \multicolumn{1}{c}{$(M_{1450}<-21)$} &
    \multicolumn{1}{c}{$(M_{1450}<-18)$} &
    \multicolumn{1}{c}{$(M_{1450}<-21)$} \\
    \hline
    10.0 & 21.85^{+0.77}_{-1.48} & 20.53^{+0.81}_{-0.77} & 20.72^{+1.16}_{-0.90} & 20.34^{+0.51}_{-1.26} & -17.79^{+0.80}_{-1.57} & -19.23^{+0.91}_{-1.09} \\
    10.1 & 21.80^{+0.80}_{-1.50} & 20.46^{+0.83}_{-0.79} & 20.63^{+1.19}_{-0.91} & 20.26^{+0.52}_{-1.29} & -17.88^{+0.83}_{-1.58} & -19.35^{+0.93}_{-1.11} \\
    10.2 & 21.74^{+0.83}_{-1.52} & 20.38^{+0.85}_{-0.81} & 20.55^{+1.23}_{-0.93} & 20.18^{+0.53}_{-1.31} & -17.97^{+0.85}_{-1.60} & -19.46^{+0.96}_{-1.13} \\
    10.3 & 21.69^{+0.85}_{-1.54} & 20.31^{+0.88}_{-0.83} & 20.47^{+1.26}_{-0.94} & 20.10^{+0.54}_{-1.34} & -18.07^{+0.87}_{-1.62} & -19.57^{+0.98}_{-1.16} \\
    10.4 & 21.63^{+0.87}_{-1.56} & 20.23^{+0.90}_{-0.85} & 20.39^{+1.29}_{-0.95} & 20.03^{+0.55}_{-1.36} & -18.16^{+0.89}_{-1.64} & -19.69^{+1.01}_{-1.18} \\
    10.5 & 21.58^{+0.89}_{-1.59} & 20.15^{+0.93}_{-0.87} & 20.31^{+1.31}_{-0.96} & 19.95^{+0.56}_{-1.39} & -18.25^{+0.90}_{-1.67} & -19.80^{+1.03}_{-1.21} \\
    10.6 & 21.52^{+0.90}_{-1.62} & 20.07^{+0.95}_{-0.89} & 20.22^{+1.35}_{-0.97} & 19.87^{+0.58}_{-1.42} & -18.34^{+0.91}_{-1.69} & -19.91^{+1.05}_{-1.23} \\
    10.7 & 21.47^{+0.90}_{-1.65} & 19.99^{+0.97}_{-0.90} & 20.13^{+1.38}_{-0.98} & 19.79^{+0.59}_{-1.44} & -18.43^{+0.92}_{-1.72} & -20.03^{+1.08}_{-1.25} \\
    10.8 & 21.42^{+0.91}_{-1.67} & 19.91^{+0.99}_{-0.92} & 20.04^{+1.42}_{-0.99} & 19.71^{+0.60}_{-1.47} & -18.52^{+0.93}_{-1.74} & -20.15^{+1.10}_{-1.27} \\
    10.9 & 21.36^{+0.93}_{-1.70} & 19.83^{+1.01}_{-0.94} & 19.95^{+1.46}_{-1.00} & 19.63^{+0.61}_{-1.50} & -18.62^{+0.95}_{-1.76} & -20.26^{+1.13}_{-1.29} \\
    11.0 & 21.30^{+0.94}_{-1.72} & 19.75^{+1.04}_{-0.96} & 19.86^{+1.50}_{-1.01} & 19.55^{+0.62}_{-1.53} & -18.71^{+0.96}_{-1.78} & -20.38^{+1.16}_{-1.32} \\
    11.1 & 21.24^{+0.96}_{-1.74} & 19.67^{+1.08}_{-0.98} & 19.78^{+1.53}_{-1.03} & 19.47^{+0.63}_{-1.56} & -18.81^{+0.98}_{-1.81} & -20.49^{+1.20}_{-1.34} \\
    11.2 & 21.18^{+0.97}_{-1.76} & 19.59^{+1.12}_{-1.00} & 19.69^{+1.56}_{-1.04} & 19.39^{+0.64}_{-1.59} & -18.90^{+0.99}_{-1.82} & -20.60^{+1.24}_{-1.36} \\
    11.3 & 21.12^{+0.99}_{-1.78} & 19.51^{+1.15}_{-1.01} & 19.60^{+1.60}_{-1.05} & 19.31^{+0.66}_{-1.62} & -18.99^{+1.01}_{-1.84} & -20.72^{+1.28}_{-1.39} \\
    11.4 & 21.06^{+1.00}_{-1.79} & 19.43^{+1.20}_{-1.03} & 19.51^{+1.63}_{-1.06} & 19.23^{+0.67}_{-1.65} & -19.09^{+1.02}_{-1.85} & -20.83^{+1.32}_{-1.40} \\
    11.5 & 21.00^{+1.02}_{-1.82} & 19.35^{+1.23}_{-1.05} & 19.42^{+1.67}_{-1.07} & 19.15^{+0.68}_{-1.68} & -19.18^{+1.04}_{-1.88} & -20.95^{+1.35}_{-1.42} \\
    11.6 & 20.94^{+1.03}_{-1.84} & 19.27^{+1.27}_{-1.06} & 19.32^{+1.71}_{-1.08} & 19.07^{+0.69}_{-1.71} & -19.27^{+1.05}_{-1.90} & -21.07^{+1.39}_{-1.44} \\
    11.7 & 20.88^{+1.05}_{-1.86} & 19.18^{+1.31}_{-1.08} & 19.23^{+1.75}_{-1.09} & 18.99^{+0.70}_{-1.74} & -19.37^{+1.07}_{-1.92} & -21.18^{+1.43}_{-1.45} \\
    11.8 & 20.82^{+1.06}_{-1.88} & 19.10^{+1.35}_{-1.09} & 19.14^{+1.79}_{-1.10} & 18.91^{+0.71}_{-1.77} & -19.46^{+1.09}_{-1.94} & -21.30^{+1.46}_{-1.47} \\
    11.9 & 20.76^{+1.08}_{-1.90} & 19.01^{+1.38}_{-1.10} & 19.04^{+1.83}_{-1.11} & 18.83^{+0.72}_{-1.80} & -19.56^{+1.10}_{-1.96} & -21.41^{+1.49}_{-1.48} \\
    12.0 & 20.70^{+1.10}_{-1.93} & 18.93^{+1.41}_{-1.11} & 18.95^{+1.86}_{-1.13} & 18.75^{+0.73}_{-1.82} & -19.65^{+1.12}_{-1.98} & -21.52^{+1.52}_{-1.50} \\
    12.1 & 20.64^{+1.12}_{-1.95} & 18.86^{+1.43}_{-1.14} & 18.87^{+1.90}_{-1.14} & 18.67^{+0.74}_{-1.85} & -19.74^{+1.14}_{-2.00} & -21.63^{+1.55}_{-1.52} \\
    12.2 & 20.58^{+1.14}_{-1.97} & 18.78^{+1.46}_{-1.16} & 18.78^{+1.93}_{-1.16} & 18.59^{+0.76}_{-1.88} & -19.83^{+1.16}_{-2.03} & -21.74^{+1.57}_{-1.54} \\
    12.3 & 20.52^{+1.16}_{-1.99} & 18.70^{+1.50}_{-1.18} & 18.69^{+1.96}_{-1.18} & 18.51^{+0.77}_{-1.91} & -19.92^{+1.18}_{-2.05} & -21.85^{+1.61}_{-1.56} \\
    12.4 & 20.46^{+1.16}_{-2.02} & 18.62^{+1.53}_{-1.20} & 18.60^{+1.99}_{-1.20} & 18.42^{+0.78}_{-1.93} & -20.01^{+1.19}_{-2.08} & -21.96^{+1.64}_{-1.58} \\
    12.5 & 20.41^{+1.17}_{-2.05} & 18.54^{+1.56}_{-1.22} & 18.52^{+2.02}_{-1.22} & 18.34^{+0.80}_{-1.96} & -20.10^{+1.20}_{-2.10} & -22.07^{+1.67}_{-1.60} \\
    12.6 & 20.35^{+1.19}_{-2.07} & 18.46^{+1.59}_{-1.24} & 18.43^{+2.05}_{-1.24} & 18.25^{+0.81}_{-1.98} & -20.19^{+1.22}_{-2.12} & -22.18^{+1.70}_{-1.63} \\
    12.7 & 20.29^{+1.21}_{-2.09} & 18.38^{+1.62}_{-1.26} & 18.35^{+2.07}_{-1.27} & 18.17^{+0.82}_{-2.01} & -20.28^{+1.24}_{-2.14} & -22.29^{+1.72}_{-1.65} \\
    12.8 & 20.23^{+1.23}_{-2.12} & 18.29^{+1.65}_{-1.28} & 18.27^{+2.10}_{-1.29} & 18.09^{+0.83}_{-2.04} & -20.37^{+1.26}_{-2.16} & -22.41^{+1.76}_{-1.66} \\
    12.9 & 20.17^{+1.25}_{-2.14} & 18.21^{+1.68}_{-1.29} & 18.19^{+2.12}_{-1.32} & 18.01^{+0.84}_{-2.07} & -20.46^{+1.28}_{-2.19} & -22.52^{+1.79}_{-1.68} \\
    13.0 & 20.11^{+1.28}_{-2.16} & 18.12^{+1.71}_{-1.31} & 18.11^{+2.15}_{-1.35} & 17.92^{+0.85}_{-2.09} & -20.55^{+1.30}_{-2.21} & -22.64^{+1.82}_{-1.69} \\
    13.1 & 20.05^{+1.30}_{-2.18} & 18.04^{+1.75}_{-1.32} & 18.02^{+2.18}_{-1.37} & 17.84^{+0.86}_{-2.12} & -20.64^{+1.31}_{-2.23} & -22.75^{+1.86}_{-1.71} \\
    13.2 & 19.99^{+1.31}_{-2.21} & 17.95^{+1.79}_{-1.34} & 17.93^{+2.21}_{-1.39} & 17.76^{+0.88}_{-2.15} & -20.73^{+1.33}_{-2.26} & -22.86^{+1.89}_{-1.72} \\
    13.3 & 19.93^{+1.32}_{-2.24} & 17.87^{+1.83}_{-1.36} & 17.84^{+2.24}_{-1.41} & 17.67^{+0.89}_{-2.17} & -20.81^{+1.34}_{-2.29} & -22.98^{+1.93}_{-1.74} \\
    13.4 & 19.88^{+1.33}_{-2.27} & 17.79^{+1.86}_{-1.37} & 17.76^{+2.27}_{-1.43} & 17.59^{+0.91}_{-2.20} & -20.90^{+1.34}_{-2.32} & -23.09^{+1.96}_{-1.75} \\
    13.5 & 19.83^{+1.34}_{-2.30} & 17.70^{+1.90}_{-1.39} & 17.66^{+2.30}_{-1.45} & 17.51^{+0.92}_{-2.23} & -20.98^{+1.35}_{-2.35} & -23.21^{+2.00}_{-1.77} \\
    13.6 & 19.78^{+1.34}_{-2.33} & 17.62^{+1.93}_{-1.41} & 17.57^{+2.33}_{-1.46} & 17.42^{+0.93}_{-2.26} & -21.06^{+1.36}_{-2.38} & -23.32^{+2.03}_{-1.78} \\
    13.7 & 19.72^{+1.35}_{-2.36} & 17.53^{+1.97}_{-1.42} & 17.48^{+2.36}_{-1.48} & 17.34^{+0.95}_{-2.29} & -21.15^{+1.37}_{-2.40} & -23.43^{+2.06}_{-1.79} \\
    13.8 & 19.67^{+1.37}_{-2.38} & 17.45^{+2.00}_{-1.44} & 17.39^{+2.39}_{-1.50} & 17.25^{+0.96}_{-2.32} & -21.24^{+1.38}_{-2.43} & -23.54^{+2.09}_{-1.81} \\
    13.9 & 19.61^{+1.38}_{-2.41} & 17.37^{+2.03}_{-1.46} & 17.30^{+2.42}_{-1.52} & 17.17^{+0.97}_{-2.35} & -21.32^{+1.39}_{-2.46} & -23.65^{+2.12}_{-1.83} \\
    14.0 & 19.55^{+1.39}_{-2.44} & 17.29^{+2.06}_{-1.48} & 17.21^{+2.45}_{-1.54} & 17.09^{+0.99}_{-2.38} & -21.41^{+1.40}_{-2.48} & -23.76^{+2.15}_{-1.84} \\
    14.1 & 19.50^{+1.40}_{-2.46} & 17.20^{+2.10}_{-1.50} & 17.12^{+2.48}_{-1.55} & 17.00^{+1.00}_{-2.41} & -21.49^{+1.41}_{-2.51} & -23.87^{+2.18}_{-1.86} \\
    14.2 & 19.44^{+1.41}_{-2.50} & 17.12^{+2.13}_{-1.52} & 17.03^{+2.51}_{-1.56} & 16.92^{+1.01}_{-2.44} & -21.58^{+1.42}_{-2.54} & -23.98^{+2.21}_{-1.87} \\
    14.3 & 19.39^{+1.42}_{-2.53} & 17.04^{+2.16}_{-1.54} & 16.94^{+2.54}_{-1.57} & 16.83^{+1.02}_{-2.47} & -21.66^{+1.43}_{-2.57} & -24.09^{+2.24}_{-1.89} \\
    14.4 & 19.34^{+1.43}_{-2.56} & 16.96^{+2.19}_{-1.56} & 16.85^{+2.57}_{-1.58} & 16.75^{+1.04}_{-2.50} & -21.75^{+1.44}_{-2.60} & -24.20^{+2.26}_{-1.91} \\
    14.5 & 19.29^{+1.44}_{-2.60} & 16.88^{+2.22}_{-1.58} & 16.76^{+2.60}_{-1.59} & 16.66^{+1.05}_{-2.53} & -21.84^{+1.45}_{-2.63} & -24.32^{+2.29}_{-1.93} \\
    14.6 & 19.24^{+1.46}_{-2.63} & 16.80^{+2.25}_{-1.60} & 16.67^{+2.63}_{-1.60} & 16.58^{+1.06}_{-2.56} & -21.93^{+1.47}_{-2.66} & -24.43^{+2.32}_{-1.94} \\
    14.7 & 19.18^{+1.47}_{-2.65} & 16.72^{+2.28}_{-1.62} & 16.57^{+2.66}_{-1.61} & 16.50^{+1.07}_{-2.59} & -22.03^{+1.48}_{-2.68} & -24.56^{+2.34}_{-1.96} \\
    14.8 & 19.13^{+1.48}_{-2.69} & 16.64^{+2.31}_{-1.64} & 16.48^{+2.69}_{-1.63} & 16.41^{+1.09}_{-2.61} & -22.15^{+1.48}_{-2.71} & -24.69^{+2.37}_{-1.97} \\
    14.9 & 19.08^{+1.48}_{-2.72} & 16.56^{+2.34}_{-1.66} & 16.39^{+2.72}_{-1.64} & 16.33^{+1.10}_{-2.64} & -22.28^{+1.48}_{-2.74} & -24.85^{+2.39}_{-1.99} \\
    15.0 & 19.02^{+1.49}_{-2.75} & 16.48^{+2.37}_{-1.68} & 16.30^{+2.75}_{-1.66} & 16.24^{+1.11}_{-2.67} & -22.50^{+1.49}_{-2.76} & -25.08^{+2.41}_{-2.00} \\
    \hline
  \end{tabular}
\end{table*}

\section{Code and data}

We make the code and data used in this work publicly available at
\url{https://github.com/gkulkarni/QLF}.  This includes homogenised AGN
catalogues and selection functions and the code used for developing
and analysing luminosity function models.
%% Make URL public. Set appropriate license.  Take permission from
%% other authors.  Set up a DOI.

\bibliographystyle{mnras}
\bibliography{refs}

\bsp
\label{lastpage}
\end{document}


