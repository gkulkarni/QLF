\documentclass[a4paper,fleqn,usenatbib]{mnras}
\usepackage[T1]{fontenc}
\usepackage{ae,aecompl}
\usepackage{graphicx}	
\usepackage{amsmath}	
\usepackage{amssymb}
\usepackage{pdflscape}
\usepackage{siunitx}

\title[AGN luminosity function]{Evolution of the AGN UV luminosity function}

\author[Kulkarni et al.]{{Girish Kulkarni$^{1}$\thanks{Email:
      kulkarni@ast.cam.ac.uk}, G\'abor Worseck$^{2}$ and Joseph
    F.~Hennawi$^{3}$} \\ $^1$Institute of Astronomy and Kavli
  Institute of Cosmology, University of Cambridge, Madingley Road,
  Cambridge CB3 0HA, UK \\ $^2$Max Planck Institute for Astronomy,
  K\"onigstuhl 17, D-69117 Heidelberg, Germany\\ $^3$Department of
  Physics, Broida Hall, UC Santa Barbara, Santa Barbara, CA 93106-9530
  USA}

\date{Accepted ---. Received ---; in original form ---}

\pubyear{2016}

\begin{document}
\label{firstpage}
\pagerange{\pageref{firstpage}--\pageref{lastpage}}
\maketitle

\begin{abstract}
  Lorem ipsum dolor sit amet, consectetur adipiscing elit. Curabitur
  eget nisi augue. Vivamus quis purus quis massa tempor posuere non
  quis magna. Aenean eleifend, metus eget facilisis faucibus, turpis
  erat suscipit tortor, ut dapibus nulla neque ac sapien. Suspendisse
  luctus eros eu quam laoreet, vel dapibus sem porttitor. Mauris nec
  massa ultrices, porttitor nulla at, euismod diam. In ultricies
  malesuada mauris ac facilisis. Nulla quis suscipit diam, vitae
  semper tortor. Proin nec nulla at massa egestas porta. Cras ac arcu
  in velit placerat facilisis fringilla in sem. Mauris finibus, mauris
  ut pretium malesuada, massa urna mattis nibh, at placerat ligula
  elit ac odio.  Nunc vel leo arcu. Nullam sagittis tincidunt
  interdum. Mauris sed lacus cursus, dapibus enim non, sagittis
  leo. Aliquam sit amet ex ut quam iaculis sollicitudin. Vivamus ut
  pharetra dolor. Vivamus id nibh leo. Vestibulum consectetur eu lorem
  eu tincidunt. Sed eget nibh orci. Pellentesque sit amet blandit
  massa, vel tempor neque.
\end{abstract}

\begin{keywords}
quasars
\end{keywords}

%% \section{Introduction}

%% Possible outline of paper:

%% \begin{enumerate}
%% \item Introduction 
%% \item Data set
%% \item QLF 
%%   \begin{enumerate}
%%   \item at specific redshifts 
%%   \item global model 
%%   \end{enumerate}
%% \item Contribution to reionizatin 
%% \item Conclusion 
%% \end{enumerate}

%% \noindent Optional possibilities:

%% \begin{enumerate}
%% \item Bolometric LF
%% \item PLE, PDE, etc?
%% \item Quantitative model fit test 
%% \item BH growth 
%% \item Comparison with galaxies 
%% \end{enumerate}

\section{Luminosity function}

In a magnitude bin $[M_\mathrm{min}, M_\mathrm{max})$, and redshift
  bin $[z_\mathrm{min}, z_\mathrm{max})$, we define the luminosity
    function as \citep{2000MNRAS.311..433P}
  \begin{equation}
    \phi \equiv \frac{N}{V_\mathrm{bin}},
  \end{equation}
  where $N$ is the number of quasars with magnitude
  $M_\mathrm{min}\leq M<M_\mathrm{max}$ and redshift
  $z_\mathrm{min}\leq z<z_\mathrm{max}$, and
  \begin{equation}
    V_\mathrm{bin} = \int_{M_\mathrm{min}}^{M_\mathrm{max}}dM\int_{z_\mathrm{min}}^{z_\mathrm{max}}dz\, f(M, z)\,\frac{dV}{dz},
    \label{eqn:vi}
  \end{equation}
  is the effective volume of the bin.  Here $f(M,z)$ is the quasar
  selection probability, which includes incompleteness.  Inclusion of
  the selection probability in Equation~(\ref{eqn:vi}) accounts for
  what have been referred to as ``incomplete bins'' in the literature.
  The comoving volume element $dV/dz$ is given by
  \begin{equation}
    \frac{dV}{dz}=\frac{dV}{dz\,d\Omega}\cdot A\cdot\frac{4\pi}{41253},
  \end{equation}
  where $A$ is the survey area in deg$^2$, and \citep{1999astro.ph..5116H}
  \begin{equation}
    \frac{dV}{dz\,d\Omega}=\frac{c}{H_0}\frac{d_L(z)^2}{(1+z)^2\sqrt{\Omega_m(1+z)^3+\Omega_\Lambda}},
    \label{eqn:dvdzdo}
  \end{equation}
  denotes the comoving volume element per unit solid angle.  The
  luminosity distance $d_L$ is given by
  \begin{equation}
    d_L(z)=(1+z)\frac{c}{H_0}\int_0^z\frac{dz}{\sqrt{\Omega_m(1+z)^3+\Omega_\Lambda}}.
    \label{eqn:dl}
  \end{equation}
  Equations~(\ref{eqn:dvdzdo}) and (\ref{eqn:dl}) assume a flat
  Universe ($\Omega_k=0$).  The luminosity function $\phi$ has units
  of $\mathrm{cMpc}^{-3}\mathrm{mag}^{-1}$.  We evaluate the double
  integral in Equation~(\ref{eqn:vi}) by the Euler method, i.e., by
  simply summing over the ``tiles'' of the selection function map.  We
  do not interpolate between the redshift and luminosity values of
  neighbouring tiles.  This may result in $V_i=0$ for some quasars, in
  which case we remove them from our analysis.  Figure~\ref{fig:boss}
  shows the luminosity function derived in this way for the BOSS DR9
  colour-selected sample of 23,301 quasars, in comparison with the
  published luminosity function of \citet{2013ApJ...773...14R}.
  Figure~\ref{fig:sdss} shows the luminosity function of a sample of
  15,179 quasars from the SDSS DR3 in comparison with the published
  luminosity function of \citet{2006AJ....131.2766R}.

  % Describe how we derive error bars.

  \section{Fitting procedure}

  We model the quasar luminosity function as a double power law given
  by \citep{1988MNRAS.235..935B, 1995ApJ...438..623P, 2000MNRAS.317.1014B}
  \begin{equation}
    \phi(M) = \frac{\phi_*}{10^{0.4(\alpha+1)(M-M_*)}+10^{0.4(\beta+1)(M-M_*)}},
\label{eqn:dpl}
\end{equation}
where all magnitudes are absolute magnitudes at rest-frame 1450~{\AA}.  
The luminosity function $\phi$ gives the comoving number density of
quasars per unit magnitude.  It usually has units of
mag$^{-1}$cMpc$^{-3}$.  Equation~(\ref{eqn:dpl}) has four free
parameters: the amplitude $\phi_*$, the ``break'' $M_*$, the faint-end
slope $\beta$, and the bright-end slope $\alpha$.  The likelihood
function can be written as \citep{2001AJ....121...54F}
\begin{equation}
  \mathcal{L}=\prod_{i,j}\frac{e^{-\mu_{ij}}\mu_{ij}^{n_{ij}}}{n_{ij}!},
  \label{eqn:lhood}
\end{equation}
where $n_{ij}$ is the number of quasars observed in the $(M_i, z_j)$
bin, and
\begin{equation}
  \mu_{ij}= \int_{M_i}^{M_{i+1}}dM\int_{z_i}^{z_{i+1}}dz\, \phi(M,z) f(M, z)\,\frac{dV}{dz},
\end{equation}
is the average number of quasars expected in the $(M_i, z_j)$ bin
given the luminosity function $\phi(M,z)$.  To obtain the maximum
likelihood solution for the luminosity functin parameters, we minimise
the function
\begin{equation}
  S = -2\ln\mathcal{L}.
\end{equation}
In the limit of infinitesimal bins, $n_{ij}=0$ or $1$, in which case
Equation~(\ref{eqn:lhood}) can be simplified and we can write \citep{1983ApJ...269...35M}
\begin{multline}
  S = -2\sum_{i=1}^N\ln\phi(M_i, z_i)\\+2\int_{M_\mathrm{min}}^{M_\mathrm{max}}dM\int_{z_\mathrm{min}}^{z_\mathrm{max}}dz\, \phi(M,z) f(M, z)\,\frac{dV}{dz},
\end{multline}
where $N$ is the total number of quasars in the sample and the
integral in the second term on the right hand side is now on the full
range of $M$ and $z$.

\section{Hydrogen photoionization rate}

We assume that all quasars have a universal SED, which can be
parameterised in some fashion.  \citet{2015MNRAS.449.4204L} have
parameterised the quasar SED as a power law with a break at 912~{\AA},
\begin{equation}
f_\nu\propto\begin{cases}
               \nu^{-0.61\pm 0.01} & \text{if}~\lambda\geq 912~\text{\AA},\\
               \nu^{-1.70\pm 0.61} & \text{if}~600~\text{\AA}<\lambda<912~\text{\AA}.\\                
               \end{cases}
\label{eqn:sed}
\end{equation}
Here $f_\nu$ is the flux density, with units of ergs s$^{-1}$
cm$^{-2}$ Hz$^{-1}$, although it is often simply called flux.  Note
that although $f_\nu$ is often the quantity of interest, e.g., when
calculating the hydrogen ionizing emissivity of quasars, what is shown
in spectra is often $f_\lambda$, the flux density
in \emph{wavelength}, which has units of ergs s$^{-1}$
cm$^{-2}$ \AA$^{-1}$.  If the flux density in frequency is power law,
then the flux density in wavelength is also power law.  The power law
index of the flux density in wavelength, $\alpha_\lambda$, is related
to the power law index of the flux density in frequency, $\alpha_\nu$,
by $\alpha_\lambda=-(\alpha_\nu+2)$.  From Equation~(\ref{eqn:sed}),
we can now see that $\alpha_\lambda=-0.3$ in the extreme UV (EUV;
600~{\AA}$<\lambda<$912~{\AA}) and $\alpha_\lambda=-1.39$ in the far
UV (FUV; $\lambda>$912~{\AA}).  The quasar SED that we have adopted is
steep in FUV and shallow in EUV.

How sensible is it to assume that the quasar SED of
Equation~(\ref{eqn:sed}) is universal?  Of the four published
composite quasar SEDs (see references in
\citealt{2015MNRAS.449.4204L}), three agree with
Equation~(\ref{eqn:sed}).  There is one published composite quasar SED
that does not agree with Equation~(\ref{eqn:sed}).  This is the
composite of \citet{2004ApJ...615..135S}, which uses more than 100
quasars at $z<0.1$.  The EUV slope of this SED is $-0.56$.  (This
means that the power law index of the flux density in wavelength is
$-1.44$, so that the spectrum actually becomes even steeper in the
EUV.  See Figure 8 of \citealt{2015MNRAS.449.4204L}.)  The quasars
considered by \citet{2004ApJ...615..135S} are much fainter that those
considered by \citet{2015MNRAS.449.4204L}, which means that it is
possible that faint quasars have a steep EUV spectrum.  So we could
say that for faint quasars,
\begin{equation}
f_\nu\propto\begin{cases}
               \nu^{-0.61\pm 0.01} & \text{if}~\lambda\geq 912~\text{\AA},\\
               \nu^{-0.56\pm 0.61} & \text{if}~600~\text{\AA}<\lambda<912~\text{\AA}.\\                
               \end{cases}
\label{eqn:sed_faint}
\end{equation}
Here, ``faint'' is defined as $M_i(z=2)>-26$.  $M_i(z=2)$ can be
related to the bolometric luminosity of the quasar (see Figure 9 of
\citealt{2015MNRAS.449.4204L}.)

The rate of emission of photons of frequency $\nu$ by quasars can be
written as
\begin{equation}
\dot n_\nu = \int dM_\nu \phi(M_\nu) \frac{L_\nu(M_\nu)}{h\nu}.
\end{equation}
The units of $\dot n_\nu$ are s$^{-1}$~cMpc$^{-3}$~Hz$^{-1}$.  Here
the integral is over some range of magnitudes, the monochromatic
luminosity $L_\nu(M)$ is related to the (absolute AB) magnitude
by \citep{1983ApJ...266..713O}
\begin{equation}
M_\nu = -2.5\log_{10}L_\nu+51.60,
\end{equation}
and $\phi$ is the luminosity function from Equation~(\ref{eqn:dpl}).
The luminosity at 1~Ry is related to that at 1450~\AA\ by
Equation~(\ref{eqn:sed}) or (\ref{eqn:sed_faint})
\begin{equation}
  L_{912}=L_{1450}\left(\frac{\nu_{912}}{\nu_{1450}}\right)^{-0.61}=L_{1450}\left(\frac{912}{1450}\right)^{0.61}
\end{equation}
The corresponding volume emissivity is conventionally written
as 
\begin{equation}
\epsilon_\nu = \dot n_\nu h\nu (1+z)^3.
\label{eqn:epsilon}
\end{equation}
This is a \emph{physical} density.  The units of $\epsilon_\nu$ are
erg~s$^{-1}$~pMpc$^{-3}$~Hz$^{-1}$.  This emissivity can now be used
to calculate the ionizing flux and photoionization rate contributed by
quasars.  The frequency dependence above 1~Ry is given by
Equation~(\ref{eqn:sed}) or (\ref{eqn:sed_faint}), so that for $\nu >
\nu_{912}$
\begin{equation}
  \epsilon_\nu = \epsilon_{912}\left(\frac{\nu}{\nu_{912}}\right)^\alpha,
  \label{eqn:epsilon_freq}
\end{equation}
where $\alpha=-1.70$ (Equation~\ref{eqn:sed}) or $-0.56$ (Equation~\ref{eqn:sed_faint}). 

The emissivity calculated in Equation~(\ref{eqn:epsilon}) is a source
property.  We are now interested in what flux this emissivity produces
in the surrounding intergalactic medium.  This brings in the medium's
properties also, via the radiative transfer equation.  The flux is
\citep{2012ApJ...746..125H}
\begin{multline}
  j(\nu_0, z_0)=\frac{1}{4\pi}\int_{z_0}^\infty dz\frac{dl}{dz}\frac{(1+z_0)^3}{(1+z)^3}\epsilon(\nu,z)\\
  \times\exp{(-\tau_\mathrm{eff}(\nu_0, z_0, z))},
  \label{eqn:flux}
\end{multline}
where
\begin{equation}
  \nu = \nu_0\left(\frac{1+z}{1+z_0}\right).
\end{equation}
We estimate the effective optical depth as
\begin{equation}
  \tau_\mathrm{eff}(\nu_0, z_0, z) = \int_{z_0}^z dz^\prime\int_0^\infty dN_\mathrm{HI} f(N_\mathrm{HI}, z^\prime) (1-e^{-\tau_\nu}),
\end{equation}
where $\tau_\nu=\sigma_\nu N_\mathrm{HI}$ and the column density
distribution $f(N_\mathrm{HI}, z)$ is taken to be a power law \citep{2013MNRAS.436.1023B}
\begin{equation}
  f(N_\mathrm{HI}, z) = \frac{A}{N_\mathrm{LL}}\left(\frac{N_\mathrm{HI}}{N_\mathrm{LL}}\right)^{-\beta_N}\left(\frac{1+z}{4.5}\right)^{-\beta_z},
\end{equation}
with $A=0.93$, $\beta_N=1.33$, $\beta_z=1.92$, and
$N_\mathrm{LL}=10^{17.2}$~cm$^{-2}$.  The hydrogen photoionization
rate is then given by
\begin{equation}
  \Gamma_\mathrm{HI}=4\pi\int_{\nu_{912}}^\infty d\nu \frac{j(\nu,z)}{h\nu} \sigma(\nu),
\end{equation}

\subsection{Local source approximation}

In the local source approximation, Equation~(\ref{eqn:flux}) becomes
\begin{equation}
  j(\nu_0, z_0) = \frac{1}{4\pi}\lambda_\mathrm{mfp}(\nu_0, z_0)\epsilon(\nu_0, z_0),
\end{equation}
where $\lambda_\mathrm{mfp}$ is the mean free path.  The hydrogen
photoionization rate is given by
\begin{equation}
  \Gamma_\mathrm{HI}=4\pi\int_{\nu_{912}}^\infty d\nu \frac{j_\nu}{h\nu} \sigma(\nu),
  \label{eqn:gammapi}
\end{equation}
where $\sigma$ is the ionization cross-section
\begin{equation}
  \sigma(\nu) = \sigma_0\left(\frac{\nu}{\nu_{912}}\right)^{-3},
  \label{eqn:sigma}
\end{equation}
with $\sigma_0=6.3\times 10^{-18}$~cm$^2$ \citep{2006agna.book.....O}.
In the redshift range $z=2.3$--$5.5$, the mean free path is measured
to be \citep{2014MNRAS.445.1745W}
\begin{equation}
  \lambda_\mathrm{mfp}(\nu, z)= \lambda_0\left(\frac{1+z}{5}\right)^{-5.4}\left(\frac{\nu}{\nu_{912}}\right)^{1.5},
  \label{eqn:mfp}
\end{equation}
where the frequency dependence comes from the assumed column density
distribution $f(N_\mathrm{HI})\propto N_\mathrm{HI}^{-1.5}$.  We
extrapolate the redshift dependence in Equation~(\ref{eqn:mfp}) at
$z<2.3$ and $z>5.5$.  Combining Equations~(\ref{eqn:epsilon_freq}),
(\ref{eqn:sigma}), and (\ref{eqn:mfp}), Equation~(\ref{eqn:gammapi})
gives
\begin{multline}
  \Gamma_\mathrm{HI}=4.6\times 10^{-13} \mathrm{s}^{-1} \left(\frac{\epsilon_{912}}{10^{24}\mathrm{erg s^{-1} Hz^{-1} cMpc^{-3}}}\right)\\
  \times\left(\frac{1}{1.5-\alpha}\right)\left(\frac{1+z}{5}\right)^{-2.4}.
\end{multline}
Note that this equation refers to the \emph{comoving} emissivity.  The
local source approximation works well for $z>3$ \citep{2013MNRAS.436.1023B}.

\begin{figure*}
  \begin{center}
    \includegraphics[width=\textwidth]{qsos.pdf}
  \end{center}
  \caption{Quasar samples.}
  \label{fig:qsos}
\end{figure*}

\begin{table*}
  % In published version of this table, we could remove file names and add data homogenisation in Notes.
  % Add z ~ 7 qso when you add it to analysis
  % See data.tex for a version of this table that includes file names
  \caption{Quasar data sets}
  \label{tab:samples}
  \begin{tabular}{crcllrSl}
    \hline
    & ID & $z$ & Survey & Reference & Number & {Area} & Notes \\
    & & & & & of quasars & {(deg$^2$)} & \\
    \hline
    1. & 13 & 0.1--2.2 & SDSS DR7 & \citet{2006AJ....131.2766R} & 48664 & 6248.0 & \\
    2. & 15 & 0.4--2.6 & 2SLAQ SGP & \citet{2009MNRAS.392...19C} & 3663 & 64.2 & \\
    3. & 15  & 0.4--2.6 & --- NGP & \citet{2009MNRAS.392...19C} & 8153 & 127.7 & \\
    4. &  1 & 2.2--3.5 & BOSS DR9 & \citet{2013ApJ...773...14R} & 23301 & 2236.0 & \\
    5. & 13 & 3.7--4.7 & SDSS DR7 & \citet{2006AJ....131.2766R} & 1785 & 6248.0 & Restricted to $z<4.7$ \\
    6. & 17 & 4.7--5.4 & SDSS+WISE & \citet{2016ApJ...829...33Y} & 99 & 14555.0 & Overlaps with 8 for $M_{1450}<-26.73$ \\
    7. &  8 & 4.7--5.1 & SDSS DR7 & \citet{2013ApJ...768..105M} & 148 & 6248.0 & \\
    8. &  8 & 4.7--5.1 & --- Stripe 82 & \citet{2013ApJ...768..105M} & 52 & 235.0 & \\
    9. &  8 & 5.1--5.5 & --- DR7 Extended & \citet{2013ApJ...768..105M} & 28 & 6248.0 & \\
    10. & 8 & 5.1--5.4 & ---  Stripe 82 Extended & \citet{2013ApJ...768..105M} & 10 & 235.0 & \\
    11. & 6 & 3.7--4.8 & NDWFS & \citet{2011ApJ...728L..26G} & 12 & 1.71 & \\
    12. & 6 & 3.8--5.1 & DLS & \citet{2011ApJ...728L..26G} & 12 & 2.05 & \\
    13. & 7 & 4.1--6.3 & CANDELS GOODS-S & \citet{2015AA...578A..83G} & 19 & 0.047 & Rescaled selection probabilities \\
    14. & 18 & 5.8--6.4 & --- Main & \citet{2016ApJ...833..222J} & 24 & 11240.0 & \\
    15. & 18 & 5.9--6.1 & --- Overlap & \citet{2016ApJ...833..222J} & 10 & 4223.0 & \\
    16. & 18 & 5.7--6.1 & --- Stripe 82 & \citet{2016ApJ...833..222J} & 13 & 277.0 & \\
    17. & 10 & 6.0 & CFHQS Deep & \citet{2010AJ....139..906W} & 1 & 4.47 & \\
    18. & 10 & 5.9--6.4 & --- Very Wide & \citet{2010AJ....139..906W} & 16 & 494.0 & \\
    19. & 11 & 6.0--6.2 & Subaru High-$z$ Quasar & \citet{2015ApJ...798...28K} & 2 & 6.5 & \\
  \end{tabular}
\end{table*}


\begin{figure*}
  \begin{center}
    \includegraphics[width=\textwidth,keepaspectratio]{boss.pdf}
  \end{center}
  \caption{Comparison of LF derived by our method with the published
    LF of \citet{2013ApJ...773...14R} for the BOSS DR9 color-selected
    sample. There are 23,301 quasars in this sample ($2.2<z<3.5$).  Of
    these, 231 quasars have $V_i=0$ and are dropped from the
    analysis.}
  \label{fig:boss}
\end{figure*}

\begin{figure*}
  \begin{center}
    \includegraphics[width=\textwidth,keepaspectratio]{sdss.pdf}
  \end{center}
  \caption{Comparison with the published LF of
    \citet{2006AJ....131.2766R} for SDSS DR3.  There are 15,343
    quasars in this sample, of which 155 quasars have $z<0.3$ and 9
    quasars have $z>5$.  Of the remaining 15179 quasars with
    $0.3<z<5$, 60 quasars have $V_i=0$ and are dropped from the
    analysis.  As a result, the green points above contain 15,119
    quasars.  The red circles contain 14,953 quasars.}
  \label{fig:sdss}
\end{figure*}

\begin{figure*}
  \begin{center}
    \includegraphics[width=0.6\textwidth,keepaspectratio]{mcgreer.pdf}
  \end{center}
  \caption{Comparison with the published LF of
    \citet{2013ApJ...768..105M} for $4.7\leq z<5.1$.  There are 148
    quasars in their DR7 sample and 52 quasars in the Stripe 82
    sample. Of these 1 quasar from the Stripe 82 sample has $V_i=0$.
    The published samples of \citet{2013ApJ...768..105M} have 146 and
    51 quasars.}
  \label{fig:mcgreer}
\end{figure*}

%% \begin{figure*}
%%   \begin{center}
%%     \includegraphics[width=0.6\textwidth,keepaspectratio]{glikman.pdf}
%%   \end{center}
%%   \caption{Comparison with the published LF of
%%     \citet{2011ApJ...728L..26G} for $3.5<z<5.2$.  There are 24 
%%     quasars in their two samples (NDWFS and DLS).}
%%   \label{fig:mcgreer}
%% \end{figure*}

\begin{figure*}
  \begin{center}
    \includegraphics[width=\textwidth]{glikman3.pdf}
  \end{center}
  \caption{Attempts at reproducing Glikman's published LF.  Panels (a)
    and (b) use her volume calculations.  Panels (c)--(e) use our
    volume calculations.  Various corrections incorporated by Gabor
    are in panels (c)--(e).  Red open circles in all panels are
    Glikman's published values; I cannot reproduce them in any of the
    panels.  All luminosity function in this version of the draft use
    Gabor's October data from panel (d).}
  \label{fig:glikman}
\end{figure*}

\begin{figure*}
  \begin{center}
    \includegraphics[width=\textwidth]{giallongo2.pdf}
  \end{center}
  \caption{Giallongo's LF when selection function is taken to be
    $\phi_\mathrm{corr}/\phi$ from their paper.  This plots shows that
    additional correction is required, which has been incorporated in
    our LFs.}
  \label{fig:giallongo}
\end{figure*}

\begin{figure*}
  \begin{center}
    \includegraphics[width=\textwidth,keepaspectratio]{mosaic.pdf}
  \end{center}
  \caption{Individual luminosity function fits for $z=0.3$ to $2.6$.}
  \label{fig:mosaic}
\end{figure*}

\begin{figure*}
  \begin{center}
    \includegraphics[width=\textwidth,keepaspectratio]{mosaic2.pdf}
  \end{center}
  \caption{Individual luminosity function fits for $z=2.6$ to $6.5$.}
  \label{fig:mosaic2}
\end{figure*}

\begin{figure*}
  \begin{center}
    \includegraphics[width=0.6\textwidth,keepaspectratio]{rhoqso_allz.pdf}
  \end{center}
  \caption{AGN number density evolution, showing the systematic offset in BOSS quasars.}
  \label{fig:rhoqso_allz}
\end{figure*}

\begin{figure*}
  \begin{center}
    \includegraphics[width=\textwidth]{evolution_alldata.pdf}
  \end{center}
  \caption{Parameter evolution from individual fits in
    Figures~\ref{fig:mosaic} and \ref{fig:mosaic2}.  Parameters from
    BOSS quasars are seen to be systematically offset. \textbf{[Show
        $z < 0.6$ point also?]}}
\end{figure*}

\begin{figure*}
  \begin{center}
    \includegraphics[width=0.6\textwidth,keepaspectratio]{rhoqso.pdf}
  \end{center}
  \caption{AGN number density evolution with some BOSS quasars
    removed.}
  \label{fig:rhoqso}
\end{figure*}

\begin{figure*}
  \begin{center}
    \includegraphics[width=\textwidth]{evolution_seldata.pdf}
  \end{center}
  \caption{Parameter evolution with some BOSS quasars removed.  We (1)
    use \emph{only} BOSS quasars for $z=2.2$--$3.5$, and (2) use only
    $p>0.9$ quasars in the $z>3.7$ sample of Richards.}
\end{figure*}

\begin{figure*}
  \begin{center}
    \includegraphics[width=\textwidth,keepaspectratio]{mosaic_withGlobal.pdf}
  \end{center}
  \caption{Individual and composite luminosity function fits for
    $z=0.3$ to $2.6$.  They do not agree for $2.2 < z < 2.6$ because
    composite fit does not consider quasars at those redshifts.}
  \label{fig:mosaic_withGlobal}
\end{figure*}

\begin{figure*}
  \begin{center}
    \includegraphics[width=\textwidth,keepaspectratio]{mosaic2_withGlobal.pdf}
  \end{center}
  \caption{Individual and composite luminosity function fits for
    $z=2.6$ to $6.5$.  They do not agree for $2.6 < z < 2.8$ because
    composite fit does not consider quasars at those redshifts.}
  \label{fig:mosaic2_withGlobal}
\end{figure*}

\begin{figure*}
  \begin{center}
    \includegraphics[width=\textwidth]{evolution_global.pdf}
  \end{center}
  \caption{Luminosity function parameter evolution in the global model.}
\end{figure*}

\begin{figure*}
  \begin{center}
    \includegraphics[width=0.6\textwidth,keepaspectratio]{rhoqso_withGlobal.pdf}
  \end{center}
  \caption{AGN number density evolution in the global model.}
  \label{fig:rhoqso}
\end{figure*}

\begin{figure*}
  \begin{center}
    \begin{tabular}{cc}
    \includegraphics[width=0.5\textwidth]{emissivity_18.pdf}
    \includegraphics[width=0.5\textwidth]{emissivity_20.pdf}
    \end{tabular}
  \end{center}
  \caption{LyC emissivity of AGN assuming 100\% escape fraction.
    Model luminosity functions are integrated down to $M_{1450}=-18$
    in the left panel and $-20$ in the right panel.}
  \label{fig:emissivity}
\end{figure*}

\begin{figure*}
  \begin{center}
    \begin{tabular}{cc}
    \includegraphics[width=0.5\textwidth]{g_18.pdf}
    \includegraphics[width=0.5\textwidth]{g_20.pdf}
    \end{tabular}
  \end{center}
  \caption{AGN contribution to the hydrogen photoionisation rate,
    assuming 100\% escape fraction.  Model luminosity functions are
    integrated down to $M_{1450}=-18$ in the left panel and $-20$ in
    the right panel.}
  \label{fig:gammapi}
\end{figure*}

\begin{figure*}
  \begin{center}
    \includegraphics[width=\textwidth,keepaspectratio]{evolution.pdf}
  \end{center}
  \caption{Parameter evolution from individual fits.  Coloured points
    show results when Giallongo quasars are not included.  Black
    points show results when Giallongo quasars are included.}
\end{figure*}

\section*{Acknowledgements}

Lorem ipsum dolor sit amet, consectetur adipiscing elit. Curabitur
eget nisi augue. Vivamus quis purus quis massa tempor posuere non quis
magna. Aenean eleifend, metus eget facilisis faucibus, turpis erat
suscipit tortor, ut dapibus nulla neque ac sapien. Suspendisse luctus
eros eu quam laoreet, vel dapibus sem porttitor. Mauris nec massa
ultrices, porttitor nulla at, euismod diam. In ultricies

\bibliographystyle{mnras}
\bibliography{refs}

\bsp
\label{lastpage}
\end{document}


