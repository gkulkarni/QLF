\documentclass[a4paper,fleqn,usenatbib]{mnras}
\usepackage[T1]{fontenc}
\usepackage{ae,aecompl}
\usepackage{graphicx}	
\usepackage{amsmath}	
\usepackage{amssymb}
\usepackage{pdflscape}

\title[AGN luminosity function]{Evolution of the AGN UV luminosity function}

\author[Kulkarni et al.]{{Girish Kulkarni$^{1}$\thanks{Email:
      kulkarni@ast.cam.ac.uk}, G\'abor Worseck$^{2}$ and Joseph
    F.~Hennawi$^{3}$} \\ $^1$Institute of Astronomy and Kavli
  Institute of Cosmology, University of Cambridge, Madingley Road,
  Cambridge CB3 0HA, UK \\ $^2$Max Planck Institute for Astronomy,
  K\"onigstuhl 17, D-69117 Heidelberg, Germany\\ $^3$Department of
  Physics, Broida Hall, UC Santa Barbara, Santa Barbara, CA 93106-9530
  USA}

\date{Accepted ---. Received ---; in original form ---}

\pubyear{2016}

\begin{document}
\label{firstpage}
\pagerange{\pageref{firstpage}--\pageref{lastpage}}
\maketitle

\begin{abstract}
  Lorem ipsum dolor sit amet, consectetur adipiscing elit. Curabitur
  eget nisi augue. Vivamus quis purus quis massa tempor posuere non
  quis magna. Aenean eleifend, metus eget facilisis faucibus, turpis
  erat suscipit tortor, ut dapibus nulla neque ac sapien. Suspendisse
  luctus eros eu quam laoreet, vel dapibus sem porttitor. Mauris nec
  massa ultrices, porttitor nulla at, euismod diam. In ultricies
  malesuada mauris ac facilisis. Nulla quis suscipit diam, vitae
  semper tortor. Proin nec nulla at massa egestas porta. Cras ac arcu
  in velit placerat facilisis fringilla in sem. Mauris finibus, mauris
  ut pretium malesuada, massa urna mattis nibh, at placerat ligula
  elit ac odio.  Nunc vel leo arcu. Nullam sagittis tincidunt
  interdum. Mauris sed lacus cursus, dapibus enim non, sagittis
  leo. Aliquam sit amet ex ut quam iaculis sollicitudin. Vivamus ut
  pharetra dolor. Vivamus id nibh leo. Vestibulum consectetur eu lorem
  eu tincidunt. Sed eget nibh orci. Pellentesque sit amet blandit
  massa, vel tempor neque.
\end{abstract}

\begin{keywords}
quasars
\end{keywords}

%% \section{Introduction}

%% Possible outline of paper:

%% \begin{enumerate}
%% \item Introduction 
%% \item Data set
%% \item QLF 
%%   \begin{enumerate}
%%   \item at specific redshifts 
%%   \item global model 
%%   \end{enumerate}
%% \item Contribution to reionizatin 
%% \item Conclusion 
%% \end{enumerate}

%% \noindent Optional possibilities:

%% \begin{enumerate}
%% \item Bolometric LF
%% \item PLE, PDE, etc?
%% \item Quantitative model fit test 
%% \item BH growth 
%% \item Comparison with galaxies 
%% \end{enumerate}

\section{Luminosity function}

In a magnitude bin $[M_\mathrm{min}, M_\mathrm{max})$, and redshift
  bin $[z_\mathrm{min}, z_\mathrm{max})$, we define the luminosity
    function as \citep{2000MNRAS.311..433P}
  \begin{equation}
    \phi \equiv \frac{N}{V_\mathrm{bin}},
  \end{equation}
  where $N$ is the number of quasars with magnitude
  $M_\mathrm{min}\leq M<M_\mathrm{max}$ and redshift
  $z_\mathrm{min}\leq z<z_\mathrm{max}$, and
  \begin{equation}
    V_\mathrm{bin} = \int_{M_\mathrm{min}}^{M_\mathrm{max}}dM\int_{z_\mathrm{min}}^{z_\mathrm{max}}dz\, f(M, z)\,\frac{dV}{dz},
    \label{eqn:vi}
  \end{equation}
  is the effective volume of the bin.  Here $f(M,z)$ is the quasar
  selection probability, which includes incompleteness.  Inclusion of
  the selection probability in Equation~(\ref{eqn:vi}) accounts for
  what have been referred to as ``incomplete bins'' in the literature.
  The comoving volume element $dV/dz$ is given by
  \begin{equation}
    \frac{dV}{dz}=\frac{dV}{dz\,d\Omega}\cdot A\cdot\frac{4\pi}{41253},
  \end{equation}
  where $A$ is the survey area in deg$^2$, and \citep{1999astro.ph..5116H}
  \begin{equation}
    \frac{dV}{dz\,d\Omega}=\frac{c}{H_0}\frac{d_L(z)^2}{(1+z)^2\sqrt{\Omega_m(1+z)^3+\Omega_\Lambda}},
    \label{eqn:dvdzdo}
  \end{equation}
  denotes the comoving volume element per unit solid angle.  The
  luminosity distance $d_L$ is given by
  \begin{equation}
    d_L(z)=(1+z)\frac{c}{H_0}\int_0^z\frac{dz}{\sqrt{\Omega_m(1+z)^3+\Omega_\Lambda}}.
    \label{eqn:dl}
  \end{equation}
  Equations~(\ref{eqn:dvdzdo}) and (\ref{eqn:dl}) assume a flat
  Universe ($\Omega_k=0$).  The luminosity function $\phi$ has units
  of $\mathrm{cMpc}^{-3}\mathrm{mag}^{-1}$.  We evaluate the double
  integral in Equation~(\ref{eqn:vi}) by the Euler method, i.e., by
  simply summing over the ``tiles'' of the selection function map.  We
  do not interpolate between the redshift and luminosity values of
  neighbouring tiles.  This may result in $V_i=0$ for some quasars, in
  which case we remove them from our analysis.  Figure~\ref{fig:boss}
  shows the luminosity function derived in this way for the BOSS DR9
  colour-selected sample of 23,301 quasars, in comparison with the
  published luminosity function of \citet{2013ApJ...773...14R}.
  Figure~\ref{fig:sdss} shows the luminosity function of a sample of
  15,179 quasars from the SDSS DR3 in comparison with the published
  luminosity function of \citet{2006AJ....131.2766R}.

  % Describe how we derive error bars.

  \section{Fitting procedure}

  We model the quasar luminosity function as a double power law given
  by \citep[e.g.,][]{2013ApJ...773...14R}
  \begin{equation}
    \phi(M) = \frac{\phi_*}{10^{0.4(\alpha+1)(M-M_*)}+10^{0.4(\beta+1)(M-M_*)}},
\label{eqn:dpl}
\end{equation}
where all magnitudes are absolute magnitudes at rest-frame 1450~{\AA}.
The luminosity function $\phi$ gives the comoving number density of
quasars per unit magnitude.  It usually has units of
mag$^{-1}$cMpc$^{-3}$.  Equation~(\ref{eqn:dpl}) has four free
parameters: the amplitude $\phi_*$, the ``break'' $M_*$, the faint-end
slope $\beta$, and the bright-end slope $\alpha$.  The likelihood
function can be written as \citep{2001AJ....121...54F}
\begin{equation}
  \mathcal{L}=\prod_{i,j}\frac{e^{-\mu_{ij}}\mu_{ij}^{n_{ij}}}{n_{ij}!},
  \label{eqn:lhood}
\end{equation}
where $n_{ij}$ is the number of quasars observed in the $(M_i, z_j)$
bin, and
\begin{equation}
  \mu_{ij}= \int_{M_i}^{M_{i+1}}dM\int_{z_i}^{z_{i+1}}dz\, \phi(M,z) f(M, z)\,\frac{dV}{dz},
\end{equation}
is the average number of quasars expected in the $(M_i, z_j)$ bin
given the luminosity function $\phi(M,z)$.  To obtain the maximum
likelihood solution for the luminosity functin parameters, we minimise
the function
\begin{equation}
  S = -2\ln\mathcal{L}.
\end{equation}
In the limit of infinitesimal bins, $n_{ij}=0$ or $1$, in which case
Equation~(\ref{eqn:lhood}) can be simplified and we can write
\begin{multline}
  S = -2\sum_{i=1}^N\ln\phi(M_i, z_i)\\+2\int_{M_\mathrm{min}}^{M_\mathrm{max}}dM\int_{z_\mathrm{min}}^{z_\mathrm{max}}dz\, \phi(M,z) f(M, z)\,\frac{dV}{dz},
\end{multline}
where $N$ is the total number of quasars in the sample and the integral in the second term on the right hand side is now on the full range of $M$ and $z$.
  

\begin{figure*}
  \begin{center}
    \includegraphics[width=\textwidth,keepaspectratio]{boss.pdf}
  \end{center}
  \caption{Comparison of LF derived by our method with the published
    LF of \citet{2013ApJ...773...14R} for the BOSS DR9 color-selected
    sample. There are 23,301 quasars in this sample ($2.2<z<3.5$).  Of
    these, 231 quasars have $V_i=0$ and are dropped from the
    analysis.}
  \label{fig:boss}
\end{figure*}

\begin{figure*}
  \begin{center}
    \includegraphics[width=\textwidth,keepaspectratio]{sdss.pdf}
  \end{center}
  \caption{Comparison with the published LF of
    \citet{2013ApJ...768..105M} for SDSS DR3.  There are 15,343
    quasars in this sample, of which 155 quasars have $z<0.3$ and 9
    quasars have $z>5$.  Of the remaining 15179 quasars with
    $0.3<z<5$, 60 quasars have $V_i=0$ and are dropped from the
    analysis.  As a result, the green points above contain 15,119
    quasars.  The red circles contain 14,953 quasars.}
  \label{fig:sdss}
\end{figure*}

\begin{figure*}
  \begin{center}
    \includegraphics[width=0.6\textwidth,keepaspectratio]{mcgreer.pdf}
  \end{center}
  \caption{Comparison with the published LF of
    \citet{2013ApJ...768..105M} for $4.7\leq z<5.1$.  There are 148
    quasars in their DR7 sample and 52 quasars in the Stripe 82
    sample. Of these 1 quasar from the Stripe 82 sample has $V_i=0$.
    The published samples of \citet{2013ApJ...768..105M} have 146 and
    51 quasars.}
  \label{fig:mcgreer}
\end{figure*}

\begin{figure*}
  \begin{center}
    \includegraphics[width=0.6\textwidth,keepaspectratio]{glikman.pdf}
  \end{center}
  \caption{Comparison with the published LF of
    \citet{2011ApJ...728L..26G} for $3.5<z<5.2$.  There are 24 
    quasars in their two samples (NDWFS and DLS).}
  \label{fig:mcgreer}
\end{figure*}

\begin{figure*}
  \begin{center}
    \includegraphics[width=\textwidth]{qsos.pdf}
  \end{center}
  \caption{Quasar samples.}
  \label{fig:qsos}
\end{figure*}

\begin{landscape}
  \begin{table}
    % In published version of this table, we could remove file names and add data homogenisation in Notes.
    % Add z ~ 7 qso when you add it to analysis
    \caption{Quasar data sets}
    \label{tab:samples}
    \begin{tabular}{cccllcclll}
      \hline
      & ID & $z$ & Survey & Reference & Number & Area & Sample file & Selection map file & Notes \\
      & & & & & of quasars & (deg$^2$) & & & \\
      \hline
      1. & 13 & 0.1--2.2 & SDSS DR7 & \citet{2006AJ....131.2766R} & 48664 & 6248.0 & \texttt{dr7z2p2\_sample.dat} & \texttt{dr7z2p2\_selfunc.dat} &  \\
      2. & 15 & 0.4--2.6 & 2SLAQ SGP & \citet{2009MNRAS.392...19C} & 3663 & 64.2 & \texttt{croom09sgp\_sample.dat} & \texttt{croom09sgp\_selfunc.dat} & \\ 
      3. &  2 & 0.4--2.6 & --- NGP & \citet{2009MNRAS.392...19C} & 8153 & 127.7 & \texttt{croom09ngp\_sample.dat} & \texttt{croom09ngp\_selfunc.dat} & \\
      4. &  1 & 2.2--3.5 & BOSS DR9 & \citet{2013ApJ...773...14R} & 23301 & 2236.0 & \texttt{bossdr9color.dat} & \texttt{ross13\_selfunc2.dat} & \\
      5. & 13 & 3.7--4.7 & SDSS DR7 & \citet{2006AJ....131.2766R} & 1785 & 6248.0 & \texttt{dr7z3p7\_sample.dat} & \texttt{dr7z3p7\_selfunc.dat} & Sample restricted to $z<4.7$ \\
      6. & 17 & 4.7--5.4 & SDSS+WISE & \citet{2016ApJ...829...33Y} & 99 & 14555.0 & \texttt{yang16\_sample.dat} & \texttt{yang16\_sel.dat} & Overlaps with 8 for $M_{1450}>-26.73$? \\
      7. &  8 & 4.7--5.1 & SDSS DR7 & \citet{2013ApJ...768..105M} & 148 & 6248.0 & \texttt{mcgreer13\_dr7sample.dat} & \texttt{mcgreer13\_dr7selfunc.dat} & \\
      8. &  4 & 4.7--5.1 & --- Stripe 82 & \citet{2013ApJ...768..105M} & 52 & 235.0 & \texttt{mcgreer13\_s82sample.dat} & \texttt{mcgreer13\_s82selfunc.dat} & \\
      9. &  8 & 5.1--5.5 & --- DR7 Extended & \citet{2013ApJ...768..105M} & 28 & 6248.0 & \texttt{mcgreer13\_dr7extend.dat} & \texttt{mcgreer13\_dr7selfunc.dat} & \\
      10. & 4 & 5.1--5.4 & ---  Stripe 82 Extended & \citet{2013ApJ...768..105M} & 10 & 235.0 & \texttt{mcgreer13\_s82extend.dat} & \texttt{mcgreer13\_s82selfunc.dat} & \\
      11. & 3 & 3.7--4.8 & NDWFS & \citet{2011ApJ...728L..26G} & 12 & 1.71 & \texttt{glikman11qso.dat} & \texttt{glikman11\_selfunc\_ndwfs\_old.dat} & \\
%         &  &  &  &  &  &  & \texttt{glikman11debug.dat} & \texttt{glikman11\_selfunc\_ndwfs.dat} & \\ 
      12. & 6 & 3.8--5.1 & DLS & \citet{2011ApJ...728L..26G} & 12 & 2.05 & \texttt{glikman11qso.dat} & \texttt{glikman11\_selfunc\_dls\_old.dat} & \\
%      &  &  &  &  &  &  & \texttt{glikman11debug.dat} & \texttt{glikman11\_selfunc\_ndwfs.dat} & \\
      13. & 7 & 4.1--6.3 & CANDELS GOODS-S & \citet{2015AA...578A..83G} & 19 & 0.047 & \texttt{giallongo15\_sample.dat} & \texttt{giallongo15\_sel.dat} &  \\
      14. & 9 & 5.9--6.1 & SDSS Deep & \citet{2008AJ....135.1057J} & 6 & 260.0 & \texttt{jiang08\_sample.dat} & \texttt{jiang08\_sel.dat} \\
      15. & 5 & 5.8--6.0 & --- Deep & \citet{2009AJ....138..305J} & 4 & 195.0 & \texttt{jiang09\_sample.dat} & \texttt{jiang09\_sel.dat} \\
      16. & 18 & 5.8--6.4 & --- Main & \citet{2016ApJ...833..222J} & 24 & 11240.0 & \texttt{jiang16main\_sample.dat} & \texttt{jiang16main\_selfunc.dat} \\
      17. & 16 & 5.8--6.4 & --- Main & \citet{2006AJ....131.1203F} & 14 & 6600.0 & \texttt{fan06\_sample.dat} & \texttt{fan06\_sel.dat} & Any overlap with 18? \\
      18. & 19 & 5.9--6.1 & --- Overlap & \citet{2016ApJ...833..222J} & 10 & 4223.0 & \texttt{jiang16overlap\_sample.dat} & \texttt{jiang16overlap\_selfunc.dat} \\
      19. & 20 & 5.7--6.1 & --- Stripe 82 & \citet{2016ApJ...833..222J} & 13 & 277.0 & \texttt{jiang16s82\_sample.dat} & \texttt{jiang16s82\_selfunc.dat} \\
      20. & 10 & 6.0 & CFHQS Deep & \citet{2010AJ....139..906W} & 1 & 4.47 & \texttt{willott10\_cfhqsdeepsample.dat} & \texttt{willott10\_cfhqsdeepsel.dat} \\
      21. & 21 & 5.9--6.4 & --- Very Wide & \citet{2010AJ....139..906W} & 16 & 494.0 & \texttt{willott10\_cfhqsvwsample.dat} & \texttt{willott10\_cfhqsvwsel.dat} \\
      22. & 11 & 6.0--6.2 & Subaru High-$z$ Quasar & \citet{2015ApJ...798...28K} & 2 & 6.5 & \texttt{kashikawa15\_sample.dat} & \texttt{kashikawa15\_sel.dat} \\ 
      
    \end{tabular}
  \end{table}
\end{landscape}

\begin{figure*}
  \begin{center}
    \begin{tabular}{cc}
      \includegraphics[width=0.4\textwidth,keepaspectratio]{{lf_z0.495}.pdf} &
      \includegraphics[width=0.4\textwidth,keepaspectratio]{{lf_z0.881}.pdf} \\
      
      \includegraphics[width=0.4\textwidth,keepaspectratio]{{lf_z1.252}.pdf} &
      \includegraphics[width=0.4\textwidth,keepaspectratio]{{lf_z1.626}.pdf} \\
      
      \includegraphics[width=0.4\textwidth,keepaspectratio]{{lf_z1.987}.pdf} &
      \includegraphics[width=0.4\textwidth,keepaspectratio]{{lf_z2.254}.pdf} \\
    \end{tabular}
  \end{center}
  \caption{Individual fits.  Something wrong with Croom's SGP data?}
\end{figure*}

\begin{figure*}
  \begin{center}
    \begin{tabular}{cc}
      \includegraphics[width=0.4\textwidth,keepaspectratio]{{lf_z2.348}.pdf} &
      \includegraphics[width=0.4\textwidth,keepaspectratio]{{lf_z2.446}.pdf} \\
      
      \includegraphics[width=0.4\textwidth,keepaspectratio]{{lf_z2.548}.pdf} &
      \includegraphics[width=0.4\textwidth,keepaspectratio]{{lf_z2.645}.pdf} \\
      
      \includegraphics[width=0.4\textwidth,keepaspectratio]{{lf_z2.746}.pdf} &
      \includegraphics[width=0.4\textwidth,keepaspectratio]{{lf_z2.901}.pdf} \\
    \end{tabular}
  \end{center}
  \caption{Individual fits.}
\end{figure*}

\begin{figure*}
  \begin{center}
    \begin{tabular}{cc}
      \includegraphics[width=0.4\textwidth,keepaspectratio]{{lf_z3.119}.pdf} &
      \includegraphics[width=0.4\textwidth,keepaspectratio]{{lf_z3.333}.pdf} \\
      
      \includegraphics[width=0.4\textwidth,keepaspectratio]{{lf_z3.831}.pdf} &
      \includegraphics[width=0.4\textwidth,keepaspectratio]{{lf_z4.196}.pdf} \\
      
      \includegraphics[width=0.4\textwidth,keepaspectratio]{{lf_z4.837}.pdf} &
      \includegraphics[width=0.4\textwidth,keepaspectratio]{{lf_z6.001}.pdf} \\
    \end{tabular}
  \end{center}
  \caption{Individual fits.}
\end{figure*}

\begin{figure*}
  \begin{center}
    \includegraphics[width=\textwidth,keepaspectratio]{evolution.pdf}
  \end{center}
  \caption{Parameter evolution from individual fits.  Giallongo's qsos not included.}
\end{figure*}


\section*{Acknowledgements}

Lorem ipsum dolor sit amet, consectetur adipiscing elit. Curabitur
eget nisi augue. Vivamus quis purus quis massa tempor posuere non quis
magna. Aenean eleifend, metus eget facilisis faucibus, turpis erat
suscipit tortor, ut dapibus nulla neque ac sapien. Suspendisse luctus
eros eu quam laoreet, vel dapibus sem porttitor. Mauris nec massa
ultrices, porttitor nulla at, euismod diam. In ultricies

\bibliographystyle{mnras}
\bibliography{refs}


\bsp
\label{lastpage}
\end{document}


